%%%%%%%%%%%%%%%%%%%%%%%%%%%%%%%%%%%%%%%%%%%%%%%%%%%%%%%%%%%%%%%%%%%%
%
%   Modern Minimalist Math Textbook Template — tcolorbox (breakable)
%
%   Diseñado por: AI Assistant
%   Motor: pdfLaTeX
%   Licencia: MIT
%
%   Cambios en esta versión:
%   - Tema cambiado a "Análisis Complejo"
%   - Paleta de colores actualizada (azules profundos + acentos magenta)
%   - Estructura y estilos: sin cambios
%
%%%%%%%%%%%%%%%%%%%%%%%%%%%%%%%%%%%%%%%%%%%%%%%%%%%%%%%%%%%%%%%%%%%%

\documentclass[12pt, a4paper, oneside]{book}

%-------------------------------------------------------------------
%   CORE PACKAGES
%-------------------------------------------------------------------
\usepackage[utf8]{inputenc}
\usepackage[T1]{fontenc}
\usepackage{lmodern}

\usepackage{amsmath, amssymb, amsthm}
\usepackage{graphicx}
\usepackage{xcolor}
\usepackage{tikz}
\usepackage{tikz-3dplot}
\usepackage{float}
\usepackage{varwidth}

% Tikz
\usetikzlibrary{patterns}
\usetikzlibrary{arrows.meta,calc,angles,quotes,intersections,angles,quotes}
\usepackage{dsfont}
% based on 
% https://tex.stackexchange.com/a/38995/121799 
% https://tex.stackexchange.com/a/76216 
% https://tex.stackexchange.com/a/59168/194703 
% https://tex.stackexchange.com/q/448920/194703 
\makeatletter 
\tikzset{ 
reuse path/.code={\pgfsyssoftpath@setcurrentpath{#1}} 
} 
\tikzset{even odd clip/.code={\pgfseteorule}, 
protect/.code={ 
\clip[overlay,even odd clip,reuse path=#1] 
(current bounding box.south west) rectangle (current bounding box.north east)
; 
}} 
\makeatother 
\usetikzlibrary{3d,arrows.meta,decorations.markings,perspective}
\tikzset{->-/.style={decoration={% https://tex.stackexchange.com/a/39282/194703
  markings,
  mark=at position #1 with {\arrow{>}}},postaction={decorate}},
  ->-/.default=0.55}

% Unidades
\usepackage{siunitx}

% Dice
\usepackage{epsdice}
\newcommand\vcdice[1]{\vcenter{\hbox{\epsdice{#1}}}}

% Varwidth
\newenvironment{centeredvarwidth}[1][\linewidth]{%
  \begin{center}
    \begin{varwidth}{#1}
}{%
    \end{varwidth}
  \end{center}
}

%-------------------------------------------------------------------
%   COLOR PALETTE (Actualizada)
%-------------------------------------------------------------------
% Paleta: primario azul profundo, acento magenta, fondos muy claros
\definecolor{primarycolor}{rgb}{0.12,0.28,0.70}    % Azul profundo
\definecolor{accentcolor}{rgb}{0.78,0.10,0.46}     % Magenta sobrio
\definecolor{boxbgcolor}{rgb}{0.975,0.98,0.995}    % Fondo muy claro azulado
\definecolor{boxlinecolor}{rgb}{0.10,0.18,0.55}    % Azul más oscuro (no esencial)

% Enfasis visual usa el color primario
\renewcommand{\emph}[1]{\textbf{\textit{\textcolor{primarycolor}{#1}}}}

% SPANISH LANGUAGE SUPPORT
\usepackage[spanish, es-noquoting, es-noshorthands]{babel}

%-------------------------------------------------------------------
%   PAGE LAYOUT & TYPOGRAPHY
%-------------------------------------------------------------------
\usepackage[
    a4paper,
    left=3cm,
    right=3cm,
    top=2.5cm,
    bottom=2.5cm,
    headheight=16pt
]{geometry}

\usepackage{microtype}

\linespread{1.241}

\usepackage{fancyhdr}
\pagestyle{fancy}
\fancyhf{}
\fancyhead[L]{\sffamily\bfseries\rightmark}
\fancyhead[R]{\sffamily\bfseries\thepage}
\renewcommand{\headrulewidth}{0.5pt}
\renewcommand{\footrulewidth}{0pt}
\fancypagestyle{plain}{
    \fancyhf{}
    \renewcommand{\headrulewidth}{0 pt}
}

%-------------------------------------------------------------------
%   CHAPTER AND SECTION TITLE DESIGN
%-------------------------------------------------------------------
\usepackage{titlesec}

% --- Chapter Title ---
\titleformat{\chapter}[display]
  {\sffamily\Huge\bfseries\color{gray!80}}
  {\filleft\fontsize{80}{80}\selectfont\thechapter}
  {20pt}
  {\filleft\color{primarycolor}}
\titlespacing*{\chapter}{0pt}{-40pt}{40pt}

% --- Section and Subsection Titles ---
\titleformat{\section}
  {\sffamily\Large\bfseries\color{primarycolor}}
  {\thesection}
  {1em}
  {}
\titleformat{\subsection}
  {\sffamily\large\bfseries\color{gray!85}}
  {\thesubsection}
  {1em}
  {}
\titleformat{\subsubsection}
  {\sffamily\bfseries\color{gray!70}}
  {\thesubsubsection}
  {1em}
  {}
  
\titlespacing{\section}{0pt}{3.5ex plus 1ex minus .2ex}{2.3ex plus .2ex}
\titlespacing{\subsection}{0pt}{3.25ex plus 1ex minus .2ex}{1.5ex plus .2ex}

%-------------------------------------------------------------------
%   THEOREM STYLES (amsthm) — unchanged
%-------------------------------------------------------------------
\newtheoremstyle{mdtheorem}
  {}
  {}
  {\itshape}
  {}
  {\sffamily\bfseries}
  {.}
  {.5em}
  {}

\theoremstyle{mdtheorem}
\newtheorem{theorem}{Teorema}[chapter]
\newtheorem{corollary}[theorem]{Corolario}
\newtheorem{lemma}[theorem]{Lema}
\newtheorem{proposition}[theorem]{Proposición}
\newtheorem{conjecture}[theorem]{Conjetura}
\newtheorem{claim}[theorem]{Afirmación}

\newtheoremstyle{mddefinition}
  {}
  {}
  {\normalfont}
  {}
  {\sffamily\bfseries}
  {.}
  {.5em}
  {}

\theoremstyle{mddefinition}
\newtheorem{definition}[theorem]{Definición}
\newtheorem{example}[theorem]{Ejemplo}
\theoremstyle{remark}
\newtheorem{remark}[theorem]{Observación}
\newtheorem{problem}[theorem]{Problema}

%-------------------------------------------------------------------
%   BOXES WITH tcolorbox (breakable)
%-------------------------------------------------------------------
\usepackage[most]{tcolorbox}
\tcbuselibrary{skins, breakable, theorems, hooks}

\tcbset{
  theobox/.style={
    enhanced, breakable,
    colback=boxbgcolor,
    frame empty, boxrule=0pt,
    borderline west={2pt}{0pt}{primarycolor!50!accentcolor},
    left=15pt, right=15pt, top=10pt, bottom=10pt,
    before upper=\vspace{2pt},
    before skip=\topsep, after skip=\topsep,
  },
  infobox/.style={
    enhanced, breakable,
    colback=boxbgcolor,
    frame empty, boxrule=0pt,
    borderline west={2pt}{0pt}{primarycolor},
    left=15pt, right=15pt, top=10pt, bottom=10pt,
    before upper=\vspace{2pt},
    before skip=\topsep, after skip=\topsep,
  },
}

\tcolorboxenvironment{theorem}{theobox}
\tcolorboxenvironment{corollary}{theobox}
\tcolorboxenvironment{lemma}{theobox}
\tcolorboxenvironment{proposition}{theobox}
\tcolorboxenvironment{conjecture}{theobox}
\tcolorboxenvironment{claim}{theobox}

\tcolorboxenvironment{definition}{infobox}
\tcolorboxenvironment{example}{infobox}
\tcolorboxenvironment{problem}{infobox}
% \tcolorboxenvironment{remark}{infobox}

%-------------------------------------------------------------------
%   HYPERLINKS (metadatos actualizados)
%-------------------------------------------------------------------
\usepackage{hyperref}
\hypersetup{
    colorlinks=true,
    linkcolor=primarycolor,
    citecolor=primarycolor,
    urlcolor=accentcolor,
    pdftitle={Análisis Complejo},
    pdfauthor={Ramiro Dibur},
    pdfsubject={Libro de texto de Matemáticas},
    pdfkeywords={matemáticas, análisis complejo, funciones holomorfas, integral de contorno},
    bookmarksopen=true,
    bookmarksnumbered=true
}

%-------------------------------------------------------------------
%   COMMANDS
%-------------------------------------------------------------------
\newcommand{\N}{\mathbb{N}}
\newcommand{\Z}{\mathbb{Z}}
\newcommand{\Q}{\mathbb{Q}}
\newcommand{\R}{\mathbb{R}}
\newcommand{\C}{\mathbb{C}}
\newcommand{\D}{\mathbb{D}}
\renewcommand{\Re}{\operatorname{Re}}
\renewcommand{\Im}{\operatorname{Im}}
\DeclareMathOperator{\area}{\acute{a}rea}
\DeclareMathOperator{\Arg}{Arg}
\DeclareMathOperator{\Log}{Log}



%-------------------------------------------------------------------
%   DOCUMENT START
%-------------------------------------------------------------------
\begin{document}

%==================================================================
%   FRONT MATTER
%==================================================================
\frontmatter

\begin{titlepage}
    \begin{center}
        \vspace*{1cm}
        
        \sffamily
        \bfseries
        \textcolor{primarycolor}{\rule{\linewidth}{2pt}}
        \vspace{0.4cm}
        
        {\Huge Análisis Complejo \par}
        
        \vspace{0.4cm}
        \textcolor{primarycolor}{\rule{\linewidth}{2pt}}
        
        \vspace{2cm}
        
        {\Large Ramiro Dibur \par}
        \vspace{1cm}
        {\large 2025 \par}
    \end{center}
\end{titlepage}

\chapter*{Aclaración}

Estas son mis notas de Análisis Complejo del segundo cuatrimestre de 2025. No son la gran cosa, pero me funcionan a \text{mí}. Por lo tanto, recomiendo cierto escepticismo al leer dado que seguramente haya errores de todo tipo ---mi mayor miedo son los de ortografía---.

También, quiero remarcar que me inspiré fuertemente en los apuntes de Luca Martínez. ¡Péguenles un vistazo!

Suerte.

\tableofcontents

%==================================================================
%   MAIN MATTER
%==================================================================

\mainmatter

\chapter{Números complejos}

Comenzamos con un repaso de la definición de los números complejos, algunas funciones relevantes, propiedades, conceptos topológicos, la esfera de Riemann y homografías.

\section{Definición y propiedades}

Recordemos cómo se definien los números comlejos.

\begin{definition}
    Definimos a los \emph{números complejos} como el conjunto
    \begin{equation*}
        \C = \{ a + bi \mid a, b \in \R \} \quad \text{donde } i^2 = -1 \text{ es la unidad imaginaria},
    \end{equation*}
    provisto de la suma
    \begin{equation*}
        (a + bi) + (c + di) = (a + b) + (c + d)i
    \end{equation*}
    y el producto
    \begin{equation*}
        (a + bi) \cdot (c + di) = (ac - bd) + (ad + bc)i.
    \end{equation*}
\end{definition}

\begin{remark}
    Si tratamos a un número complejo como una expresión algebraica y luego utilizamos que $i^2 = -1$, obtenemos el producto definido.
\end{remark}

Naturalmente, para un número complejo $z = a + bi$, decimos que $\Re(z) = a$ es la \textit{parte real} e $\Im(z) = b$ la \textit{parte imaginaria}. También, definimos el \textit{módulo} (o \textit{valor absoluto}) como $|z| = \sqrt{a^2 + b^2}$. Por último, definimos el conjugado.

\begin{definition}
    El \emph{conjugado} de un número complejo $z = a + bi \in \C$ es
    \begin{equation*}
        \overline{z} = a - bi.
    \end{equation*}
\end{definition}

Nos interesa averiguar cómo es el inverso de un número complejo.

\begin{proposition}
    Sea $z \in \C \setminus \{0\}$. Entonces, el inverso es
    \begin{equation*}
        z^{-1} = \frac{\overline{z}}{|z|^2}.
    \end{equation*}
\end{proposition}

\begin{proof}
    Basta con probar que $z \overline{z} = |z|^2$. Escribimos a $z = a + bi$ y entonces 
    \begin{align*}
        z \overline{z} &= (a + bi) (a - bi) \\
        &= (a)^2 - (bi)^2 \\
        &= a^2 + b^2 \\
        &= |z|^2.
    \end{align*}
    Reordenando la ecaución, obtenemos el resultado.
\end{proof}


\section{Forma polar}

Una de las formas más útiles de expresar a un número complejo es en su forma polar.

\begin{definition}
    Sea $z = a + bi \in \C \setminus \{0\}$. Definimos el \emph{argumento} de un número complejo como todo $\theta \in \R$ que cumple
    \begin{equation*}
        \Re(z) = |z| \cos \theta \quad\text{e}\quad \Im(z) = |z| \sin \theta
    \end{equation*}
    y lo denotamos $\arg(z)$. También, si $\theta \in [0, 2\pi)$, lo llamamos el \emph{argumento principal} y lo denotamos $\Arg(z)$.
\end{definition}

\begin{remark}
    Para el cero las condiciones se cumplen trivialmente.
\end{remark}

El argumento y el módulo se trasladan intuitivamente a una representación gráfica.

\begin{figure}[H]
\centering
\begin{tikzpicture}[scale=1.1]
  % Parámetros editables
  \def\r{2.2}      % módulo = |z|
  \def\ang{40}     % ángulo (en grados): Arg(z)

  % Puntos
  \coordinate (O) at (0,0);
  \coordinate (X) at (1,0);        % punto en el eje real
  \coordinate (Z) at (\ang:\r);

  % Ejes
  \draw[->] (-0.6,0) -- (3.0,0) node[below right] {$\Re$};
  \draw[->] (0,-0.6) -- (0,2.6) node[above left] {$\Im$};

  % Círculos
  \draw[densely dotted] (O) circle (1);
  \draw[dashed] (O) circle (\r);

  % Vector z
  \draw[very thick,-{Latex}] (O) -- (Z)
    node[pos=1, above right] {$z$};

  % Proyecciones
  \path let \p1 = (Z) in
    coordinate (Xproj) at (\x1,0)
    coordinate (Yproj) at (0,\y1);
  \draw[dotted] (Z) -- (Xproj) node[below] {$r\cos\theta$};
  \draw[dotted] (Z) -- (Yproj) node[left]  {$r\sin\theta$};

  % Punto y ángulo
  \fill (Z) circle (0.03);
  \pic [draw, ->, "$\theta$", angle radius=11mm, angle eccentricity=1.15]
      {angle = X--O--Z};
\end{tikzpicture}
\end{figure}

Por trigonometría, podemos deducir la siguiente expresión:
\begin{equation*}
    z = r (\cos \theta + i \sen \theta),
\end{equation*}
donde $r = |z|$ y $\theta = \Arg(z)$. 

Así también surge la \textit{fórmula de Euler}.

\begin{theorem}[Fórmula de Euler]
    Sea $z \in \C$. Entonces, podemos expresarlo como
    \begin{equation*}
        z = |z| e^{i \Arg(z)} = r e^{i \theta}.
    \end{equation*}
\end{theorem}

La demostración de esto la vemos más adelante.


\section{Raíces}

Para encontrar la raíz $n$-ésima de un número complejo, lo más fácil es escribirlo en forma de Euler y luego elevar a $\frac{1}{n}$. (No voy a ir más en detalle que esto.)

\begin{proposition}
    Sean $z_0, z_1, \dots, z_{n-1} \in \C$ las raíces $n$-ésimas de $w \in \C \setminus \{0\}$, con $m,n \in \mathbb{N}$ y $n \geq 2$. Entonces
    \begin{equation}
        \sum_{i=0}^{n-1} z_i^m \;=\;
        \begin{cases}
            n\,w^k & \text{si } m = k n,\; k \in \mathbb{N},\\
            0 & \text{en caso contrario}.
        \end{cases}
    \end{equation}
\end{proposition}

(No lo demuestro, pero sale expresando las raíces en forma de Euler y utilizando la expresión de la suma de la progresión geométrica.)


\section{Topología y continuidad}

{\color{red} TODO: ESCRIBIR PROPIEDADES}


\section{Esfera de Riemann}

Consideramos antes a $\C$ junto con el infinito.

\begin{definition}
    Definimos el \emph{plano complejo extendido} $\widehat{\C} = \C \cup \{\infty\}$.
\end{definition}

El plano complejo extendido tiene una interpretación intuitiva dada por la \textit{proyección estereográfica}. (En realidad, voy a arrancar con la inversa de la proyección estereográfica porque me resulta más intuitivo.) 

Nos situamos en $\R^3$ y consideramos a la esfera unitaria $\mathbb{S}^2$. Identificamos el plano $xy$ con $\C$ y definimos la biyección $\varphi : \C \to \mathbb{S}^2 \setminus \{N\}$ de la siguiente forma: si $z = x + iy$,
\begin{equation*}
    \varphi(z) = \left(\frac{2x}{|z|^2 + 1}, \frac{2y}{|z|^2 + 1}, \frac{|z|^2 - 1}{|z|^2 + 1}\right).
\end{equation*}

\begin{figure}[H]
    \centering
    % Código obtenido de https://tex.stackexchange.com/questions/546979/sphere-shell-using-tikz
    \begin{tikzpicture}[declare function={%
            stereox(\x,\y)=2*\x/(1+\x*\x+\y*\y);%
            stereoy(\x,\y)=2*\y/(1+\x*\x+\y*\y);%
            stereoz(\x,\y)=(-1+\x*\x+\y*\y)/(1+\x*\x+\y*\y);
            Px=1.75;Py=-1.5;Qx=-1.5;Qy=-1.25;amax=2.5;},scale=2.5,
            line join=round,line cap=round,
            dot/.style={circle,fill,inner sep=1pt},>={Stealth[length=1.2ex]}, scale=0.8]
    \pgfmathsetmacro{\myaz}{15}
    \path[save path=\pathSphere]   (0,0) circle[radius=1];
    \begin{scope}[3d view={\myaz}{18}]
    \draw (-amax,amax) -- (-amax,-amax) coordinate (bl) -- (amax,-amax) 
    coordinate (br)-- (amax,amax)
    %node[above left]{$z=0$}
    ;
    \begin{scope}
    \tikzset{protect=\pathSphere}
    \draw (-amax,amax) -- (amax,amax) node[below left,xshift=-2em]{$\C$};
    \end{scope}
    \begin{scope}
    \clip[reuse path=\pathSphere];
    \draw[dashed] (-amax,amax) -- (amax,amax);
    \end{scope}
    \begin{scope}[canvas is xy plane at z=0]
    \draw[dashed] (\myaz:1) arc[start angle=\myaz,end angle=\myaz+180,radius=1];
    \draw (\myaz:1) arc[start angle=\myaz,end angle=\myaz-180,radius=1];
    \path[save path=\pathPlane] (\myaz:amax) -- (\myaz+180:amax) --(bl) -- (br) -- cycle;
    \begin{scope}
        \clip[use path=\pathPlane];
        \draw[dashed,use path=\pathSphere];
    \end{scope}
    \begin{scope}
        \tikzset{protect=\pathPlane}
        \draw[use path=\pathSphere];
    \end{scope}
    \end{scope}
    \draw[-=0.3] (Px,Py,0) node[dot,label=below:{$w$}](w){}
    -- node[auto,pos=0.3,swap]{} ({stereox(Px,Py)},{stereoy(Px,-1)},{stereoz(Px,Py)})
    node[dot,label=left:{$\varphi(w)$}](w*){};
    \draw[-] (Qx,Qy,0) node[dot,label=below:{$z$}](z){}
    -- node[auto,pos=0.5]{} ({stereox(Qx,Qy)},{stereoy(Qx,-1)},{stereoz(Qx,Qy)})
    node[dot,label=above left:{$\varphi(z)$}](z*){};
    \draw[dashed] (w*) -- (0,0,1) node[dot,label=above:{$N$}](zeta){}
    -- (z*);
    \node at (0.5,0,0.8) [label=above right:{$\mathbb{S}$}] {};
    \end{scope}
    \end{tikzpicture}
\end{figure}

Observamos que para conseguir $\varphi(z)$, se traza una recta de $z$ a $N$ y $\Phi(z)$ es el punto donde $\mathbb{S}^2$ y la recta se intersecan.  Sin embargo, ningún punto de $\C$ termina en $N$. ¿Podemos extender $\varphi$? La respuesta es sí y se relaciona con el plano complejo extendido.

Si $|z| \to \infty$, entonces $\varphi(z) \to N$. Entonces, en $\widehat{\C}$, definimos $\varphi(\infty) = N$.

\begin{proposition}
    La proyección estereográfica envía circunferencias en $\mathbb{S}^2$ a circunferencias o rectas en $\C$.
\end{proposition}

\begin{proof}
    Sea $\mathcal C\subset\mathbb S^2$ una circunferencia y sea $\Pi:\; ax+by+cz=d$ el plano que la contiene.
    Escribimos la inversa de la proyección estereográfica:
    \begin{equation*}
    x=\frac{2u}{|w|^2+1},\quad
    y=\frac{2v}{|w|^2+1},\quad
    z=\frac{|w|^2-1}{|w|^2+1},\qquad w=u+iv\in\C.
    \end{equation*}
    Sustituyendo en la ecuación de $\Pi$ y multiplicando por $|w|^2+1$ llegamos a
    \begin{equation*}
    (c-d)|w|^2-2au-2bv-(c+d)=0.
    \end{equation*}
    Si $c\neq d$, esto equivale a $|w|^2+\alpha u+\beta v+\gamma=0$, que describe una circunferencia en $\C$.
    Si $c=d$, obtenemos $2au+2bv+(c+d)=0$, que describe una recta en $\C$.
\end{proof}

\begin{remark}
    Observemos que que $c=d$ equivale a que el polo norte $(0,0,1)$ pertenezca a $\Pi$, \textit{i.e.} a que $\mathcal C$ pase por el polo norte. (El caso tangencial $a=b=0,\ c=d$ no produce circunferencia en la esfera y queda excluido por hipótesis.)
\end{remark}

Si bien la demostración es puramente algebraica, tenemos una interpretación gráfica.

\begin{figure}[H]
    \centering
    % Código obtenido de https://tex.stackexchange.com/questions/546979/sphere-shell-using-tikz
    \begin{tikzpicture}[declare function={%
            stereox(\x,\y)=2*\x/(1+\x*\x+\y*\y);%
            stereoy(\x,\y)=2*\y/(1+\x*\x+\y*\y);%
            stereoz(\x,\y)=(-1+\x*\x+\y*\y)/(1+\x*\x+\y*\y);
            Px=1.75;Py=-1.5;Qx=-1.5;Qy=-1.25;amax=2.5;},scale=2.5,
            line join=round,line cap=round,
            dot/.style={circle,fill,inner sep=1pt},>={Stealth[length=1.2ex]}, scale=0.8]
    \pgfmathsetmacro{\myaz}{15}
    \path[save path=\pathSphere]   (0,0) circle[radius=1];
    \begin{scope}[3d view={\myaz}{18}]
    \draw (-amax,amax) -- (-amax,-amax) coordinate (bl) -- (amax,-amax) 
    coordinate (br)-- (amax,amax);

    %%%%%%%%%%%%%%%%%%%%%%%%%%%%%%%%%%%%%%%%%%%%%%%%%%%%%%%%%%%%%
    %%% PLANO por N y la recta wz en \C (como ya tenías)
    \pgfmathsetmacro{\dx}{Qx-Px}
    \pgfmathsetmacro{\dy}{Qy-Py}
    \pgfmathsetmacro{\dn}{sqrt(\dx*\dx+\dy*\dy)}
    \pgfmathsetmacro{\ux}{\dx/\dn}
    \pgfmathsetmacro{\uy}{\dy/\dn}

    \pgfmathsetmacro{\mx}{(Px+Qx)/2}
    \pgfmathsetmacro{\my}{(Py+Qy)/2}

    % Vector hacia N desde M=(mx,my,0)
    \pgfmathsetmacro{\vx}{-\mx}
    \pgfmathsetmacro{\vy}{-\my}
    \pgfmathsetmacro{\vz}{1}

    % Lámina del plano (opcional, ya la tenés: podés dejarla o quitarla)
    \pgfmathsetmacro{\L}{1.2*amax}
    \pgfmathsetmacro{\H}{1.6}
    % Normal del plano n = u x v
    \pgfmathsetmacro{\nx}{\uy*1 - 0*\vy}
    \pgfmathsetmacro{\ny}{0*\vx - \ux*1}
    \pgfmathsetmacro{\nz}{\ux*\vy - \uy*\vx}
    \pgfmathsetmacro{\nn}{sqrt(\nx*\nx+\ny*\ny+\nz*\nz)}

    %%%%%%%%%%%%%%%%%%%%%%%%%%%%%%%%%%%%%%%%%%%%%%%%%%%%%%%%%%%%%
    %%% CIRCUNFERENCIA: intersección (esfera ∩ plano) pasa por N, φ(w), φ(z)
    % Centro C = proyección ortogonal del 0 sobre el plano: C = ( (n·M)/||n||^2 ) n
    \pgfmathsetmacro{\ndotM}{\nx*\mx + \ny*\my + \nz*0}
    \pgfmathsetmacro{\Cx}{(\ndotM/(\nn*\nn))*\nx}
    \pgfmathsetmacro{\Cy}{(\ndotM/(\nn*\nn))*\ny}
    \pgfmathsetmacro{\Cz}{(\ndotM/(\nn*\nn))*\nz}

    % Radio r = sqrt(1 - d^2), con d = |n·M|/||n||
    \pgfmathsetmacro{\rC}{sqrt(max(0,1 - (\ndotM*\ndotM)/(\nn*\nn)))}

    % Base ortonormal en el plano: a = u (ya unitario), b = (n_unit x a) normalizado
    \pgfmathsetmacro{\nxu}{\nx/\nn}
    \pgfmathsetmacro{\nyu}{\ny/\nn}
    \pgfmathsetmacro{\nzu}{\nz/\nn}
    \pgfmathsetmacro{\bx}{- \nzu*\uy}
    \pgfmathsetmacro{\by}{  \nzu*\ux}
    \pgfmathsetmacro{\bz}{  \nxu*\uy - \nyu*\ux}
    \pgfmathsetmacro{\bn}{sqrt(\bx*\bx+\by*\by+\bz*\bz)}
    \pgfmathsetmacro{\bx}{\bx/\bn}
    \pgfmathsetmacro{\by}{\by/\bn}
    \pgfmathsetmacro{\bz}{\bz/\bn}

    % Dibujo por aproximación poligonal
    \def\step{4}
    \foreach \t in {0,\step,...,356}{
      \pgfmathsetmacro{\ct}{cos(\t)}
      \pgfmathsetmacro{\st}{sin(\t)}
      \pgfmathsetmacro{\ctb}{cos(\t+\step)}
      \pgfmathsetmacro{\stb}{sin(\t+\step)}
      \draw[teal!70!black, thick]
        ({\Cx + \rC*(\ux*\ct + \bx*\st)},
         {\Cy + \rC*(\uy*\ct + \by*\st)},
         {\Cz + \rC*( 0*\ct + \bz*\st)})
        -- 
        ({\Cx + \rC*(\ux*\ctb + \bx*\stb)},
         {\Cy + \rC*(\uy*\ctb + \by*\stb)},
         {\Cz + \rC*( 0*\ctb + \bz*\stb)});
    }
    %%%%%%%%%%%%%%%%%%%%%%%%%%%%%%%%%%%%%%%%%%%%%%%%%%%%%%%%%%%%%

    \begin{scope}
    \tikzset{protect=\pathSphere}
    \draw (-amax,amax) -- (amax,amax) node[below left,xshift=-2em]{$\C$};
    \end{scope}
    \begin{scope}
    \clip[reuse path=\pathSphere];
    \draw[dashed] (-amax,amax) -- (amax,amax);
    \end{scope}
    \begin{scope}[canvas is xy plane at z=0]
      \draw[dashed] (\myaz:1) arc[start angle=\myaz,end angle=\myaz+180,radius=1];
      \draw        (\myaz:1) arc[start angle=\myaz,end angle=\myaz-180,radius=1];
      \path[save path=\pathPlane] (\myaz:amax) -- (\myaz+180:amax) --(bl) -- (br) -- cycle;
      \begin{scope}
        \clip[use path=\pathPlane];
        \draw[dashed, use path=\pathSphere];
      \end{scope}
      \begin{scope}
        \tikzset{protect=\pathPlane}
        \draw[use path=\pathSphere];
      \end{scope}
    \end{scope}

    % Segmentos de proyección y puntos
    \draw[-=0.3] (Px,Py,0) node[dot,label=below:{$w$}](w){}
      -- ({stereox(Px,Py)},{stereoy(Px,Py)},{stereoz(Px,Py)})
         node[dot,label=left:{$\varphi(w)$}](w*){};
    \draw[-] (Qx,Qy,0) node[dot,label=below:{$z$}](z){}
      -- ({stereox(Qx,Qy)},{stereoy(Qx,Qy)},{stereoz(Qx,Qy)})
         node[dot,label=above left:{$\varphi(z)$}](z*){};
    \draw[dashed] (w*) -- (0,0,1) node[dot,label=above:{$N$}](N){} -- (z*);
    \node at (0.5,0,0.8) [label=above right:{$\mathbb{S}$}] {};

    %%%%%%%%%%%%%%%%%%%%%%%%%%%%%%%%%%%%%%%%%%%%%%%%%%%%%%%%%%%%%
    %%% PLANO: pasa por N y por la recta wz en C
    % Dirección de la recta en C: d = (dx,dy,0)
    \pgfmathsetmacro{\dx}{Qx-Px}
    \pgfmathsetmacro{\dy}{Qy-Py}
    \pgfmathsetmacro{\dn}{sqrt(\dx*\dx+\dy*\dy)}
    \pgfmathsetmacro{\ux}{\dx/\dn}
    \pgfmathsetmacro{\uy}{\dy/\dn}

    % Un punto de la recta (el medio entre w y z)
    \pgfmathsetmacro{\mx}{(Px+Qx)/2}
    \pgfmathsetmacro{\my}{(Py+Qy)/2}

    % Vector hacia N desde ese punto: v = N - M = (-mx,-my,1)
    \pgfmathsetmacro{\vx}{-\mx}
    \pgfmathsetmacro{\vy}{-\my}
    \pgfmathsetmacro{\vz}{1}
    \pgfmathsetmacro{\vn}{sqrt(\vx*\vx+\vy*\vy+\vz*\vz)}
    \pgfmathsetmacro{\vxu}{\vx/\vn}
    \pgfmathsetmacro{\vyu}{\vy/\vn}
    \pgfmathsetmacro{\vzu}{\vz/\vn}

    % Tamaños de la "lámina" (largos a gusto)
    \pgfmathsetmacro{\L}{1.2*amax}
    \pgfmathsetmacro{\H}{1.6}

    % Esquinas del rectángulo en el plano: M ± L*u ± H*vhat
    \coordinate (P1) at ({\mx+\L*\ux+\H*\vxu}, {\my+\L*\uy+\H*\vyu}, {\H*\vzu});
    \coordinate (P2) at ({\mx-\L*\ux+\H*\vxu}, {\my-\L*\uy+\H*\vyu}, {\H*\vzu});
    \coordinate (P3) at ({\mx-\L*\ux-\H*\vxu}, {\my-\L*\uy-\H*\vyu}, {-\H*\vzu});
    \coordinate (P4) at ({\mx+\L*\ux-\H*\vxu}, {\my+\L*\uy-\H*\vyu}, {-\H*\vzu});

    % Lámina del plano (ligeramente translúcida)
    \fill[teal!20, opacity=0.12] (P1)--(P2)--(P3)--(P4)--cycle;

    % Intersección con C (z=0): la misma recta por M con dirección u
    \begin{scope}[canvas is xy plane at z=0]
      \draw[teal!50!black, thick]
        ({\mx-\L*\ux},{\my-\L*\uy}) -- ({\mx+\L*\ux},{\my+\L*\uy});
    \end{scope}
    %%%%%%%%%%%%%%%%%%%%%%%%%%%%%%%%%%%%%%%%%%%%%%%%%%%%%%%%%%%%%

    \end{scope}
    \end{tikzpicture}
\end{figure}


Dado que la circunferencia pasa por $N$, la proyección es una recta en vez de una circunferencia.


\section{Homografías}

Definimos las homografías que luego van a ser útiles.

\begin{definition}
    Una \textbf{homografía} es una función $f : \widehat{\C} \to \widehat{\C}$
    \begin{equation*}
        f(z) = \frac{az+b}{cz+d},
    \end{equation*}
    donde $a,b,c,d \in \C$ y $ad - bc \neq 0$.
\end{definition}

Hay dos valores de $f$ que quizás no son inmediatamente obvios: para $c \neq 0$,
\begin{equation*}
    \begin{cases}
        f(-\frac{d}{c}) = \infty, \\
        f(\infty) = \frac{a}{c}.
    \end{cases}
\end{equation*}

Las \textit{homografías básicas} son:
\begin{enumerate}
    \item \textbf{Traslación:} 
    \begin{equation*}
        T_a(z) = z + a, \qquad a \in \C.
    \end{equation*}
    \item \textbf{Homotecia:} 
    \begin{equation*}
        H_r(z) = r z, \qquad r > 0.
    \end{equation*}
    \item \textbf{Rotación:} 
    \begin{equation*}
        R_\alpha(z) = \alpha z, \qquad |\alpha| = 1.
    \end{equation*}
    \item \textbf{Inversión:} 
    \begin{equation*}
        I(z) = \frac{1}{z}.
    \end{equation*}
\end{enumerate}

\begin{proposition}
    Sea $f:\widehat{\C}\to\widehat{\C}$ una homografía con $c\neq 0$. Entonces, existen aplicaciones afines $g,h:\C\to\C$ tales que
    \begin{equation*}
        f = g \circ i \circ h, \qquad \text{donde } i(z)=\frac{1}{z}.
    \end{equation*}
\end{proposition}

\begin{proof}
    Sea $q \in \C$ tal que $ad - qc = 0$, o sea $q = \frac{ad}{c}$. Entonces
    \begin{align*}
        f(z) &= \frac{az+b}{cz+d} 
        = \frac{az+b+q-q}{cz+d} 
        = \frac{az+\tfrac{ad}{c}}{cz+d} + \frac{b-q}{cz+d}.
    \end{align*}
    De este modo,
    \begin{align*}
        f(z) &= \frac{az+\tfrac{d}{c}\,c}{cz+d} + \frac{b-\tfrac{ad}{c}}{cz+d} \\
             &= \frac{a}{c} + \Bigl(b-\frac{ad}{c}\Bigr)(cz+d)^{-1}.
    \end{align*}
    Luego, definiendo
    \[
        h(z) = cz+d, 
        \qquad g(w) = \frac{a}{c} + \Bigl(b-\frac{ad}{c}\Bigr)w,
    \]
    se obtiene que $f = g \circ i \circ h$.
\end{proof}

\begin{proposition}
    La imagen de cualquier circunferencia o recta por una homografía es una circunferencia o una recta.
\end{proposition}

\begin{proof}
    El único caso complicado es el de las inversiones. Consideramos una circunferencia dada por la ecuación
    \begin{equation*}
        |z - z_0| = r, \qquad r \in \R_{\geq 0}.
    \end{equation*}
    Equivalentemente,
    \begin{equation*}
        (z - z_0)(\overline{z} - \overline{z_0}) = r^2.
    \end{equation*}
    Como en la inversión $i(z)=\dfrac{1}{z}$ tenemos $\overline{z}=\dfrac{1}{\overline{i(z)}}=\dfrac{1}{\overline{w}}$ con $w=i(z)$, resulta
    \begin{equation*}
        \Bigl(\frac{1}{w}-z_0\Bigr)\Bigl(\frac{1}{\overline{w}}-\overline{z_0}\Bigr)=r^2.
    \end{equation*}
    Multiplicando ambos lados por $|w|^2$, se obtiene
    \begin{equation*}
        (1-z_0w)(1-\overline{z_0}\,\overline{w})=r^2|w|^2.
    \end{equation*}
    Desarrollando,
    \begin{equation*}
        1 - z_0 w - \overline{z_0}\,\overline{w} + |z_0|^2 |w|^2 = r^2 |w|^2.
    \end{equation*}
    Es decir,
    \begin{equation*}
        (|z_0|^2-r^2)|w|^2 - z_0 w - \overline{z_0}\,\overline{w} + 1 = 0,
    \end{equation*}
    que es la ecuación de una circunferencia o de una recta en el plano complejo, según que $|z_0|^2-r^2\neq 0$ o $|z_0|^2-r^2=0$. 
    Por lo tanto, la inversión envía circunferencias en circunferencias o rectas, y lo mismo vale para toda homografía.
\end{proof}

Veamos que toda homografía está determinada unívocamente por tres puntos.

\begin{proposition}
    Sean $z_1, z_2, z_3 \in \C$. Hay una única homografía que satisface
    \begin{equation*}
        f(z_1) = 0, \quad f(z_2) = 1, \quad f(z_3) = \infty.
    \end{equation*}
\end{proposition}

\begin{proof}
    {\color{red} COMPLETAR}
\end{proof}

\chapter{Probabilidad condicional}

\section{Probabilidad condicional}

Ya vimos cómo calcular probabilidades de eventos. Ahora queremos formalizar qué significa calcular probabilidades \textit{condicionadas} a que cierto evento ya ocurrió.

\begin{example}
    Tiro dos dados y sumo los resultados. Consideremos los eventos
    \begin{align*}
        A &= \{\text{la suma es $5$}\}, \\
        B &= \{\text{el segundo dado es par}\}.
    \end{align*}
    El espacio muestral es
    \begin{equation*}
        \Omega = \{(a,b) \mid a,b \in \{1,\ldots,6\}\}.
    \end{equation*}
    En particular,
    \begin{equation*}
        A = \{(1,4),(4,1),(2,3),(3,2)\}.
    \end{equation*}
    Luego,
    \begin{equation*}
        P(A \mid B) = \frac{\# (A \cap B)}{\# B} = \frac{2}{18} = \frac{1}{9}.
    \end{equation*}
\end{example}

El cálculo anterior nos motiva a introducir la definición general de probabilidad condicional.

\begin{definition}
    Dados $A$ y $B$ eventos, con $P(B) \neq 0$, la \emph{probabilidad condicional} de $A$ dado $B$ se define como
    \begin{equation*}
        P(A \mid B) = \frac{P(A \cap B)}{P(B)}.
    \end{equation*}
\end{definition}

Veamos cómo aplica la definición en un ejemplo.

\begin{example}
    Consideremos una urna con $9$ bolitas de la siguiente forma:
    \begin{itemize}
        \item $5$ naranjas,
        \item $4$ violetas, de las cuales $3$ tienen una cruz.
    \end{itemize}
    Denotemos $N=$ \{\text{naranja}\}, $V=$ \{\text{violeta}\}, $C=$ \{\text{tiene cruz}\}. Entonces,
    \begin{align*}
        P(N) &= \frac{5}{9}, &
        P(N \mid V) &= \frac{1}{3}, &
        P(C \mid V) &= \frac{1}{2}.
    \end{align*}
\end{example}

A partir de la definición, podemos verificar fácilmente que $P(\,\cdot \mid B)$ cumple los axiomas de probabilidad.

\begin{proposition}
    Sea $B$ un evento con $P(B) \neq 0$. Entonces la aplicación
    \begin{equation*}
        A \mapsto P(A \mid B)
    \end{equation*}
    define una probabilidad sobre el espacio muestral condicionado.
\end{proposition}

\begin{proof}
    Para todo $B$ con $P(B)>0$:
    \begin{itemize}
        \item[(P1)] $P(\Omega \mid B) = \frac{P(\Omega \cap B)}{P(B)} = \frac{P(B)}{P(B)}=1$.
        \item[(P2)] Como $P$ es una probabilidad, $P(A \mid B) \geq 0$.
        \item[(P3)] Si $\{A_n\}_{n \in \N}$ son eventos disjuntos dos a dos,
        \begin{align*}
            P\!\left(\bigcup_{n \in \N} A_n \,\middle|\, B\right)
            &= \frac{P\!\left(\bigcup_{n \in \N}(A_n \cap B)\right)}{P(B)} \\
            &= \frac{\sum_{n \in \N} P(A_n \cap B)}{P(B)} \\
            &= \sum_{n \in \N} \frac{P(A_n \cap B)}{P(B)} \\
            &= \sum_{n \in \N} P(A_n \mid B).
        \end{align*}
    \end{itemize}
\end{proof}

\begin{remark}
    Es importante aclarar que $A \mid B$ no es un evento. Por lo tanto, expresiones como $P(A \mid B \mid C)$ no tienen sentido.
\end{remark}

\begin{proposition}
    Para todo par de eventos $A, B$ con $P(B) \neq 0$, se cumple
    \begin{equation*}
        P(A \mid B) \, P(B) = P(A \cap B).
    \end{equation*}
\end{proposition}

Este resultado es útil para calcular probabilidades conjuntas a partir de probabilidades condicionales.

\begin{example}
    Sacamos dos cartas de un mazo de $52$ cartas sin reemplazo. Sea
    \begin{align*}
        A_1 &= \{\text{la primera carta es trébol}\}, \\
        A_2 &= \{\text{la segunda carta es trébol}\}.
    \end{align*}
    Entonces,
    \begin{align*}
        P(A_1) &= \frac{13}{52}, \\
        P(A_2 \mid A_1) &= \frac{12}{51}.
    \end{align*}
    Aplicando la proposición anterior,
    \begin{equation*}
        P(A_1 \cap A_2) = P(A_1)\,P(A_2 \mid A_1) = \frac{13}{52} \cdot \frac{12}{51}.
    \end{equation*}
\end{example}

\begin{proposition}
    Para eventos $A_1, A_2, \dots, A_n$ se cumple
    \begin{multline*}
        P(A_1 \cap A_2 \cap \cdots \cap A_n)
        \\= P(A_1) P(A_2 \mid A_1) P(A_3 \mid A_1 \cap A_2) \cdots P(A_n \mid A_1 \cap \cdots \cap A_{n-1}).
    \end{multline*}
\end{proposition}

\begin{proof}[Idea de demostración]
    La demostración se hace fácilmente por inducción en $n$.
\end{proof}

\section{Probabilidad total}

La noción de probabilidad condicional nos permite descomponer probabilidades en términos de eventos disjuntos.

\begin{proposition}[Probabilidad total en dos partes]
    Sea $B \subseteq \Omega$ un evento. Entonces, para todo $A \subseteq \Omega$,
    \begin{align*}
        P(A) &= P(A \cap B) + P(A \cap B^{\mathrm c}) \\
             &= P(A \mid B)\,P(B) + P(A \mid B^{\mathrm c})\,P(B^{\mathrm c}).
    \end{align*}
\end{proposition}

Más en general, si $\{A_i\}_{i \in I}$ es una partición numerable de $\Omega$, se cumple la \textit{ley de la probabilidad total}:
\begin{equation*}
    P(B) = \sum_{i \in I} P(B \mid A_i)\,P(A_i).
\end{equation*}

\begin{example}
    Consideremos el siguiente experimento:
    \begin{centeredvarwidth}
        \begin{enumerate}
            \item En una caja hay $3$ monedas: una con dos caras y dos equilibradas.
            \item Se elige una moneda al azar.
            \item Se lanza la moneda.
        \end{enumerate}
    \end{centeredvarwidth}

    Definamos los eventos
    \begin{align*}
        C &= \{\text{sale cara}\}, \\
        A &= \{\text{la moneda elegida es la de dos caras}\}.
    \end{align*}

    Entonces,
    \begin{align*}
        P(C) &= P(C \mid A)\,P(A) + P(C \mid A^{\mathrm{c}})\,P(A^{\mathrm{c}}) \\
             &= 1 \cdot \frac{1}{3} + \frac{1}{2} \cdot \frac{2}{3} \\
             &= \frac{2}{3}.
    \end{align*}
\end{example}

Este cálculo es un primer ejemplo de la \textit{fórmula de la probabilidad total}, que nos permite descomponer la probabilidad de un evento en función de una partición del espacio.

\begin{example}
    Ahora consideremos otro experimento. Lanzamos un dado. Sea
    \begin{align*}
        A &= \{\text{saco un $5$}\}, \\
        B &= \{\text{saco un $7$}\}, \\
        C &= \{\text{ni $5$ ni $7$}\} = (A \cup B)^{\mathrm c}.
    \end{align*}

    Supongamos que ganar depende de estos eventos: 
    si ocurre $A$ se gana siempre, 
    si ocurre $B$ nunca se gana, 
    y si ocurre $C$ se gana con cierta probabilidad.

    Denotemos $G=\{\text{ganar}\}$. Entonces, por probabilidad total,
    \begin{align*}
        P(G) &= P(G \mid A)\,P(A) + P(G \mid B)\,P(B) + P(G \mid C)\,P(C).
    \end{align*}

    Como $P(G \mid A)=1$, $P(G \mid B)=0$, y $P(A)=P(B)=\frac{1}{6}$, $P(C)=\frac{4}{6}$, se obtiene
    \begin{align*}
        P(G) &= 1 \cdot \frac{1}{6} + 0 \cdot \frac{1}{6} + P(G \mid C)\,\frac{4}{6}.
    \end{align*}

    Si además sabemos que $P(G \mid C)=\frac{1}{2}$, queda
    \begin{equation*}
        P(G) = \frac{1}{6} + \frac{1}{2}\cdot\frac{4}{6} = \frac{6}{15}.
    \end{equation*}
\end{example}

Estos ejemplos muestran cómo las probabilidades condicionales y la probabilidad total se combinan para calcular probabilidades en situaciones más complejas.


\section{Independencia de eventos}

La independencia de eventos se relaciona íntimamente con la probabilidad condicional. En esencia, si la probabilidad de un evento $A$ dado un evento $B$ no cambia, entonces son independientes.

\begin{definition}
    Decimos que dos eventos $A$ y $B$ son independientes si
    \begin{equation*}
        P(A \cap B) = P(A) P(B).
    \end{equation*}
\end{definition}

\begin{remark}
    Si $P(A) = 0$ o $P(A) = 1$, entones es independiente de cualquier evento.
\end{remark}

Esta noción la podemos generalizar a más eventos.

\begin{definition}
    Decimos que los eventos $A_1, A_2, \ldots, A_n$ son independientes si, para todo $I \subseteq \{1, 2, \ldots, n\}$,
    \begin{equation*}
        P\left(\bigcap_{i \in I} A_i\right) = \prod_{i \in I} P(A_i).
    \end{equation*}
\end{definition}

Y para familias de eventos:

\begin{definition}
    Decimos que una familia de eventos $\mathcal{A}$ es independiente si todo subconjunto finito de $\mathcal{A}$ es independiente.
\end{definition}

\chapter{Series en el plano complejo}

\section{Series numéricas y criterios de convergencia}

\begin{definition}
    Sea $(a_n)_{n \in \N_0}$ una sucesión de números complejos. 
    Decimos que la serie $\sum a_n$ \emph{converge} si existe el límite
    \begin{equation*}
        \lim_{N \to \infty} \sum_{n=0}^N a_n,
    \end{equation*}
    y lo denotamos por $\sum_{n=0}^\infty a_n$. Además, si la serie de valores absolutos converge,
    \begin{equation*}
        \sum_{n=0}^\infty |a_n| < +\infty,
    \end{equation*}
    decimos que $\sum a_n$ \emph{converge absolutamente}.
\end{definition}

Un primer hecho importante es que la convergencia absoluta implica la convergencia usual.  

\begin{proposition}
    Si $(a_n)_{n \in \N_0} \subseteq \C$ y la serie $\sum_{n=0}^{\infty} |a_n|$ converge, 
    entonces $\sum_{n=0}^{\infty} a_n$ converge.
\end{proposition}

\begin{proof}
    Sean $m<n$ y
    \begin{equation*}
        S_n = \sum_{j=0}^n a_j, \qquad 
        \sigma_n = \sum_{j=0}^n |a_j|.
    \end{equation*}
    Entonces
    \begin{equation*}
        |S_n - S_m|
        = \left|\sum_{j=0}^n a_j - \sum_{j=0}^m a_j\right|
        = \left|\sum_{j=m+1}^n a_j\right|
        \leq \sum_{j=m+1}^n |a_j|
        = \sigma_n - \sigma_m.
    \end{equation*}
    Como la serie de valores absolutos converge, 
    la diferencia $\sigma_n - \sigma_m$ se puede hacer arbitrariamente pequeña. 
    Por lo tanto, $(S_n)$ es de Cauchy y la serie converge.
\end{proof}

A continuación presentamos algunos criterios prácticos para decidir la convergencia de series.  

\begin{proposition}[Criterio de la comparación]
    Sean $(a_n)$, $(b_n)$ sucesiones de números reales positivos tales que $a_n \leq b_n$. Entonces:
    \begin{enumerate}
        \item Si $\sum b_n$ converge, también lo hace $\sum a_n$.
        \item Si $\sum a_n = +\infty$, entonces $\sum b_n = +\infty$.
    \end{enumerate}
\end{proposition}

Este criterio se usa sobre todo para comparar con series conocidas (geométrica, armónica, $p$-series, etc.).  

\begin{proposition}[Criterio de la raíz $n$-ésima de Cauchy]
    Sea $(a_n)$ sucesión de números complejos y sea
    \begin{equation*}
        \alpha = \limsup_{n \to \infty} |a_n|^{1/n}.
    \end{equation*}
    Entonces:
    \begin{enumerate}
        \item Si $\alpha < 1$, la serie $\sum a_n$ converge absolutamente.
        \item Si $\alpha > 1$, la serie $\sum a_n$ diverge.
        \item Si $\alpha = 1$, el criterio no es concluyente.
    \end{enumerate}
\end{proposition}

El criterio de la raíz es particularmente útil cuando $a_n$ involucra potencias elevadas en $n$, por ejemplo $a_n = \frac{n^k}{m^n}$.  

\begin{proposition}[Criterio de D’Alembert]
    Sea $(a_n)$ sucesión de números complejos y sea
    \begin{equation*}
        \beta = \limsup_{n \to \infty} \left|\frac{a_{n+1}}{a_n}\right|.
    \end{equation*}
    Entonces:
    \begin{enumerate}
        \item Si $\beta < 1$, la serie $\sum a_n$ converge absolutamente.
        \item Si $\beta > 1$, la serie $\sum a_n$ diverge.
        \item Si $\beta = 1$, el criterio no es concluyente.
    \end{enumerate}
\end{proposition}

Este último criterio, también llamado del \textit{cociente}, es uno de los más usados en la práctica, sobre todo cuando los términos de la serie contienen factoriales o potencias encadenadas.

\begin{proposition}[Criterio de la comparación en el límite]
    Sean $(a_n)$, $(b_n)$ sucesiones de números reales positivos.
    \begin{enumerate}
        \item Si $\lim_{n \to \infty} \tfrac{a_n}{b_n} = c > 0$, entonces $\sum a_n$ y $\sum b_n$ 
        convergen o divergen simultáneamente.
        \item Si $\lim_{n \to \infty} \tfrac{a_n}{b_n} = 0$ y $\sum b_n$ converge, entonces 
        $\sum a_n$ también converge.
        \item Si $\lim_{n \to \infty} \tfrac{a_n}{b_n} = +\infty$ y $\sum b_n = +\infty$, 
        entonces $\sum a_n = +\infty$.
    \end{enumerate}
\end{proposition}

Este criterio refina al de comparación directa y es muy útil cuando la razón de los términos se estabiliza.

\begin{proposition}[Criterio de la integral]
    Sea $(a_n)$ una sucesión de números reales no negativos y 
    $f : [0,+\infty) \to [0,+\infty)$ una función continua y decreciente tal que $f(n) = a_n$
    para todo $n \in \N_0$. Entonces, la serie $\sum a_n$ converge si y sólo si 
    converge la integral impropia
    \begin{equation*}
        \int_0^{\infty} f(t)\,dt.
    \end{equation*}
\end{proposition}

Este criterio se conoce también como \textit{criterio de la integral de Cauchy} y permite conectar 
la teoría de series con el análisis de integrales impropias.

\begin{definition}
    Sean $(a_n)$ y $(b_n)$ dos sucesiones de números complejos. Definimos el 
    \emph{producto de Cauchy} como la sucesión $(c_k)$ dada por
    \begin{equation*}
        c_k = \sum_{j=0}^k a_{k-j} b_j.
    \end{equation*}
\end{definition}

Este producto aparece de manera natural al multiplicar series de potencias, 
y será especialmente relevante en el estudio de funciones analíticas.

\begin{proposition}[Convolución de series absolutamente convergentes]
    Sean $(a_n)$ y $(b_n)$ sucesiones de números complejos tales que 
    $\sum a_n$ y $\sum b_n$ son absolutamente convergentes. 
    Entonces, la serie formada por el producto de Cauchy $(c_k)$ es absolutamente convergente y además
    \begin{equation*}
        \sum_{k=0}^\infty |c_k| \;\leq\;
        \left( \sum_{k=0}^\infty |a_k| \right)
        \left( \sum_{k=0}^\infty |b_k| \right).
    \end{equation*}
\end{proposition}

\begin{proof}
    Recordemos que el producto de Cauchy está dado por
    \begin{equation*}
        c_n = \sum_{j=0}^n a_j b_{n-j}.
    \end{equation*}
    En particular,
    \begin{align*}
        |c_0| &\leq |a_0 b_0|, \\
        |c_1| &\leq |a_0 b_1| + |a_1 b_0|, \\
        |c_2| &\leq |a_0 b_2| + |a_1 b_1| + |a_2 b_0|, \\
              &\;\;\vdots \\
        |c_n| &\leq |a_0 b_n| + |a_1 b_{n-1}| + \dots + |a_n b_0|.
    \end{align*}

    Sumando las desigualdades anteriores, obtenemos
    \begin{align*}
        \sum_{k=0}^n |c_k|
        &\leq \sum_{j=0}^n \sum_{m=0}^{\,n-j} |a_j|\,|b_m| \\
        &= \sum_{j=0}^n |a_j| \left( \sum_{m=0}^{\,n-j} |b_m| \right).
    \end{align*}
    Como $\sum b_n$ converge absolutamente, el sumando interno está acotado por
    $\sum_{m=0}^\infty |b_m|$. Así,
    \begin{equation*}
        \sum_{k=0}^n |c_k|
        \leq \left(\sum_{j=0}^n |a_j|\right) \left(\sum_{m=0}^\infty |b_m|\right).
    \end{equation*}

    Finalmente, dejando $n \to \infty$,
    \begin{equation*}
        \sum_{k=0}^\infty |c_k|
        \leq \left(\sum_{j=0}^\infty |a_j|\right)\left(\sum_{m=0}^\infty |b_m|\right).
    \end{equation*}
    Esto prueba tanto la convergencia absoluta de la serie de Cauchy como la desigualdad deseada.
\end{proof}

\begin{proposition}[Suma por partes]
    Sean $(a_n)$, $(b_n)$ sucesiones de números complejos y definamos
    \begin{equation*}
        A_n = \sum_{k=0}^n a_k.
    \end{equation*}
    Entonces, para todo $N>k$ se cumple
    \begin{equation*}
        \sum_{n=k}^N a_n b_n
        = A_N b_N - A_{k-1} b_k - \sum_{n=k}^{N-1} A_n (b_{n+1}-b_n).
    \end{equation*}
\end{proposition}

\begin{proof}
    Comenzamos desarrollando la suma:
    \begin{align*}
        \sum_{n=k}^N a_n b_n
        &= \sum_{n=k}^N (A_n - A_{n-1}) b_n \\
        &= \sum_{n=k}^N A_n b_n - \sum_{n=k}^N A_{n-1} b_n \\
        &= A_N b_N - A_{k-1} b_k + \sum_{n=k}^{N-1} A_n b_n - \sum_{n=k+1}^N A_{n-1} b_n.
    \end{align*}
    Reordenando,
    \begin{align*}
        \sum_{n=k}^N a_n b_n
        &= A_N b_N - A_{k-1} b_k + \sum_{n=k}^{N-1} A_n b_n - \sum_{n=k}^{N-1} A_n b_{n+1} \\
        &= A_N b_N - A_{k-1} b_k - \sum_{n=k}^{N-1} A_n (b_{n+1}-b_n),
    \end{align*}
    que es la identidad buscada.
\end{proof}

Este resultado se conoce también como \textit{fórmula de Abel} y es muy útil para analizar series con términos oscilantes.  

\begin{proposition}[Criterio de Dirichlet]
    Sea $(a_n)$ una sucesión de números reales con $a_{n+1} \leq a_n$ y $a_n \to 0$.  
    Sea $(b_n)$ una sucesión de números complejos tal que
    \begin{equation*}
        B_n = \left|\sum_{k=0}^n b_k\right| \leq C \quad \text{para todo } n.
    \end{equation*}
    Entonces, la serie $\sum a_n b_n$ es convergente.
\end{proposition}

\begin{proof}
    Usando suma por partes, para todo $N$ se tiene
    \begin{align*}
        \left|\sum_{k=0}^N a_k b_k\right|
        &\leq |B_N a_N| + \sum_{n=0}^{N-1} |B_n|\,|a_n - a_{n+1}| \\
        &\leq C|a_N| + C \sum_{n=0}^{N-1} (a_n - a_{n+1}).
    \end{align*}
    Como $a_n \to 0$, al tomar $N \to \infty$ resulta
    \begin{equation*}
        \left|\sum_{k=0}^\infty a_k b_k\right| \leq C a_0,
    \end{equation*}
    lo cual prueba la convergencia de la serie.
\end{proof}

\begin{remark}
    Sea $(a_n)$ una sucesión de reales no negativos, decreciente y tal que $a_n \to 0$. 
    Entonces, la serie $\sum a_n z^n$ converge absolutamente si $|z| < 1$, y converge también si $|z|=1$ pero $z \neq 1$.
\end{remark}

Este resultado es una aplicación directa del criterio de Dirichlet al caso de las llamadas 
\textit{series de Dirichlet}, y nos servirá para analizar ejemplos clásicos como la serie armónica alternada.


\section{Series de funciones}

Sea $(f_n: D \subseteq \C \to \C)_{n \in \N}$ una sucesión de funciones. Denotamos por
\begin{equation*}
    S_n(z) = \sum_{i=0}^n f_i(z)
\end{equation*}
a las sumas parciales.

\begin{definition}
    Decimos que una serie de funciones $\sum f_n$ \emph{converge puntualmente} si $S_n(z)$ converge para todo $z \in \D$. Además, si $S_n$ converge uniformemente, decimos que la serie $\sum f_n$ converge \emph{uniformemente}.
\end{definition}

\begin{remark}
    Notemos que la convergencia puntual habla sobre la convergencia de una sucesión numérica; en cambio, la convergencia uniforme habla sobre sucesión de funciones.
\end{remark}

\begin{proposition}
    Sea $(f_n: D \subseteq \C \to \C)_{n \in \N}$ una sucesión de funciones continuas tal que $\sum f_n$ converge uniformemente. Entonces, $\sum f_n$ es continua.
\end{proposition}

\begin{proof}
    consideremos las sumas parciales $S_n$. En sí, la proposición vale para sucesiones de funciones en general.
\end{proof}

Para la convergencia puntual, simplemente podemos fijar algún $z \in \C$ y utilizar los criterios de las series numéricas. Sin embargo, no tenemos un criterio para la convergencia uniforme.

\begin{proposition}[Criterio de Weierstrass]
    Sea $(f_n: D \subseteq \C \to \C)_{n \in \N}$ una sucesión de funciones que cumplen
    \begin{enumerate}
        \item Para cada $n \in \N$, existe $M_n > 0$ tal que $|f_n| < M_n$.
        \item La serie $\sum M_n$ converge.
    \end{enumerate}
    Entonces, $\sum f_n$ converge uniformemente.
\end{proposition}

Otra noción de convergencia:

\begin{definition}
    Sea $(f_n: D \subseteq \C \to \C)_{n \in \N}$ una sucesión de funciones continuas. Decimos que $\sum f_n$ \emph{converge normalmente} si, para todo $K \subseteq D$ compacto,
    \begin{equation*}
        \sum_{n = 0}^{\infty } \Vert f_n \Vert_K \text{ converge}.
    \end{equation*}
\end{definition}


\section{Series de potencias}

\begin{definition}
    Sea $(a_n)_{n \in \N}$ una sucesión en $\C$ y $z_0 \in \C$. Le decimos \emph{series de potencias} a las series de la forma
    \begin{equation*}
        \sum_{n=0}^{\infty} a_n (z - z_0)^n.
    \end{equation*}
\end{definition}

Nos limitamos a las seires de potencia centradas en $0$.

\begin{lemma}[Abel]
    Sea $(a_n)_{n \in \N}$ una sucesión en $\C$ y $z_0 \in \C \setminus \{0\}$ tal que $\sum a_n z_0^n$ converge. Entonces, $\sum a_n z^n$ converge uniforme y absolutamente en toda bola $\overline{B}(0, r)$, con $r < |z_0|$.
\end{lemma}

\begin{proof}
    Sea $(a_n)_{n \in \N}$ una sucesión en $\C$ y $z_0 \in \C \setminus \{0\}$ tal que $\sum a_n z_0^n$ converge. Necesariamente, $a_n z_0^n \to 0$; por lo tanto, existe $C > 0$ tal que $|a_n z_0^n|$, para todo $n \in \N$. Sea $r < |z_0|$ y $z \in \overline{B}(0, r)$. Entonces,
    \begin{equation*}
        |a_n z^n| = |a_n| |z|^n \leq |a_n| |z_0|^n \left( \frac{|z|}{|z_0|} \right)^n \leq C \left( \frac{|z|}{|z_0|} \right)^n.
    \end{equation*}
    La serie geométrica
    \begin{equation*}
       \sum_{n=0}^{\infty} C \left( \frac{|z|}{|z_0|} \right)^n
    \end{equation*}
    es convergente. Por el criterio de Weierstrass, $\sum a_n z^n$ converge uniforme y absolutamente en $\overline{B}(0, r)$.
\end{proof}

\begin{definition}
    Sea $(a_n)_{n \in \N}$ una sucesión en $\C$. El \emph{radio de convergencia} de la series $\sum a_n z^n$ es
    \begin{equation*}
        R = \frac{1}{\limsup_{n \to \infty} |a_n|^{\frac{1}{n}}}.
    \end{equation*}
\end{definition}

Una de las propiedades más importantes del radio de convergencia es que, justamente, la serie converge para todo $z \in \C$ tal que $|z| < R$. (Si $R = \infty$, entonces converge en $\C$.)

\begin{lemma}
    \label{lem:desig-binom}
    Sean $n \in \N$ y $z, h \in \C$. Entonces se cumple la desigualdad
    \begin{equation*}
        \left| \sum_{j=2}^{n} \binom{n}{j} z^{\,k-j} h^j \right|
        \leq |h|^2 \, \frac{n(n-1)}{2} \, (|z|+|h|)^{n-2}.
    \end{equation*}
\end{lemma}

\begin{proof}
    Antes de continuar vamos a probar una desigualdad que nos va a ser de utilidad. 
    Consideremos
    \begin{align*}
        \left| \sum_{j=2}^{n} \binom{n}{j} z^{\,k-j} h^j \right|
        &= |h|^2 \, \left| \sum_{j=2}^{n} 
            \frac{n(n-1)}{j(j-1)} \binom{n-2}{j-2} \, z^{\,n-j} h^{\,j-2} \right|.
    \end{align*}
    Como $j \geq 2$, se cumple que $\tfrac{1}{j(j-1)} \leq \tfrac{1}{2}$. Así, 
    \begin{align*}
        \left| \sum_{j=2}^{n} \binom{n}{j} z^{\,k-j} h^j \right|
        &\leq |h|^2 \, \frac{n(n-1)}{2} 
            \left| \sum_{j=2}^{n} \binom{n-2}{j-2} \, z^{\,n-j} h^{\,j-2} \right|.
    \end{align*}
    Haciendo el cambio de índice $k = j-2$, queda
    \begin{align*}
        \left| \sum_{j=2}^{n} \binom{n}{j} z^{\,k-j} h^j \right|
        &\leq |h|^2 \, \frac{n(n-1)}{2} 
            \sum_{k=0}^{n-2} \binom{n-2}{k} |z|^{\,n-k-2} |h|^k \\
        &= |h|^2 \, \frac{n(n-1)}{2} (|z|+|h|)^{n-2}.
    \end{align*}
    Esto concluye la demostración.
\end{proof}

\begin{theorem}
    Sea $(a_n)_{n \in \N}$ una sucesión en $\C$ y sea $S = \sum a_n z^n$ la serie de potencias asociada, con radio de convergencia $R$. Entonces, la derivada de $S$ es
    \begin{equation*}
        S' = \left(\sum a_n z^n\right)' = \sum n a_n z^{n-1}
    \end{equation*}
\end{theorem}

\begin{proof}
    Primero, como
    \begin{equation*}
        \limsup_{n \to \infty} |n+1|^{\frac{1}{n}} |a_{n+1}|^{\frac{1}{n}} = \limsup_{n \to \infty} |a_{n+1}|^{\frac{1}{n+1}},
    \end{equation*}
    ambas $S'$ y $\sum n a_n z^{n-1}$ tienen el mismo radio de convergencia.

    Sean $z, h \in \C$ tales que $|z| + |h| < R$. Sea $\varepsilon > 0$. Buscamos un $\delta > 0$ tal que
    \begin{equation*}
        \text{si }|h| < \delta, \text{ entonces } |S(z + h) - S(z) - S'(z)h| < \varepsilon |h|.
    \end{equation*}

    Consideremos 
    \begin{align*}
        S(z + h) &= \sum_{k=0}^\infty a_k (z+h)^k \\
        &= \sum_{k=0}^\infty \sum_{j=0}^k \binom{k}{j} z^{k-j}h^j \\
        &= \sum_{k=0}^\infty a_k \left( z^k + kz^{k-1} h + \sum_{j=2}^k \binom{k}{j} z^{k-j}h^j\right)
    \end{align*}
    y también
    \begin{align*}
        S(z+h) &= \sum_{k=0}^{\infty} a_k z^k 
        + \sum_{k=0}^{\infty} a_{k+1}(k+1) z^k h
        + \sum_{k=0}^{\infty} a_k \sum_{j=2}^{k} \binom{k}{j} z^{k-j} h^j \\
        &= S(z) 
        + S'(z)h
        + \sum_{k=0}^{\infty} a_k \sum_{j=2}^{k} \binom{k}{j} z^{k-j} h^j.
    \end{align*}
    Por lo tanto,
    \begin{align*}
        S(z + h) - S(z) - S'(z)h &= \sum_{k=0}^{\infty} a_k \sum_{j=2}^{k} \binom{k}{j} z^{\,k-j} \\ 
        &=
    \end{align*}
\end{proof}

\end{document}
