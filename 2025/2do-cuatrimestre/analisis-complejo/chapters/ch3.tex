\chapter{Series en el plano complejo}

\section{Series numéricas y criterios de convergencia}

\begin{definition}
    Sea $(a_n)_{n \in \N_0}$ una sucesión de números complejos. 
    Decimos que la serie $\sum a_n$ \emph{converge} si existe el límite
    \begin{equation*}
        \lim_{N \to \infty} \sum_{n=0}^N a_n,
    \end{equation*}
    y lo denotamos por $\sum_{n=0}^\infty a_n$. Además, si la serie de valores absolutos converge,
    \begin{equation*}
        \sum_{n=0}^\infty |a_n| < +\infty,
    \end{equation*}
    decimos que $\sum a_n$ \emph{converge absolutamente}.
\end{definition}

Un primer hecho importante es que la convergencia absoluta implica la convergencia usual.  

\begin{proposition}
    Si $(a_n)_{n \in \N_0} \subseteq \C$ y la serie $\sum_{n=0}^{\infty} |a_n|$ converge, 
    entonces $\sum_{n=0}^{\infty} a_n$ converge.
\end{proposition}

\begin{proof}
    Sean $m<n$ y
    \begin{equation*}
        S_n = \sum_{j=0}^n a_j, \qquad 
        \sigma_n = \sum_{j=0}^n |a_j|.
    \end{equation*}
    Entonces
    \begin{equation*}
        |S_n - S_m|
        = \left|\sum_{j=0}^n a_j - \sum_{j=0}^m a_j\right|
        = \left|\sum_{j=m+1}^n a_j\right|
        \leq \sum_{j=m+1}^n |a_j|
        = \sigma_n - \sigma_m.
    \end{equation*}
    Como la serie de valores absolutos converge, 
    la diferencia $\sigma_n - \sigma_m$ se puede hacer arbitrariamente pequeña. 
    Por lo tanto, $(S_n)$ es de Cauchy y la serie converge.
\end{proof}

A continuación presentamos algunos criterios prácticos para decidir la convergencia de series.  

\begin{proposition}[Criterio de la comparación]
    Sean $(a_n)$, $(b_n)$ sucesiones de números reales positivos tales que $a_n \leq b_n$. Entonces:
    \begin{enumerate}
        \item Si $\sum b_n$ converge, también lo hace $\sum a_n$.
        \item Si $\sum a_n = +\infty$, entonces $\sum b_n = +\infty$.
    \end{enumerate}
\end{proposition}

Este criterio se usa sobre todo para comparar con series conocidas (geométrica, armónica, $p$-series, etc.).  

\begin{proposition}[Criterio de la raíz $n$-ésima de Cauchy]
    Sea $(a_n)$ sucesión de números complejos y sea
    \begin{equation*}
        \alpha = \limsup_{n \to \infty} |a_n|^{1/n}.
    \end{equation*}
    Entonces:
    \begin{enumerate}
        \item Si $\alpha < 1$, la serie $\sum a_n$ converge absolutamente.
        \item Si $\alpha > 1$, la serie $\sum a_n$ diverge.
        \item Si $\alpha = 1$, el criterio no es concluyente.
    \end{enumerate}
\end{proposition}

El criterio de la raíz es particularmente útil cuando $a_n$ involucra potencias elevadas en $n$, por ejemplo $a_n = \frac{n^k}{m^n}$.  

\begin{proposition}[Criterio de D’Alembert]
    Sea $(a_n)$ sucesión de números complejos y sea
    \begin{equation*}
        \beta = \limsup_{n \to \infty} \left|\frac{a_{n+1}}{a_n}\right|.
    \end{equation*}
    Entonces:
    \begin{enumerate}
        \item Si $\beta < 1$, la serie $\sum a_n$ converge absolutamente.
        \item Si $\beta > 1$, la serie $\sum a_n$ diverge.
        \item Si $\beta = 1$, el criterio no es concluyente.
    \end{enumerate}
\end{proposition}

Este último criterio, también llamado del \textit{cociente}, es uno de los más usados en la práctica, sobre todo cuando los términos de la serie contienen factoriales o potencias encadenadas.

\begin{proposition}[Criterio de la comparación en el límite]
    Sean $(a_n)$, $(b_n)$ sucesiones de números reales positivos.
    \begin{enumerate}
        \item Si $\lim_{n \to \infty} \tfrac{a_n}{b_n} = c > 0$, entonces $\sum a_n$ y $\sum b_n$ 
        convergen o divergen simultáneamente.
        \item Si $\lim_{n \to \infty} \tfrac{a_n}{b_n} = 0$ y $\sum b_n$ converge, entonces 
        $\sum a_n$ también converge.
        \item Si $\lim_{n \to \infty} \tfrac{a_n}{b_n} = +\infty$ y $\sum b_n = +\infty$, 
        entonces $\sum a_n = +\infty$.
    \end{enumerate}
\end{proposition}

Este criterio refina al de comparación directa y es muy útil cuando la razón de los términos se estabiliza.

\begin{proposition}[Criterio de la integral]
    Sea $(a_n)$ una sucesión de números reales no negativos y 
    $f : [0,+\infty) \to [0,+\infty)$ una función continua y decreciente tal que $f(n) = a_n$
    para todo $n \in \N_0$. Entonces, la serie $\sum a_n$ converge si y sólo si 
    converge la integral impropia
    \begin{equation*}
        \int_0^{\infty} f(t)\,dt.
    \end{equation*}
\end{proposition}

Este criterio se conoce también como \textit{criterio de la integral de Cauchy} y permite conectar 
la teoría de series con el análisis de integrales impropias.

\begin{definition}
    Sean $(a_n)$ y $(b_n)$ dos sucesiones de números complejos. Definimos el 
    \emph{producto de Cauchy} como la sucesión $(c_k)$ dada por
    \begin{equation*}
        c_k = \sum_{j=0}^k a_{k-j} b_j.
    \end{equation*}
\end{definition}

Este producto aparece de manera natural al multiplicar series de potencias, 
y será especialmente relevante en el estudio de funciones analíticas.

\begin{proposition}[Convolución de series absolutamente convergentes]
    Sean $(a_n)$ y $(b_n)$ sucesiones de números complejos tales que 
    $\sum a_n$ y $\sum b_n$ son absolutamente convergentes. 
    Entonces, la serie formada por el producto de Cauchy $(c_k)$ es absolutamente convergente y además
    \begin{equation*}
        \sum_{k=0}^\infty |c_k| \;\leq\;
        \left( \sum_{k=0}^\infty |a_k| \right)
        \left( \sum_{k=0}^\infty |b_k| \right).
    \end{equation*}
\end{proposition}

\begin{proof}
    Recordemos que el producto de Cauchy está dado por
    \begin{equation*}
        c_n = \sum_{j=0}^n a_j b_{n-j}.
    \end{equation*}
    En particular,
    \begin{align*}
        |c_0| &\leq |a_0 b_0|, \\
        |c_1| &\leq |a_0 b_1| + |a_1 b_0|, \\
        |c_2| &\leq |a_0 b_2| + |a_1 b_1| + |a_2 b_0|, \\
              &\;\;\vdots \\
        |c_n| &\leq |a_0 b_n| + |a_1 b_{n-1}| + \dots + |a_n b_0|.
    \end{align*}

    Sumando las desigualdades anteriores, obtenemos
    \begin{align*}
        \sum_{k=0}^n |c_k|
        &\leq \sum_{j=0}^n \sum_{m=0}^{\,n-j} |a_j|\,|b_m| \\
        &= \sum_{j=0}^n |a_j| \left( \sum_{m=0}^{\,n-j} |b_m| \right).
    \end{align*}
    Como $\sum b_n$ converge absolutamente, el sumando interno está acotado por
    $\sum_{m=0}^\infty |b_m|$. Así,
    \begin{equation*}
        \sum_{k=0}^n |c_k|
        \leq \left(\sum_{j=0}^n |a_j|\right) \left(\sum_{m=0}^\infty |b_m|\right).
    \end{equation*}

    Finalmente, dejando $n \to \infty$,
    \begin{equation*}
        \sum_{k=0}^\infty |c_k|
        \leq \left(\sum_{j=0}^\infty |a_j|\right)\left(\sum_{m=0}^\infty |b_m|\right).
    \end{equation*}
    Esto prueba tanto la convergencia absoluta de la serie de Cauchy como la desigualdad deseada.
\end{proof}

\begin{proposition}[Suma por partes]
    Sean $(a_n)$, $(b_n)$ sucesiones de números complejos y definamos
    \begin{equation*}
        A_n = \sum_{k=0}^n a_k.
    \end{equation*}
    Entonces, para todo $N>k$ se cumple
    \begin{equation*}
        \sum_{n=k}^N a_n b_n
        = A_N b_N - A_{k-1} b_k - \sum_{n=k}^{N-1} A_n (b_{n+1}-b_n).
    \end{equation*}
\end{proposition}

\begin{proof}
    Comenzamos desarrollando la suma:
    \begin{align*}
        \sum_{n=k}^N a_n b_n
        &= \sum_{n=k}^N (A_n - A_{n-1}) b_n \\
        &= \sum_{n=k}^N A_n b_n - \sum_{n=k}^N A_{n-1} b_n \\
        &= A_N b_N - A_{k-1} b_k + \sum_{n=k}^{N-1} A_n b_n - \sum_{n=k+1}^N A_{n-1} b_n.
    \end{align*}
    Reordenando,
    \begin{align*}
        \sum_{n=k}^N a_n b_n
        &= A_N b_N - A_{k-1} b_k + \sum_{n=k}^{N-1} A_n b_n - \sum_{n=k}^{N-1} A_n b_{n+1} \\
        &= A_N b_N - A_{k-1} b_k - \sum_{n=k}^{N-1} A_n (b_{n+1}-b_n),
    \end{align*}
    que es la identidad buscada.
\end{proof}

Este resultado se conoce también como \textit{fórmula de Abel} y es muy útil para analizar series con términos oscilantes.  

\begin{proposition}[Criterio de Dirichlet]
    Sea $(a_n)$ una sucesión de números reales con $a_{n+1} \leq a_n$ y $a_n \to 0$.  
    Sea $(b_n)$ una sucesión de números complejos tal que
    \begin{equation*}
        B_n = \left|\sum_{k=0}^n b_k\right| \leq C \quad \text{para todo } n.
    \end{equation*}
    Entonces, la serie $\sum a_n b_n$ es convergente.
\end{proposition}

\begin{proof}
    Usando suma por partes, para todo $N$ se tiene
    \begin{align*}
        \left|\sum_{k=0}^N a_k b_k\right|
        &\leq |B_N a_N| + \sum_{n=0}^{N-1} |B_n|\,|a_n - a_{n+1}| \\
        &\leq C|a_N| + C \sum_{n=0}^{N-1} (a_n - a_{n+1}).
    \end{align*}
    Como $a_n \to 0$, al tomar $N \to \infty$ resulta
    \begin{equation*}
        \left|\sum_{k=0}^\infty a_k b_k\right| \leq C a_0,
    \end{equation*}
    lo cual prueba la convergencia de la serie.
\end{proof}

\begin{remark}
    Sea $(a_n)$ una sucesión de reales no negativos, decreciente y tal que $a_n \to 0$. 
    Entonces, la serie $\sum a_n z^n$ converge absolutamente si $|z| < 1$, y converge también si $|z|=1$ pero $z \neq 1$.
\end{remark}

Este resultado es una aplicación directa del criterio de Dirichlet al caso de las llamadas 
\textit{series de Dirichlet}, y nos servirá para analizar ejemplos clásicos como la serie armónica alternada.


\section{Series de funciones}

Sea $(f_n: D \subseteq \C \to \C)_{n \in \N}$ una sucesión de funciones. Denotamos por
\begin{equation*}
    S_n(z) = \sum_{i=0}^n f_i(z)
\end{equation*}
a las sumas parciales.

\begin{definition}
    Decimos que una serie de funciones $\sum f_n$ \emph{converge puntualmente} si $S_n(z)$ converge para todo $z \in \D$. Además, si $S_n$ converge uniformemente, decimos que la serie $\sum f_n$ converge \emph{uniformemente}.
\end{definition}

\begin{remark}
    Notemos que la convergencia puntual habla sobre la convergencia de una sucesión numérica; en cambio, la convergencia uniforme habla sobre sucesión de funciones.
\end{remark}

\begin{proposition}
    Sea $(f_n: D \subseteq \C \to \C)_{n \in \N}$ una sucesión de funciones continuas tal que $\sum f_n$ converge uniformemente. Entonces, $\sum f_n$ es continua.
\end{proposition}

\begin{proof}
    consideremos las sumas parciales $S_n$. En sí, la proposición vale para sucesiones de funciones en general.
\end{proof}

Para la convergencia puntual, simplemente podemos fijar algún $z \in \C$ y utilizar los criterios de las series numéricas. Sin embargo, no tenemos un criterio para la convergencia uniforme.

\begin{proposition}[Criterio de Weierstrass]
    Sea $(f_n: D \subseteq \C \to \C)_{n \in \N}$ una sucesión de funciones que cumplen
    \begin{enumerate}
        \item Para cada $n \in \N$, existe $M_n > 0$ tal que $|f_n| < M_n$.
        \item La serie $\sum M_n$ converge.
    \end{enumerate}
    Entonces, $\sum f_n$ converge uniformemente.
\end{proposition}

Otra noción de convergencia:

\begin{definition}
    Sea $(f_n: D \subseteq \C \to \C)_{n \in \N}$ una sucesión de funciones continuas. Decimos que $\sum f_n$ \emph{converge normalmente} si, para todo $K \subseteq D$ compacto,
    \begin{equation*}
        \sum_{n = 0}^{\infty } \Vert f_n \Vert_K \text{ converge}.
    \end{equation*}
\end{definition}


\section{Series de potencias}

\begin{definition}
    Sea $(a_n)_{n \in \N}$ una sucesión en $\C$ y $z_0 \in \C$. Le decimos \emph{series de potencias} a las series de la forma
    \begin{equation*}
        \sum_{n=0}^{\infty} a_n (z - z_0)^n.
    \end{equation*}
\end{definition}

Nos limitamos a las seires de potencia centradas en $0$.

\begin{lemma}[Abel]
    Sea $(a_n)_{n \in \N}$ una sucesión en $\C$ y $z_0 \in \C \setminus \{0\}$ tal que $\sum a_n z_0^n$ converge. Entonces, $\sum a_n z^n$ converge uniforme y absolutamente en toda bola $\overline{B}(0, r)$, con $r < |z_0|$.
\end{lemma}

\begin{proof}
    Sea $(a_n)_{n \in \N}$ una sucesión en $\C$ y $z_0 \in \C \setminus \{0\}$ tal que $\sum a_n z_0^n$ converge. Necesariamente, $a_n z_0^n \to 0$; por lo tanto, existe $C > 0$ tal que $|a_n z_0^n|$, para todo $n \in \N$. Sea $r < |z_0|$ y $z \in \overline{B}(0, r)$. Entonces,
    \begin{equation*}
        |a_n z^n| = |a_n| |z|^n \leq |a_n| |z_0|^n \left( \frac{|z|}{|z_0|} \right)^n \leq C \left( \frac{|z|}{|z_0|} \right)^n.
    \end{equation*}
    La serie geométrica
    \begin{equation*}
       \sum_{n=0}^{\infty} C \left( \frac{|z|}{|z_0|} \right)^n
    \end{equation*}
    es convergente. Por el criterio de Weierstrass, $\sum a_n z^n$ converge uniforme y absolutamente en $\overline{B}(0, r)$.
\end{proof}

\begin{definition}
    Sea $(a_n)_{n \in \N}$ una sucesión en $\C$. El \emph{radio de convergencia} de la series $\sum a_n z^n$ es
    \begin{equation*}
        R = \frac{1}{\limsup_{n \to \infty} |a_n|^{\frac{1}{n}}}.
    \end{equation*}
\end{definition}

Una de las propiedades más importantes del radio de convergencia es que, justamente, la serie converge para todo $z \in \C$ tal que $|z| < R$. (Si $R = \infty$, entonces converge en $\C$.)

\begin{lemma}
    \label{lem:desig-binom}
    Sean $n \in \N$ y $z, h \in \C$. Entonces se cumple la desigualdad
    \begin{equation*}
        \left| \sum_{j=2}^{n} \binom{n}{j} z^{\,k-j} h^j \right|
        \leq |h|^2 \, \frac{n(n-1)}{2} \, (|z|+|h|)^{n-2}.
    \end{equation*}
\end{lemma}

\begin{proof}
    Antes de continuar vamos a probar una desigualdad que nos va a ser de utilidad. 
    Consideremos
    \begin{align*}
        \left| \sum_{j=2}^{n} \binom{n}{j} z^{\,k-j} h^j \right|
        &= |h|^2 \, \left| \sum_{j=2}^{n} 
            \frac{n(n-1)}{j(j-1)} \binom{n-2}{j-2} \, z^{\,n-j} h^{\,j-2} \right|.
    \end{align*}
    Como $j \geq 2$, se cumple que $\tfrac{1}{j(j-1)} \leq \tfrac{1}{2}$. Así, 
    \begin{align*}
        \left| \sum_{j=2}^{n} \binom{n}{j} z^{\,k-j} h^j \right|
        &\leq |h|^2 \, \frac{n(n-1)}{2} 
            \left| \sum_{j=2}^{n} \binom{n-2}{j-2} \, z^{\,n-j} h^{\,j-2} \right|.
    \end{align*}
    Haciendo el cambio de índice $k = j-2$, queda
    \begin{align*}
        \left| \sum_{j=2}^{n} \binom{n}{j} z^{\,k-j} h^j \right|
        &\leq |h|^2 \, \frac{n(n-1)}{2} 
            \sum_{k=0}^{n-2} \binom{n-2}{k} |z|^{\,n-k-2} |h|^k \\
        &= |h|^2 \, \frac{n(n-1)}{2} (|z|+|h|)^{n-2}.
    \end{align*}
    Esto concluye la demostración.
\end{proof}

\begin{theorem}
    Sea $(a_n)_{n \in \N}$ una sucesión en $\C$ y sea $S = \sum a_n z^n$ la serie de potencias asociada, con radio de convergencia $R$. Entonces, la derivada de $S$ es
    \begin{equation*}
        S' = \left(\sum a_n z^n\right)' = \sum n a_n z^{n-1}
    \end{equation*}
\end{theorem}

\begin{proof}
    Primero, como
    \begin{equation*}
        \limsup_{n \to \infty} |n+1|^{\frac{1}{n}} |a_{n+1}|^{\frac{1}{n}} = \limsup_{n \to \infty} |a_{n+1}|^{\frac{1}{n+1}},
    \end{equation*}
    ambas $S'$ y $\sum n a_n z^{n-1}$ tienen el mismo radio de convergencia.

    Sean $z, h \in \C$ tales que $|z| + |h| < R$. Sea $\varepsilon > 0$. Buscamos un $\delta > 0$ tal que
    \begin{equation*}
        \text{si }|h| < \delta, \text{ entonces } |S(z + h) - S(z) - S'(z)h| < \varepsilon |h|.
    \end{equation*}

    Consideremos 
    \begin{align*}
        S(z + h) &= \sum_{k=0}^\infty a_k (z+h)^k \\
        &= \sum_{k=0}^\infty \sum_{j=0}^k \binom{k}{j} z^{k-j}h^j \\
        &= \sum_{k=0}^\infty a_k \left( z^k + kz^{k-1} h + \sum_{j=2}^k \binom{k}{j} z^{k-j}h^j\right)
    \end{align*}
    y también
    \begin{align*}
        S(z+h) &= \sum_{k=0}^{\infty} a_k z^k 
        + \sum_{k=0}^{\infty} a_{k+1}(k+1) z^k h
        + \sum_{k=0}^{\infty} a_k \sum_{j=2}^{k} \binom{k}{j} z^{k-j} h^j \\
        &= S(z) 
        + S'(z)h
        + \sum_{k=0}^{\infty} a_k \sum_{j=2}^{k} \binom{k}{j} z^{k-j} h^j.
    \end{align*}
    Por lo tanto,
    \begin{align*}
        S(z + h) - S(z) - S'(z)h &= \sum_{k=0}^{\infty} a_k \sum_{j=2}^{k} \binom{k}{j} z^{\,k-j} \\ 
        &=
    \end{align*}
\end{proof}