\chapter{Probabilidad condicional}

\section{Probabilidad condicional}

Ya vimos cómo calcular probabilidades de eventos. Ahora queremos formalizar qué significa calcular probabilidades \textit{condicionadas} a que cierto evento ya ocurrió.

\begin{example}
    Tiro dos dados y sumo los resultados. Consideremos los eventos
    \begin{align*}
        A &= \{\text{la suma es $5$}\}, \\
        B &= \{\text{el segundo dado es par}\}.
    \end{align*}
    El espacio muestral es
    \begin{equation*}
        \Omega = \{(a,b) \mid a,b \in \{1,\ldots,6\}\}.
    \end{equation*}
    En particular,
    \begin{equation*}
        A = \{(1,4),(4,1),(2,3),(3,2)\}.
    \end{equation*}
    Luego,
    \begin{equation*}
        P(A \mid B) = \frac{\# (A \cap B)}{\# B} = \frac{2}{18} = \frac{1}{9}.
    \end{equation*}
\end{example}

El cálculo anterior nos motiva a introducir la definición general de probabilidad condicional.

\begin{definition}
    Dados $A$ y $B$ eventos, con $P(B) \neq 0$, la \emph{probabilidad condicional} de $A$ dado $B$ se define como
    \begin{equation*}
        P(A \mid B) = \frac{P(A \cap B)}{P(B)}.
    \end{equation*}
\end{definition}

Veamos cómo aplica la definición en un ejemplo.

\begin{example}
    Consideremos una urna con $9$ bolitas de la siguiente forma:
    \begin{itemize}
        \item $5$ naranjas,
        \item $4$ violetas, de las cuales $3$ tienen una cruz.
    \end{itemize}
    Denotemos $N=$ \{\text{naranja}\}, $V=$ \{\text{violeta}\}, $C=$ \{\text{tiene cruz}\}. Entonces,
    \begin{align*}
        P(N) &= \frac{5}{9}, &
        P(N \mid V) &= \frac{1}{3}, &
        P(C \mid V) &= \frac{1}{2}.
    \end{align*}
\end{example}

A partir de la definición, podemos verificar fácilmente que $P(\,\cdot \mid B)$ cumple los axiomas de probabilidad.

\begin{proposition}
    Sea $B$ un evento con $P(B) \neq 0$. Entonces la aplicación
    \begin{equation*}
        A \mapsto P(A \mid B)
    \end{equation*}
    define una probabilidad sobre el espacio muestral condicionado.
\end{proposition}

\begin{proof}
    Para todo $B$ con $P(B)>0$:
    \begin{itemize}
        \item[(P1)] $P(\Omega \mid B) = \frac{P(\Omega \cap B)}{P(B)} = \frac{P(B)}{P(B)}=1$.
        \item[(P2)] Como $P$ es una probabilidad, $P(A \mid B) \geq 0$.
        \item[(P3)] Si $\{A_n\}_{n \in \N}$ son eventos disjuntos dos a dos,
        \begin{align*}
            P\!\left(\bigcup_{n \in \N} A_n \,\middle|\, B\right)
            &= \frac{P\!\left(\bigcup_{n \in \N}(A_n \cap B)\right)}{P(B)} \\
            &= \frac{\sum_{n \in \N} P(A_n \cap B)}{P(B)} \\
            &= \sum_{n \in \N} \frac{P(A_n \cap B)}{P(B)} \\
            &= \sum_{n \in \N} P(A_n \mid B).
        \end{align*}
    \end{itemize}
\end{proof}

\begin{remark}
    Es importante aclarar que $A \mid B$ no es un evento. Por lo tanto, expresiones como $P(A \mid B \mid C)$ no tienen sentido.
\end{remark}

\begin{proposition}
    Para todo par de eventos $A, B$ con $P(B) \neq 0$, se cumple
    \begin{equation*}
        P(A \mid B) \, P(B) = P(A \cap B).
    \end{equation*}
\end{proposition}

Este resultado es útil para calcular probabilidades conjuntas a partir de probabilidades condicionales.

\begin{example}
    Sacamos dos cartas de un mazo de $52$ cartas sin reemplazo. Sea
    \begin{align*}
        A_1 &= \{\text{la primera carta es trébol}\}, \\
        A_2 &= \{\text{la segunda carta es trébol}\}.
    \end{align*}
    Entonces,
    \begin{align*}
        P(A_1) &= \frac{13}{52}, \\
        P(A_2 \mid A_1) &= \frac{12}{51}.
    \end{align*}
    Aplicando la proposición anterior,
    \begin{equation*}
        P(A_1 \cap A_2) = P(A_1)\,P(A_2 \mid A_1) = \frac{13}{52} \cdot \frac{12}{51}.
    \end{equation*}
\end{example}

\begin{proposition}
    Para eventos $A_1, A_2, \dots, A_n$ se cumple
    \begin{multline*}
        P(A_1 \cap A_2 \cap \cdots \cap A_n)
        \\= P(A_1) P(A_2 \mid A_1) P(A_3 \mid A_1 \cap A_2) \cdots P(A_n \mid A_1 \cap \cdots \cap A_{n-1}).
    \end{multline*}
\end{proposition}

\begin{proof}[Idea de demostración]
    La demostración se hace fácilmente por inducción en $n$.
\end{proof}

\section{Probabilidad total}

La noción de probabilidad condicional nos permite descomponer probabilidades en términos de eventos disjuntos.

\begin{proposition}[Probabilidad total en dos partes]
    Sea $B \subseteq \Omega$ un evento. Entonces, para todo $A \subseteq \Omega$,
    \begin{align*}
        P(A) &= P(A \cap B) + P(A \cap B^{\mathrm c}) \\
             &= P(A \mid B)\,P(B) + P(A \mid B^{\mathrm c})\,P(B^{\mathrm c}).
    \end{align*}
\end{proposition}

Más en general, si $\{A_i\}_{i \in I}$ es una partición numerable de $\Omega$, se cumple la \textit{ley de la probabilidad total}:
\begin{equation*}
    P(B) = \sum_{i \in I} P(B \mid A_i)\,P(A_i).
\end{equation*}

\begin{example}
    Consideremos el siguiente experimento:
    \begin{centeredvarwidth}
        \begin{enumerate}
            \item En una caja hay $3$ monedas: una con dos caras y dos equilibradas.
            \item Se elige una moneda al azar.
            \item Se lanza la moneda.
        \end{enumerate}
    \end{centeredvarwidth}

    Definamos los eventos
    \begin{align*}
        C &= \{\text{sale cara}\}, \\
        A &= \{\text{la moneda elegida es la de dos caras}\}.
    \end{align*}

    Entonces,
    \begin{align*}
        P(C) &= P(C \mid A)\,P(A) + P(C \mid A^{\mathrm{c}})\,P(A^{\mathrm{c}}) \\
             &= 1 \cdot \frac{1}{3} + \frac{1}{2} \cdot \frac{2}{3} \\
             &= \frac{2}{3}.
    \end{align*}
\end{example}

Este cálculo es un primer ejemplo de la \textit{fórmula de la probabilidad total}, que nos permite descomponer la probabilidad de un evento en función de una partición del espacio.

\begin{example}
    Ahora consideremos otro experimento. Lanzamos un dado. Sea
    \begin{align*}
        A &= \{\text{saco un $5$}\}, \\
        B &= \{\text{saco un $7$}\}, \\
        C &= \{\text{ni $5$ ni $7$}\} = (A \cup B)^{\mathrm c}.
    \end{align*}

    Supongamos que ganar depende de estos eventos: 
    si ocurre $A$ se gana siempre, 
    si ocurre $B$ nunca se gana, 
    y si ocurre $C$ se gana con cierta probabilidad.

    Denotemos $G=\{\text{ganar}\}$. Entonces, por probabilidad total,
    \begin{align*}
        P(G) &= P(G \mid A)\,P(A) + P(G \mid B)\,P(B) + P(G \mid C)\,P(C).
    \end{align*}

    Como $P(G \mid A)=1$, $P(G \mid B)=0$, y $P(A)=P(B)=\frac{1}{6}$, $P(C)=\frac{4}{6}$, se obtiene
    \begin{align*}
        P(G) &= 1 \cdot \frac{1}{6} + 0 \cdot \frac{1}{6} + P(G \mid C)\,\frac{4}{6}.
    \end{align*}

    Si además sabemos que $P(G \mid C)=\frac{1}{2}$, queda
    \begin{equation*}
        P(G) = \frac{1}{6} + \frac{1}{2}\cdot\frac{4}{6} = \frac{6}{15}.
    \end{equation*}
\end{example}

Estos ejemplos muestran cómo las probabilidades condicionales y la probabilidad total se combinan para calcular probabilidades en situaciones más complejas.


\section{Independencia de eventos}

La independencia de eventos se relaciona íntimamente con la probabilidad condicional. En esencia, si la probabilidad de un evento $A$ dado un evento $B$ no cambia, entonces son independientes.

\begin{definition}
    Decimos que dos eventos $A$ y $B$ son independientes si
    \begin{equation*}
        P(A \cap B) = P(A) P(B).
    \end{equation*}
\end{definition}

\begin{remark}
    Si $P(A) = 0$ o $P(A) = 1$, entones es independiente de cualquier evento.
\end{remark}

Esta noción la podemos generalizar a más eventos.

\begin{definition}
    Decimos que los eventos $A_1, A_2, \ldots, A_n$ son independientes si, para todo $I \subseteq \{1, 2, \ldots, n\}$,
    \begin{equation*}
        P\left(\bigcap_{i \in I} A_i\right) = \prod_{i \in I} P(A_i).
    \end{equation*}
\end{definition}

Y para familias de eventos:

\begin{definition}
    Decimos que una familia de eventos $\mathcal{A}$ es independiente si todo subconjunto finito de $\mathcal{A}$ es independiente.
\end{definition}