% GRÁFICO
\begin{tikzpicture}[xscale=5, yscale=3]
	% Configuración de los ejes
	\draw[->] (-0.1,0) -- (1.1,0) node[below] {$x$};
	\draw[->] (0,-0.1) -- (0,1.1) node[left] {$y$};

	% Marcas en los ejes
	% Se envuelve \xtext en $...$ para asegurar que el contenido matemático se interprete correctamente.
	% Se usa {1} en lugar de 1 para que el foreach lo trate como un argumento agrupado.
	\foreach \x/\xtext in {1/{1}, 0.5/{\frac{1}{2}}, 0.333/{\frac{1}{3}}, 0.25/{\frac{1}{4}}, 0.2/{\frac{1}{5}}}
	{
	\draw (\x,0.02) -- (\x,-0.02) node[below] {$\xtext$};
	}

	% Para indicar que se acumulan hacia 0
	\draw (0.1,0.02) -- (0.1,-0.02);
	\draw (0.05,0.02) -- (0.05,-0.02);

	\draw (0.02,1) -- (-0.02,1) node[left] {$1$};
	\draw (0.02,0.5) -- (-0.02,0.5) node[left] {$0.5$};

	% Parte 2: El segmento horizontal (0,1) en el eje x
	% Interpretamos (0,1) como el intervalo abierto en el eje x: {(x,0) | 0 < x < 1}
	\draw[line width=1.5pt, accentcolor] (0,0) -- (1,0);

	% Parte 1: Los segmentos verticales {1/n} x [0,1]
	% Dibujamos los primeros N dientes (por ejemplo, hasta n=5)
	\foreach \n in {1,2,3,4,5} {
			\pgfmathsetmacro{\xcoord}{1/\n}
			\draw[line width=1.5pt, accentcolor] (\xcoord,0) -- (\xcoord,1);
		}

	% Indicamos la acumulación de los dientes hacia el eje Y
	% con puntos suspensivos o una línea punteada que se difumina
	\node[blue, align=center, scale=0.8] at (0.11,0.5) {$\ldots \ldots$};
	\draw[line width=1.5pt, accentcolor, opacity=0.5] (0.02,0) -- (0.02,1); % Un "diente" simbólico muy cerca del eje Y

	% Etiqueta del conjunto
	\node[below right, accentcolor, scale=1.2] at (0.6,0.8) {$X$};

\end{tikzpicture}
