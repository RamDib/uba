\chapter{Ejercicios Surtidos}

Este capítulo es en preparación para el final. Contiene la resolución de distintos ejercicios que me fui encontrando en el camino y que me parecieron interesantes.

\begin{exercise}
    Todas las normas en $\mathbb{R}^n$ son equivalentes.
\end{exercise}

\begin{proof}
    Sean $\| \cdot \|$ y $\| \cdot \|'$ normas arbitrarias de $\mathbb{R}^n$. Recordemos que dos normas son equivalentes si existen constantes $C_1, C_2 > 0$ tales que para todo $x \in \mathbb{R}^n$ se cumple:
    \begin{equation*}
        C_1 \| x \| \leq \| x \|' \leq C_2 \| x \|.
    \end{equation*}
    Un camino más fácil es probar que toda norma es equivalente a $\| \cdot \|_{\infty}$, ya que la equivalencia de normas es transitiva.

    Veamos primero que existe una constante $C_1 > 0$ tal que para todo $x \in \mathbb{R}^n$ se cumple:
    \begin{equation*}
        \| x \| \leq C_1 \| x \|_{\infty}.
    \end{equation*}
    Consideramos
    \begin{align*}
        \| x \| = \| x_1 \cdot e_1 + \dots + x_n \cdot e_n \| &\leq |x_1| \cdot \| e_1 \| + \dots + |x_n| \cdot \| e_n \| \\
        &\leq \| x \|_{\infty} \cdot (\| e_1 \| + \dots + \| e_n \|),
    \end{align*}
    y como $\| e_1 \| + \dots + \| e_n \|$ es constante, entonces podemos tomar $C_1 = \| e_1 \| + \dots + \| e_n \|$, obteniendo así 
    \begin{equation*}
        \| x \| \leq C_1 \| x \|_{\infty}.
    \end{equation*}
    
    Ahora, veamos que existe una constante $C_2 > 0$ tal que para todo $x \in \mathbb{R}^n$ se cumple:
    \begin{equation*}
        \| x \|_{\infty} \leq C_2 \| x \|.
    \end{equation*}
    Definimos al conjunto $S = \{ x \in \mathbb{R}^n : \| x \| = 1 \}$, que es cerrado y acotado, por lo que es compacto. Definimos también la función $\varphi : S \to \mathbb{R}$ dada por $\varphi(x) = \| x \|_{\infty}$. Esta función es continua porque $\| x \|_{\infty}$ es una norma, y por lo tanto es continua en $\mathbb{R}^n$. Como $S$ es compacto, $\varphi$ alcanza su máximo en $S$ en un punto $x_0$.

    Todo $x \in \mathbb{R}^n$ puede escribirse como $x = \| x \| \cdot \frac{x}{\| x \|}$ si $\| x \| \neq 0$, y por lo tanto
    \begin{align*}
        \| x \|_{\infty} &= \left\| \frac{x}{\| x \|} \right\|_{\infty} \cdot \| x \| \\
        &\leq \varphi(x_0) \cdot \| x \|.
    \end{align*}
    Basta con tomar $C_2 = \varphi(x_0)$, y así obtenemos la desigualdad buscada.
\end{proof}

\begin{exercise}
    Hallar el cardinal de $C(\mathbb{R}, \mathbb{R})$, el espacio de funciones continuas de $\mathbb{R}$ a $\mathbb{R}$.
\end{exercise}

\begin{proof}
    Dado que $\mathbb{Q}$ es denso en $\mathbb{R}$, las funciones continuas en $\mathbb{R}$ están completamente determinadas por sus valores en $\mathbb{Q}$. Es decir, si $f, g \in C(\mathbb{R}, \mathbb{R})$ y $f(q) = g(q)$ para todo $q \in \mathbb{Q}$, entonces $f = g$. Por lo tanto, existe un biyección entre $C(\mathbb{R}, \mathbb{R})$ y $C(\mathbb{Q}, \mathbb{R})$. Entonces, basta con calcular el cardinal de $C(\mathbb{Q}, \mathbb{R})$.

    Enumeramos a $\mathbb{Q}$ como $\{ q_1, q_2, \ldots \}$. Para definir una función en $C(\mathbb{Q}, \mathbb{R})$, necesitamos asignar un valor real a cada $q_i$. Como $\mathbb{R}$ tiene cardinalidad $2^{\aleph_0}$, el número de funciones de $\mathbb{Q}$ a $\mathbb{R}$ es $(2^{\aleph_0})^{\aleph_0} = 2^{\aleph_0 \cdot \aleph_0} = 2^{\aleph_0}$. Por lo tanto, el cardinal de $|C(\mathbb{R}, \mathbb{R})| \leq 2^{\aleph_0}$. Y para ver que el cardinal es igual, el conjunto de funciones constantes es un subconjunto de $C(\mathbb{R}, \mathbb{R})$ y tiene cardinalidad $2^{\aleph_0}$, ya que cada función constante puede ser identificada con un número real. Por lo tanto, el cardinal de $C(\mathbb{R}, \mathbb{R})$ es exactamente $2^{\aleph_0}$.
\end{proof}

\begin{exercise}
    Sea $(X, d)$ un espacio métrico completo sin puntos aislados y sea $D \subseteq X$ un subconjunto denso numerable. Probar que $D$ no es un $G_\delta$.
\end{exercise}

\begin{proof}
    Lo probamos por absurdo. Supongamos que $D = \{x_n\}_{n \in \mathbb{N}}$ es un conjunto $G_\delta$, entonces
    \begin{equation*}
        D = \bigcap_{n \in \mathbb{N}} U_n,
    \end{equation*}
    donde cada $U_n$ es un conjunto abierto y como $D \subseteq U_n$, también es denso. Consideramos ahora a la sucesión de abiertos dada por
    \begin{equation*}
        V_n = U_n \setminus \bigcup_{k=1}^{n} x_k.
    \end{equation*}
    Cabe aclarar que $V_n$ es abierto porque es la diferencia de un abierto y un conjunto finito. Además, como no hay puntos aislados en $X$, cada $V_n$ sigue siendo denso. Entonces, consideramos
    \begin{equation*}
        \bigcap_{n \in \mathbb{N}} V_n = \bigcap_{n \in \mathbb{N}} (U_n \setminus \bigcup_{k=1}^{n} x_k) = \varnothing.
    \end{equation*}
    Sin embargo, esto contradice el Teorema de Baire, dado que la intersección numerable de conjuntos abiertos densos es densa; en particular, no puede ser vacía. Por lo tanto, $D$ no puede ser un conjunto $G_\delta$.
\end{proof}

\begin{exercise}
    Probar que:
    \begin{enumerate}
        \item[(a)] Las componentes conexas de un espacio métrico son cerradas.
        \item[(b)] Si el espacio tiene finitas componentes conexas, las componentes son abiertas.
        \item[(c)] Dar un ejemplo de un espacio métrico con una componente conexa no abierta.
        \item[(d)] Si $f : E \to \mathbb{Z}$ es continua, entonces $f$ es constante en cada componente conexa de $E$.
    \end{enumerate}
\end{exercise}

\begin{proof}
    (a) Sea $C$ una componente cerrada de un espacio métrico $X$. Probamos que la clausura de un conjunto conexo es conexo. 

    Sea $A \subseteq X$ un conjunto conexo. Para ver que $\overline{A}$ es conexo, basta con probar que $\overline{A}$ no puede ser escrito como la unión disjunta de dos conjuntos abiertos no vacíos. Supongamos que $\overline{A} = U \cup V$ con $U$ y $V$ abiertos y no vacíos, y $U \cap V = \emptyset$. Entonces, $U \cap A$ y $V \cap A$ son conjuntos abiertos no vacíos en $A$, y por lo tanto, $A$ se puede escribir como la unión disjunta de estos dos conjuntos, lo cual contradice la conexidad de $A$. Por lo tanto, $\overline{A}$ es conexo.

    Usando esto, dado que $C$ es una componente conexa, como $C \cap \overline{C} \neq \emptyset$ y $\overline{C}$ es conexo, entonces $\overline{C} \subseteq C$. Por lo tanto, $C$ es cerrado.

    (b) Sea $X$ un espacio métrico con finitas componentes conexas. Podemos escribir a $X$ como la unión disjunta de sus componentes conexas,
    \begin{equation*}
        X = C_1 \cup C_2 \cup \dots \cup C_n,
    \end{equation*}
    donde cada $C_i$ es una componente conexa. En particular, como cada $C_i$ es cerrado y hay finitas componentes conexas, $\bigcup_{j \neq i} C_j$ es cerrado. Por lo tanto, su complemento 
    \begin{equation*}
        X \setminus \bigcup_{j \neq i} C_j = C_i
    \end{equation*}
    es abierto. Entonces, cada componente conexa es abierta.

    (c) Un ejemplo de un espacio métrico con una componente conexa no abierta es el conjunto $\{ \frac{1}{n} \mid n \in \mathbb{N} \} \cup \{ 0 \}$. Sea $r > 0$. La bola abierta $B(0, r)$ contiene a algún $\frac{1}{N}$ para $N$ suficientemente grande, por lo tanto $\{ \frac{1}{n} \mid n \in \mathbb{N} \} \cup \{ 0 \}$ no es abierto.
    
    (d) Sea $f : E \to \mathbb{Z}$ continua. Sea $C$ una componente conexa de $E$. Como $C$ es conexa, no se puede expresar como unión de abiertos disjuntos no vacíos. Recordemos que todo subconjunto de $\mathbb{Z}$ es abierto, por lo tanto su preimagen es abierta. Supongamos que $f|_C$ no es constante, entonces podemos expresar a $C$ como la unión disjunta de dos conjuntos abiertos no vacíos, $f|_C^{-1}(a)$ y $f|_C^{-1}(b)$, donde $a, b \in \mathbb{Z}$ y $a \neq b$. Esto contradice que $C$ sea conexo. Por lo tanto, $f|_C$ es constante.
\end{proof}

\begin{exercise}
    Sean $E$ y $F$ espacios de Banach y $D \subseteq E$ un subconjunto denso. Sea $(T_n)_{n \in \mathbb{N}} \subseteq \mathcal{L}(E, F)$ una sucesión de operadores tal que $\sup_{n \in \mathbb{N}} \| T_n \| < +\infty$. Supongamos que $(T_n(x))_{n \in \mathbb{N}}$ converge para todo $x \in D$. Probar que $(T_n)_{n \in \mathbb{N}}$ converge puntualmente a un $T \in \mathcal{L}(E, F)$.
\end{exercise}

\begin{proof}
    Consideremos la función $\widetilde{T} : D \to F$ definida por
    \begin{equation*}
        \widetilde{T}(x) = \lim_{n \to \infty} T_n(x).
    \end{equation*}
    Esta función está bien definida porque la sucesión $(T_n(x))_{n \in \mathbb{N}}$ converge para todo $x \in D$. Consideramos ahora la extensión de $\widetilde{T}$ a todo $E$. Definimos $T : E \to F$ como
    \begin{equation*}
        T(x) = \lim_{n \to \infty} \widetilde{T}(x_n),
    \end{equation*}
    donde $(x_n)_{n \in \mathbb{N}}$ es una sucesión tal que $x_n \to x$. 

    Veamos que existe el límite. Sea $(x_n)_{n \in \mathbb{N}}$ una sucesión que converge a $x \in X$. Queremos ver que $(\widetilde{T}(x_n))_{n \in \mathbb{N}}$ es convergente. Como estamos en un completo, basta ver que es de Cauchy. Sean $m, n \in \mathbb{N}$. Entonces,
    \begin{align*}
        \| \widetilde{T}(x_n) - \widetilde{T}(x_m) \|  &= \| \widetilde{T}(x_n - x_m) \| \\
        &= \| \lim_{k \to \infty} T_k (x_n - x_m) \\
        &\leq \sup_{k \in \mathbb{N}} \| T_k \| \cdot \| x_n - x_m \| \longrightarrow 0.
    \end{align*}
    Por lo tanto, es de Cauchy.
    
    Veamos que está bien definida. Sean $x \in E$, y consideremos dos sucesiones convergentes $(x_n)_{n \in \mathbb{N}}$ y $(y_n)_{n \in \mathbb{N}}$ en $D$ que convergen a $x$. Entonces, para todo $n$, tenemos que
    \begin{align*}
        \| \widetilde{T}(x_n) - \widetilde{T}(y_n) \| &= \left\| \lim_{k \to \infty} T_k(x_n) - \lim_{k \to \infty} T_k(y_n) \right\| \\
        &\leq \lim_{k \to \infty} \| T_k(x_n) - T_k(y_n) \| \\
        &\leq \sup_{k \in \mathbb{N}} \| T_k \| \cdot \| x_n - y_n \|.
    \end{align*}
    Por hipótesis, $\sup_{k \in \mathbb{N}} \| T_k \| < +\infty$, y como $x_n \to x$ y $y_n \to x$, entonces $\| x_n - y_n \| \to 0$. Por lo tanto, $T$ está bien definida.

    Ahora, veamos que $T$ es lineal. Sea $x, y \in E$ y $\alpha, \beta \in \mathbb{R}$. Entonces,
    \begin{align*}
        T(\alpha x + \beta y) &= \lim_{n \to \infty} \widetilde{T}(\alpha x_n + \beta y_n) \\
        &= \lim_{n \to \infty} \lim_{k \to \infty} T_k(\alpha x_n + \beta y_n) \\
        &= \lim_{n \to \infty} (\alpha \lim_{k \to \infty} T_k(x_n) + \beta \lim_{k \to \infty} T_k(y_n)) \\
        &= \alpha T(x) + \beta T(y).
    \end{align*}
    Por ende, $T$ es lineal.

    Veamos que $T$ es continua. Basta con probar que es acotada. Consideremos $x \in E$ y una sucesión $(x_n)_{n \in \mathbb{N}}$ en $D$ tal que $x_n \to x$. Entonces,
    \begin{align*}
        \| T(x) \| &= \left\| \lim_{n \to \infty} \widetilde{T}(x_n) \right\| \\
        &\leq \lim_{n \to \infty} \| \widetilde{T}(x_n) \| \\
        &\leq \lim_{n \to \infty} \sup_{k \in \mathbb{N}} \| T_k \| \cdot \| x_n \| \\
        &= \sup_{k \in \mathbb{N}} \| T_k \| \cdot \lim_{n \to \infty} \| x_n \| \\
        &= \sup_{k \in \mathbb{N}} \| T_k \| \cdot \| x \|.
    \end{align*}
    Por lo tanto, $T$ es acotada, y como es lineal y acotada, entonces $T \in \mathcal{L}(E, F)$.
\end{proof}

\begin{exercise}
    Sea $X$ compacto y $f : X \to X$ una isometría. Probar que $f$ es sobreyectiva.
\end{exercise}

\begin{proof}
    Como $f$ es una isometría, preserva las distancias, es decir, para todo $x, y \in X$, se cumple que $d(x, y) = d(f(x), f(y))$.

    Supongamos que $f$ no es sobreyectiva, entonces existe un $x_0 \in X$ tal que no está en la imagen de $f$. Por lo tanto, dado que $X$ es cerrado, su imagen también, y entonces $d(x_0, f(X)) > \varepsilon$, con $\varepsilon > 0$. Pero como $f$ es una isometría,
    \begin{equation*}
        \varepsilon < d(x_0, f(x_0)) = d(f(x_0), f^2(x_0)) = d(f^2(x_0), f^3(x_0)) = \cdots
    \end{equation*}
    También notemos que 
    \begin{equation*}
        \varepsilon < d(x_0, f(X)) < d(x_0, f^k(x_0)) = d(f^n(x_0), f^{n + k}(x_0)).
    \end{equation*}
    Por lo tanto, la sucesión $(f^n(x_0))_{n \in \mathbb{N}}$ no tiene subsucesión convergente, pero como estamos en un espacio compacto, esto es absurdo.
\end{proof}

\begin{exercise}
    Sea $X$ un espacio métrico completo y $\{F_n\}_{n \in \mathbb{N}}$ una familia de cerrados tal que $X = \bigcup_{n \in \mathbb{N}} F_n$. Probar que $\bigcup_{n \in \mathbb{N}} (F_n)^{\circ}$ es denso.
\end{exercise}

\begin{proof}
    Veamos que $\bigcup_{n \in \mathbb{N}} (F_n)^{\circ}$ es denso. 
    
    Sea $x \in X$ y $r > 0$. Necesitamos probar que $B(x, r) \cap \bigcup_{n \in \mathbb{N}} (F_n)^{\circ} \neq \varnothing$. Supongamos que la intersección es vacía. Como $\overline{B}\left(x, \frac{r}{2}\right)$ es un cerrado en un completo, es un subespacio completo. Consideramos ahora la intersección $F_n \cap \overline{B}\left(x, \frac{r}{2}\right)$ que es cerrada. Y también sabemos que $(F_n)^{\circ} \cap \overline{B}\left(x, \frac{r}{2}\right) = \varnothing$. Por lo tanto, $F_n \cap \overline{B}\left(x, \frac{r}{2}\right)$ es cerrado de interior vacío en el subespacio $\overline{B}\left(x, \frac{r}{2}\right)$. Entonces,
    \begin{equation*}
        \bigcup_{n \in \mathbb{N}} \left(F_n \cap \overline{B}\left(x, \frac{r}{2}\right)\right) = \overline{B}\left(x, \frac{r}{2}\right),
    \end{equation*}
    y por el Teorema de Baire, la unión numerable de nunca densos tiene interior vacío, lo cual es absurdo dado que $(\overline{B}\left(x, \frac{r}{2}\right))^{\circ} \neq \varnothing$. Por lo tanto, $\bigcup_{n \in \mathbb{N}} (F_n)^{\circ}$ es denso en $X$.
\end{proof}

\begin{exercise}
    Sea $E$ un espacio normado ---no lo aclara el ejercicio, pero consideremos al espacio sobre un cuerpo completo $K$--- y sea $S$ un subespacio vectorial de $E$. Probar:
    \begin{enumerate}
        \item[(a)] Si $\dim(S) < \infty$, entonces $S$ es cerrado en $E$.
        \item[(b)] Dar un ejemplo de $E$ normado y $S$ subespacio propio de $E$ tal que $S$ es denso en $E$.
    \end{enumerate}
\end{exercise}

\begin{proof}
    (a) Procedemos por inducción. Consideremos el subespacio $\{0\}$ de dimensión $0$, es trivialmente cerrado. 

    Sea $S$ un subespacio vectorial cerrado y sea $x_0 \not\in S$. Queremos ver que $S \oplus \langle x_0 \rangle$ es cerrado. Sea $(v_n)_{n \in \mathbb{N}}$ una sucesión en $S \oplus \langle x_0 \rangle$ que converge a $v \in E$. Dado que $v_n \in S \oplus \langle x_0 \rangle$, se puede expresar como
    \begin{equation*}
        v_n = s_n + \lambda_n x_0,
    \end{equation*}
    donde $s_n \in S$ y $\lambda_n \in K$. Consideramos la función $\varphi : S \oplus \langle x_0 \rangle \to K$ tal que $\varphi(v) = \lambda$, donde $v = s + \lambda x_0$, con $s \in S$ y $\lambda \in K$. La función está bien definida ya que están en suma directa. No es difícil ver que $\varphi$ es lineal y $\ker \varphi = S$, y como $\ker \varphi$ es cerrado, $\varphi$ es continua, o sea, uniformemente continua.

    Consideramos ahora $(\varphi(v_n))_{n \in \mathbb{N}}$. Dado que $(v_n)_{n \in \mathbb{N}}$ es convergente, es de Cauchy. Por lo tanto, $(\varphi(v_n))_{n \in \mathbb{N}}$ es de Cauchy, por continuidad uniforme de $\varphi$. Por último, como $K$ es completo, $(\varphi(v_n))_{n \in \mathbb{N}}$ converge a algún $\lambda \in K$. Por ende,
    \begin{align*}
        \lim_{n \to \infty} s_n &= \lim_{n \to \infty} (v_n - \lambda_n x_0) \\
        &= \lim_{n \to \infty} v_n - \lim_{n \to \infty} \lambda_n x_0 \\
        &= v - \lambda x_0,
    \end{align*}
    entonces $(s_n)_{n \in \mathbb{N}}$ converge a $s = v - \lambda x_0$. En particular,
    \begin{align*}
        \varphi(s) &= \varphi(v - \lambda x_0) \\
        &= \varphi(v) - \varphi(\lambda x_0) \\
        &= \lambda - \lambda \\
        &= 0,
    \end{align*}
    entonces $s \in \ker \varphi$, o sea $s \in S$. Por lo tanto, $v = s + \lambda x_0$, donde $s \in S$ y $\lambda \in K$, entonces $v \in S \oplus \langle x_0 \rangle$. Esto prueba que $S \oplus \langle x_0 \rangle$ es cerrado.

    Siguiendo inductivamente, cada subespacio de dimensión finita es cerrado.

    (b) Consideramos al espacio $(c, \| \cdot \|_{\infty})$ de sucesiones convergentes y al subespacio $A = \{(a_n)_{n \in \mathbb{N}} \mid \text{eventualmente constante} \}$ (ver que es un subespacio son cuentas aburridas). Por lo tanto, sea $(x_n)_{n \in \mathbb{N}}$ una sucesión convergente a $x$. Entonces, consideramos la sucesión de sucesiones eventualmente constantes $(a^k)_{k \in \mathbb{N}}$ tal que 
    \begin{align*}
        a_n^k = \begin{cases}
            x_n &\text{si } n \leq k, \\
            x &\text{si } n > k. \\
        \end{cases}
    \end{align*}
    Claramente $a^k \to (x_n)$, por lo que $A$ es un subespacio propio denso en $c$.
\end{proof}

\begin{exercise}
    Sea $X$ un espacio métrico compacto. Demostrar que $X$ es completo y que todo subconjunto infinito $A$ de $X$ verifica que $A' \neq \varnothing$.
\end{exercise}

\begin{proof}
    Probamos que toda sucesión posee una subsucesión convergente. Sea $(x_n)_{n \in \mathbb{N}}$ una sucesión en $X$ tal que no posee una subsucesión convergente. Por lo tanto, para todo $x \in X$, existe un $\varepsilon_x > 0$ tal que $B(x, \varepsilon_x)$ interseca con finitos puntos de la sucesión; de lo contrario, tendriamos un punto de acumulación de la sucesión y, por lo tanto, una subsucesión convergente. Consideramos al cubrimiento abierto $\{ B(x, \varepsilon_x) \}_{x \in X}$. Por compacidad de $X$, podemos tomar un subcubrimiento finito $\{ B(y_i, \varepsilon_i) \}_{1 \leq i \leq N}$, con $N \in \mathbb{N}$. Sin embargo, esto provoca un absurdo. O bien, la sucesión tiene finitos términos distintos, y por el Principio de Palomar, alguno se repite infinitas veces y obtenemos una subsucesión constante; o bien, posee toma infinitos valores distintos y entonces alguna de las bolas del cubrimiento finito contiene infinitos términos de la sucesión, lo cual es absurdo. Por lo tanto, toda sucesión posee una subsucesión convergente.

    Ahora sí, probamos las dos afirmaciones del ejercicio. Sea $(x_n)_{n \in \mathbb{N}}$ una succesión de Cauchy. Como es de Cauchy y tiene una subsucesión convergente, entonces converge. Es decir, $X$ es completo. Sea $A$ un subconjunto infinito. Entonces, tiene un subconjunto numerable. Consideramos la sucesión asociada con el subconjunto numerable y obtenemos una subsucesión convergente. Por lo tanto, el límite de la subsucesión es un punto de acumulación del numerable y, en consecuencia, de $A$. Entonces, $A' \neq \varnothing$.
\end{proof}

\begin{exercise}
    Asumiendo que ya se demostró que $(C([0, 1]), \| \cdot \|_{\infty})$ es un espacio normado, demostrar que es un espacio de Banach.
\end{exercise}

\begin{proof}
    Sea $(f_n)_{n \in \mathbb{N}}$ una sucesión de Cauchy en $(C([0, 1]), \| \cdot \|_{\infty})$. 
    Es decir, para todo $\varepsilon > 0$ existe $N \in \mathbb{N}$ tal que
    \begin{equation*}
        \| f_n - f_m \|_{\infty} < \varepsilon, \quad \forall m,n \ge N.
    \end{equation*}

    Sea $x \in [0,1]$. La sucesión $(f_n(x))_{n \in \mathbb{N}}$ es de Cauchy en $\mathbb{R}$ porque
    \begin{equation*}
        |f_n(x) - f_m(x)| \leq \| f_n - f_m \|_{\infty} \xrightarrow[m,n\to\infty]{} 0.
    \end{equation*}
    Como $\mathbb{R}$ es completo, definimos
    \begin{equation*}
        f(x) = \lim_{n \to \infty} f_n(x), \quad x \in [0,1].
    \end{equation*}

    Fijamos $\varepsilon>0$. Por ser $(f_n)$ de Cauchy, existe $N\in\mathbb{N}$ tal que
    \begin{equation*}
        \| f_n - f_m \|_{\infty} < \varepsilon, \quad \forall n,m \ge N.
    \end{equation*}
    Sea $m \ge N$ fijo. Para todo $x \in [0,1]$ tenemos
    \begin{align*}
        |f_m(x) - f(x)| 
        &= \lim_{n \to \infty} | f_m(x) - f_n(x) | \\
        &\le \limsup_{n \to \infty} \| f_m - f_n \|_{\infty} \\
        &\le \varepsilon.
    \end{align*}
    Por lo tanto,
    \begin{equation*}
        \| f_m - f \|_{\infty} \le \varepsilon, \quad \forall m \ge N,
    \end{equation*}
    y concluimos que $f_n \to f$ uniformemente.

    Como cada $f_n$ es continua y la convergencia es uniforme, $f$ también es continua.  
    En particular, $f \in C([0,1])$.

    La convergencia uniforme implica que
    \begin{equation*}
        \| f_n - f \|_{\infty} \xrightarrow[n \to \infty]{} 0.
    \end{equation*}

    Hemos probado que toda sucesión de Cauchy converge en $(C([0,1]), \| \cdot \|_\infty)$.
    Por lo tanto, es un espacio de Banach.
\end{proof}

\begin{exercise}
    Sea $X$ un espacio métrico totalmente acotado y sea $f : X \to X$ una isometría. Demostrar que $f(X)$ es denso en $X$.
\end{exercise}

\begin{proof}
    Como $X$ es totalmente acotado, para todo $\varepsilon > 0$, existe un cubrimiento finito por bolas abiertas de radio menor que $\varepsilon$. Supongamos que existe $x_0 \in X$ y $\varepsilon_0 > 0$ tal que $B(x_0, \varepsilon_0) \cap f(X) = \varnothing$. Consideramos la sucesión $(f^n(x_0))_{n \in \mathbb{N}}$. Por hipótesis, 
    \begin{equation*}
        d(x_0, f^n(x_0)) \geq \varepsilon_0, \quad \text{para todo } n \in \mathbb{N}.
    \end{equation*}
    De lo contrario, tendríamos un absurdo. Por lo tanto,
    \begin{equation*}
        d(f^m(x_0), f^n(x_0)) \geq \varepsilon_0, \quad \text{para todo } m, n \in \mathbb{N}.
    \end{equation*}
    Sin embargo, esto produce un absurdo ya que $X$ es totalmente acotado.
\end{proof}