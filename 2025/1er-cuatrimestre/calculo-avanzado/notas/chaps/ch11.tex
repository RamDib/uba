\chapter{Espacios de funciones}

En este capítulo, cuando hablamos de funciones nos referimos únicamente a funciones acotadas. Recordemos cómo se definía este espacio métrico.

\begin{definition}
	Sea $Y$ un conjunto y $(Z, d^Z)$ un espacio métrico. Definimos el espacio de funciones de $Y$ a $Z$ acotadas como
	\begin{equation*}
		B(Y, Z) = \left\{ f : Y \to Z \mid f \text{ es acotada} \right\}.
	\end{equation*}
\end{definition}

De ahora en adelante, consideramos al espacio de funciones $(B(Y, Z), d_{\infty})$.

\section{Función evaluación y Ley Exponencial}

\begin{definition}
	Sea $X = B(Y, Z)$. Definimos las funciones
	\begin{enumerate}
		\item La evaluación: $\ev : X \times Y \to Z$ tal que $\ev(f, y) = f(y)$.
		\item La evaluación en un punto $y$: $\ev_y : X \to Z$ tal que $\ev_y(f) = f(y)$.
	\end{enumerate}
\end{definition}

\begin{remark}
	La función $\ev_y$ es continua.
\end{remark}

Ahora sea $Y$ también un espacio métrico. Entonces, podemos considerar alguna de las métricas
\begin{equation*}
	d_{X \times Y} ((f, y), (f', y')) =
	\begin{cases}
		d_X (f, f') + d_Y (y, y'), \text{ o,}
		\max \left\{d_X (f, f'), d_Y (y, y')  \right\}.
	\end{cases}
\end{equation*}
Cualquiera de estas dos verifica que una sucesión es convergente en $X \times Y$ si y sólo si es convergente coordenada a coordenada en $X$ e $Y$, respectivamente.

\begin{proposition}
	Sea $X$ el espacio de funciones de $Y$ a $Z$ continuas y acotadas. Entonces, $\ev : X \times Y \to Z$ es continua.
\end{proposition}

\begin{proof}
	Veamos la continuidad mediante sucesiones. Sea $(f_n, y_n)_{n \in \mathbb{N}}$ una sucesión convergente en $X \times Y$. Recordemos que $(f_n, y_n)_{n \in \mathbb{N}}$ converge si y sólo si $(f_n)_{n \in \mathbb{N}}$ e $(y_n)_{n \in \mathbb{N}}$ convergen. Sean respectivamente $f$ e $y$ los límites.

	Veamos que $\ev(f_n, y_n) \longrightarrow \ev(f, y)$. Calculamos
	\begin{align*}
		d_Z(f_n(y_n), f(y)) & \leq d_Z(f_n(y_n), f(y_n)) + d_Z(f(y_n), f(y))            \\
		                    & \leq \sup_{t \in Y} d_Z(f_n(t), f(t)) + d_Z(f(y_n), f(y)) \\
		                    & = d_X(f_n, f) + d_Z(f(y_n), f(y))
	\end{align*}

	Dado que $f_n \to f$ en $X$, el primer término $d_X(f_n, f) \longrightarrow 0$, y como $f$ es continua e $y_n \longrightarrow y$, $d_Z(f(y_n), f(y)) \longrightarrow 0$.
\end{proof}

Veamos la ley exponencial.

\begin{proposition}
	Sean $X$ e $Y$ espacios métricos y sea $K$ compacto. Entonces, existe una biyección natural entre $C(X \times K, Y)$ y $C(X, C(K, Y))$. En particular es
	\begin{align*}
		f : X \times K \to Y   & \mapsto \tilde{f} : X \to C(K, Y) \\
		\text{tal que }f(x, k) & = \tilde{f}(x)(k).
	\end{align*}
\end{proposition}

\begin{proof}
	\color{red} COMPLETAR
\end{proof}

\section{Equicontinuidad}

\begin{definition}
	Sean $X$ e $Y$ espacios métricos y sea $\mathcal{F} \subseteq C(X, Y)$ una familia de funciones continuas de $X$ a $Y$. Decimos que $\mathcal{F}$ es \emph{equicontinua} en $x_0 \in X$ si
	\begin{center}
		\begin{minipage}{0.9\linewidth}
			para todo $\varepsilon > 0$, existe $\delta > 0$ tal que $f(B_X(x_0, \varepsilon)) \subseteq B_Y(f(x_0, \varepsilon))$, para todo $f \in \mathcal{F}$.
		\end{minipage}
	\end{center}
	Si $\mathcal{F}$ es equicontinua en todo $X$, decimos que $\mathcal{F}$ es \emph{equicontinua}.
\end{definition}

\begin{remark}
	Como con la continuidad usual, podemos usar la distancia en vez de bolas.
\end{remark}

\begin{definition}
	Sean $X$ e $Y$ espacios métricos y sea $\mathcal{F} \subseteq C(X, Y)$. Decimos que $\mathcal{F}$ es \emph{uniformemente equicontinua} si
	\begin{center}
		\begin{minipage}{0.9\linewidth}
			para todo $\varepsilon > 0$, existe $\delta > 0$ tal que $d_X(x, x') < \delta$ implica $d_Y(f(x), f(x')) < \varepsilon$ para todo $x, x' \in X$ y $f \in \mathcal{F}$.
		\end{minipage}
	\end{center}
\end{definition}

\begin{remark}
	Si para todo $f \in \mathcal{F}$, $f$ es Lipschitz con la misma constante, entonces $\mathcal{F}$ es equicontinua.
\end{remark}

\begin{proposition}
	Sea $\left\{ f_n \right\}_{n \in \mathbb{N}}$ equicontinua que converge puntualmente a $f : X \to Y$. Entonces, $\overline{\left\{ f_n \right\}}_{n \in \mathbb{N}}$ es equicontinua.
\end{proposition}

\begin{proof}
	Sean $x_0 \in X$ y $\varepsilon > 0$. Por equicontinuidad de $\overline{\left\{ f_n \right\}}_{n \in \mathbb{N}}$, existe $\delta > 0$ tal que
	\begin{equation*}
		d(x_0, x) < \delta \implies d(f_n(x_0), f_n(x)) < \varepsilon, \quad \text{para todo }n \in \mathbb{N}.
	\end{equation*}
	Tomamos límite en $n \to \infty$,
	\begin{equation*}
		d(f(x_0), f(x)) \leq \varepsilon.
	\end{equation*}
	Por lo tanto, $\overline{\left\{ f_n \right\}}_{n \in \mathbb{N}}$ es equicontinua.
\end{proof}

\begin{remark}
	La familia $\mathcal{F}$ es equicontinua en $x_0$ si y sólo si
	\begin{equation*}
		\sup_{f \in \mathcal{F}} d(f(x), f(x_0)) \xrightarrow{x \to x_0} 0.
	\end{equation*}
\end{remark}

\begin{lemma}
	Sea $X$ un espacio métrico e $Y$ un espacio métrico completo y sea $A \subseteq X$ denso. Sea $\left\{ f_n \right\}_{n \in \mathbb{N}} \subseteq C(X, Y)$ equicontinua. Si $(f_n)_{n \in \mathbb{N}}$ converge puntualmente en $A$, entonces converge puntualmente en $X$.
\end{lemma}

\begin{proof}
	Sea $x \in X$. Veamos que $(f_n(x))_{n \in \mathbb{N}}$ es de Cauchy. Calculamos
	\begin{equation*}
		d(f_n(x), f_m(x)) \leq d(f_n(x), f_n(a)) + d(f_n(a), f_m(a)) + d(f_m(a), f_m(x)).
	\end{equation*}
	Por equicontinuidad de $(f_n(x))_{n \in \mathbb{N}}$, existe $\delta$ tal que
	\begin{equation*}
		d(x, a) < \delta \implies d(f_n(x), f_n(a)), d(f_m(a), f_m(x)) < \frac{\varepsilon}{3}.
	\end{equation*}
	También, como $(f_n(a))_{n \in \mathbb{N}}$ es de Cauchy, existe $N \in \mathbb{N}$ tal que
	\begin{equation*}
		d(f_n(a), f_m(a)) < \frac{\varepsilon}{3}, \quad \forall m, n \geq N.
	\end{equation*}
	Entonces,
	\begin{equation*}
		d(f_n(x), f_m(x)) \leq \frac{\varepsilon}{3} + \frac{\varepsilon}{3} + \frac{\varepsilon}{3},
	\end{equation*}
	probando que existe el límite puntual.
\end{proof}

\begin{lemma}
	Sea $K$ compacto e $Y$ un espacio métrico y sea $\left\{ f_n \right\}_{n \in \mathbb{N}}$ equicontinua. Si $\left( f_n \right)_{n \in \mathbb{N}}$ converge puntualmente, entonces converge uniformemente.
\end{lemma}

\begin{proof}
	Supongamos que $f$ no converge uniformemente. Entonces, existe $\varepsilon > 0$ y $\left\{ f_{n_k} \right\}_{k \in \mathbb{N}}$ y $(x_k)_{k \in \mathbb{N}}$ tal que
	\begin{equation*}
		d(f_{n_k}(x_k), f(x_k)) \geq \varepsilon.
	\end{equation*}
	Tomando una subsucesión convergente, podemos suponer que $(x_k)_{k \in \mathbb{N}}$ converge. Entonces,
	\begin{align*}
		\varepsilon \leq d(f_{n_k}(x_k), f_{n_k}(x_0)) + d(f_{n_k}(x_0), f(x_0)) + d(f(x_0), f(x_k)).
	\end{align*}
	Y siguiendo pasos similares a la demostración anterior, llegamos a un absurdo.
\end{proof}

Un resultado no tan sorprendente.

\begin{proposition}
	Sea $K$ compacto e $Y$ un espacio métrico y sea $\mathcal{F} \subseteq C(K, Y)$ equicontinua. Entonces, $\mathcal{F}$ es uniformemente equicontinua.
\end{proposition}

\begin{proof}
	Sea $\varepsilon > 0$. Como $\mathcal{F}$ es equicontinua, para cada $x \in K$, existe $\delta_x > 0$ tal que
	\begin{equation*}
		d_K(x, x') < \delta_x \implies d_Y(f(x), f(x')) < \varepsilon, \quad \forall x' \in K, f \in \mathcal{F}.
	\end{equation*}
	Dado que $K$ es compacto y $\{B(x, \delta_x)\}_{x \in K}$ forma un cubrimiento abierto, tomamos un subcubrimiento finito $\{B(x_n, \delta_{x_n})\}_{1 \leq n \leq N}$ y consideramos $\delta = \min_{1 \leq n \leq N} \delta_{x_n}$ que sirve uniformemente para todo $x \in K$. Entonces, para todo $x \in K$,
	\begin{equation*}
		d_K(x, x') < \delta_x \implies d_Y(f(x), f(x')) < \varepsilon, \quad \forall x' \in K, f \in \mathcal{F}.
	\end{equation*}
	Por lo tanto, $\mathcal{F}$ es uniformemente continua.
\end{proof}

\begin{definition}
	Sea $\mathcal{F}$ una familia de funciones de $X$ a $Y$. Decimos que $F$ es \emph{equiacotada} si
	\begin{center}
		para todo $x \in X$, $\mathcal{F}(x) = \left\{ f(x) \mid f \in \mathcal{F} \right\}$ es totalmente acotado.
	\end{center}
	Además, decimos que $\mathcal{F}$ es \emph{totalmente equiacotada} si $\mathcal{F}(X) = \left\{ f(x) \mid x \in X, f \in \mathcal{F} \right\}$ es totalmente acotado.
\end{definition}

\begin{remark}
	Un espacio es totalmente acotado si y sólo si su clausura es compacta.
\end{remark}

Ahora vemos el teorema de Arzela--Ascoli.

\begin{theorem}
	Sea $K$ compacto e $Y$ completo. Sea $\mathcal{F} \subseteq C(K, Y)$. Son equivalentes:
	\begin{enumerate}
		\item $\mathcal{F}$ es equicontinua y equiacotada.
		\item $\mathcal{F}$ es totalmente acotado.
		\item $\mathcal{F}$ es uniformemente equicontinua y uniformemente equiacotado.
	\end{enumerate}
\end{theorem}

\begin{proof}
	(1 $\Rightarrow$ 2) Sea $\mathcal{F}$ equicontinua y equiacotada. Basta con ver que $\overline{\mathcal{F}}$ es compacto.

	Sea $(f_n)_{n \in \mathbb{K}}$ una sucesión en $\mathcal{F}$. Como $K$ es compacto, en particular, es separable. Por lo tanto, consideramos $A \subseteq K$ denso numerable. Queremos ver que existe una subsucesión $(f_{n_k})_{k \in \mathbb{N}}$ tal que converge puntualmente en $A$. Tomamos las sucesiones $(f_n(a_i))_{n \in \mathbb{N}}$ con $i \in \mathbb{N}$ fijo. Dado que $\mathcal{F}$ es equiacotada, $\left\{ f_n(a_i) \right\}_{n \in \mathbb{N}}$ es totalmente acotado y entonces $\overline{\left\{ f_n(a_i) \right\}}_{n \in \mathbb{N}}$ es compacto. Por lo tanto, podemos tomar un subconjunto $\mathcal{M} \subseteq \mathbb{N}$ tal que $(f_m(a_i))_{m \in \mathcal{M}}$ converge para todo $i \in \mathbb{N}$. Entonces, la subsucesión $(f_m)_{m \in \mathcal{M}}$ converge puntualmente en $A$.

	Dado que $(f_m)_{m \in \mathcal{M}}$ converge puntualmente en $A$ denso, entonces converge uniformemente.

	(2 $\Rightarrow$ 3) Sea $\mathcal{F}$ totalmente acotado. Entonces, $\overline{\mathcal{F}}$ es compacto. Por lo tanto, consideramos
	\begin{equation*}
		\mathcal{F}(K) = \ev (K \times \mathcal{F}) \subseteq \ev (K \times \overline{\mathcal{F}}).
	\end{equation*}
	Dado que $\ev$ es continua y $K \times \overline{\mathcal{F}}$ es compacto, $\mathcal{F}$ es totalmente acotado.

	Por la ley exponencial,
	\begin{equation*}
		C(K \times \overline{\mathcal{F}}, Y) \cong C(K, C(\overline{\mathcal{F}}, Y)).
	\end{equation*}
	Como $\ev : K \to C(\overline{F}, Y)$ es continua sobre un compacto, entonces es uniformemente continua. Por lo tanto, $\mathcal{F}$ es uniformemente equicontinua.

	(3 $\Rightarrow$ 1) Inmediato.
\end{proof}

\section{Sucesiones y series de funciones}

Recordemos las definiciones de convergencia puntual y uniforme.

\begin{definition}
	Sea $(f_n)_{n \in \mathbb{R}}$ una sucesión de funciones de $\mathbb{R}$ en $\mathbb{R}$ y $f : \mathbb{R} \to \mathbb{R}$. Decimos que converge \emph{puntualmente} a $f$ si para todo $x \in \mathbb{R}$ y $\varepsilon > 0$, existe $N \in \mathbb{N}$ tal que
	\begin{equation*}
		\left\lvert f(x) - f_n(x) \right\rvert < \varepsilon, \quad \text{para todo }n \geq N.
	\end{equation*}
\end{definition}

Y convergencia uniforme.

\begin{definition}
	Sea $(f_n)_{n \in \mathbb{R}}$ una sucesión de funciones de $\mathbb{R}$ en $\mathbb{R}$ y $f : \mathbb{R} \to \mathbb{R}$. Decimos que converge \emph{uniformemente} a $f$ si para todo $\varepsilon > 0$, existe $N \in \mathbb{N}$ tal que
	\begin{equation*}
		\left\lvert f(x) - f_n(x) \right\rvert < \varepsilon, \quad \text{para todo }n \geq N, x \in \mathbb{R}.
	\end{equation*}
\end{definition}

\begin{remark}
	La convergencia uniforme implica la convergencia puntual.
\end{remark}

Veamos algunas propiedades útiles.

\begin{proposition}
	Sea $(f_n)_{n \in \mathbb{N}}$ una sucesión de funciones reales.
	\begin{enumerate}
		\item \textbf{Continuidad del Límite Uniforme:} Si $(f_n)_{n \in \mathbb{N}}$ es una sucesión de funciones continuas definida en un conjunto $E$ y $(f_n)_{n \in \mathbb{N}}$ converge uniformemente a $f$ en $E$, entonces la función límite $f$ es continua en $E$.
		\item \textbf{Intercambio de Derivación y Límite:} Sea $(f_n)_{n \in \mathbb{N}}$ una sucesión de funciones diferenciables en un intervalo $[a,b]$. Si la sucesión de las derivadas $(f_n')_{n \in \mathbb{N}}$ converge uniformemente a una función $g$ en $[a,b]$ y existe un punto $x_0 \in [a,b]$ tal que la sucesión numérica $(f_n(x_0))_{n \in \mathbb{N}}$ converge (puntualmente), entonces la sucesión $(f_n)_{n \in \mathbb{N}}$ converge uniformemente a una función $f$ en $[a,b]$, $f$ es diferenciable en $[a,b]$, y su derivada es $f'(x) = g(x)$. Es decir, se puede intercambiar el límite con la derivación:
		      $$\left( \lim_{n \to \infty} f_n(x) \right)' = \lim_{n \to \infty} f_n'(x)$$
		\item \textbf{Intercambio de Integración y Límite:} Si $(f_n)_{n \in \mathbb{N}}$ es una sucesión de funciones integrables en un intervalo $[a,b]$ y $(f_n)_{n \in \mathbb{N}}$ converge uniformemente a $f$ en $[a,b]$, entonces la función límite $f$ es integrable en $[a,b]$ y se puede intercambiar el límite con la integración:
		      $$\lim_{n \to \infty} \int_a^b f_n(x) \, dx = \int_a^b \lim_{n \to \infty} f_n(x) \, dx = \int_a^b f(x) \, dx$$
	\end{enumerate}
\end{proposition}

\begin{proof}
	No lo demuestro.
\end{proof}

\section{Series de funciones}

\begin{definition}
	Sea $(f_n)_{n \in \mathbb{N}}$ una sucesión de funciones. Entonces, la serie $\sum_{n = 1}^{\infty} f_n(x)$ converge \emph{puntualmente} si la sucesión $(S_{N}(x))_{N \in \mathbb{N}}$ lo hace. Análogo para convergencia uniforme. Y converge absolutamente si la serie de $(\left\lvert f_n \right\rvert)_{n \in \mathbb{N}})$ lo hace.
\end{definition}

\begin{proposition}
	Sea $(f_n)_{n \in \mathbb{N}}$ una sucesión de funciones continuas tal que $\sum f_n(x)$ converge uniformemente. Entonces, $\sum f_n(x)$ es continua.
\end{proposition}

\begin{proof}
	Como $f_n$ es continua para todo $n \in \mathbb{N}$, entonces $S_n$ también lo es ya que es suma de continuas.
\end{proof}

Veamos criterios de convergencia puntual.

\begin{proposition}
	Sea $(f_n)_{n \in \mathbb{N}}$ una sucesión de funciones continuas. Consideramos $a_n = f_n(x)$.
	\begin{enumerate}
		\item \textbf{Criterios de D'Alembert y de la raíz:} \\Sea $L = \lim_{n \to \infty} \left\lvert \frac{a_{n+1}}{a_n} \right\rvert$ o $L = \limsup \sqrt[n]{\left\lvert a_n \right\rvert}$.
		      \begin{itemize}
			      \item Si $L < 1$, entonces la serie converge absolutamente.
			      \item Si $L > 1$, entonces la serie diverge.
			      \item Si $L = 1$, no sabemos nada.
		      \end{itemize}

		\item \textbf{Criterio de Liebniz:} Si $\lim_{n \to \infty} a_n = 0$ y es monotónica decreciente, entonces la serie alternada converge.
	\end{enumerate}
\end{proposition}


Y uno para la convergencia uniforme.

\begin{proposition}
	Sea $(f_n)_{n \in \mathbb{N}}$ una sucesión de funciones tal que existe $(M_n)_{n \in \mathbb{N}}$ una sucesión de reales tales que $\sum M_n$ converge y, para todo $x$, $\left\lvert f_n(x) \right\rvert \leq M_n$. Entonces, $\sum f_n$ converge uniformemente.
\end{proposition}

Por último, para series de potencias.

\begin{proposition}
	Sea $\sum a_n x^n$ una serie de potencias. Entonces, el radio de convergencia $R = \frac{1}{\limsup \sqrt[n]{\left\lvert a_n \right\rvert}}$. Es decir, en el intervalo $[-R, R]$, la serie converge uniformemente.
\end{proposition}

\section{Teorema de Stone--Weierstrass}

Para enunciar y probar el Teorema de Stone--Weierstrass, haremos lo siguiente:
\begin{itemize}
	\item Álgebras de funciones.
	\item Lemas preliminares.
	\item Enunciado y demostración del teorema versión real.
	\item Enunciado y demostración del teorema versión compleja.
\end{itemize}

Empezamos definiendo un álgebra de funciones.

\begin{definition}
	Sea $X$ un espacio métrico y sea $A \subseteq C(X, \mathbb{R})$. Decimos que $A$ es un \emph{álgebra de funciones} (o simplemente \emph{álgebra}) si $A$ es un subespacio vectorial de $C(X, \mathbb{R})$ y
	\begin{center}
		si $f, g \in A$, entonces $f \cdot g \in A$.
	\end{center}

	Además, decimos que un álbebra $A$ \emph{contiene a las constantes} si $1 \in A$. Y decimos que un conjunto $A$ \emph{separa puntos} si para todo $x, y \in X$ distintos, existe $f \in A$ tal que $f(x) \neq f(y)$.
\end{definition}

\begin{remark}
	Si $A$ es un álgebra que contiene a las constantes, entonces para todo $f \in A$ y $p \in \mathbb{R}[x]$, $p(f) \in A$.
\end{remark}

Y el lema.

\begin{lemma}
	Si $A \subseteq C(X, \mathbb{R})$ es un álgebra, entonces su clausura $\overline{A}$ también es un álgebra.
\end{lemma}

\begin{proof}
	Queremos ver que $\overline{A}$ es un subespacio vectorial cerrado bajo multiplicación puntual. Como $A$ es un álgebra, entonces $0 \in A$ y por lo tanto $0 \in \overline{A}$.

	Sean $f, g \in \overline{A}$. Entonces, existen sucesiones $(f_n)_{n \in \mathbb{N}}$ y $(g_n)_{n \in \mathbb{N}}$ en $A$ tales que $f_n \rightrightarrows f$ y $g_n \rightrightarrows g$. Como $A$ es un álgebra, se cumple que $f_n + g_n \in A$ y $f_n \cdot g_n \in A$ para todo $n \in \mathbb{N}$.

	Veamos primero que $f + g \in \overline{A}$. Se tiene que
	\[
		\|f_n + g_n - (f + g)\|_\infty \leq \|f_n - f\|_\infty + \|g_n - g\|_\infty \to 0.
	\]
	Por lo tanto, $f_n + g_n \rightrightarrows f + g$, y entonces $f + g \in \overline{A}$.

	Veamos ahora que $f \cdot g \in \overline{A}$. Como las sucesiones $(f_n)$ y $(g_n)$ convergen uniformemente, en particular son uniformemente acotadas. Es decir, existe $M > 0$ tal que
	\[
		\|f_n\|_\infty \leq M, \quad \|g_n\|_\infty \leq M, \quad \text{para todo } n \in \mathbb{N}.
	\]
	Calculamos, para todo $x \in X$,
	\begin{align*}
		|f_n(x)g_n(x) - f(x)g(x)| 
		&\leq |f_n(x)g_n(x) - f(x)g_n(x)| + |f(x)g_n(x) - f(x)g(x)| \\
		&= |f_n(x) - f(x)| \cdot |g_n(x)| + |f(x)| \cdot |g_n(x) - g(x)| \\
		&\leq \|f_n - f\|_\infty \cdot M + \|f\|_\infty \cdot \|g_n - g\|_\infty.
	\end{align*}
	Como $f_n \to f$ uniformemente, entonces $\|f\|_\infty \leq M + 1$ para $n$ suficientemente grande. Luego,
	\[
		\|f_n g_n - f g\|_\infty < \leq \|f_n - f\|_\infty \cdot M + (M + 1) \cdot \|g_n - g\|_\infty \to 0.
	\]
	Por lo tanto, $f_n g_n \rightrightarrows f g$ y como $f_n g_n \in A$, obtenemos que $f \cdot g \in \overline{A}$.

	Con esto, $\overline{A}$ es cerrado bajo suma, producto por escalar (por linealidad), y producto puntual. Por lo tanto, $\overline{A}$ es un álgebra.
\end{proof}

Ahora sí enunciamos el teorema de Stone--Weierstrass.

\begin{theorem}[Stone--Weierstrass]
	Sea $X$ un espacio métrico compacto y sea $A \subseteq C(X, \mathbb{R})$ un álgebra que contiene a las constantes y separa puntos. Entonces, $\overline{A} = C(X, \mathbb{R})$.
\end{theorem}

Antes de probarlo, veamos cómo vamos a estructurar la demostración. Probamos que:
\begin{enumerate}
	\item Si $f \in \overline{A}$, entonces $|f| \in \overline{A}$.
	\item Si $f, g \in \overline{A}$, entonces las funciones mínimo y máximo también están en $\overline{A}$.
	\item Existe una función que interpola a dos puntos distintos de $X$.
	\item Existe una función $g_x \in \overline{A}$ tal que $g_x(x) = f(x)$ y $g_x \leq f + \varepsilon$.
	\item Existe $g \in \overline{A}$ tal que $g(x) = f(x)$ para todo $x \in X$ y $\| f - g \| \leq \varepsilon$.
\end{enumerate}

\begin{proof}
	(1.) Sea $f \in \overline{A}$. Sea $M > 0$ tal que $\|f\| < M$. Entonces, como $\overline{A}$ es un álgebra, entonces $\frac{f^2}{M^2} \in \overline{A}$. Por lo tanto, $|f| = \sqrt{\frac{f^2}{M^2}} \in \overline{A}$. Por el lema anteriormente visto, existe una sucesión tal que converge uniformemente a $\sqrt{\frac{f^2}{M^2}} = \frac{\|f\|}{M}$.
\end{proof}