\chapter{Nociones básicas}

Formalizamos algunas conceptos de probabilidad que vienen de la intuición. 


\section{Modelo probabilístico}

Consideremos un experimento con distintos posibles resultados. 

\begin{definition}
    El \emph{espacio muestral} de un experimento es el conjunto de posibles resultados del experimento. 
\end{definition}

Usualmente denotamos un espacio muestral con $\Omega$.

\begin{remark}
    Todo resultado corresponde con un único elemento $\omega \in \Omega$.
\end{remark}

Veamos algunos ejemplos.

\begin{example}
    \label{ex:dados}
    Consideremos el siguiente experimento:
    \begin{centeredvarwidth}
        \begin{enumerate}
            \item Se tira un dado balanceado de $6$ caras.
            \item Se graba el resultado.
        \end{enumerate}
    \end{centeredvarwidth}

    En este caso, el espacio muestral es el conjunto
    \begin{equation*}
        \Omega = \{ 1, 2, 3, 4, 5, 6 \}.
    \end{equation*}
\end{example}

Cabe aclarar que no importa de qué manera escribimos los resultados siempre y cuando la correspondencia con el resultado sea clara. Por ejemplo, podríamos haber definido el espacio muestral como
\begin{equation*}
    \Omega = \{ \vcdice{1}, \vcdice{2}, \vcdice{3}, \vcdice{4}, \vcdice{5}, \vcdice{6}\}.
\end{equation*}

\begin{example}
    Consideremos el siguiente experimento:
    \begin{centeredvarwidth}
        \begin{enumerate}
            \item Se tira una moneda $3$ veces.
            \item Se graba el resultado.
        \end{enumerate}
    \end{centeredvarwidth}

    El espacio muestral es
    \begin{equation*}
        \Omega = \{ CCC, CCS, \ldots, SSC, SSS \}.
    \end{equation*}
\end{example}

Nótese que $\Omega$ se puede escribir como $\{ C, S \}^3$.

\begin{example}
    Consideremos el siguiente experimento:
    \begin{centeredvarwidth}
        \begin{enumerate}
            \item Se elije un habitante de Buenos Aires al azar.
            \item Se mide su altura en metros.
        \end{enumerate}
    \end{centeredvarwidth}

    El espacio muestral podría ser
    \begin{equation*}
        \Omega = \R.
    \end{equation*}
\end{example}

Uno podría argumentar que el espacio muestral debería ser
\begin{equation*}
    \Omega = [0, 3],
\end{equation*}
ya que es imposible que alguien mida $\SI{-1}{m}$ o $\SI{100}{m}$. Sin embargo, lo único que nos interesa es que, al medir a alguien, caiga dentro de $\Omega$.


\section{Eventos}

\begin{definition}
    Sea $\Omega$ un espacio muestral. Un \emph{evento} es un subconjunto de $\Omega$.
\end{definition}

Veámoslo en algunos ejemplos.

\begin{example}
    Consideramos el experimento del ejemplo \ref{ex:dados}. El conjunto
    \begin{equation*}
        A = \{ \text{el resultado es un número par} \} = \{2, 4, 6\}
    \end{equation*}
    es un evento dado que $A \subseteq \Omega$.
\end{example}

Por ahora, usemos la noción intuitiva de probabilidades. 

La probabilidad se le asigna a un evento, no a un resultado. Por ejemplo, cuando decimos
\begin{equation*}
    P(\vcdice{3}) = \frac{1}{6}
\end{equation*}
en realidad queremos decir
\begin{equation*}
    P(\{ \vcdice{3} \}) = \frac{1}{6}.
\end{equation*}
No obstante, por practicidad acudiremos a la primera notación.

Usualmente calculamos la probabilidad de un evento de la siguiente manera:
\begin{equation*}
    P(A) = \frac{\# \text{ casos donde sucede } A}{\# \text{ casos totales}}.
\end{equation*}

Veamos por qué esto no es generalizable.

\begin{example}
    Consideremos el siguiente experimento:
    \begin{centeredvarwidth}
        \begin{enumerate}
            \item Se tiran $2$ dados balanceados de $6$ caras.
            \item Se suman los números de las caras.
            \item Se graba el resultado.
        \end{enumerate}
    \end{centeredvarwidth}

    Un espacio muestral podría ser
    \begin{equation*}
        \Omega = \{ 2, 3, \ldots, 12 \}.
    \end{equation*}
    Sin embargo, $P(2) \neq \frac{1}{10}$. Esto se puede resolver tomando el espacio muestral
    \begin{equation*}
        \Omega = \{ \vcdice{1} \vcdice{1}, \vcdice{1} \vcdice{2}, \ldots, \vcdice{6} \vcdice{5}, \vcdice{6} \vcdice{6} \}.
    \end{equation*}
    Por lo tanto, para todo resultado $\omega \in \Omega$, 
    \begin{equation*}
        P(\omega) = \frac{\# \text{ casos donde sucede } A}{\# \text{ casos totales}} = \frac{1}{36}.
    \end{equation*}
\end{example}

A partir de este ejemplo surge una definición.

\begin{definition}
    Sea $\Omega$ un espacio muestral. Si $\Omega$ es finito y todos sus elementos tienen la misma probabilidad, decimos que $\Omega$ es un \emph{espacio muestral equiprobable}.
\end{definition}

Sin embargo, hasta ahora únicamente tratamos con espacios muestrales finitos. Veamos qué pasa con los infinitos.

\begin{example}
    Consideremos el siguiente experimento:
    \begin{centeredvarwidth}
        \begin{enumerate}
            \item Se elije un punto del disco unitario $D = \{ (x, y) \in \R^2 \mid x^2 + y^2 = 1\}$ al azar.
            \item Se graba el resultado.
        \end{enumerate}
    \end{centeredvarwidth}

    Sea $A = \{ (x, y) \in \R^2 \mid x^2 + y^2 \leq \frac{1}{2} \}$.

    \begin{figure}[H]
        \centering
        \begin{tikzpicture}[scale=0.6]
            % Dibuja y rellena el círculo grande (espacio muestral Omega)
            \fill[black!10] (0,0) circle (3cm);
            \draw (0,0) circle (3cm);
            \node[above right] at (3,0) {$\Omega$};

            % Dibuja y rellena el círculo más pequeño (evento A)
            % Paso 1: Rellenar el círculo con un color sólido para el fondo
            \fill[primarycolor!20] (0,0) circle (2.1213cm);
            % Paso 2: Rellenar el mismo círculo con un patrón de líneas diagonales
            \fill[pattern=north east lines, pattern color=primarycolor!50] (0,0) circle (2.1213cm);
            % Borde del círculo A
            \draw[primarycolor, thick] (0,0) circle (2.1213cm);
            \node[below right] at (2,0) {$A$};
        \end{tikzpicture}
    \end{figure}

    Por lo que 
    \begin{equation*}
        P(A) = \frac{\area(A)}{\area(\Omega)}.
    \end{equation*}
\end{example}


\section{Eventos aleatorios}

No podemos definirle una probablidad a todos los eventos de un espacio muestral.

\begin{definition}
    Un evento al cual le podemos definir una probabilidad es llamado un \emph{evento aleatorios}.
\end{definition}

Agregamos algunas reglas adicionales.

\begin{definition}
    Llamamos $\mathcal{F}$ a una familia de eventos a los cuales podemos calcularles su probabilidad si cumple los siguientes axiomas:
    \begin{itemize}
        \item[(F1)] $\Omega \in \mathcal{F}$.
        \item[(F2)] Si $A \in \mathcal{F}$, entonces $A^{\textrm{c}} \in \mathcal{F}$.
        \item[(F3)] Si $\{A_n\}_{n \in \N} \subseteq \mathcal{F}$, entonces $\bigcup_{n \in \N} A_n \in \mathcal{F}$. 
    \end{itemize}
\end{definition}

\begin{remark}
    Si una familia cumple los axiomas (F1), (F2) y (F3), entonces se llama una $\sigma$-álgebra de conjuntos.
\end{remark}

\begin{example}
    En el ejemplo \ref{ex:dados}, el espacio muestral
    \begin{equation*}
        \Omega = \{ 1, 2, 3, 4, 5, 6 \}
    \end{equation*}
    tiene a la familia de eventos $\mathcal{F} = \mathcal{P}(\Omega)$ que se les puede asignar una probabilidad.
\end{example}

Veamos algunas propiedades que podemos deducir.

\begin{proposition}
    \label{prop:familia-eventos}
    Sea $\Omega$ un espacio muestral y $\mathcal{F} \subseteq \mathcal{P}(\Omega)$ una $\sigma$-álgebra. Las siguientes proposiciones son verdaderas:
    \begin{enumerate}
        \item $\varnothing \in \mathcal{F}$.
        \item Si $\{A_n\}_{1 \leq n \leq N} \subseteq \mathcal{F}$, entonces $\bigcup_{1 \leq n \leq N} A_n \in \mathcal{F}$.
        \item Si $\{A_n\}_{n \in \N} \subseteq \mathcal{F}$, entonces $\bigcap_{n \in \N} A_n \in \mathcal{F}$. (También la versión finita.)
        \item Si $A, B \in \mathcal{F}$, entonces $A \setminus B \in \mathcal{F}$.
    \end{enumerate}
\end{proposition}

\begin{proof}
    (\textit{1.}) Dado que $\Omega \in \mathcal{F}$ por (F1), obtenemos que $\varnothing = \Omega^{\textrm{c}} \in \mathcal{F}$ por (F2).

    (\textit{2.}) Sea $\{A_n\}_{1 \leq n \leq N} \subseteq \mathcal{F}$. Consideremos la familia $\{B_n\}_{n \in \N}$ tal que 
    \begin{equation*}
        B_n = \begin{cases}
            A_n & \text{si } 1 \leq n \leq N, \\
            \varnothing & \text{si } n \geq N. \\
        \end{cases}
    \end{equation*}
    Entonces, por (F3),
    \begin{equation*}
        \bigcup_{1 \leq n \leq N} A_n = \bigcup_{n \in \N} B_n \in \mathcal{F}.
    \end{equation*}

    (\textit{3.}) Sea $\{A_n\}_{n \in \N} \subseteq \mathcal{F}$. Por (F3),
    \begin{equation*}
        \bigcap_{n \in \N} A_n = \left(\bigcup_{n \in \N} A_n^{\mathrm{c}}\right)^{\textrm{c}} \in \mathcal{F}.
    \end{equation*}

    (\textit{4.}) Sean $A, B \in \mathcal{F}$. Dado que $B^{\mathrm{c}}$,
    \begin{equation*}
        A \setminus B = A \cap B^{\mathrm{c}} \in \mathcal{F}.
    \end{equation*}
\end{proof}

\begin{example}
    Consideremos el siguiente experimento:
    \begin{centeredvarwidth}
        \begin{enumerate}
            \item Se elije un número real del intervalo $[0, 1]$ al azar.
            \item Se graba el resultado.
        \end{enumerate}
    \end{centeredvarwidth}

    Sea $\Omega = [0, 1]$ el espacio muestral y sea $\mathcal{F}$ la familia de eventos a los cuales les podemos asignar una probabilidad. Para un intervalo $[a, b]$ la probabilidad se puede calcular como
    \begin{equation*}
        P([a, b]) = b - a.
    \end{equation*}

    Aplicando las propiedades de la proposición \ref{prop:familia-eventos}, podemos deducir que en $\mathcal{F}$ están los eventos:
    \begin{itemize}
        \item Los intervalos abiertos y cerrados.
        \item Uniones e intersecciones numerables de cerrados y/o abiertos.
        \item Los puntos $\{x\}$ con $x \in [0, 1]$.
        \item Los números racionales $\Q$.
    \end{itemize}

    ¿Cuál es la probabilidad de $\Q$? Basta con tomar $\{B_m\} = \{B(q_m, \frac{\varepsilon}{2^{m+1}})\}_{m \in \N}$ y ver que
    \begin{align*}
        P\left(\bigcup_{m \in \N} B_m\right) &\leq \sum_{m \in \N} P(B_m) \\
        &\leq \sum_{m \in \N} \frac{\varepsilon}{2^m} \\
        &\leq \varepsilon.
    \end{align*}
    Tomando $\varepsilon \to 0$, obtenemos que $P(\Q) = 0$.
\end{example}


\section{Definición de probabilidad}

La \textit{idea de Laplace} de probabilidad consta en que la probabilidad de un evento es el límite de la frecuencia con la que sucede cuando la cantidad de ensayos tiende a infinito.

Por otro lado, está la \textit{axiomatización de Kolmogorov}:

\begin{definition}
    Sea $\Omega$ un espacio muestra y $\mathcal{F} \subseteq \mathcal{P}(\Omega)$ una $\sigma$-álgebra. Una función probabilidad es una función $P: \mathcal{F} \to [0, 1]$ que cumple los siguientes axiomas:
    \begin{itemize}
        \item[(P1)] $P(\Omega) = 1$.
        \item[(P2)] $P(A) \geq 0$ para todo $A \in \mathcal{F}$.
        \item[(P3)] Si $\{A_n\}_{n \in \N} \subseteq \mathcal{F}$ una familia de eventos disjuntos, entonces $P(\bigcup_{n \in \N} A_n) = \sum_{n = 1}^{\infty} P(A_n)$.
    \end{itemize}
\end{definition}

Con esto podemos definir un espacio de probabilidad.

\begin{definition}
    Sea $\Omega$ un espacio muestral, $\mathcal{F} \subseteq \mathcal{P}(\Omega)$ una familia de eventos y $P: \mathcal{F} \to [0, 1]$ una función probabilidad. Entonces, la terna $(\Omega, \mathcal{F}, P)$ es un \emph{espacio de probabilidad}.
\end{definition}

%
% TODO: Propiedad de suma = 1 => terna espacio de proba
%

Probamos algunos resultados inmediatos.

\begin{proposition}
    Sea $(\Omega, \mathcal{F}, P)$ un espacio de probabilidad. Entonces, las siguientes proposiciones son verdaderas:
    \begin{enumerate}
        \item $P(\varnothing) = 0$.
        \item Si $A$ y $B$ son eventos disjuntos, entonces $P(A \cup B) = P(A) + P(B)$.
        \item Si $\{A_n\}_{1 \leq n \leq N}$ es una familia de eventos disjuntos, entonces $P(\bigcup_{1 \leq n \leq N} A_n) = \sum_{n = 1}^N P(A_n)$.
        \item $P(A^{\textrm{c}}) = 1 - P(A)$.
        \item Si $A \subseteq B$, entonces $P(A) \leq P(B)$.
        \item $P(\bigcup_{n \in \N} A_n) \leq \sum_{n = 1}^{\infty} P(A_n)$.
    \end{enumerate}
\end{proposition}

\begin{proof}
    (\textit{1.}) Consideremos la familia $\{ \Omega, \varnothing, \ldots \}$. Por (P3), 
    \begin{align*}
        P(\Omega) &= P(\Omega \cup \varnothing \cup \cdots) \\
        &= P(\Omega) + \underbrace{P(\varnothing) + \cdots}_{=0}
    \end{align*}

    (\textit{2.}) La propiedad sale utilizando (P3) y tomando la familia $\{ A, B, \varnothing, \ldots \}$.

    (\textit{3.}) Se prueba por inducción y ussando la proposición anterior.

    (\textit{4.}) Podemos escribir como unión disjunta $\Omega = A \cup A^{\mathrm{c}}$. Y con la propiedad \textit{2.} obtenemos que
    \begin{equation*}
        1 = P(\Omega) = P(A) + P(A^{\mathrm{c}}),
    \end{equation*}
    entonces
    \begin{equation*}
        P(A^{\mathrm{c}}) = 1 - P(A).
    \end{equation*}

    (\textit{5.}) Como $B = (B \setminus A) \cup A$,
    \begin{equation*}
        P(A) \leq P(B \setminus A) + P(A) = P(B).
    \end{equation*}

    (\textit{6.}) Sea $\{A_n\}_{n \in \N}$ una familia de eventos. Consideremos $\{B_n\}_{n \in \N}$ tal que
    \begin{equation*}
        B_n = A_n \setminus \bigcup_{1 \leq k \leq n-1} A_k.
    \end{equation*}
    Como los eventos $B_n$ son disjuntos dos a dos,
    \begin{equation*}
        P\left(\bigcup_{n \in \N} A_n\right) = P\left(\bigcup_{n \in \N} B_n\right) = \sum_{n = 1}^{\infty} P(B_n) \leq \sum_{n = 1}^{\infty} P(A_n).
    \end{equation*}
\end{proof}



