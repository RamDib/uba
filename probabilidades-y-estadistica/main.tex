%%%%%%%%%%%%%%%%%%%%%%%%%%%%%%%%%%%%%%%%%%%%%%%%%%%%%%%%%%%%%%%%%%%%
%
%   Modern Minimalist Math Textbook Template — tcolorbox (breakable)
%
%   Designed by: AI Assistant
%   Engine: pdfLaTeX
%   License: MIT
%
%   Changes:
%   - Replaced mdframed with tcolorbox (breakable)
%   - Kept same look: left vertical rule, soft background
%
%%%%%%%%%%%%%%%%%%%%%%%%%%%%%%%%%%%%%%%%%%%%%%%%%%%%%%%%%%%%%%%%%%%%

\documentclass[11pt, a4paper, oneside]{book}

%-------------------------------------------------------------------
%   CORE PACKAGES
%-------------------------------------------------------------------
\usepackage[utf8]{inputenc}
\usepackage[T1]{fontenc}
\usepackage{lmodern} % A modern version of Computer Modern font

\usepackage{amsmath, amssymb, amsthm} % Essential math packages
\usepackage{graphicx} % For including images
\usepackage{xcolor} % For custom colors
\usepackage{tikz}
\usepackage{float}
\usepackage{varwidth}

% Tikz
\usetikzlibrary{patterns}

% Unidades
\usepackage{siunitx}

% Dice
\usepackage{epsdice}
\newcommand\vcdice[1]{\vcenter{\hbox{\epsdice{#1}}}}

% Varwidth
\newenvironment{centeredvarwidth}[1][\linewidth]{%
  \begin{center}
    \begin{varwidth}{#1}
}{%
    \end{varwidth}
  \end{center}
}

% Emph sequence
\definecolor{primarycolor}{rgb}{0.1, 0.5, 0.3} % Soft green
\renewcommand{\emph}[1]{\textbf{\textit{\textcolor{primarycolor}{#1}}}}

% SPANISH LANGUAGE SUPPORT
\usepackage[spanish, es-noquoting, es-noshorthands]{babel} % For Spanish hyphenation and automatic text

%-------------------------------------------------------------------
%   PAGE LAYOUT & TYPOGRAPHY
%-------------------------------------------------------------------
\usepackage[
    a4paper,
    left=3cm,
    right=3cm,
    top=2.5cm,
    bottom=2.5cm,
    headheight=15pt % To avoid warning with fancyhdr
]{geometry}

\usepackage{microtype} % Improves typography (justification and spacing)

\usepackage{fancyhdr} % For custom headers and footers
\pagestyle{fancy}
\fancyhf{} % Clear all header and footer fields
\fancyhead[L]{\sffamily\bfseries\rightmark} % Section title on the left
\fancyhead[R]{\sffamily\bfseries\thepage}   % Page number on the right
\renewcommand{\headrulewidth}{0.5pt}
\renewcommand{\footrulewidth}{0pt}
\fancypagestyle{plain}{ % Style for chapter starting pages
    \fancyhf{}
    \renewcommand{\headrulewidth}{0 pt}
}

%-------------------------------------------------------------------
%   COLORS
%-------------------------------------------------------------------
\definecolor{boxbgcolor}{rgb}{0.97, 0.97, 0.98}
\definecolor{boxlinecolor}{rgb}{0.1, 0.1, 0.4}

%-------------------------------------------------------------------
%   CHAPTER AND SECTION TITLE DESIGN
%-------------------------------------------------------------------
\usepackage{titlesec}

% --- Chapter Title ---
% Note: The babel package automatically changes \chaptername to "Capítulo"
\titleformat{\chapter}[display]
  {\sffamily\huge\bfseries\color{gray!80}} % Format for the whole thing
  {\filleft\fontsize{80}{80}\selectfont\thechapter} % Chapter number
  {20pt} % Space between number and title
  {\filleft\color{primarycolor}} % Title format
\titlespacing*{\chapter}{0pt}{-40pt}{40pt}

% --- Section and Subsection Titles ---
\titleformat{\section}
  {\sffamily\Large\bfseries\color{primarycolor}}
  {\thesection}
  {1em}
  {}
\titleformat{\subsection}
  {\sffamily\large\bfseries\color{gray!85}}
  {\thesubsection}
  {1em}
  {}
\titleformat{\subsubsection}
  {\sffamily\bfseries\color{gray!70}}
  {\thesubsubsection}
  {1em}
  {}
  
% Adjust spacing for sections
\titlespacing{\section}{0pt}{3.5ex plus 1ex minus .2ex}{2.3ex plus .2ex}
\titlespacing{\subsection}{0pt}{3.25ex plus 1ex minus .2ex}{1.5ex plus .2ex}

%-------------------------------------------------------------------
%   THEOREM STYLES (amsthm) — unchanged
%-------------------------------------------------------------------
\newtheoremstyle{mdtheorem}
  {} % Space above
  {} % Space below
  {\itshape} % Body font
  {} % Indent amount
  {\sffamily\bfseries} % Theorem head font
  {.} % Punctuation after theorem head
  {.5em} % Space after theorem head
  {} % Theorem head spec

\theoremstyle{mdtheorem}
\newtheorem{theorem}{Teorema}[chapter]
\newtheorem{corollary}[theorem]{Corolario}
\newtheorem{lemma}[theorem]{Lema}
\newtheorem{proposition}[theorem]{Proposición}
\newtheorem{conjecture}[theorem]{Conjetura}
\newtheorem{claim}[theorem]{Afirmación}

\newtheoremstyle{mddefinition}
  {}
  {}
  {\normalfont} % Not italic
  {}
  {\sffamily\bfseries}
  {.}
  {.5em}
  {}

\theoremstyle{mddefinition}
\newtheorem{definition}[theorem]{Definición}
\newtheorem{example}[theorem]{Ejemplo}
\theoremstyle{remark}
\newtheorem{remark}[theorem]{Observación}
\newtheorem{problem}[theorem]{Problema}

%-------------------------------------------------------------------
%   BOXES WITH tcolorbox (breakable)
%-------------------------------------------------------------------
\usepackage[most]{tcolorbox}
\tcbuselibrary{skins, breakable, theorems, hooks}

% Unified, descriptive style names (no 'mdframed' wording).
% - 'theobox'  for theorems/lemmas/etc. (teal-ish left rule)
% - 'infobox'  for definitions/examples/problems (primary left rule)
% Added breathing room at the top via top=8pt + a tiny before upper vspace.
\tcbset{
  theobox/.style={
    enhanced, breakable,
    colback=boxbgcolor,
    frame empty, boxrule=0pt,
    borderline west={2pt}{0pt}{primarycolor!50!cyan},
    left=10pt, right=10pt, top=8pt, bottom=10pt,
    before upper=\vspace{2pt},
    before skip=\topsep, after skip=\topsep,
  },
  infobox/.style={
    enhanced, breakable,
    colback=boxbgcolor,
    frame empty, boxrule=0pt,
    borderline west={2pt}{0pt}{primarycolor},
    left=15pt, right=10pt, top=8pt, bottom=10pt,
    before upper=\vspace{2pt},
    before skip=\topsep, after skip=\topsep,
  },
}

% Apply styles to environments
\tcolorboxenvironment{theorem}{theobox}
\tcolorboxenvironment{corollary}{theobox}
\tcolorboxenvironment{lemma}{theobox}
\tcolorboxenvironment{proposition}{theobox}
\tcolorboxenvironment{conjecture}{theobox}
\tcolorboxenvironment{claim}{theobox}

\tcolorboxenvironment{definition}{infobox}
\tcolorboxenvironment{example}{infobox}
\tcolorboxenvironment{problem}{infobox}
% If you want 'Observación' boxed as well, uncomment:
% \tcolorboxenvironment{remark}{infobox}

%-------------------------------------------------------------------
%   HYPERLINKS
%-------------------------------------------------------------------
\usepackage{hyperref}
\hypersetup{
    colorlinks=true,
    linkcolor=primarycolor,
    citecolor=primarycolor,
    urlcolor=blue,
    pdftitle={Probabilidades y Estadística},
    pdfauthor={Ramiro Dibur},
    pdfsubject={Libro de texto de Matemáticas},
    pdfkeywords={matemáticas, libro de texto, plantilla},
    bookmarksopen=true,
    bookmarksnumbered=true
}

%-------------------------------------------------------------------
%   COMMANDS
%-------------------------------------------------------------------
\newcommand{\N}{\mathbb{N}}
\newcommand{\Z}{\mathbb{Z}}
\newcommand{\Q}{\mathbb{Q}}
\newcommand{\R}{\mathbb{R}}
\newcommand{\C}{\mathbb{C}}
\renewcommand{\Re}{\operatorname{Re}}
\renewcommand{\Im}{\operatorname{Im}}
\DeclareMathOperator{\area}{\acute{a}rea}

%-------------------------------------------------------------------
%   DOCUMENT START
%-------------------------------------------------------------------
\begin{document}

%==================================================================
%   FRONT MATTER
%==================================================================
\frontmatter

\begin{titlepage}
    \begin{center}
        \vspace*{1cm}
        
        \sffamily
        \bfseries
        \textcolor{primarycolor}{\rule{\linewidth}{2pt}}
        \vspace{0.4cm}
        
        {\Huge Probabilidades y Estadística \par}
        
        \vspace{0.4cm}
        \textcolor{primarycolor}{\rule{\linewidth}{2pt}}
        
        \vspace{2cm}
        
        {\Large Ramiro Dibur \par}
        \vspace{1cm}
        {\large 2025 \par}
    \end{center}
\end{titlepage}

% Note: The babel package automatically changes "Contents" to "Índice"
\tableofcontents


%==================================================================
%   MAIN MATTER
%==================================================================
\mainmatter

\chapter*{Prefacio}

Estas son mis notas de Probabilidades y Estadística del segundo cuatrimentre de 2025. Las escribo más que nada para estudiar yo, pero las publico por si le llegan a ser útil a alguien.

\chapter{Números complejos}

Comenzamos con un repaso de la definición de los números complejos, algunas funciones relevantes, propiedades, conceptos topológicos, la esfera de Riemann y homografías.

\section{Definición y propiedades}

Recordemos cómo se definien los números comlejos.

\begin{definition}
    Definimos a los \emph{números complejos} como el conjunto
    \begin{equation*}
        \C = \{ a + bi \mid a, b \in \R \} \quad \text{donde } i^2 = -1 \text{ es la unidad imaginaria},
    \end{equation*}
    provisto de la suma
    \begin{equation*}
        (a + bi) + (c + di) = (a + b) + (c + d)i
    \end{equation*}
    y el producto
    \begin{equation*}
        (a + bi) \cdot (c + di) = (ac - bd) + (ad + bc)i.
    \end{equation*}
\end{definition}

\begin{remark}
    Si tratamos a un número complejo como una expresión algebraica y luego utilizamos que $i^2 = -1$, obtenemos el producto definido.
\end{remark}

Naturalmente, para un número complejo $z = a + bi$, decimos que $\Re(z) = a$ es la \textit{parte real} e $\Im(z) = b$ la \textit{parte imaginaria}. También, definimos el \textit{módulo} (o \textit{valor absoluto}) como $|z| = \sqrt{a^2 + b^2}$. Por último, definimos el conjugado.

\begin{definition}
    El \emph{conjugado} de un número complejo $z = a + bi \in \C$ es
    \begin{equation*}
        \overline{z} = a - bi.
    \end{equation*}
\end{definition}

Nos interesa averiguar cómo es el inverso de un número complejo.

\begin{proposition}
    Sea $z \in \C \setminus \{0\}$. Entonces, el inverso es
    \begin{equation*}
        z^{-1} = \frac{\overline{z}}{|z|^2}.
    \end{equation*}
\end{proposition}

\begin{proof}
    Basta con probar que $z \overline{z} = |z|^2$. Escribimos a $z = a + bi$ y entonces 
    \begin{align*}
        z \overline{z} &= (a + bi) (a - bi) \\
        &= (a)^2 - (bi)^2 \\
        &= a^2 + b^2 \\
        &= |z|^2.
    \end{align*}
    Reordenando la ecaución, obtenemos el resultado.
\end{proof}


\section{Forma polar}

Una de las formas más útiles de expresar a un número complejo es en su forma polar.

\begin{definition}
    Sea $z = a + bi \in \C \setminus \{0\}$. Definimos el \emph{argumento} de un número complejo como todo $\theta \in \R$ que cumple
    \begin{equation*}
        \Re(z) = |z| \cos \theta \quad\text{e}\quad \Im(z) = |z| \sin \theta
    \end{equation*}
    y lo denotamos $\arg(z)$. También, si $\theta \in [0, 2\pi)$, lo llamamos el \emph{argumento principal} y lo denotamos $\Arg(z)$.
\end{definition}

\begin{remark}
    Para el cero las condiciones se cumplen trivialmente.
\end{remark}

El argumento y el módulo se trasladan intuitivamente a una representación gráfica.

\begin{figure}[H]
\centering
\begin{tikzpicture}[scale=1.1]
  % Parámetros editables
  \def\r{2.2}      % módulo = |z|
  \def\ang{40}     % ángulo (en grados): Arg(z)

  % Puntos
  \coordinate (O) at (0,0);
  \coordinate (X) at (1,0);        % punto en el eje real
  \coordinate (Z) at (\ang:\r);

  % Ejes
  \draw[->] (-0.6,0) -- (3.0,0) node[below right] {$\Re$};
  \draw[->] (0,-0.6) -- (0,2.6) node[above left] {$\Im$};

  % Círculos
  \draw[densely dotted] (O) circle (1);
  \draw[dashed] (O) circle (\r);

  % Vector z
  \draw[very thick,-{Latex}] (O) -- (Z)
    node[pos=1, above right] {$z$};

  % Proyecciones
  \path let \p1 = (Z) in
    coordinate (Xproj) at (\x1,0)
    coordinate (Yproj) at (0,\y1);
  \draw[dotted] (Z) -- (Xproj) node[below] {$r\cos\theta$};
  \draw[dotted] (Z) -- (Yproj) node[left]  {$r\sin\theta$};

  % Punto y ángulo
  \fill (Z) circle (0.03);
  \pic [draw, ->, "$\theta$", angle radius=11mm, angle eccentricity=1.15]
      {angle = X--O--Z};
\end{tikzpicture}
\end{figure}

Por trigonometría, podemos deducir la siguiente expresión:
\begin{equation*}
    z = r (\cos \theta + i \sen \theta),
\end{equation*}
donde $r = |z|$ y $\theta = \Arg(z)$. 

Así también surge la \textit{fórmula de Euler}.

\begin{theorem}[Fórmula de Euler]
    Sea $z \in \C$. Entonces, podemos expresarlo como
    \begin{equation*}
        z = |z| e^{i \Arg(z)} = r e^{i \theta}.
    \end{equation*}
\end{theorem}

La demostración de esto la vemos más adelante.


\section{Raíces}

Para encontrar la raíz $n$-ésima de un número complejo, lo más fácil es escribirlo en forma de Euler y luego elevar a $\frac{1}{n}$. (No voy a ir más en detalle que esto.)

\begin{proposition}
    Sean $z_0, z_1, \dots, z_{n-1} \in \C$ las raíces $n$-ésimas de $w \in \C \setminus \{0\}$, con $m,n \in \mathbb{N}$ y $n \geq 2$. Entonces
    \begin{equation}
        \sum_{i=0}^{n-1} z_i^m \;=\;
        \begin{cases}
            n\,w^k & \text{si } m = k n,\; k \in \mathbb{N},\\
            0 & \text{en caso contrario}.
        \end{cases}
    \end{equation}
\end{proposition}

(No lo demuestro, pero sale expresando las raíces en forma de Euler y utilizando la expresión de la suma de la progresión geométrica.)


\section{Topología y continuidad}

{\color{red} TODO: ESCRIBIR PROPIEDADES}


\section{Esfera de Riemann}

Consideramos antes a $\C$ junto con el infinito.

\begin{definition}
    Definimos el \emph{plano complejo extendido} $\widehat{\C} = \C \cup \{\infty\}$.
\end{definition}

El plano complejo extendido tiene una interpretación intuitiva dada por la \textit{proyección estereográfica}. (En realidad, voy a arrancar con la inversa de la proyección estereográfica porque me resulta más intuitivo.) 

Nos situamos en $\R^3$ y consideramos a la esfera unitaria $\mathbb{S}^2$. Identificamos el plano $xy$ con $\C$ y definimos la biyección $\varphi : \C \to \mathbb{S}^2 \setminus \{N\}$ de la siguiente forma: si $z = x + iy$,
\begin{equation*}
    \varphi(z) = \left(\frac{2x}{|z|^2 + 1}, \frac{2y}{|z|^2 + 1}, \frac{|z|^2 - 1}{|z|^2 + 1}\right).
\end{equation*}

\begin{figure}[H]
    \centering
    % Código obtenido de https://tex.stackexchange.com/questions/546979/sphere-shell-using-tikz
    \begin{tikzpicture}[declare function={%
            stereox(\x,\y)=2*\x/(1+\x*\x+\y*\y);%
            stereoy(\x,\y)=2*\y/(1+\x*\x+\y*\y);%
            stereoz(\x,\y)=(-1+\x*\x+\y*\y)/(1+\x*\x+\y*\y);
            Px=1.75;Py=-1.5;Qx=-1.5;Qy=-1.25;amax=2.5;},scale=2.5,
            line join=round,line cap=round,
            dot/.style={circle,fill,inner sep=1pt},>={Stealth[length=1.2ex]}, scale=0.8]
    \pgfmathsetmacro{\myaz}{15}
    \path[save path=\pathSphere]   (0,0) circle[radius=1];
    \begin{scope}[3d view={\myaz}{18}]
    \draw (-amax,amax) -- (-amax,-amax) coordinate (bl) -- (amax,-amax) 
    coordinate (br)-- (amax,amax)
    %node[above left]{$z=0$}
    ;
    \begin{scope}
    \tikzset{protect=\pathSphere}
    \draw (-amax,amax) -- (amax,amax) node[below left,xshift=-2em]{$\C$};
    \end{scope}
    \begin{scope}
    \clip[reuse path=\pathSphere];
    \draw[dashed] (-amax,amax) -- (amax,amax);
    \end{scope}
    \begin{scope}[canvas is xy plane at z=0]
    \draw[dashed] (\myaz:1) arc[start angle=\myaz,end angle=\myaz+180,radius=1];
    \draw (\myaz:1) arc[start angle=\myaz,end angle=\myaz-180,radius=1];
    \path[save path=\pathPlane] (\myaz:amax) -- (\myaz+180:amax) --(bl) -- (br) -- cycle;
    \begin{scope}
        \clip[use path=\pathPlane];
        \draw[dashed,use path=\pathSphere];
    \end{scope}
    \begin{scope}
        \tikzset{protect=\pathPlane}
        \draw[use path=\pathSphere];
    \end{scope}
    \end{scope}
    \draw[-=0.3] (Px,Py,0) node[dot,label=below:{$w$}](w){}
    -- node[auto,pos=0.3,swap]{} ({stereox(Px,Py)},{stereoy(Px,-1)},{stereoz(Px,Py)})
    node[dot,label=left:{$\varphi(w)$}](w*){};
    \draw[-] (Qx,Qy,0) node[dot,label=below:{$z$}](z){}
    -- node[auto,pos=0.5]{} ({stereox(Qx,Qy)},{stereoy(Qx,-1)},{stereoz(Qx,Qy)})
    node[dot,label=above left:{$\varphi(z)$}](z*){};
    \draw[dashed] (w*) -- (0,0,1) node[dot,label=above:{$N$}](zeta){}
    -- (z*);
    \node at (0.5,0,0.8) [label=above right:{$\mathbb{S}$}] {};
    \end{scope}
    \end{tikzpicture}
\end{figure}

Observamos que para conseguir $\varphi(z)$, se traza una recta de $z$ a $N$ y $\Phi(z)$ es el punto donde $\mathbb{S}^2$ y la recta se intersecan.  Sin embargo, ningún punto de $\C$ termina en $N$. ¿Podemos extender $\varphi$? La respuesta es sí y se relaciona con el plano complejo extendido.

Si $|z| \to \infty$, entonces $\varphi(z) \to N$. Entonces, en $\widehat{\C}$, definimos $\varphi(\infty) = N$.

\begin{proposition}
    La proyección estereográfica envía circunferencias en $\mathbb{S}^2$ a circunferencias o rectas en $\C$.
\end{proposition}

\begin{proof}
    Sea $\mathcal C\subset\mathbb S^2$ una circunferencia y sea $\Pi:\; ax+by+cz=d$ el plano que la contiene.
    Escribimos la inversa de la proyección estereográfica:
    \begin{equation*}
    x=\frac{2u}{|w|^2+1},\quad
    y=\frac{2v}{|w|^2+1},\quad
    z=\frac{|w|^2-1}{|w|^2+1},\qquad w=u+iv\in\C.
    \end{equation*}
    Sustituyendo en la ecuación de $\Pi$ y multiplicando por $|w|^2+1$ llegamos a
    \begin{equation*}
    (c-d)|w|^2-2au-2bv-(c+d)=0.
    \end{equation*}
    Si $c\neq d$, esto equivale a $|w|^2+\alpha u+\beta v+\gamma=0$, que describe una circunferencia en $\C$.
    Si $c=d$, obtenemos $2au+2bv+(c+d)=0$, que describe una recta en $\C$.
\end{proof}

\begin{remark}
    Observemos que que $c=d$ equivale a que el polo norte $(0,0,1)$ pertenezca a $\Pi$, \textit{i.e.} a que $\mathcal C$ pase por el polo norte. (El caso tangencial $a=b=0,\ c=d$ no produce circunferencia en la esfera y queda excluido por hipótesis.)
\end{remark}

Si bien la demostración es puramente algebraica, tenemos una interpretación gráfica.

\begin{figure}[H]
    \centering
    % Código obtenido de https://tex.stackexchange.com/questions/546979/sphere-shell-using-tikz
    \begin{tikzpicture}[declare function={%
            stereox(\x,\y)=2*\x/(1+\x*\x+\y*\y);%
            stereoy(\x,\y)=2*\y/(1+\x*\x+\y*\y);%
            stereoz(\x,\y)=(-1+\x*\x+\y*\y)/(1+\x*\x+\y*\y);
            Px=1.75;Py=-1.5;Qx=-1.5;Qy=-1.25;amax=2.5;},scale=2.5,
            line join=round,line cap=round,
            dot/.style={circle,fill,inner sep=1pt},>={Stealth[length=1.2ex]}, scale=0.8]
    \pgfmathsetmacro{\myaz}{15}
    \path[save path=\pathSphere]   (0,0) circle[radius=1];
    \begin{scope}[3d view={\myaz}{18}]
    \draw (-amax,amax) -- (-amax,-amax) coordinate (bl) -- (amax,-amax) 
    coordinate (br)-- (amax,amax);

    %%%%%%%%%%%%%%%%%%%%%%%%%%%%%%%%%%%%%%%%%%%%%%%%%%%%%%%%%%%%%
    %%% PLANO por N y la recta wz en \C (como ya tenías)
    \pgfmathsetmacro{\dx}{Qx-Px}
    \pgfmathsetmacro{\dy}{Qy-Py}
    \pgfmathsetmacro{\dn}{sqrt(\dx*\dx+\dy*\dy)}
    \pgfmathsetmacro{\ux}{\dx/\dn}
    \pgfmathsetmacro{\uy}{\dy/\dn}

    \pgfmathsetmacro{\mx}{(Px+Qx)/2}
    \pgfmathsetmacro{\my}{(Py+Qy)/2}

    % Vector hacia N desde M=(mx,my,0)
    \pgfmathsetmacro{\vx}{-\mx}
    \pgfmathsetmacro{\vy}{-\my}
    \pgfmathsetmacro{\vz}{1}

    % Lámina del plano (opcional, ya la tenés: podés dejarla o quitarla)
    \pgfmathsetmacro{\L}{1.2*amax}
    \pgfmathsetmacro{\H}{1.6}
    % Normal del plano n = u x v
    \pgfmathsetmacro{\nx}{\uy*1 - 0*\vy}
    \pgfmathsetmacro{\ny}{0*\vx - \ux*1}
    \pgfmathsetmacro{\nz}{\ux*\vy - \uy*\vx}
    \pgfmathsetmacro{\nn}{sqrt(\nx*\nx+\ny*\ny+\nz*\nz)}

    %%%%%%%%%%%%%%%%%%%%%%%%%%%%%%%%%%%%%%%%%%%%%%%%%%%%%%%%%%%%%
    %%% CIRCUNFERENCIA: intersección (esfera ∩ plano) pasa por N, φ(w), φ(z)
    % Centro C = proyección ortogonal del 0 sobre el plano: C = ( (n·M)/||n||^2 ) n
    \pgfmathsetmacro{\ndotM}{\nx*\mx + \ny*\my + \nz*0}
    \pgfmathsetmacro{\Cx}{(\ndotM/(\nn*\nn))*\nx}
    \pgfmathsetmacro{\Cy}{(\ndotM/(\nn*\nn))*\ny}
    \pgfmathsetmacro{\Cz}{(\ndotM/(\nn*\nn))*\nz}

    % Radio r = sqrt(1 - d^2), con d = |n·M|/||n||
    \pgfmathsetmacro{\rC}{sqrt(max(0,1 - (\ndotM*\ndotM)/(\nn*\nn)))}

    % Base ortonormal en el plano: a = u (ya unitario), b = (n_unit x a) normalizado
    \pgfmathsetmacro{\nxu}{\nx/\nn}
    \pgfmathsetmacro{\nyu}{\ny/\nn}
    \pgfmathsetmacro{\nzu}{\nz/\nn}
    \pgfmathsetmacro{\bx}{- \nzu*\uy}
    \pgfmathsetmacro{\by}{  \nzu*\ux}
    \pgfmathsetmacro{\bz}{  \nxu*\uy - \nyu*\ux}
    \pgfmathsetmacro{\bn}{sqrt(\bx*\bx+\by*\by+\bz*\bz)}
    \pgfmathsetmacro{\bx}{\bx/\bn}
    \pgfmathsetmacro{\by}{\by/\bn}
    \pgfmathsetmacro{\bz}{\bz/\bn}

    % Dibujo por aproximación poligonal
    \def\step{4}
    \foreach \t in {0,\step,...,356}{
      \pgfmathsetmacro{\ct}{cos(\t)}
      \pgfmathsetmacro{\st}{sin(\t)}
      \pgfmathsetmacro{\ctb}{cos(\t+\step)}
      \pgfmathsetmacro{\stb}{sin(\t+\step)}
      \draw[teal!70!black, thick]
        ({\Cx + \rC*(\ux*\ct + \bx*\st)},
         {\Cy + \rC*(\uy*\ct + \by*\st)},
         {\Cz + \rC*( 0*\ct + \bz*\st)})
        -- 
        ({\Cx + \rC*(\ux*\ctb + \bx*\stb)},
         {\Cy + \rC*(\uy*\ctb + \by*\stb)},
         {\Cz + \rC*( 0*\ctb + \bz*\stb)});
    }
    %%%%%%%%%%%%%%%%%%%%%%%%%%%%%%%%%%%%%%%%%%%%%%%%%%%%%%%%%%%%%

    \begin{scope}
    \tikzset{protect=\pathSphere}
    \draw (-amax,amax) -- (amax,amax) node[below left,xshift=-2em]{$\C$};
    \end{scope}
    \begin{scope}
    \clip[reuse path=\pathSphere];
    \draw[dashed] (-amax,amax) -- (amax,amax);
    \end{scope}
    \begin{scope}[canvas is xy plane at z=0]
      \draw[dashed] (\myaz:1) arc[start angle=\myaz,end angle=\myaz+180,radius=1];
      \draw        (\myaz:1) arc[start angle=\myaz,end angle=\myaz-180,radius=1];
      \path[save path=\pathPlane] (\myaz:amax) -- (\myaz+180:amax) --(bl) -- (br) -- cycle;
      \begin{scope}
        \clip[use path=\pathPlane];
        \draw[dashed, use path=\pathSphere];
      \end{scope}
      \begin{scope}
        \tikzset{protect=\pathPlane}
        \draw[use path=\pathSphere];
      \end{scope}
    \end{scope}

    % Segmentos de proyección y puntos
    \draw[-=0.3] (Px,Py,0) node[dot,label=below:{$w$}](w){}
      -- ({stereox(Px,Py)},{stereoy(Px,Py)},{stereoz(Px,Py)})
         node[dot,label=left:{$\varphi(w)$}](w*){};
    \draw[-] (Qx,Qy,0) node[dot,label=below:{$z$}](z){}
      -- ({stereox(Qx,Qy)},{stereoy(Qx,Qy)},{stereoz(Qx,Qy)})
         node[dot,label=above left:{$\varphi(z)$}](z*){};
    \draw[dashed] (w*) -- (0,0,1) node[dot,label=above:{$N$}](N){} -- (z*);
    \node at (0.5,0,0.8) [label=above right:{$\mathbb{S}$}] {};

    %%%%%%%%%%%%%%%%%%%%%%%%%%%%%%%%%%%%%%%%%%%%%%%%%%%%%%%%%%%%%
    %%% PLANO: pasa por N y por la recta wz en C
    % Dirección de la recta en C: d = (dx,dy,0)
    \pgfmathsetmacro{\dx}{Qx-Px}
    \pgfmathsetmacro{\dy}{Qy-Py}
    \pgfmathsetmacro{\dn}{sqrt(\dx*\dx+\dy*\dy)}
    \pgfmathsetmacro{\ux}{\dx/\dn}
    \pgfmathsetmacro{\uy}{\dy/\dn}

    % Un punto de la recta (el medio entre w y z)
    \pgfmathsetmacro{\mx}{(Px+Qx)/2}
    \pgfmathsetmacro{\my}{(Py+Qy)/2}

    % Vector hacia N desde ese punto: v = N - M = (-mx,-my,1)
    \pgfmathsetmacro{\vx}{-\mx}
    \pgfmathsetmacro{\vy}{-\my}
    \pgfmathsetmacro{\vz}{1}
    \pgfmathsetmacro{\vn}{sqrt(\vx*\vx+\vy*\vy+\vz*\vz)}
    \pgfmathsetmacro{\vxu}{\vx/\vn}
    \pgfmathsetmacro{\vyu}{\vy/\vn}
    \pgfmathsetmacro{\vzu}{\vz/\vn}

    % Tamaños de la "lámina" (largos a gusto)
    \pgfmathsetmacro{\L}{1.2*amax}
    \pgfmathsetmacro{\H}{1.6}

    % Esquinas del rectángulo en el plano: M ± L*u ± H*vhat
    \coordinate (P1) at ({\mx+\L*\ux+\H*\vxu}, {\my+\L*\uy+\H*\vyu}, {\H*\vzu});
    \coordinate (P2) at ({\mx-\L*\ux+\H*\vxu}, {\my-\L*\uy+\H*\vyu}, {\H*\vzu});
    \coordinate (P3) at ({\mx-\L*\ux-\H*\vxu}, {\my-\L*\uy-\H*\vyu}, {-\H*\vzu});
    \coordinate (P4) at ({\mx+\L*\ux-\H*\vxu}, {\my+\L*\uy-\H*\vyu}, {-\H*\vzu});

    % Lámina del plano (ligeramente translúcida)
    \fill[teal!20, opacity=0.12] (P1)--(P2)--(P3)--(P4)--cycle;

    % Intersección con C (z=0): la misma recta por M con dirección u
    \begin{scope}[canvas is xy plane at z=0]
      \draw[teal!50!black, thick]
        ({\mx-\L*\ux},{\my-\L*\uy}) -- ({\mx+\L*\ux},{\my+\L*\uy});
    \end{scope}
    %%%%%%%%%%%%%%%%%%%%%%%%%%%%%%%%%%%%%%%%%%%%%%%%%%%%%%%%%%%%%

    \end{scope}
    \end{tikzpicture}
\end{figure}


Dado que la circunferencia pasa por $N$, la proyección es una recta en vez de una circunferencia.


\section{Homografías}

Definimos las homografías que luego van a ser útiles.

\begin{definition}
    Una \textbf{homografía} es una función $f : \widehat{\C} \to \widehat{\C}$
    \begin{equation*}
        f(z) = \frac{az+b}{cz+d},
    \end{equation*}
    donde $a,b,c,d \in \C$ y $ad - bc \neq 0$.
\end{definition}

Hay dos valores de $f$ que quizás no son inmediatamente obvios: para $c \neq 0$,
\begin{equation*}
    \begin{cases}
        f(-\frac{d}{c}) = \infty, \\
        f(\infty) = \frac{a}{c}.
    \end{cases}
\end{equation*}

Las \textit{homografías básicas} son:
\begin{enumerate}
    \item \textbf{Traslación:} 
    \begin{equation*}
        T_a(z) = z + a, \qquad a \in \C.
    \end{equation*}
    \item \textbf{Homotecia:} 
    \begin{equation*}
        H_r(z) = r z, \qquad r > 0.
    \end{equation*}
    \item \textbf{Rotación:} 
    \begin{equation*}
        R_\alpha(z) = \alpha z, \qquad |\alpha| = 1.
    \end{equation*}
    \item \textbf{Inversión:} 
    \begin{equation*}
        I(z) = \frac{1}{z}.
    \end{equation*}
\end{enumerate}

\begin{proposition}
    Sea $f:\widehat{\C}\to\widehat{\C}$ una homografía con $c\neq 0$. Entonces, existen aplicaciones afines $g,h:\C\to\C$ tales que
    \begin{equation*}
        f = g \circ i \circ h, \qquad \text{donde } i(z)=\frac{1}{z}.
    \end{equation*}
\end{proposition}

\begin{proof}
    Sea $q \in \C$ tal que $ad - qc = 0$, o sea $q = \frac{ad}{c}$. Entonces
    \begin{align*}
        f(z) &= \frac{az+b}{cz+d} 
        = \frac{az+b+q-q}{cz+d} 
        = \frac{az+\tfrac{ad}{c}}{cz+d} + \frac{b-q}{cz+d}.
    \end{align*}
    De este modo,
    \begin{align*}
        f(z) &= \frac{az+\tfrac{d}{c}\,c}{cz+d} + \frac{b-\tfrac{ad}{c}}{cz+d} \\
             &= \frac{a}{c} + \Bigl(b-\frac{ad}{c}\Bigr)(cz+d)^{-1}.
    \end{align*}
    Luego, definiendo
    \[
        h(z) = cz+d, 
        \qquad g(w) = \frac{a}{c} + \Bigl(b-\frac{ad}{c}\Bigr)w,
    \]
    se obtiene que $f = g \circ i \circ h$.
\end{proof}

\begin{proposition}
    La imagen de cualquier circunferencia o recta por una homografía es una circunferencia o una recta.
\end{proposition}

\begin{proof}
    El único caso complicado es el de las inversiones. Consideramos una circunferencia dada por la ecuación
    \begin{equation*}
        |z - z_0| = r, \qquad r \in \R_{\geq 0}.
    \end{equation*}
    Equivalentemente,
    \begin{equation*}
        (z - z_0)(\overline{z} - \overline{z_0}) = r^2.
    \end{equation*}
    Como en la inversión $i(z)=\dfrac{1}{z}$ tenemos $\overline{z}=\dfrac{1}{\overline{i(z)}}=\dfrac{1}{\overline{w}}$ con $w=i(z)$, resulta
    \begin{equation*}
        \Bigl(\frac{1}{w}-z_0\Bigr)\Bigl(\frac{1}{\overline{w}}-\overline{z_0}\Bigr)=r^2.
    \end{equation*}
    Multiplicando ambos lados por $|w|^2$, se obtiene
    \begin{equation*}
        (1-z_0w)(1-\overline{z_0}\,\overline{w})=r^2|w|^2.
    \end{equation*}
    Desarrollando,
    \begin{equation*}
        1 - z_0 w - \overline{z_0}\,\overline{w} + |z_0|^2 |w|^2 = r^2 |w|^2.
    \end{equation*}
    Es decir,
    \begin{equation*}
        (|z_0|^2-r^2)|w|^2 - z_0 w - \overline{z_0}\,\overline{w} + 1 = 0,
    \end{equation*}
    que es la ecuación de una circunferencia o de una recta en el plano complejo, según que $|z_0|^2-r^2\neq 0$ o $|z_0|^2-r^2=0$. 
    Por lo tanto, la inversión envía circunferencias en circunferencias o rectas, y lo mismo vale para toda homografía.
\end{proof}

Veamos que toda homografía está determinada unívocamente por tres puntos.

\begin{proposition}
    Sean $z_1, z_2, z_3 \in \C$. Hay una única homografía que satisface
    \begin{equation*}
        f(z_1) = 0, \quad f(z_2) = 1, \quad f(z_3) = \infty.
    \end{equation*}
\end{proposition}

\begin{proof}
    {\color{red} COMPLETAR}
\end{proof}

\end{document}
