\documentclass[11pt, a4paper, oneside]{book}

\usepackage[utf8]{inputenc}
\usepackage[spanish]{babel}
\usepackage{lmodern}
\usepackage{microtype}

\usepackage{amsmath}
\usepackage{amssymb}
\usepackage{amsthm}
\usepackage{amsfonts}
\usepackage{stmaryrd}

\usepackage{graphicx}
\graphicspath{{images/}}

\usepackage[margin=1in, headheight=14.68782pt]{geometry}
\linespread{1.1}
\usepackage{indentfirst}

\usepackage{xcolor}
\definecolor{accentcolor}{HTML}{005691}
\definecolor{complementarycolor}{HTML}{560091}
\definecolor{lightgray}{gray}{0.95}

\usepackage{tikz}
\usepackage{pgfplots}
\pgfplotsset{compat=1.18}
\usetikzlibrary{babel}
\usetikzlibrary{calc}
\usetikzlibrary{positioning}
\usetikzlibrary{svg.path}

\usepackage{hyperref}
\hypersetup{
    colorlinks=true,
    linkcolor=accentcolor,
    filecolor=magenta,
    urlcolor=accentcolor,
    citecolor=accentcolor,
    hidelinks
}

\usepackage{fancyhdr}
\pagestyle{fancy}
\fancyhf{}
\fancyhead[L]{\color{accentcolor}\scshape \leftmark}
\fancyfoot[C]{\thepage}
\renewcommand{\headrulewidth}{0.8pt}
\renewcommand{\footrulewidth}{0pt}

\usepackage{titlesec}
\titleformat{\chapter}[display]
    {\normalfont\bfseries\Huge\color{accentcolor}}
    {\chaptertitlename\ \thechapter}
    {20pt}
    {\Huge}
    [\vspace{10pt}\hrulefill]
\titlespacing*{\chapter}{0pt}{50pt}{40pt}

\titleformat{\section}
    {\normalfont\Large\bfseries\color{accentcolor}\sffamily}
    {\thesection}
    {1em}
    {}
\titlespacing*{\section}{0pt}{3.5ex plus 1ex minus .2ex}{2.3ex plus .2ex}

\titleformat{\subsection}
    {\normalfont\large\bfseries\sffamily}
    {\thesubsection}
    {1em}{}
\titlespacing*{\subsection}{0pt}{3.25ex plus 1ex minus .2ex}{1.5ex plus .2ex}

\usepackage[most]{tcolorbox}
\tcbuselibrary{theorems, skins}

\tcbset{
    mathbookbox/.style={
        breakable,
        colback=lightgray,
        colframe=accentcolor,
        boxrule=0.8pt,
        arc=3mm,
        leftrule=3pt,
        rightrule=0pt,toprule=0pt,bottomrule=0pt,
        boxsep=4pt,
        parbox=false,
    }
}

\theoremstyle{plain}
\newtheorem{theorem}{Teorema}[chapter]
\newtheorem{lemma}[theorem]{Lema}
\newtheorem{corollary}[theorem]{Corolario}
\newtheorem{proposition}[theorem]{Proposición}
\newtheorem{axiom}{Axioma}[chapter]

\theoremstyle{definition}
\newtheorem{definition}{Definición}[chapter]
\newtheorem{example}{Ejemplo}[chapter]
\newtheorem{exercise}{Ejercicio}[chapter]

\theoremstyle{remark}
\newtheorem*{remark}{Observación}

\tcolorboxenvironment{theorem}{mathbookbox, colframe=accentcolor!50!cyan}
\tcolorboxenvironment{axiom}{mathbookbox, colframe=accentcolor!50!cyan}
\tcolorboxenvironment{lemma}{mathbookbox, colframe=accentcolor!50!cyan}
\tcolorboxenvironment{corollary}{mathbookbox, colframe=accentcolor!50!cyan, boxrule=0.4pt, colback=white} \tcolorboxenvironment{proposition}{mathbookbox, colframe=accentcolor!50!cyan}
\tcolorboxenvironment{definition}{mathbookbox}
\tcolorboxenvironment{example}{mathbookbox, boxrule=0.4pt, colback=white}
\tcolorboxenvironment{exercise}{mathbookbox, boxrule=0.4pt, colback=white}

\usepackage{enumitem}
\setlist[enumerate]{itemsep=0.5ex, parsep=0.5ex, topsep=0.5ex}
\setlist[itemize]{itemsep=0.5ex, parsep=0.5ex, topsep=0.5ex}

\newcommand{\R}{\mathbb{R}}
\newcommand{\C}{\mathbb{C}}
\newcommand{\N}{\mathbb{N}}
\newcommand{\Z}{\mathbb{Z}}
\newcommand{\Q}{\mathbb{Q}}

\DeclareMathOperator{\sen}{sen}
\DeclareMathOperator{\arctg}{arctg}
\DeclareMathOperator{\sgn}{sgn}
\DeclareMathOperator{\diam}{diam}
\DeclareMathOperator{\ev}{ev}

\newcommand{\diff}[1]{\text{d}#1}
\newcommand{\der}[2]{\frac{\diff{#1}}{\diff{#2}} }
\newcommand{\pder}[2]{\frac{\partial #1}{\partial #2}}

\DeclareEmphSequence{\color{accentcolor}\bfseries\itshape}

\begin{document}

\thispagestyle{empty}
\vspace*{\fill}
\begin{center}
	{\fontsize{50pt}{52pt}\selectfont\color{accentcolor}\bfseries{Notas de Cálculo Avanzado}\MakeUppercase{}\par}
	\vspace{2.5cm}
	{\Huge\sffamily Ramiro Dibur \par}
	\vspace{0.8cm}
	\vspace{1.5cm}
	{\Large 2025 \par}
\end{center}
\vspace*{\fill}

\newpage

\frontmatter

\section*{Prefacio}

Hola, ¿cómo estás? Te voy a contar un poco de qué se tratan estas notas. En primer lugar, el propósito principal de estas notas es ayudarme a aprender a mí. Sin dudas, escribir pensando en cómo explicar una demostración o planificar el orden de los contenidos es una de las maneras más poderosas para aprender. Segundo, si a alguien ---me refiero a vos--- le es de ayuda, entonces ya me conformo. No pretendo que estás notas sean un reemplazo de la cursada; sólo busco compartir mi camino. Tercero y por último, estoy aburrido.

La estructura y contenido de las notas va a ser bastante relajado, pero voy a intentar que sea completo. Seguramente haya errores, así que tené cuidado con lo que leés. La idea es que puedas ir a un capitulo cualquiera y con leerlo ya más o menos tengas una idea de qué se trata. Consideralos como mini notas individuales compiladas.

No sé si en esta versión o en otra de más adelante, pero voy a intentar de agregar los resueltos de las guías. Esto es un trabajo monstruoso, pero si llego a poder lo subo.

Nada más. ¡Éxitos!

\hfill --- Rama :)

\tableofcontents

\mainmatter

\part{Teoría de conjuntos}

\chapter{Relaciones y orden}

Este capítulo realmente no es complicado. Te recomiendo pasarlo bastante rápido y cualquier cosa, si no te acordás de alguna definición, volvé a leerla en el momento que la necesites.

\section{Relaciones binarias}

Hacemos un breve repaso de la definición de relación y relaciones de orden ---visto en Álgebra I---.

\begin{definition}
	Definimos una \emph{relación} entre dos conjuntos $A$ y $B$ como un subconjunto $\mathcal{R} \subseteq A \times B$. Para $a \in A$ y $b \in B$ decimos que $a \mathcal{R} b$ si $(a, b) \in \mathcal{R}$.
\end{definition}

Recordemos también algunas de las propiedades que podía tener una relación.

\begin{definition}
	Sea $\mathcal{R}$ una relación en $A \times A$ (o simplemente en $A$). Entonces, decimos que $\mathcal{R}$ es:
	\begin{itemize}
		\item \textbf{Reflexiva} si $a \mathcal{R} a$, para todo $a \in A$.
		\item \textbf{Simétrica} si $a \mathcal{R} b$ implica $b \mathcal{R} a$, para todo $a, b \in A$.
		\item \textbf{Antisimétrica} si $a \mathcal{R} b$ y $b \mathcal{R} a$ implica $a = b$, para todo $a, b \in \mathcal{R}$.
		\item \textbf{Transitiva} si $a \mathcal{R} b$ y $b \mathcal{R} c$ implica $a \mathcal{R} c$, para todo $a, b, c \in \mathcal{R}$.
	\end{itemize}
\end{definition}

Seguramente recordarás ---de Álgebra I también--- que hay distintos tipos de relaciones. En particular, nos van a interesar las relaciones de \textit{orden}.

\begin{definition}
	Decimos que una relación en $A$ es de \emph{orden} si es reflexiva, antisimétrica y transitiva.
\end{definition}

Veamos algunos ejemplos de relaciones de orden.

\begin{example}
	\begin{enumerate}
		\item La relación $\leq$ en $\mathbb{R}$ es de orden.
		\item La relación $\mid$ en $\mathbb{N}$ es de orden.
		\item Sin embargo, la relación $\mid$ en $\mathbb{Z}$ no es de orden.
	\end{enumerate}
\end{example}


\section{Conjuntos ordenados}

Este concepto puede parecer nuevo, pero simplemente es considerar la relación y al conjunto en el que está definida.

\begin{definition}
	Un \emph{conjunto ordenado} (o \emph{poset}) es un par $(A, \preceq)$ donde $A$ es un conjunto y $\preceq$ es una relación de orden en $A$.
\end{definition}

Además de \textit{conjunto ordenado}, también podemos decir \textit{conjunto parcialmente ordenado} (de acá viene \textit{poset}).

\begin{definition}
	Sea $(A, \preceq)$ un conjunto ordenado tal que, para cualesquiera $a, b \in A$,
	\begin{equation*}
		a \preceq b \quad\text{ó}\quad b \preceq a.
	\end{equation*}
	Entonces, decimos que $(A, \preceq)$ es un \emph{conjunto totalmente ordenado} (o una \emph{cadena}).
\end{definition}

Con $\preceq$ nos referimos a una relación de orden \textit{arbitraria} en $A$. Si en algún momento se presta a confusión en qué conjunto estamos trabajando, también podemos notarlo como $\preceq_{A}$.

\begin{example}
	\noindent
	\begin{enumerate}
		\item $(\mathbb{R}, \leq)$ es un conjunto totalmente ordenado.
		\item $(\mathbb{N}, \mid)$ es un conjunto parcialmente ordenado.
		\item $(\mathcal{P}(X), \subseteq)$ es un conjunto parcialmente ordenado.
	\end{enumerate}
\end{example}

\section{Cotas, supremo e ínfimo}

Dentro de los conjuntos ordenados, nos interesa ver cotas superiores, inferiores, supremos e ínfimos.

\begin{definition}
	Sea $(A, \preceq)$ un conjunto ordenado y sea $B \subseteq A$. Entonces, decimos que:
	\begin{itemize}
		\item Un elemento $s \in A$ es una \emph{cota superior} de $B$ si $b \preceq s$, para todo $b \in B$.
		\item Un elemento $t \in A$ es una \emph{cota inferior} de $B$ si $t \preceq b$, para todo $b \in B$.
		\item Una cota superior $s \in A$ es el \emph{supremo} de $B$ si $s \preceq s'$, para toda cota superior $s'$ de $B$.
		\item Una cota inferior $t \in A$ es el \emph{ínfimo} de $B$ si $t' \preceq t$, para toda cota inferior $t'$ de $B$.
	\end{itemize}
\end{definition}

A continuación demostraremos la unicidad del supremo y el ínfimo, permitiéndonos decir \textit{el} supremo y \textit{el} ínfimo.

\begin{proposition}
	Sea $(A, \preceq)$ un conjunto ordenado y sea $B \subseteq A$. Si existe un supremo (ínfimo) de $B$, entonces es único.
\end{proposition}

\begin{proof}
	Sea $s, s' \in A$ supremos de $B$. Por definición de supremo,
	\begin{equation*}
		s \preceq s' \quad \text{y} \quad s' \preceq s.
	\end{equation*}
	Por lo tanto, $s = s'$. (El argumento es idéntico para el ínfimo.)
\end{proof}

Utilizamos la notación $\sup B$ para el supremo de $B$ e $\inf B$ para el ínfimo de $B$.

\begin{remark}
	Cuando el supremo (ínfimo) se realiza es igual al máximo (mínimo).
\end{remark}

\begin{example}
	Consideremos los siguientes conjuntos:
	\begin{itemize}
		\item $\sup [0, 1] = \max [0, 1] 1$ e $\inf [0, 1] = \min [0, 1] = 0$.
		\item $\sup (0, 1) = 1$ e $\inf (0, 1) = 0$.
		\item $\sup\{ \frac{1}{n} \mid n \in \mathbb{N} \} = 1$ e $\inf\{ \frac{1}{n} \mid n \in \mathbb{N} \} = 0$.
	\end{itemize}
\end{example}

\section{Funciones crecientes e isomorfismos de orden}

Vamos a definir morfismos e isomorfismos de orden que más adelante los vamos a utilizar para enunciar el teorema del punto fijo.

\begin{definition}
	Sean $(A, \preceq_A)$ y $(B, \preceq_B)$ conjuntos ordenados. Decimos que una función $f : (A, \preceq_A) \to (B, \preceq_B)$ es \emph{creciente} (o un \emph{morfismo de orden}) si
	\begin{equation*}
		a \preceq_A b \implies f(a) \preceq_B f(b),
	\end{equation*}
	para todo $a, b \in A$. Asimismo, decimos que $f$ es \emph{decreciente} si
	\begin{equation*}
		a \preceq_A b \implies f(b) \preceq_B f(a),
	\end{equation*}
	para todo $a, b \in A$.
\end{definition}

Por lo general, no hace falta aclarar qué orden estamos usando; suele estar implícito.

\begin{example}
	La función identidad $\mathrm{Id} : (mathbb{N}, \mid) \to (\mathbb{N}, \leq)$ es creciente.
\end{example}

Y si la inversa es cretiente, obtenemos la siguiente definición.

\begin{definition}
	Sean $(A, \preceq_A)$ y $(B, \preceq_B)$ conjuntos ordenados. Decimos que una función $f : (A, \preceq_A) \to (B, \preceq_B)$ es un \emph{isomorfismo de orden} si es biyectiva y tanto $f$ como $f^{-1}$ son crecientes.
\end{definition}

\begin{example}
	La biyección $f : \mathbb{N} \to 2\mathbb{N}$ tal que $f(n) = 2n$ es un isomorfismo de orden.
\end{example}










\chapter{Reticulados}

Otro capítulo corto. Lo único que vamos a ver es reticulados y el teorema del punto fijo.

\section{Definición y ejemplos}

\begin{definition}
	Decimos que un conjunto ordenado $(\mathcal{O}, \preceq)$ es un \emph{reticulado} si para todo $a, b \in \mathcal{O}$ existen el supremo y el ínfimo de $\{ a, b \}$.
\end{definition}

\begin{example}
	El conjunto ordenado $(\mathcal{P}(X), \subseteq)$ es un reticulado.
\end{example}

\begin{proof}[Solución]
	Sean $A, B \subseteq X$. Veamos que
	\begin{equation*}
		\sup \{ A, B \} = A \cup B \quad \text{e} \quad \inf \{ A, B \} = A \cap B,
	\end{equation*}
	así probando que existe el supremo y el ínfimo y entonces $(\mathcal{P}(X), \subseteq)$ es un reticulado.

	Claramente, $A \cup B$ es una cota superior de $\{ A, B \}$, ya que $A, B \subseteq A \cup B$ por definición. Sea $S \subseteq X$ una cota superior de $\{ A, B \}$. Entonces, por definición de cota superior,
	\begin{equation*}
		A, B \subseteq S.
	\end{equation*}
	Entonces, $A \cup B \subseteq S$. Lo cual demuestra que $\sup \{ A, B \} = A \cup B$.

	La demostración para el ínfimo es análoga.
\end{proof}

\begin{remark}
	Este mismo argumento lo podemos extender a
	\begin{equation*}
		\sup \{ A_i \}_{i \in I} = \bigcup_{i \in I} A_i \quad \text{e} \quad \inf \{ A_i \}_{i \in I} = \bigcap_{i \in I} A_i.
	\end{equation*}
\end{remark}

\begin{example}
	El conjunto ordenado $(\mathbb{N},\mid)$ es un reticulado. En particular,
	\begin{equation*}
		\sup \{ a, b \} = \mathrm{mcd}\{ a, b \} \quad \text{e} \quad \inf \{ a, b \} = \mathrm{gcd}\{ a, b \}.
	\end{equation*}
\end{example}

\section{Reticulados completos}

\begin{definition}
	Sea $(\mathcal{O}, \preceq)$ un reticulado. Decimos que es \emph{completo} si para todo subconjunto de $\mathcal{O}$ tiene supremo e ínfimo.
\end{definition}

\begin{example}
	El conjunto ordenado $(\mathcal{P}(X), \preceq)$ es un reticulado completo.
\end{example}

\begin{proof}[Solución]
	Utilizando la observación previa, ya probamos que todo subconjunto de $\mathcal{P}(X)$ tiene supremo e ínfimo.
\end{proof}

\begin{proposition}
	Sea $(\mathcal{O}, \preceq)$ un reticulado. Entonces, las siguientes propiedades son verdaderas:
	\begin{enumerate}
		\item \textbf{Idempotencia:} $a \vee a = a$ y $a \wedge a = a$.
		\item \textbf{Conmutatividad:} $a \vee b = b \vee a$ y $a \wedge b = b \wedge a$.
		\item \textbf{Asociatividad:} $(a \vee b) \vee c = a \vee (b \vee c)$ y $(a \wedge b) \wedge c = a \wedge (b \wedge c)$.
		\item \textbf{Absorción:} $a \vee (a \wedge b) = a$ y $a \wedge (a \vee b) = a$.
	\end{enumerate}
\end{proposition}

\begin{proof}
	La demostración no es particularmente complicada, sólo molesta.
\end{proof}

\section{Teorema del Punto Fijo}

Finalmente, enunciamos y probamos el teorema del punto fijo.

\begin{theorem}
	Sea $(\mathcal{O}, \preceq)$ un reticulado \textit{completo} y sea $f: \mathcal{O} \to  \mathcal{O}$ \textit{creciente}. Entonces, existe $x \in \mathcal{O}$ tal que $f(x)= x$.
\end{theorem}

\begin{proof}
	Sea $A = \{ x \in \mathcal{O} \mid x \preceq f(x) \}$. Como $\mathcal{O}$ es un reticulado completo, considero $s = \sup A$. Recordemos que, para todo $a \in A$, $a \preceq s$. Y como $f$ es creciente, para todo $a \in A$,
	$$
		f(a) \preceq f(s).
	$$
	Por definición de $A$, $a \preceq f(a)$ y a su vez $f(a) \preceq f(s)$, entonces, para todo $a \in A$,
	$$
		a \preceq f(s).
	$$
	O sea, $f(s)$ es cota superior de $A$. Por definición del supremo, $s \preceq f(s)$ y entonces $s \in A$. A la desigualdad $s \preceq f(s)$ le aplicamos $f$ y obtenemos
	$$
		f(s) \preceq f(f(s)).
	$$
	Por lo tanto, $f(s) \in A$; entonces tenemos $f(s) \preceq s$ y $s \preceq f(s)$, lo cual implica que $f(s) = s$.
\end{proof}




\chapter{Cardinalidad}

Ahora sí arranca el primer capítulo importante de la materia.

\section{Conjuntos finitos y numerables}

Como se puede esperar, la definición de conjunto finito es bastante intuitiva.

\begin{definition}
	Decimos que un conjunto $A$ es \emph{finito} si es vacío o existe una biyección $f: \llbracket n \rrbracket \to  A$. Definimos el $ \lvert A \rvert = n$ y $ \lvert \emptyset \rvert = 0$. Además, $A$ es \emph{infinito} si no es finito.
\end{definition}

Por si no lo habías visto antes, la notación $\llbracket n \rrbracket$ es el conjunto de números naturales hasta $n$.

\begin{definition}
	Decimos que $A$ es \emph{numerable} si existe una biyección $f: \mathbb{N} \to  A$. Además, si $A$ es numerable o finito decimos que es \emph{contable}.
\end{definition}

La idea detrás de armar biyecciones entre conjuntos para probar su cardinalidad se debe a que es imposible ``contar'' de manera tradicional los elementos de un conjunto.

\begin{example}
	El conjunto $\mathbb{Z}$ es numerable.
\end{example}

\begin{proof}[Solución]
	Para demostrar que $\mathbb{Z}$ es numerable, necesitamos encontrar una biyección $f : \mathbb{N} \to \mathbb{Z}$. La definimos como
	\begin{equation*}
		f(n) = \begin{cases}
			\frac{n}{2}    & \text{si } n \text{ es par},   \\
			-\frac{n-1}{2} & \text{si } n \text{ es impar}.
		\end{cases}
	\end{equation*}
	Aunque a simple vista sea difícil verlo, esta función manda a los pares a $\mathbb{N}$ y a los impares a $\mathbb{Z}_{\leq 0}$.
\end{proof}

\begin{example}
	El conjunto $\mathbb{Q}$ es numerable.
\end{example}

\begin{proof}[Solución]
	Buscamos una función $f : \mathbb{N} \to \mathbb{Q}$ biyectiva. Primero, hacemos una biyección de $\mathbb{N}$ a $\mathbb{Q}^+$. Definimos la función de la siguiente manera:
	\begin{center}
		\begin{tikzpicture}[
		scale=1.1, % Escala general
		font=\small, % Fuente para las fracciones
		node_style/.style={minimum size=2.2em, inner sep=1pt}, % Estilo para nodos de fracciones
		path_arrow/.style={->, >=stealth, thick, accentcolor, shorten >=2pt, shorten <=2pt}
	]
	% Coordenadas base para la tabla
	\def\xstart{0}
	\def\ystart{0}
	\def\xstep{1.3} % Separación horizontal
	\def\ystep{-1.2} % Separación vertical

	% Dibujar la cuadrícula de fracciones p/q (todas)
	% Fila q=1
	\node[node_style] (f-1-1) at (\xstart + 1*\xstep, \ystart + 1*\ystep) {$\frac{1}{1}$};
	\node[node_style] (f-2-1) at (\xstart + 2*\xstep, \ystart + 1*\ystep) {$\frac{2}{1}$};
	\node[node_style] (f-3-1) at (\xstart + 3*\xstep, \ystart + 1*\ystep) {$\frac{3}{1}$};
	\node[node_style] (f-4-1) at (\xstart + 4*\xstep, \ystart + 1*\ystep) {$\frac{4}{1}$};
	\node at (\xstart + 4.8*\xstep, \ystart + 1*\ystep) {$\dots$};

	% Fila q=2
	\node[node_style] (f-1-2) at (\xstart + 1*\xstep, \ystart + 2*\ystep) {$\frac{1}{2}$};
	\node[node_style] (f-2-2) at (\xstart + 2*\xstep, \ystart + 2*\ystep) {$\frac{2}{2}$};
	\node[node_style] (f-3-2) at (\xstart + 3*\xstep, \ystart + 2*\ystep) {$\frac{3}{2}$};
	\node[node_style] (f-4-2) at (\xstart + 4*\xstep, \ystart + 2*\ystep) {$\frac{4}{2}$};
	\node at (\xstart + 4.8*\xstep, \ystart + 2*\ystep) {$\dots$};

	% Fila q=3
	\node[node_style] (f-1-3) at (\xstart + 1*\xstep, \ystart + 3*\ystep) {$\frac{1}{3}$};
	\node[node_style] (f-2-3) at (\xstart + 2*\xstep, \ystart + 3*\ystep) {$\frac{2}{3}$};
	\node[node_style] (f-3-3) at (\xstart + 3*\xstep, \ystart + 3*\ystep) {$\frac{3}{3}$};
	\node[node_style] (f-4-3) at (\xstart + 4*\xstep, \ystart + 3*\ystep) {$\frac{4}{3}$};
	\node at (\xstart + 4.8*\xstep, \ystart + 3*\ystep) {$\dots$};

	% Fila q=4
	\node[node_style] (f-1-4) at (\xstart + 1*\xstep, \ystart + 4*\ystep) {$\frac{1}{4}$};
	\node[node_style] (f-2-4) at (\xstart + 2*\xstep, \ystart + 4*\ystep) {$\frac{2}{4}$};
	\node[node_style] (f-3-4) at (\xstart + 3*\xstep, \ystart + 4*\ystep) {$\frac{3}{4}$};
	\node[node_style] (f-4-4) at (\xstart + 4*\xstep, \ystart + 4*\ystep) {$\frac{4}{4}$};
	\node at (\xstart + 4.8*\xstep, \ystart + 4*\ystep) {$\dots$};

	% Puntos suspensivos verticales
	\foreach \pxval in {1,...,4}{
			\node at (\xstart + \pxval*\xstep, \ystart + 4.8*\ystep + 0.1*\ystep) {$\vdots$};
		}

	% Dibujar el camino de enumeración diagonal con flechas curvas en los "giros"

	% Diagonal 1 (p+q=2)
	% 1/1 -> (inicio de la siguiente diagonal) 2/1
	\draw[path_arrow] (f-1-1) to [out=0, in=180, looseness=0.8] (f-2-1); % Curva suave a la derecha

	% Diagonal 2 (p+q=3)
	% 2/1 -> 1/2
	\draw[path_arrow] (f-2-1) to [out=-135, in=45, looseness=0.8] (f-1-2); % Curva hacia abajo e izquierda
	% 1/2 -> (inicio de la siguiente diagonal) 3/1
	\draw[path_arrow] (f-1-2) to [out=0, in=180, looseness=0.8] (f-3-1); % Curva suave a la derecha

	% Diagonal 3 (p+q=4)
	% 3/1 -> 2/2
	\draw[path_arrow] (f-3-1) to [out=-135, in=45, looseness=0.8] (f-2-2); % Curva hacia abajo e izquierda
	% 2/2 -> 1/3
	\draw[path_arrow] (f-2-2) to [out=-135, in=45, looseness=0.8] (f-1-3); % Curva hacia abajo e izquierda
	% 1/3 -> (inicio de la siguiente diagonal) 4/1
	\draw[path_arrow] (f-1-3) to [out=0, in=180, looseness=0.8] (f-4-1); % Curva suave a la derecha

	% Diagonal 4 (p+q=5)
	% 4/1 -> 3/2
	\draw[path_arrow] (f-4-1) to [out=-135, in=45, looseness=0.8] (f-3-2); % Curva hacia abajo e izquierda
	% 3/2 -> 2/3
	\draw[path_arrow] (f-3-2) to [out=-135, in=45, looseness=0.8] (f-2-3); % Curva hacia abajo e izquierda
	% 2/3 -> 1/4
	\draw[path_arrow] (f-2-3) to [out=-135, in=45, looseness=0.8] (f-1-4); % Curva hacia abajo e izquierda

	% Nota sobre el patrón
	\node[align=center, text width=7cm, font=\scriptsize, below=0.8cm of f-1-4] at (\xstart + 2.5*\xstep, \ystart + 5.2*\ystep) {
		Patrón de recorrido diagonal de todas las fracciones $\frac{p}{q}$.
		Para $\mathbb{Q}^+$, se omiten las fracciones reducibles.
	};

\end{tikzpicture}

	\end{center}
	Se puede ver a simple vista que es biyectiva.
\end{proof}

En esencia, acá probamos que $\mathbb{N} \times \mathbb{N}$ es numerable. Es más, a continuación vamos a ver una notación muy útil.

\section{Coordinabilidad}

La idea de la notación de coordinabilidad es expresar que dos conjuntos tienen la misma cardinalidad.

\begin{definition}
	Si existe $f: A \to  B$ biyectiva, entonces se dice que $A$ y $B$ son \emph{coordinables} y lo denotamos como $A \sim B$.
\end{definition}

Notemos que por definición, todo conjunto numerable es coordinable con $\mathbb{N}$.

\begin{lemma}
	Todo subconjunto de $\mathbb{N}$ es contable.
\end{lemma}

\begin{proof}
	Si $A$ es finito, entonces ya estamos. Supongamos que $A$ es infinito. Defino la función $f: \mathbb{N} \to A$ tal que $f(1) = \min A$ y
	\begin{equation*}
		f(n+1) = \min (A \setminus \{ f(1), f(2), \dots, f(n) \}).
	\end{equation*}
	Como $A$ es infinito, $f$ está bien definida. Además, $f$ es claramente una biyección. Por lo tanto, $A$ es contable.
\end{proof}

Entonces, el conjunto de los números primos es numerable, así como $2 \mathbb{N}$.

\begin{proposition}
	Sea $X$ un conjunto \textit{numerable}. Si existe $f: X \twoheadrightarrow Y$ sobreyectiva, entonces $Y$ es contable.
\end{proposition}

\begin{proof}
	Si $X \neq \mathbb{N}$ simplemente consideramos una biyección de $\mathbb{N}$ a $X$. Entonces, podemos suponer que $X = \mathbb{N}$.

	Sea $f: X \twoheadrightarrow Y$ sobreyectiva. Definimos la función
	$$
		g: Y \to \mathbb{N} \text{ tal que } y \mapsto \min f^{-1}(y).
	$$
	La función $g$ está bien definida porque $f$ es sobreyectiva, lo que garantiza que $f^{-1}(y) \neq \emptyset$ para todo $y \in Y$. Como $g$ es inyectiva, la restricción $g|_{\operatorname{Im}g}$ es biyectiva. Dado que encontramos una biyección entre $Y$ y un subconjunto de $\mathbb{N}$, demostramos que $Y$ es contable.
\end{proof}

\begin{remark}
	Un conjunto $Y$ es contable si y sólo si existe $f : \mathbb{N} \twoheadrightarrow Y$ sobreyectiva.
\end{remark}

Ahora vamos a ver, probablemente, la proposición más útil de este capítulo.

\begin{proposition}
	La unión numerable de conjuntos numerables es numerable.
\end{proposition}

\begin{proof}
	Sea $\{ A_{n} \}_{n \in \mathbb{N}}$ una famlia de conjuntos numerables. Dado que cada $A_n$ es numerable, en particular es coordinable con $\mathbb{N} \times \{ n \}$. Por lo tanto,
	\begin{equation*}
		\bigcup_{n \in \mathbb{N}} A_n \sim \bigcup_{n \in \mathbb{N}} \mathbb{N} \times \{ n \} = \mathbb{N} \times \mathbb{N} \sim \mathbb{N}.
	\end{equation*}
	Por lo tanto, la unión numerable de conjuntos numerables es numerable.
\end{proof}

\begin{example}
	El conjunto de los polinomio con coeficientes racionales $\mathbb{Q}[x]$ es numerable.
\end{example}

\begin{proof}[Solución]
	Veamos que $\mathbb{Q}[x]$ es numerable. Recordemos que un polinomio es de la forma
	\begin{equation*}
		a_{n} x^n + a_{n-1} x^{n-1} + \dots + a_{1} x + a_0,
	\end{equation*}
	donde $a_0, a_1, \ldots, a_n \in \mathbb{Q}$ y $n \in \mathbb{N}_0$. Por lo tanto, podemos pensar un polinomio como
	\begin{equation*}
		(a_0, a_1, \ldots, a_n) \in \mathbb{Q}^{n+1}.
	\end{equation*}
	Entonces,
	\begin{equation*}
		\mathbb{Q}[x] \sim \bigcup_{n \in \mathbb{N}_0} \mathbb{Q}^{n+1}
	\end{equation*}
	y como $\mathbb{Q}^n$ es numerable para todo $n \in \mathbb{N}$, $\mathbb{Q}[x]$ es numerable.
\end{proof}

Notemos que omití algunos detalles en la demostración. Por ejemplo, la asociación que dí no es exactamente una biyección, ya que $a_n \neq 0$. Pero no es tan dramático el problema, simplemente con saber que $\mathbb{Q} \setminus \left\{ 0 \right\} \sim \mathbb{Q}$, entonces podemos encontrar la biyección.


\section{Conjuntos no numerables}

El primer conjunto no numerable que vamos a ver es $\mathbb{R}$.

\begin{proposition}
	El conjunto de los numeros reales no es numerable.
\end{proposition}

\begin{proof}[Solución]
	Para facilitarnos la vida, podemos considerar la biyección $g: \mathbb{R} \to (0, 1)$ dada por $g(x) = \frac{\tan^{-1} (x)}{\pi} + \frac{1}{2}$.
	\begin{center}
		\color{red} INSERTAR GRÁFICO
	\end{center}
	Esto da una biyección entre $\mathbb{R}$ y $(0, 1)$. Por lo tanto, basta con demostrar que no existe una biyección entre $\mathbb{N}$ y $(0, 1)$ para ver que $\mathbb{R}$ no es cordinable con $\mathbb{N}$.

	Supongamos que $\mathbb{N}$ es coordinable con $(0, 1)$. Sea $f: \mathbb{N} \to (0, 1)$ una biyección tal que
	\begin{align*}
		f(1) & = 0.\textcolor{accentcolor}{a_{11}} a_{12}a_{13}\dots \\
		f(2) & = 0.a_{21}\textcolor{accentcolor}{a_{22}}a_{23}\dots  \\
		f(3) & = 0.a_{31}a_{32}\color{accentcolor}{a_{33}}\dots      \\
		     & \dots
	\end{align*}
	Consideramos el número definido por agarrar los dígitos de la diagonal y sumarle $1$ si es menor que $9$ o restarle $1$ si es $9$. Nos queda el número
	$$
		0.a'_{11}a'_{22}a'_{33}\dots
	$$
	Este número es distinto a $f(n)$ en el $n$-ésimo dígito decimal. Lo cual es absurdo, ya que si $f$ es biyectiva debería existir un natural $m$ tal que $f(m) = 0.a'_{11}a'_{22}a'_{33}\dots$. Por lo tanto, demostramos que $\mathbb{N}$ no es coordinable con $\mathbb{R}$.
\end{proof}

Aclaro una pequeña notación. Cuando escribimos $Y^X$ nos referimos al conjunto de funciones de $X$ a $Y$:
\begin{equation*}
	Y^{X} = \left\{ f : X \to Y \right\}.
\end{equation*}

\begin{proposition}
	El conjunto partes $\mathcal{P}(X)$ es coordinable con $2^X$.
\end{proposition}

\begin{proof}
	Definimos la función $f : \mathcal{P}(X) \to \{0, 1\}^X$ como
	\begin{equation*}
		A \subseteq X \mapsto g_A,
	\end{equation*}
	donde $g_A : X \to \{ 0, 1 \}$ tal que
	\begin{equation*}
		g_A(x) = \begin{cases}
			1, & \text{ si }x \in A      \\
			0, & \text{ si } x \not\in A
		\end{cases}.
	\end{equation*}
	Esta función claramente es bitectiva.
\end{proof}

Ahora vamos a ver el Teorema de Cantor.

\begin{theorem}
	Ningún conjunto es coordinable con su conjunto de partes.
\end{theorem}

\begin{proof}
	Sea $X$ un conjunto. Queremos ver que no es coordinable con $\mathcal{P}(X)$. Supongamos que existe una biyección $f : X \to \mathcal{P}(X)$. Consideremos al siguiente subconjunto:
	\begin{equation*}
		A = \{ x \in X \mid x \not \in f(x) \}.
	\end{equation*}
	La idea de este subconjunto es forzar a que sea distinto a cada elemento de la imagen de $f$. Por lo tanto, para todo $x \in X$, $A \neq f(x)$ ya que $A$ siempre tiene un elemento que no pertenece a $f(x)$, en particular $x$. Acabamos de probar que $f$ no es sobreyectiva, por lo tanto no es biyectiva y entonces $X$ y $\mathcal{P}(X)$ no son coordinables.
\end{proof}


\section{Cardinales y Axioma de Elección}

En estas notas (y en la cursada) se acepta el axioma de elección y el principio de buena ordenación. El axioma de elección postula lo siguiente.

\begin{axiom}[Axioma de elección]
	Sea $\mathcal{F}$ una familia de conjuntos no vacíos y disjuntos dos a dos. Entonces, existe una función $f: \mathcal{F} \to \bigcup \mathcal{F}$ tal que $f(A) \in A$ para todo $A \in \mathcal{F}$.
\end{axiom}

Otras formulaciones equivalentes son:

\begin{enumerate}[label=(\roman*)]
	\item Sea $X$ un conjunto. Existe una función
	      $$f: \mathcal{P}(X) - \{ \emptyset \} \to X$$
	      tal que $f(A) \in A$ para todo $A \in \mathcal{P}(X)$.
	\item El producto cartesiano de conjuntos no vacíos es no\\ vacío.
\end{enumerate}

El axioma de elección también es equivalente al principio de buena ordenación.

\begin{axiom}[Principio de buena ordenación]
	Todo conjunto puede ser dotado de un orden tal que todo subconjunto no vacío tiene un elemento mínimo. A este tipo de orden se le llama \textit{orden bueno}.
\end{axiom}

Aceptando el axioma de elección (y en consecuencia el principio de buen ordenamiento) se puede demostrar el lema de Zorn.

\begin{axiom}[Lema de Zorn]
	Son equivalentes:
	\begin{enumerate}
		\item Sea $(A, \preceq)$ un conjunto parcialmente ordenado tal que toda cadena de $A$ tiene cota superior. Entonces, $A$ tiene un elemento maximal.
		\item Sea $(A, \preceq)$ un conjunto parcialmente ordenado tal que toda cadena de $A$ tiene supremo. Entonces, $A$ tiene un elemento maximal.
		\item Todo conjunto parcialmente ordenado tiene una cadena maximal.
	\end{enumerate}
\end{axiom}

Ahora sí, veamos un teorema que vamos a usar bastante.

\begin{theorem}
	Sean $f: X \hookrightarrow Y$ y $g: Y \hookrightarrow X$ funciones \textit{inyectivas}. Entonces, existe una biyección $h: X \to  Y$.
\end{theorem}

\begin{proof}
	La idea detrás de esta demostración es particionar a los conjuntos $X$ e $Y$ en $A$, $B$ y $A'$, $B'$, respectivamente, de forma tal que
	$$
		A \cap B = A' \cap B' = \emptyset,
	$$
	y
	$$
		A \cup B = X \quad \text{y} \quad A' \cup B' = Y,
	$$
	y además
	$$
		f(A) = A' \quad \text{y} \quad f(B) = B'.
	$$
	\begin{center}
		\begin{tikzpicture}[scale=0.65]
	% Sets X and Y
	\draw[thick] (0,0) ellipse (2 and 3);
	\draw[thick] (6,0) ellipse (2 and 3);

	% Labels for X and Y
	\node at (0,-3.5) {$X$};
	\node at (6,-3.5) {$Y$};

	% Subsets A and B in X
	\node at (0,1) {$A$};
	\node at (0,-1) {$B$};

	% Subsets A' and B' in Y
	\node at (6,1) {$A'$};
	\node at (6,-1) {$B'$};

	% Diagonal line between X and Y
	\draw (-2, 0) -- (2, 0);
	\draw (4, 0) -- (8, 0);

	\draw[-to, bend left] (0.25,1.25) to node[midway, above] {$f$} (5.75,1.25);
	\draw[-to, bend left] (5.75,-1.35) to node[midway, below] {$g$} (0.25,-1.35);
\end{tikzpicture}


	\end{center}

	Esto se reduce a encontrar $A \subseteq X$ tal que
	$$
		X - g(Y - f(A)) = A.
	$$
	Definimos $\Phi : \mathcal{P}(X) \to  \mathcal{P}(X)$ tal que $\Phi(A) = X - g(Y - f(A))$. Lo cual nos da la siguiente ecuación,
	$$
		\Phi (A) = A.
	$$
	Veamos que $\Phi$ es un morfismo de orden con en $(\mathcal{P}(X), \subseteq )$. Sean $X_{0}$ y $X_{1}$ subconjuntos de $X$. Supongamos que $X_{0} \subseteq X_{1}$. Entonces,
	\begin{align*}
		X_{0}               & \subseteq X_{1}               \\
		f(X_{0})            & \subseteq f(X_{1})            \\
		Y - f(X_{0})        & \supseteq Y - f(X_{1})        \\
		g(Y - f(X_{0}))     & \supseteq g(Y - f(X_{1}))     \\
		X - g(Y - f(X_{0})) & \subseteq X - g(Y - f(X_{1})) \\
		\Phi(X_{0})         & \subseteq \Phi(X_{1})
	\end{align*}
	probando que $\Phi$ es un morfismo de orden. Y como ya sabemos, $(\mathcal{P}(X), \subseteq)$ es un reticulado completo. Por lo tanto, podemos utilizar el teorema del punto fijo, dándonos un $A \subseteq X$ tal que $\Phi (A) = A$.
\end{proof}

Cuando trabajamos con cardinales de conjuntos, para probar que dos conjuntos tienen el mismo cardinal es necesario establecer una biyección o inyecciones hacia ambos lados. Sin embargo, encontrar la función en particular puede ser molesto a veces. Para esto sirve la aritmética de cardinales. Nos ahorra el trabajo de encontrar una función en particular y nos permite trabajar con cardinales generales.

\begin{definition}
	Decimos que dos conjuntos $A$ y $B$ tienen el mismo \emph{cardinal} si existe una función biyectiva $f: A \to  B$ y lo denotamos como $\lvert A \rvert = \lvert B \rvert$.
\end{definition}

Por ejemplo, si $A$ es un conjunto numerable, entonces
$$
	\lvert A \rvert = \lvert \mathbb{N} \rvert.
$$
En general, como hay algunos cardinales que son más ocurrentes, los denotamos de una forma especial.

\begin{definition}
	Denotamos
	\begin{itemize}
		\item El cardinal de $\mathbb{N}$ como $\aleph_0$.
		\item El cardinal de $\mathbb{R}$ como $\mathfrak{c}$.
	\end{itemize}
\end{definition}

\begin{remark}
	Si $A$ es numerable, entonces su cardinal es $\aleph_0$.
\end{remark}

\begin{definition}
	Sean $A$ y $B$ conjuntos. Si existe una función inyectiva $f: A \hookrightarrow B$, entonces $\lvert A \rvert \leq \lvert B \rvert$.
\end{definition}

Si bien tratamos con funciones inyectivas, podríamos haber utilizado funciones sobreyectivas para las definiciones; ya que, si existe $f: A \hookrightarrow B$ inyectiva y $A \neq \emptyset $, entonces existe $g: B \twoheadrightarrow A$ sobreyectiva.

A continuación veremos una suerte de tricotomía pero para los cardinales. Para la demostración de este teorema, es necesario el lema de Zorn.

\begin{theorem}
	Sean $X$ e $Y$ conjuntos no vacíos. Existe $f: X \hookrightarrow Y$ inyectiva o existe $g: Y \hookrightarrow X$ inyectiva.
\end{theorem}

\begin{proof}
	Consideramos el conjunto ordenado
	$$
		\mathcal{A} = \{ (A, f_{A}) \mid f_{A}: A \subseteq X \hookrightarrow Y  \text{ es inyectiva}\}
	$$
	con el orden $(A, f_{A}) \preceq (B, f_{B})$ si $A \subseteq B$ y $f_{B}|_{A} = f_{A}$. Queremos utilizar el lema de Zorn, para eso necesitamos probar que toda cadena de $\mathcal{A}$ tiene cota superior.

	Sea $\mathcal{C} = \{ (A_{i}, f_{A_{i}})\}_{i \in I}$ una cadena de $\mathcal{A}$. Consideremos el par $(A, f_{A})$ donde
	$$
		A = \bigcup_{i \in I} A_{i}
	$$
	y
	$$
		f_{A} : A \to Y \text{ donde } f_{A}(a) = f_{A_{i}}(a) \text{ si } a \in A_{i}.
	$$
	Es evidente que $(A_{i}, f_{A_{i}}) \preceq (A, f_{A})$ para todo $i \in I$.

	Probemos que $f_{A}$ está bien definida ya que, para cualesquiera $A_{i}, A_{j}$ de $\mathcal{C}$ tales que $A_{i} \subseteq A_{j}$ (sin pérdida de generalidad), $f_{A_{j}|_{A_{i}} = f_{A_{i}}}$.

	Veamos que $A \in \mathcal{A}$. Claramente $A \subseteq X$. Probemos que $f_{A}$ es inyectiva. Sean $a, a' \in A$ tales que $f_{A}(a) = f_{A}(a')$. Entonces, existe un $i \in I$ tal que $a, a' \in A_{i}$, ya que $\mathcal{C}$ es una cadena. Por lo tanto, obtenemos
	$$
		f_{A_{i}}(a) =f_{A_{i}}(a'),
	$$
	y por inyectividad de $f_{A_{i}}$, $a = a'$. Lo cual prueba que $f_{A}$ es inyectiva. Entonces, como $A \in \mathcal{A}$, tenemos que $\mathcal{C}$ está acotado superiormente.

	Como toda cadena de $\mathcal{A}$ está acotada superiormente, por el lema de Zorn, existe un elemento maximal. Sea $(B, f_{B})$ un elemento maximal. Necesariamente, $\operatorname{dom} f_{B} = X$ o
	$\operatorname{Im} Y$, del caso contrario $(B, f_{B})$ no sería maximal. Si $\operatorname{dom} f_{B} = X$, ya conseguimos nuestra función inyectiva. Si $\operatorname{Im} Y$, entonces definimos $f: Y \hookrightarrow X$ donde $f(y) = f_{B}^{-1}(y)$.
\end{proof}

\begin{remark}
	Esto es equivalente a decir que, para dos conjuntos $X$ e $Y$ cualesquiera, $\lvert X \rvert \leq \lvert Y \rvert$ o $\lvert X \rvert \geq \lvert Y \rvert$.
\end{remark}

\begin{proposition}
	Sea $A$ un conjunto \textit{infinito}. Entonces, existe una partición de $A$ tal que todas las partes son numerables. Es decir,
	$$
		A = \bigsqcup_{i \in I} A_i,
	$$
	donde $A_i$ es numerable para todo $i \in I$.
\end{proposition}

\begin{proof}
	{\color{red} La demostración de esta proposición no la incluyo.}
\end{proof}

Volviendo a los cardinales.

\begin{definition}
	Sean $A$ y $B$ conjuntos disjuntos (por praciticidad). Si $a = \lvert A \rvert$ y $b = \lvert B \rvert$, entonces
	\begin{itemize}
		\item $a + b = |A \cup B|$.
		\item $a \cdot b = |A \times B|$.
		\item $a^{b} = |A^{B}|$.
	\end{itemize}
\end{definition}

Podemos operar como lo esperaríamos.

\begin{proposition}
	Sean $a, b, c$ cardinales. Entonces:
	\begin{itemize}
		\item $a + b = b + a$.
		\item $a \cdot b = b \cdot a$.
		\item $(a + b) + c = a + (b + c)$.
		\item $(a \cdot b) \cdot c = a \cdot (b \cdot c)$.
		\item $a \cdot (b + c) = a \cdot b + a \cdot c$.
		\item $a^{b+c} = a^{b} \cdot a^{c}$.
		\item $a^{b^{c}} = a^{b \cdot c}$.
	\end{itemize}
\end{proposition}

\begin{proof}
	Las propiedades se deducen directamente de las definiciones de suma y producto de cardinales, utilizando las propiedades correspondientes de las uniones disjuntas y los productos cartesianos de conjuntos.
\end{proof}


\part{Espacios métricos y topología}

\chapter{Nociones básicas}

Va haber casi infinitas definiciones, así que preparate.

\section{Definición y ejemplos}

\begin{definition}
	Un \emph{espacio métrico} es un par $(X, d)$, donde $X$ es un conjunto y $d: X \times X \to \mathbb{R}_{\geq 0}$ una función llamada \emph{distancia} (o \emph{métrica}), que satisface las siguientes propiedades para todo $x, y, z \in X$:
	\begin{enumerate}
		\item $d(x, y) = 0$ si y sólo si $x = y$.
		\item $d(x, y) = d(y, x)$.
		\item $d(x, z) \leq  d(x, y) + d(y, z)$.
	\end{enumerate}
\end{definition}

\begin{remark}
	Como se cumple la desigualdad triangular, también se cumple
	\begin{itemize}
		\item $d(x_{1}, x_{n}) \leq d(x_{1}, x_{2}) + d(x_{2}, x_{3}) + \dots + d(x_{n-1}, x_{n})$.
		\item $\lvert d(x, z) - d(y, z) \rvert \leq d(x, y)$.
	\end{itemize}
\end{remark}

También tenemos la noción de un subespacio métrico.

\begin{definition}
	Sea $(X, d)$ un espacio métrico y $A \subseteq X$. Se dice que $A$ es un \emph{subespacio métrico} de $X$ si $(A, d_{A \times A})$ es un espacio métrico.
\end{definition}

Veamos algunos ejemplos de espacios métricos.

\begin{example}
	\begin{enumerate}
		\item El espacio $(\mathbb{R}^{n}, \lVert \cdot \rVert_2)$ es un espacio métrico con la norma euclídea, donde la métrica inducida es $d(x, y) = \lVert x - y \rVert_2 = \sqrt{\sum_{i=1}^n (x_i - y_i)^2}$.

		\item Para cualquier conjunto $X$, la métrica discreta está definida por $$d(x, y) = \delta_{xy} = \begin{cases} 0 & \text{si } x = y \\ 1 & \text{si } x \neq y \end{cases}.$$

		\item Si $(X, d)$ es un espacio métrico, entonces
		      \begin{equation*}
			      d'(x, y) = \min(d(x, y), 1)
		      \end{equation*}
		      también es una métrica en $X$, llamada métrica acotada equivalente. Esta métrica garantiza que la distancia entre dos puntos nunca excede $1$.

		\item Si $(V, \langle \cdot, \cdot \rangle)$ es un espacio vectorial con producto interno (sobre $\mathbb{R}$ o $\mathbb{C}$), entonces $\|x\| = \sqrt{\langle x, x \rangle}$ es una norma, y $d(x, y) = \|x - y\|$ es una métrica inducida por la norma. Un ejemplo clásico es el espacio de funciones de cuadrado integrable $L^2([a, b])$ con el producto interno $\langle f, g \rangle = \int_a^b f(x)\overline{g(x)} dx$.

		\item El espacio $C([a, b], \mathbb{R})$ de funciones continuas $f: [a, b] \to \mathbb{R}$. Se pueden definir varias normas en este espacio:
		      \begin{itemize}
			      \item La norma del supremo (o norma $L^\infty$): $\|f\|_\infty = \sup_{x \in [a, b]} |f(x)|$. La métrica inducida es $d(f, g) = \sup_{x \in [a, b]} |f(x) - g(x)|$.
			      \item La norma $L^1$: $\|f\|_1 = \int_a^b |f(x)| dx$. La métrica inducida es $d(f, g) = \int_a^b |f(x) - g(x)| dx$.
			      \item La norma $L^2$: $\|f\|_2 = \left( \int_a^b |f(x)|^2 dx \right)^{1/2}$. La métrica inducida es $d(f, g) = \left( \int_a^b |f(x) - g(x)|^2 dx \right)^{1/2}$.
		      \end{itemize}
		\item El espacio de secuencias $\ell^p$ para $1 \le p < \infty$: $\ell^p = \{ (x_n)_{n=1}^\infty \subset \mathbb{R} : \sum_{n=1}^\infty |x_n|^p < \infty \}$. La norma está definida por $\|x\|_p = \left( \sum_{n=1}^\infty |x_n|^p \right)^{1/p}$. La métrica es $d(x, y) = \|x - y\|_p$.
		\item El espacio de secuencias $\ell^\infty$: $\ell^\infty = \{ (x_n)_{n=1}^\infty \subset \mathbb{R} : \sup_{n \in \mathbb{N}} |x_n| < \infty \}$. La norma es $\|x\|_\infty = \sup_{n \in \mathbb{N}} |x_n|$. La métrica es $d(x, y) = \|x - y\|_\infty$.
	\end{enumerate}
\end{example}

De los ejemplos que di, los más importantes de saber son el espacio de funciones continuas con la norma del supremo y el espacio de secuencias con la norma infinito.


\section{Bolas, abiertos y cerrados}

Vamos a definir los conjuntos básicos que luego vamos a utilizar con mucha frecuencia. Vamos a denotar a un espacio métrico arbitrario con $(X, d)$.

\begin{definition}
	Definimos
	$$
		B(x, r) = \{ y \in X \mid d(x, y) < r\}
	$$
	como la \emph{bola abierta} centrada en $x$ con radio $r$. Análogamente, a
	$$
		\overline{B}(x, r) = \{ y \in X \mid d(x, y) \leq  r\}
	$$
	la \emph{bola cerrada} centrada en $x$ con radio $r$.
\end{definition}

Voy a ir agregando diagramas que ilustren más o menos la idea.

\begin{center}
	\begin{tikzpicture}
	% Dibujar la bola abierta
	\draw[accentcolor, thick, dashed] (0,0) circle (1.5cm);
	\fill[accentcolor!10, opacity=0.7] (0,0) circle (1.5cm);
	\node at (0,0) [fill=black, circle, inner sep=1.5pt, label=below:$x$] {};
	\draw[] (0,0) -- node[above, black] {$r$} (1.5,0);
	\node[above, accentcolor] at (0, 1.8) {Bola Abierta $B(x, r)$};

	% Dibujar la bola cerrada
	\begin{scope}[xshift=4cm]
		\draw[accentcolor, thick] (0,0) circle (1.5cm);
		\fill[accentcolor!10, opacity=0.7] (0,0) circle (1.5cm);
		\node at (0,0) [fill=black, circle, inner sep=1.5pt, label=below:$x$] {};
		\draw (0,0) -- node[above, black] {$r$} (1.5,0); % Línea continua para radio
		\node[above, accentcolor] at (0, 1.8) {Bola Cerrada $\overline{B}(x, r)$};
	\end{scope}
\end{tikzpicture}

\end{center}

¡Ojo! Si bien le decimos ``bola'', el conjunto puede tener una forma totalmente distinta. Por ejemplo, en $\mathbb{R}^2$, pensá qué pasa con las métricas $d_1$, $d_2$ y $d^{\infty}$.

\begin{example}
	Consideremos $C([a, b], \mathbb{R})$ con la norma del supremo. ¿Cómo se ve la bola $B(0, 1)$?

	\begin{center}
		\begin{tikzpicture}[x=0.75pt,y=0.75pt,yscale=-1,xscale=1]
	%uncomment if require: \path (0,300); %set diagram left start at 0, and has height of 300

	%Shape: Axis 2D [id:dp0672040940975357] 
	\draw  (0,76.5) -- (180,76.5)(28.8,0) -- (28.8,150) (173,71.5) -- (180,76.5) -- (173,81.5) (23.8,7) -- (28.8,0) -- (33.8,7)  ;
	%Straight Lines [id:da19847713850781956] 
	\draw [color={rgb, 255:red, 0; green, 86; blue, 145 }  ,draw opacity=1 ][fill={rgb, 255:red, 0; green, 86; blue, 145 }  ,fill opacity=1 ] [dash pattern={on 4.5pt off 4.5pt}]  (0,120) -- (150,120) ;
	%Straight Lines [id:da26723242026257843] 
	\draw [color={rgb, 255:red, 0; green, 86; blue, 145 }  ,draw opacity=1 ][fill={rgb, 255:red, 0; green, 86; blue, 145 }  ,fill opacity=1 ] [dash pattern={on 4.5pt off 4.5pt}]  (0,30) -- (150,30) ;
	%Shape: Rectangle [id:dp7430552976593023] 
	\draw  [draw opacity=0][fill={rgb, 255:red, 0; green, 86; blue, 145 }  ,fill opacity=0.1 ] (0,30) -- (150,30) -- (150,120) -- (0,120) -- cycle ;
	%Curve Lines [id:da49050279449110856] 
	\draw [color={rgb, 255:red, 86; green, 0; blue, 145 }  ,draw opacity=1 ]   (0.23,51.95) .. controls (20.23,36.95) and (23.25,80.01) .. (38,89.21) .. controls (52.75,98.41) and (69.09,66.23) .. (92,59.21) .. controls (114.91,52.18) and (140.23,64.45) .. (150.23,56.95) ;

	% Text Node
	\draw (170,84) node [anchor=north west][inner sep=0.75pt]    {$x$};
	% Text Node
	\draw (5,1) node [anchor=north west][inner sep=0.75pt]    {$y$};
	% Text Node
	\draw (5,22) node [anchor=north west][inner sep=0.75pt]    {$1$};
	% Text Node
	\draw (69,55) node [anchor=north west][inner sep=0.75pt]  [font=\footnotesize,color={rgb, 255:red, 86; green, 0; blue, 145 }  ,opacity=1 ]  {$f$};
	% Text Node
	\draw (146,84) node [anchor=north west][inner sep=0.75pt]    {$1$};


\end{tikzpicture}

	\end{center}

	Como podemos ver, la función $f \in B(0, 1)$ ya que siempre $-1 \leq f(x) \leq 1$, para todo $x \in [0, 1]$. En general, la bola $B(0, 1)$ es el conjunto de funciones que están siempre en esa franja.
\end{example}

A continuación definimos lo que es un entorno de un punto.

\begin{definition}
	Un \emph{entorno} de $x \in X$ es un subconjunto $V \subseteq X$ tal que $x \in V$ y existe una bola $B(x, r) \subseteq V$.
\end{definition}

\begin{center}
	\begin{tikzpicture}
	% The main outer shape (U)
	% Corrected the option syntax: draw/fill options separated from plot options
	\draw[draw=complementarycolor, fill=complementarycolor!10, opacity=0.7] % General draw/fill options
	plot [smooth cycle, tension=0.6]                   % Plot-specific options
	coordinates {
			(1, 1)
			(-2, 1)
			(-1.5, -1)
			(2, -2)
		};
	% Added label for the outer neighborhood set V
	\node[complementarycolor] at (-1.5, 0.5) {$V$};

	% The neighborhood circle (dashed outline and fill)
	\draw[accentcolor, thick, dashed] (0,0) circle (0.5);
	\fill[accentcolor!10, opacity=0.7] (0,0) circle (0.5);

	% The center point x and radius r
	\node at (0,0) [fill=black, circle, inner sep=1.5pt, label=below:$x$] {};
	\draw[] (0,0) -- node[above, black] {$r$} (0.5,0);

	\node[complementarycolor] at (-0.3, 1.75) {Entorno de $x$};
\end{tikzpicture}

























\end{center}

\begin{remark}
	Notemos que $B(x, r)$ siempre es un entorno de $x$.
\end{remark}

\begin{definition}
	Sea $A \subseteq X$. Definimos
	\begin{itemize}
		\item El \emph{interior} de $A$ como
		      $$
			      A^{\circ} = \{ x \in X \mid \exists r > 0 \text{ tal que } B(x, r) \subseteq A\}.
		      $$
		\item La \emph{clausura} de $A$ como
		      $$
			      \overline{A} = \{ x \in X \mid \forall r > 0, B(x, r) \cap A \neq \emptyset\}.
		      $$
		\item La \emph{frontera} de $A$ como
		      $$
			      \partial A = \overline{A} - A^{\circ}.
		      $$
		\item El \emph{exterior} de $A$ como
		      $$
			      \mathrm{ext} A = (X - A)^{\circ}.
		      $$
	\end{itemize}
\end{definition}

\begin{center}
	\begin{tikzpicture}[x=0.75pt,y=0.75pt,yscale=-1,xscale=1]
	%uncomment if require: \path (0,300); %set diagram left start at 0, and has height of 300

	%Shape: Polygon Curved [id:ds678130924119981] 
	\draw  [color={rgb, 255:red, 0; green, 86; blue, 145 }  ,draw opacity=1 ][fill={rgb, 255:red, 0; green, 86; blue, 145 }  ,fill opacity=0.1 ][dash pattern={on 4.5pt off 4.5pt}][line width=0.75]  (49.56,38.17) .. controls (64.04,31.89) and (132.48,30.63) .. (118,43.2) .. controls (103.53,55.77) and (92.59,64.15) .. (104.25,75.88) .. controls (115.91,87.61) and (102.23,122.38) .. (81.41,119.87) .. controls (60.58,117.36) and (37.36,96.23) .. (31.71,88.87) .. controls (26.05,81.5) and (35.08,44.45) .. (49.56,38.17) -- cycle ;
	%Shape: Polygon Curved [id:ds19735692429512375] 
	\draw  [color={rgb, 255:red, 0; green, 86; blue, 145 }  ,draw opacity=1 ][fill={rgb, 255:red, 0; green, 86; blue, 145 }  ,fill opacity=0.1 ][line width=0.75]  (199.56,38.17) .. controls (214.04,31.89) and (282.48,30.63) .. (268,43.2) .. controls (253.53,55.77) and (242.59,64.15) .. (254.25,75.88) .. controls (265.91,87.61) and (252.23,122.38) .. (231.41,119.87) .. controls (210.58,117.36) and (187.36,96.23) .. (181.71,88.87) .. controls (176.05,81.5) and (185.08,44.45) .. (199.56,38.17) -- cycle ;
	%Shape: Polygon Curved [id:ds4025351553833618] 
	\draw  [color={rgb, 255:red, 0; green, 86; blue, 145 }  ,draw opacity=1 ][line width=0.75]  (49.56,184.63) .. controls (64.04,178.34) and (132.48,177.09) .. (118,189.66) .. controls (103.53,202.23) and (92.59,210.6) .. (104.25,222.34) .. controls (115.91,234.07) and (102.23,268.84) .. (81.41,266.33) .. controls (60.58,263.82) and (37.36,242.69) .. (31.71,235.32) .. controls (26.05,227.96) and (35.08,190.91) .. (49.56,184.63) -- cycle ;
	%Shape: Path Data [id:dp28607428735171037] 
	\draw  [draw opacity=0][fill={rgb, 255:red, 0; green, 86; blue, 145 }  ,fill opacity=0.1 ][dash pattern={on 4.5pt off 4.5pt}] (300,150) -- (300,300) -- (150,300) -- (150,150) -- (300,150) -- cycle (199.56,188.17) .. controls (185.08,194.45) and (176.05,231.5) .. (181.71,238.87) .. controls (187.36,246.23) and (210.58,267.36) .. (231.41,269.87) .. controls (252.23,272.38) and (265.91,237.61) .. (254.25,225.88) .. controls (242.59,214.15) and (253.53,205.77) .. (268,193.2) .. controls (282.48,180.63) and (214.04,181.89) .. (199.56,188.17) -- cycle ;
	%Shape: Square [id:dp6423377198062709] 
	\draw  [color={rgb, 255:red, 0; green, 86; blue, 145 }  ,draw opacity=1 ] (150,150) -- (300,150) -- (300,300) -- (150,300) -- cycle ;
	%Shape: Polygon Curved [id:ds7262645597364767] 
	\draw  [color={rgb, 255:red, 0; green, 86; blue, 145 }  ,draw opacity=1 ][dash pattern={on 4.5pt off 4.5pt}][line width=0.75]  (199.56,188.17) .. controls (214.04,181.89) and (282.48,180.63) .. (268,193.2) .. controls (253.53,205.77) and (242.59,214.15) .. (254.25,225.88) .. controls (265.91,237.61) and (252.23,272.38) .. (231.41,269.87) .. controls (210.58,267.36) and (187.36,246.23) .. (181.71,238.87) .. controls (176.05,231.5) and (185.08,194.45) .. (199.56,188.17) -- cycle ;

	% Text Node
	\draw (44.33,47.33) node [anchor=north west][inner sep=0.75pt]  [font=\footnotesize,color={rgb, 255:red, 0; green, 86; blue, 145 }  ,opacity=1 ]  {$A^{\circ }$};
	% Text Node
	\draw (194.33,47.33) node [anchor=north west][inner sep=0.75pt]  [font=\footnotesize,color={rgb, 255:red, 0; green, 86; blue, 145 }  ,opacity=1 ]  {$\overline{A}$};
	% Text Node
	\draw (44.33,193.79) node [anchor=north west][inner sep=0.75pt]  [font=\footnotesize,color={rgb, 255:red, 0; green, 86; blue, 145 }  ,opacity=1 ]  {$\partial A$};
	% Text Node
	\draw (194.33,197.33) node [anchor=north west][inner sep=0.75pt]  [font=\footnotesize,color={rgb, 255:red, 0; green, 86; blue, 145 }  ,opacity=1 ]  {$\mathrm{ext} A$};


\end{tikzpicture}

\end{center}

Establecidos los conceptos de interior y clausura, podemos definir los conjuntos abiertos y cerrados.

\begin{definition}
	Sea $A \subseteq X$. Decimos que
	\begin{itemize}
		\item $A$ es \emph{abierto} si $A^{\circ} = A$.
		\item $A$ es \emph{cerrado} si $\overline{A} = A$.
	\end{itemize}
\end{definition}

\begin{remark}
	Hay que tener cuidado con $\overline{B}(x, r)$ y $\overline{B(x, r)}$, ya que no necesariamente son iguales. Lo que sí vale es que $\overline{B(x, r)} \subseteq \overline{B}(x, r)$
\end{remark}

\begin{proposition}
	Sea $A \subseteq X$. Entonces,
	\begin{itemize}
		\item $A^\circ \subseteq A \subseteq \overline{A}$.
		\item $(A^\circ)^\circ = A^\circ$, $\overline{\overline{A}} = \overline{A}$.
		\item $\overline{A \cup B} = \overline{A} \cup \overline{B}$, $(A \cap B)^\circ = A^\circ \cap B^\circ$.
	\end{itemize}
\end{proposition}

\begin{proof}
	No voy a demostrarlo, pero pensá por qué tiene sentido esto. (Mira los diagramas de arriba.)
\end{proof}

La siguiente proposición es extremadamente útil.

\begin{proposition}
	Sea $\{ A_i \}_{i \in I}$ una familia de abiertos. Entonces,
	\begin{enumerate}
		\item La unión $\bigcup_{i \in I} A_i$ es abierta.
		\item Si $I$ es finito, entonces la unión $\bigcap_{i \in I} A_i$ es abierta.
	\end{enumerate}
\end{proposition}

\begin{proof}
	\framebox{1.} Sea $U = \bigcup_{i \in I} A_i$. Si $x \in U$, entonces existe un $i \in I$ tal que $x \in A_i$. Por lo tanto, existe una bola $B(x, r)$ que cumple que $B(x, r) \subseteq A_i \subseteq U$. Por lo tanto, $U \subseteq U^{\circ}$. Entonces, $U$ es abierto.

	\framebox{2.} Sea $I$ finito y $V = \bigcap_{i \in I} A_i$. Si $x \in V$, entonces $x \in A_i$ para todo $i \in I$. Por lo tanto, existen bolas $B(x, r_i) \subseteq A_i$ para todo $i \in I$. Definimos $r = \min_{i \in I} r_i$, entonces $B(x, r) \subseteq \bigcap_{i \in I} A_i = V$. Por lo tanto, $V \subseteq V^{\circ}$. Entonces, $V$ es abierto.
\end{proof}

La proposición análoga para cerrados vale intercambiando las intersecciones por uniones y \textit{viceversa}.

\begin{proposition}
	Sea $\{ A_i \}_{i \in I}$ una familia de cerrados. Entonces,
	\begin{enumerate}
		\item La intersección $\bigcap_{i \in I} A_i$ es cerrado.
		\item Si $I$ es finito, entonces la intersección $\bigcup_{i \in I} A_i$ es cerrado.
	\end{enumerate}
\end{proposition}

\begin{proof}
	\framebox{1.} Sea $C = \bigcap_{i \in I} A_i$. Queremos demostrar que $C$ es cerrado. Equivalentemente, demostraremos que su complemento $C^c$ es abierto. Tenemos que $C^c = (\bigcap_{i \in I} A_i)^c = \bigcup_{i \in I} A_i^c$. Como cada $A_i$ es cerrado, su complemento $A_i^c$ es abierto. Por la Proposición anterior (parte 1), la unión de una familia arbitraria de abiertos es abierta. Por lo tanto, $C^c = \bigcup_{i \in I} A_i^c$ es abierto. En consecuencia, $C = \bigcap_{i \in I} A_i$ es cerrado.

	\framebox{2.} Sea $I$ finito y $D = \bigcup_{i \in I} A_i$. Queremos demostrar que $D$ es cerrado. Equivalentemente, demostraremos que su complemento $D^c$ es abierto. Tenemos que $D^c = (\bigcup_{i \in I} A_i)^c = \bigcap_{i \in I} A_i^c$. Como cada $A_i$ es cerrado, su complemento $A_i^c$ es abierto. Por la Proposición anterior (parte 2), la intersección de una familia finita de abiertos es abierta. Por lo tanto, $D^c = \bigcap_{i \in I} A_i^c$ es abierto. En consecuencia, $D = \bigcup_{i \in I} A_i$ es cerrado.
\end{proof}


\section{Densos}

\begin{definition}
	Sea $A \subseteq X$. Decimos que $A$ es \emph{denso} (en $X$) si $\overline{A} = X$.
\end{definition}

\begin{remark}
	Otra formulación equivalente es que para todo $U \subseteq X$ abierto no vacío se cumple que $A \cap U \neq \varnothing$.
\end{remark}

El ejemplo típico de subconjunto denso es $\mathbb{Q}$ en $\mathbb{R}$. (Agarramos el punto medio entre dos puntos de $\mathbb{R}$ y lo truncamos en algún momento para que esté en $\mathbb{Q}$.)

\begin{definition}
	Sea $A \subseteq X$. Se dice que $x \in X$ es un \emph{punto de acumulación} de $A$ si
	\begin{center}
		para todo $\varepsilon > 0$, $\left| B(x, \varepsilon) \cap A \right| \geq \aleph_0$.
	\end{center}
	El conjunto
	$$
		A' = \{ x \in X \mid x \text{ es un \emph{punto de acumulaci\'on} de } A \}
	$$
	se llama el \emph{conjunto derivado} de $A$.
\end{definition}

\begin{center}
	\begin{tikzpicture}
	% Define the start and end points for the line
	\coordinate (start) at (0,0);
	\coordinate (end) at (6,0); % Adjust 6 for desired length of the line

	% Mark 0
	\node[below] at (start) {$0$};
	\draw (start) -- +(0,2.5mm); % Tick mark for 0



	\node[below] at (1/2,0) {$\dots$};
	\draw (1/2,0) -- +(0,2.5mm); % Tick mark for 1/3


	\node[below] at (6/6,0) {$\frac{1}{6}$};
	\draw (6/6,0) -- +(0,2.5mm); % Tick mark for 1/3

	% % Mark 1/3
	% Calculate position: 1/3 of the total length (6 units) is 2 units
	\node[below] at (6/5,0) {$\frac{1}{5}$};
	\draw (6/5,0) -- +(0,2.5mm); % Tick mark for 1/3

	% Mark 1/3
	% Calculate position: 1/3 of the total length (6 units) is 2 units
	\node[below] at (6/4,0) {$\frac{1}{4}$};
	\draw (6/4,0) -- +(0,2.5mm); % Tick mark for 1/3


	% Mark 1/3
	% Calculate position: 1/3 of the total length (6 units) is 2 units
	\node[below] at (6/3,0) {$\frac{1}{3}$};
	\draw (6/3,0) -- +(0,2.5mm); % Tick mark for 1/3

	% Mark 1/2
	% Calculate position: 1/2 of the total length (6 units) is 3 units
	\node[below] at (6/2,0) {$\frac{1}{2}$};
	\draw (6/2,0) -- +(0,2.5mm); % Tick mark for 1/2

	% Mark 1
	\node[below] at (end) {$1$};
	\draw (end) -- +(0,2.5mm); % Tick mark for 1
\end{tikzpicture}

\end{center}

\begin{remark}
	Notemos que $\overline{A} = A \cup A'$.
\end{remark}


\section{Distancia a conjuntos}

\begin{definition}
	Sea $x \in X$ y $A \subseteq X$. Definimos la \emph{distancia} entre $x$ y $A$ como
	$$
		d(x, A) = \inf_{a \in A} d(x, a).
	$$
\end{definition}

\begin{remark}
	Se cumple la desigualdad triangular con conjuntos. O sea,
	$$
		d(x, A) \leq d(x, y) + d(y, A).
	$$
\end{remark}

\begin{definition}
	Sean $A, B \subseteq X$ no vacíos. Definimos la \emph{distancia} entre $A$ y $B$ como
	$$
		d(A, B) = \inf_{a \in A, b \in B} d(a, b).
	$$
\end{definition}

Y una última definición.

\begin{definition}
	Sea $A \subseteq X$. Definimos el \emph{diámetro} de $A$ como
	$$
		\diam(A) = \sup_{a, b \in A} d(a, b).
	$$
\end{definition}

A menudo nos va a servir la siguiente proposición.

\begin{proposition}
	Sea $A \subseteq X$ y $x \in X$. Entonces,
	\begin{equation*}
		d(x, A) = 0 \quad \text{ si y sólo si } \quad x \in \overline{A}.
	\end{equation*}
\end{proposition}

\begin{proof}
	($\Rightarrow$) Supongamos que $d(x, A) = 0$. Consideramos una bola $B(x, r)$ con $r > 0$. Como $d(x, A) = 0$, existe un $a \in A$ tal que $d(x, a) < r$. Por lo tanto, $B(x, r) \cap A \neq \varnothing$. Por lo tanto, $x \in \overline{A}$.

	($\Leftarrow$) Supongamos que $x \in \overline{A}$. Entonces, para toda bola $B(x, \frac{1}{n})$ con $n \in \mathbb{N}$, existe $a_n \in A$ tal que $d(x, a_n) < \frac{1}{n}$. Por lo tanto, consideramos a $d(a_n, x) \longrightarrow 0$. Entonces, $d(x, A) = 0$.
\end{proof}




\chapter{Sucesiones y convergencia}

Seguimos considerando a $(X, d)$ un espacio métrico arbitrario.

\section{Convergencia de sucesiones}

\begin{definition}
	Sea $(x_n)_{n \in \mathbb{N}}$ una sucesión en $X$. Decimos que $\lim_{n \to \infty} x_{n} = x$ (o $x_{n} \xrightarrow[n \to \infty]{} x$) si
	\begin{center}
		\begin{minipage}{0.9\linewidth}
			para todo $\varepsilon > 0$, existe $N \in \mathbb{N}$ tal que $n \geq N$ implica $d(x_n, x) < \varepsilon$.
		\end{minipage}
	\end{center}
\end{definition}

\begin{remark}
	En general, no es necesario demostrar exactamente que $d(x_n, x) < \varepsilon$. Podemos reemplazar $\varepsilon$ por cualquier expresión que ``genere'' a los reales positivos. Por ejemplo, si probamos que $d(x_n, x) \leq \frac{\varepsilon}{2}$, prácticamente estamos demostrando que $d(x_n, x) < \varepsilon$.
\end{remark}

\begin{proposition}
	Sea $(x_n)_{n \in \mathbb{N}}$ una sucesión en $X$. Si
	$$
		\lim_{n \to \infty} x_n = x \quad\text{y}\quad \lim_{n \to \infty} x_n = y,
	$$
	entonces $x = y$.
\end{proposition}


\begin{proof}
	Sea $\varepsilon > 0$. Entonces, existe $N \in \mathbb{N}$ tal que
	$$
		d(x_n, x) < \frac{\varepsilon}{2} \quad \text{y} \quad d(x_n, y) < \frac{\varepsilon}{2},
	$$
	para todo $n \geq N$. Por lo tanto,
	\begin{align*}
		0 \leq  d(x, y) \leq d(x, x_n) + d(x_n, y) < \varepsilon.
	\end{align*}
	Entonces, $d(x, y) = 0$ por lo tanto $x = y$.
\end{proof}

Además, valen todas las proposiciones que vimos en Taller de Cálculo Avanzado.

\section{Sucesiones de Cauchy}

\begin{definition}
	Una sucesión $(x_n)_{n \in \mathbb{N}}$ en $X$ es una \emph{sucesión de Cauchy} si
	\begin{center}
		\begin{minipage}{0.9\linewidth}
			para todo $\varepsilon > 0$, existe $N \in \mathbb{N}$ tal que $n, m \geq N$ implica $d(x_n, x_m) < \varepsilon$.
		\end{minipage}
	\end{center}
\end{definition}

\begin{remark}
	Una sucesión de Cauchy no necesariamente tiene que converger. Por ejemplo, la sucesión $(\frac{1}{n})_{n \in \mathbb{N}}$ no converge en $\mathbb{Q} \setminus \{ 0 \}$.
\end{remark}

\begin{proposition}
	Sea $(x_n)_{n \in \mathbb{N}}$ una sucesión en $X$. Si $(x_n)$ converge, entonces es de Cauchy.
\end{proposition}

\begin{proof}
	Sea $\lim_{n \to \infty} x_n = x$ y sea $\varepsilon > 0$. Por definición de límite, existe $N \in \mathbb{N}$ tal que para todo $n \geq N$, vale que $d(x_n, x) < \frac{\varepsilon}{2}$. Entonces, si $n, m \geq N$, tenemos que
	$$
		d(x_n, x_m) \leq d(x_n, x) + d(x, x_m) < \frac{\varepsilon}{2} + \frac{\varepsilon}{2} = \varepsilon.
	$$
	Por lo tanto, $(x_n)_{n \in \mathbb{N}}$ es una sucesión de Cauchy.
\end{proof}

\begin{remark}
	El recíproco no siempre es cierto. Por ejemplo, en el espacio $\mathbb{Q}$ con la métrica usual, la sucesión de las aproximaciones decimales de $\sqrt{2}$ (e.g., $1, 1.4, 1.41, 1.414, \dots$) es de Cauchy pero no converge en $\mathbb{Q}$.
\end{remark}
\begin{proposition}
	Toda sucesión de Cauchy está acotada.
\end{proposition}

\begin{proof}
	Sea $(x_n)_{n \in \mathbb{N}}$ una sucesión de Cauchy. Por definición, para $\varepsilon = 1$, existe un $N \in \mathbb{N}$ tal que para todo $n, m \geq N$, vale que $d(x_n, x_m) < 1$.

	Fijando $m=N$, tenemos que $d(x_n, x_N) < 1$ para todo $n \ge N$. Esto significa que todos los términos de la sucesión a partir de $x_N$ están contenidos en la bola $B(x_N, 1)$. Consideremos el conjunto finito de distancias de los primeros términos a $x_N$: $\{d(x_1, x_N), d(x_2, x_N), \dots, d(x_{N-1}, x_N)\}$. Sea $R = \max\{1, d(x_1, x_N), \dots, d(x_{N-1}, x_N)\}$.

	Entonces, para cualquier $n \in \mathbb{N}$, se cumple que $d(x_n, x_N) \leq R$. Esto demuestra que la sucesión entera está contenida en la bola $\overline{B}(x_N, R)$, y por lo tanto, está acotada.
\end{proof}

\begin{remark}
	(Formulaciones equivalentes de convergencia). Una sucesión $(x_n)_{n \in \mathbb{N}}$ converge a $x$ si y solo si para todo entorno $V$ de $x$, existe un $N \in \mathbb{N}$ tal que $x_n \in V$ para todo $n \geq N$.
\end{remark}

\section{Relación entre sucesiones y clausura}

Quizás una de las mayores utilidades de las sucesiones es que proveen una caracterización alternativa de la clausura de un conjunto.

\begin{proposition}
	Sea $A \subseteq X$. Un punto $x \in X$ pertenece a $\overline{A}$ si y sólo si existe una sucesión $(a_{n})_{n \in \mathbb{N}}$ de puntos de $A$ tal que $a_{n} \longrightarrow x$.
\end{proposition}

\begin{proof}
	($\Rightarrow$) Supongamos que $x \in \overline{A}$. Por definición de clausura, para todo entorno de $x$, su intersección con $A$ es no vacía. En particular, para cada $n \in \mathbb{N}$, la bola $B(x, 1/n)$ es un entorno de $x$, por lo que $B(x, 1/n) \cap A \neq \emptyset$.

	Podemos entonces construir una sucesión $(a_n)_{n \in \mathbb{N}}$ eligiendo para cada $n$ un elemento $a_n \in B(x, 1/n) \cap A$. Por construcción, $(a_n)$ es una sucesión de puntos de $A$ y además cumple que $d(a_n, x) < 1/n$.

	Dado $\varepsilon > 0$, por la propiedad arquimediana existe $N \in \mathbb{N}$ tal que $1/N < \varepsilon$. Luego, para todo $n \geq N$, tenemos $d(a_n, x) < 1/n \leq 1/N < \varepsilon$. Esto prueba que $a_n \to x$.

	($\Leftarrow$) Supongamos que existe una sucesión $(a_n)_{n \in \mathbb{N}}$ en $A$ tal que $a_n \to x$. Queremos ver que $x \in \overline{A}$. Para ello, debemos probar que toda bola centrada en $x$ intersecta a $A$.

	Sea $B(x, \varepsilon)$ una bola arbitraria con $\varepsilon > 0$. Como $a_n \to x$, por definición de límite, existe $N \in \mathbb{N}$ tal que para todo $n \ge N$, $d(a_n, x) < \varepsilon$. Esto significa que $a_n \in B(x, \varepsilon)$. Como además $a_n \in A$ por hipótesis, tenemos que $a_n \in B(x, \varepsilon) \cap A$.

	Por lo tanto, la intersección es no vacía, y concluimos que $x \in \overline{A}$.
\end{proof}


\chapter{Separabilidad}

Previo a definir un espacio métrico separable, veamos qué es la topología de un espacio.

\section{Topología del espacio}

\begin{definition}
	Sea $(X, d)$ un espacio métrico. La \emph{topología} de $X$ es la familia de subconjuntos abiertos de $X$. Una \emph{base} de la topología de $X$ es una familia de abiertos $\mathcal{B}$ tal que
	\begin{center}
		\begin{minipage}{0.9\linewidth}
			todo abierto $A \subseteq X$ se puede expresar como $A = \bigcup_{i \in I} B_i$ con $B_i \in \mathcal{B}$.
		\end{minipage}
	\end{center}
\end{definition}
\begin{remark}
	La condición de una base puede ser reemplazada por:
	\begin{center}
		\begin{minipage}{0.9\linewidth}
			Para todo $A \subseteq X$ abierto se cumple que para todo $a \in A$, existe $B \in \mathcal{B}$ tal que $a \in B \subseteq A$.
		\end{minipage}
	\end{center}
\end{remark}

\begin{example}
	Algunas bases son:
	\begin{enumerate}
		\item La base $\{ B(x, r) \mid x \in X \text{ y } r > 0 \}$.
		\item La familia de abiertos es una base.
		\item La base $\{ B(x, \frac{1}{n}) \mid x \in X \text{ y } n \in \mathbb{N} \}$.
	\end{enumerate}
\end{example}

\begin{definition}
	Decimos que dos espacios métricos tienen \emph{topologías equivalentes} si inducen los mismos abiertos.
\end{definition}

\begin{example}
	En $\mathbb{R}^n$, las métricas $\lVert \cdot \rVert_p$ donde $p \in (0, + \infty]$ inducen la misma topología.
\end{example}

Recordemos la definición de subespacio métrico: subconjunto del espacio con la métrica restringida.

\begin{remark}
	Definimos la bola en $Y \subseteq X$ como
	$$
		B_Y (y, r) = \{ x \in Y \mid d(x, y) < r\}.
	$$
\end{remark}

\begin{proposition}
	Sea $(X, d)$ un espacio métrico e $Y \subseteq X$. Entonces, un subconjunto $A \subseteq Y$ es abierto en $Y$ si y sólo si existe un subconjunto abierto $B \subseteq X$ tal que $A = B \cap Y$.
\end{proposition}

\begin{proof}
	($\Rightarrow$) Sea $A \subseteq Y$ un subconjunto abierto. Entonces, $A$ es unión de bolas abiertas
	$$
		A = \bigcup_{i \in I} B_Y (y_i, r_i).
	$$
	Definimos al conjunto
	$$
		B = \bigcup_{i \in I} B (y_i, r_i).
	$$
	Dado que $B$ es unión de abiertos,  también es abierto. Consideramos
	$$
		B \cap Y = \bigcup_{i \in I} B_Y (y_i, r_i) = A.
	$$

	($\Leftarrow$) Sea $A \subseteq Y$ y $B \subseteq X$ un subconjunto abierto que cumplen que $A = B \cap Y$. Como $B$ es abierto, se puede expresar como la unión de bolas abiertas
	$$
		B = \bigcup_{i \in I} B(x_i, r_i).
	$$
	Por lo tanto,
	$$
		A = B \cap Y = \bigcup_{i \in I} B_Y (x_i, r_i).
	$$
	Y como $A$ es unión de bolas abiertas, $A$ es abierto.
\end{proof}

\begin{remark}
	La proposición análoga para cerrados también es válida.
\end{remark}

\section{Espacios separables}

\begin{definition}
	Un espacio métrico $(X, d)$ es \emph{separable} si existe un subconjunto denso $D \subseteq X$ tal que $\left| D \right| \leq \aleph_0$.
\end{definition}

\begin{example}
	Algunos espacios separables son:
	\begin{enumerate}
		\item $X$ contable es separable con $D = X$.
		\item $(\mathbb{R}, |\cdot|)$ es separable con $D = \mathbb{Q}$.
		\item $(\mathbb{R} - \mathbb{Q}, |\cdot|)$ es separable con $D = \mathbb{Q} + \sqrt{2}$.
		\item $(\ell^{p} (\mathbb{R}), \lVert \cdot \rVert)$ es separable para $p < + \infty$.
		\item $C([a, b], \mathbb{R})$ es separable con $D = \mathbb{Q}[x]$.
	\end{enumerate}
\end{example}

\begin{definition}
	\begin{itemize}
		\item Decimos que una familia $\{ A_i \}_{i \in I}$ es un \emph{cubrimiento} de $X$ si $X \subseteq \bigcup_{i \in I} A_i$.
		\item Si además $B_i$ es abierto para todo $i \in I$, decimos que $\{ B_i \}_{i \in I}$ es un \emph{cubrimiento abierto} de $X$.
		\item La familia $\{ A_j \}_{j \in J}$ es un \emph{subcubrimiento} de $\{ A_i \}_{i \in I}$ si es un cubrimiento y $J \subseteq I$.
	\end{itemize}
\end{definition}

A continuación enunciamos y demostramos el Teorema de Lindelöf.

\begin{theorem}
	Sea $(X, d)$ un espacio métrico. Entonces, son equivalentes:
	\begin{enumerate}
		\item $X$ es separable.
		\item Existe una base contable de $X$.
		\item Todo cubrimiento abierto de $X$ admite un subcubrimiento contable.
	\end{enumerate}
\end{theorem}

\begin{proof}
	\framebox{1. $\Rightarrow$ 2.} Sea $(X, d)$ un espacio métrico separable. Entonces, existe un subconjunto denso $D \subseteq X$ contable. Veamos que
	$$
		\mathcal{B} = \left\{ B\left(x, \frac{1}{n}\right) \right\}_{x \in D, n \in \mathbb{N}}
	$$
	es una base contable. Dado que $D \times \mathbb{N}$ es contable, $\mathcal{B}$ es contable, basta con ver que para todo $A \subseteq X$ abierto y todo $a \in A$, existe $B \in \mathcal{B}$ tal que $a \in B \subseteq A$.

	Sea $A \subseteq X$ un subconjunto abierto y sea $a \in A$. Por definición de abierto, existe una bola $B(a, r)$ con $r > 0$ tal que $B(a, r) \subseteq A$. Queremos encontrar $x \in D$ y $r' > 0$ tales que $a \in B(x, r') \subseteq B(a, r)$. Elegimos $r' = \frac{1}{n}$ tal que $\frac{1}{n} < \frac{r}{2}$. Consideramos ahora la bola $B(a, \frac{1}{n})$. Dado que $D$ es denso, existe $x \in D$ tal que $x \in B(a, \frac{1}{n})$.

	Veamos que $a \in B(x, \frac{1}{n}) \subseteq B(a, r)$. Sea $x' \in B(x, \frac{1}{n})$. Entonces,
	$$
		d(x', a) \leq d(x', x) + d(x, a) < \frac{2}{n} < r.
	$$
	\begin{center}
		\begin{tikzpicture}[x=0.75pt,y=0.75pt,yscale=-1,xscale=1]
	%uncomment if require: \path (0,300); %set diagram left start at 0, and has height of 300

	%Shape: Polygon Curved [id:ds7613739725483922] 
	\draw  [color={rgb, 255:red, 0; green, 86; blue, 145 }  ,draw opacity=1 ][fill={rgb, 255:red, 0; green, 86; blue, 145 }  ,fill opacity=0.1 ][dash pattern={on 4.5pt off 4.5pt}][line width=0.75]  (57.97,34.79) .. controls (80.65,24.22) and (187.88,22.11) .. (165.2,43.24) .. controls (143.21,63.73) and (126.42,77.59) .. (142.1,96.4) .. controls (142.59,96.98) and (143.1,97.57) .. (143.66,98.17) .. controls (161.93,117.89) and (140.5,176.33) .. (107.87,172.11) .. controls (75.24,167.89) and (38.86,132.38) .. (30,120) .. controls (21.14,107.62) and (35.29,45.35) .. (57.97,34.79) -- cycle ;
	%Shape: Ellipse [id:dp42634731485483457] 
	\draw  [color={rgb, 255:red, 86; green, 0; blue, 145 }  ,draw opacity=1 ][fill={rgb, 255:red, 86; green, 0; blue, 145 }  ,fill opacity=0.1 ][dash pattern={on 4.5pt off 4.5pt}] (65.5,92) .. controls (65.5,83.72) and (72.1,77) .. (80.25,77) .. controls (88.4,77) and (95,83.72) .. (95,92) .. controls (95,100.28) and (88.4,107) .. (80.25,107) .. controls (72.1,107) and (65.5,100.28) .. (65.5,92) -- cycle ;
	%Shape: Ellipse [id:dp1598998823581017] 
	\draw  [color={rgb, 255:red, 86; green, 0; blue, 145 }  ,draw opacity=1 ][dash pattern={on 4.5pt off 4.5pt}] (80.2,92.05) .. controls (80.2,92.02) and (80.22,92) .. (80.25,92) .. controls (80.28,92) and (80.3,92.02) .. (80.3,92.05) .. controls (80.3,92.08) and (80.28,92.1) .. (80.25,92.1) .. controls (80.22,92.1) and (80.2,92.08) .. (80.2,92.05) -- cycle ;

	%Shape: Ellipse [id:dp6929882431707403] 
	\draw  [color={rgb, 255:red, 0; green, 145; blue, 105 }  ,draw opacity=1 ][fill={rgb, 255:red, 0; green, 145; blue, 105 }  ,fill opacity=0.1 ][dash pattern={on 4.5pt off 4.5pt}] (77.5,105) .. controls (77.5,96.72) and (84.1,90) .. (92.25,90) .. controls (100.4,90) and (107,96.72) .. (107,105) .. controls (107,113.28) and (100.4,120) .. (92.25,120) .. controls (84.1,120) and (77.5,113.28) .. (77.5,105) -- cycle ;
	%Shape: Ellipse [id:dp7063709012266387] 
	\draw  [color={rgb, 255:red, 0; green, 145; blue, 105 }  ,draw opacity=1 ][fill={rgb, 255:red, 0; green, 145; blue, 105 }  ,fill opacity=0.1 ][dash pattern={on 4.5pt off 4.5pt}] (92.2,105.05) .. controls (92.2,105.02) and (92.22,105) .. (92.25,105) .. controls (92.28,105) and (92.3,105.02) .. (92.3,105.05) .. controls (92.3,105.08) and (92.28,105.1) .. (92.25,105.1) .. controls (92.22,105.1) and (92.2,105.08) .. (92.2,105.05) -- cycle ;

	%Shape: Circle [id:dp3999102181776214] 
	\draw  [color={rgb, 255:red, 155; green, 155; blue, 155 }  ,draw opacity=1 ][dash pattern={on 4.5pt off 4.5pt}] (38.55,92.05) .. controls (38.55,69.05) and (57.2,50.4) .. (80.2,50.4) .. controls (103.2,50.4) and (121.85,69.05) .. (121.85,92.05) .. controls (121.85,115.05) and (103.2,133.7) .. (80.2,133.7) .. controls (57.2,133.7) and (38.55,115.05) .. (38.55,92.05) -- cycle ;

	% Text Node
	\draw (117.65,44.65) node [anchor=north west][inner sep=0.75pt]  [font=\footnotesize,color={rgb, 255:red, 0; green, 86; blue, 145 }  ,opacity=1 ]  {$A$};
	% Text Node
	\draw (74,82) node [anchor=north west][inner sep=0.75pt]  [font=\scriptsize,color={rgb, 255:red, 86; green, 0; blue, 145 }  ,opacity=1 ]  {$a$};
	% Text Node
	\draw (92.25,101) node [anchor=north west][inner sep=0.75pt]  [font=\footnotesize,color={rgb, 255:red, 0; green, 145; blue, 105 }  ,opacity=1 ]  {$x$};

\end{tikzpicture}

	\end{center}

	\framebox{2. $\Rightarrow$ 3.} Sea $\mathcal{B} = \{ B_n \}_{n \in N}$ una base contable de $X$ y sea $\mathcal{U} = \{ U_i \}_{i \in I}$ un cubrimiento abierto. Definimos la función
	\begin{align*}
		\Phi: N & \to I                                         \\
		n       & \mapsto i \text{ tal que } B_n \subseteq U_i.
	\end{align*}
	Y consideramos $J = \operatorname{Im} \Phi$. Seguro que $J$ es contable ya que $\Phi$ es sobreyectiva hacia $J$. Entonces, basta con ver que el subconjunto $\mathcal{U}' = \{ U_j \}_{j \in J}$ cubre a $X$.

	Sea $x \in X$. Como $\mathcal{B}$ es una base, existe algún $n \in N$ tal que $x \in B_n$. Consideramos $j \in J$ tal que $B_n \subseteq U_j$, entonces $x \in U_j$. Por lo tanto, para todo $x \in X$, existe un $j \in J$ tal que $x \in U_j$. Por lo tanto, $\mathcal{U}'$ cubre a $X$.

	\framebox{3. $\Rightarrow$ 1.} Sea $\mathcal{U}_n = \{ B(x, \frac{1}{n}) \mid x \in X\}$. Como $\mathcal{U}_n$ es un cubrimiento abierto, existe un subcubrimiento contable
	$$
		\mathcal{U}'_n = \{ B\left(x, \frac{1}{n}\right) \mid x \in D_n \subseteq X\}.
	$$
	Dado que $\mathcal{U}'_n$ es contable, $D_n$ tiene que serlo también. Sea $D = \bigcup_{n \in \mathbb{N}} D_n$. Veamos que $D$ es denso en $X$.

	Sea $x \in X$ y $r > 0$. Queremos ver que $B(x, r) \cap D \neq \emptyset$. Por Arquimedianidad, existe $N \in \mathbb{N}$ tal que $\frac{1}{N} < r$. Dado que $U'_N$ es un cubrimiento, existe $a \in D_N$ tal que $x \in B(a, \frac{1}{N})$. Entonces, $d(x, a) < \frac{1}{N} < r$. Entonces, $B(x, r) \cap D \neq \emptyset$.
\end{proof}

\begin{corollary}
	Sea $(X, d)$ un espacio métrico separable y sea $Y \subseteq X$ un subespacio. Entonces, $Y$ es separable.
\end{corollary}

\begin{proof}
	Por el Teorema anterior, $(X, d)$ es separable si y sólo si existe una base contable de $X$. Basta con ver que $Y$ tiene una base contable. Consideramos $\mathcal{B}|_Y = \{ B_n \cap Y\}_{n \in \mathbb{N}}$. Esto define una base contable de $Y$.
\end{proof}

\begin{proposition}
	Un espacio métrico $(X, d)$ es separable si y sólo si toda familia de disjuntos abiertos de $X$ es contable.
\end{proposition}

\begin{proof}
	($\Rightarrow$) Sea $(X, d)$ un espacio métrico separable. Sea $\mathcal{U} = \{ U_i \}_{i \in I}$ una familia de disjuntos abiertos de $X$. Por el Teorema anterior, existe una base contable $\mathcal{B}$. Definimos la función
	\begin{align*}
		\varphi: I & \to  N                                         \\
		i          & \mapsto n \text{ tal que } B_n \subseteq U_i .
	\end{align*}
	La función está bien definida ya que, dado que $\mathcal{B}$ es una base, todo elemento de $\mathcal{U}$ se puede expresar como unión de elementos de $\mathcal{B}$ (para la elección podemos simplemente tomar el mínimo de los $n$). Veamos que $\varphi$ es inyectiva. Sean $i, j \in I$ tales que $\varphi(i) = \varphi(j) = n$. Entonces, $B_n \subseteq U_i \cap  U_j$, lo cual es absurdo. Por lo tanto, $\varphi$ es inyectiva.

	Como $\left| N \right| \leq \aleph_0$ y $\left| I \right| \leq \left| N \right| $, obtenemos que $\left| I \right| \leq \aleph_0$, es decir, $I$ es contable.

	($\Leftarrow$) Sea $\mathcal{F}_n$ una familia disjunta de bolas abiertas de radio $\frac{1}{n}$ maximal. Es decir, toda bola de radio $\frac{1}{n}$ interseca con algún elemento de $\mathcal{F}_n$. Por hipótesis, $\mathcal{F}_n$ es contable. Definimos $D_n$ como el conjunto de centros de las bolas de $\mathcal{F}_n$.

	Sea $D = \bigcup_{n \in \mathbb{N}} D_n$. Veamos que $D$ es denso en $X$. Sea $x \in X$, $r > 0$. Basta con ver que $B(x, r) \cap D \neq \emptyset$. Sea $n \in \mathbb{N}$ tal que $\frac{1}{n} < \frac{r}{2}$.

	Supongamos que $B(x, r) \cap D = \emptyset$. Entonces, $B(x, r) \cap D_n = \emptyset$. Notemos que dos bolas cualesquiera se intersecan si y sólo si la distancia entre sus centros es menor a la suma de sus radios.

	Teniendo esto en cuenta, consideremos la bola $B(x, \frac{1}{n})$. Dado que $(x, r) \cap D_n = \emptyset$, seguro que $d(x, a) > r$. Pero como $r > \frac{2}{n}$, entonces $d(x, a) > \frac{1}{n} + \frac{1}{n}$. Por lo que $B(x, \frac{1}{n})$ no interseca con ningún elemento de $\mathcal{F}_n$. Esto es absurdo; por lo tanto, $B(x, r) \cap D \neq \emptyset$. Entonces, $D$ es denso en $X$.
\end{proof}




\chapter{Continuidad}

\section{Funciones Continuas}

\begin{definition}
	Sean $(X, d_X)$ e $(Y, d_Y)$ espacios métricos. Una función $f : X \to Y$ se dice \emph{continua en} $x$ si
	\begin{center}
		\begin{minipage}{0.9\linewidth}
			para todo $\varepsilon > 0$, existe un $\delta > 0$ tal que $d_X (x', x) < \delta$ implica $d_Y (f(x'), f(x)) < \varepsilon$, para todo $x' \in X$.
		\end{minipage}
	\end{center}
	Decimos que $f$ es \emph{continua} si es continua en todo punto de su dominio.
\end{definition}

\begin{remark}
	O más compacto:
	\begin{equation*}
		\forall \varepsilon > 0, \exists \delta > 0 \mid d_X (x', x) < \delta \Rightarrow d_Y (f(x'), f(x)), \forall x' \in X.
	\end{equation*}
\end{remark}

Es muy útil pensar la continuidad de forma intuitiva utilizando la noción de entornos. Si $f$ es continua en $x$, entonces para todo entorno $V$ de $f(x)$, $f^{-1}(V)$ es un entorno de $x$.

\begin{center}
	\begin{tikzpicture}[x=0.75pt,y=0.75pt,yscale=-1,xscale=1]
	%uncomment if require: \path (0,300); %set diagram left start at 0, and has height of 300

	%Shape: Polygon Curved [id:ds9863252903529107] 
	\draw  [color={rgb, 255:red, 0; green, 86; blue, 145 }  ,draw opacity=1 ][fill={rgb, 255:red, 0; green, 86; blue, 145 }  ,fill opacity=0.1 ][dash pattern={on 4.5pt off 4.5pt}][line width=0.75]  (57.97,33.79) .. controls (80.65,23.22) and (187.88,21.11) .. (165.2,42.24) .. controls (142.52,63.36) and (123.85,78.74) .. (139.52,97.54) .. controls (155.2,116.35) and (116.63,128.1) .. (84,123.88) .. controls (51.37,119.65) and (38.86,131.38) .. (30,119) .. controls (21.14,106.62) and (35.29,44.35) .. (57.97,33.79) -- cycle ;
	%Shape: Ellipse [id:dp7045194170512884] 
	\draw  [color={rgb, 255:red, 86; green, 0; blue, 145 }  ,draw opacity=1 ][fill={rgb, 255:red, 86; green, 0; blue, 145 }  ,fill opacity=0.1 ][dash pattern={on 4.5pt off 4.5pt}] (54.5,79.77) .. controls (54.5,62.52) and (68.16,48.54) .. (85.01,48.54) .. controls (101.86,48.54) and (115.52,62.52) .. (115.52,79.77) .. controls (115.52,97.02) and (101.86,111) .. (85.01,111) .. controls (68.16,111) and (54.5,97.02) .. (54.5,79.77) -- cycle ;
	%Shape: Ellipse [id:dp17380935073260606] 
	\draw  [color={rgb, 255:red, 86; green, 0; blue, 145 }  ,draw opacity=1 ][dash pattern={on 4.5pt off 4.5pt}] (84.91,79.87) .. controls (84.91,79.82) and (84.96,79.77) .. (85.01,79.77) .. controls (85.07,79.77) and (85.11,79.82) .. (85.11,79.87) .. controls (85.11,79.93) and (85.07,79.98) .. (85.01,79.98) .. controls (84.96,79.98) and (84.91,79.93) .. (84.91,79.87) -- cycle ;

	%Straight Lines [id:da5395275905863334] 
	\draw [line width=0.75]    (143.52,81.54) -- (223,81.54) ;
	\draw [shift={(225,81.54)}, rotate = 180] [color={rgb, 255:red, 0; green, 0; blue, 0 }  ][line width=0.75]    (4.37,-1.32) .. controls (2.78,-0.56) and (1.32,-0.12) .. (0,0) .. controls (1.32,0.12) and (2.78,0.56) .. (4.37,1.32)   ;
	%Shape: Polygon Curved [id:ds45640730182556577] 
	\draw  [color={rgb, 255:red, 0; green, 86; blue, 145 }  ,draw opacity=1 ][fill={rgb, 255:red, 0; green, 86; blue, 145 }  ,fill opacity=0.1 ][dash pattern={on 4.5pt off 4.5pt}][line width=0.75]  (263.97,36.79) .. controls (286.65,26.22) and (295.46,-5.49) .. (336.33,36.88) .. controls (377.2,79.24) and (329.85,81.74) .. (345.52,100.54) .. controls (361.2,119.35) and (322.63,131.1) .. (290,126.88) .. controls (257.37,122.65) and (250.2,126.26) .. (241.33,113.88) .. controls (232.47,101.49) and (241.29,47.35) .. (263.97,36.79) -- cycle ;
	%Shape: Ellipse [id:dp21595437817711993] 
	\draw  [color={rgb, 255:red, 86; green, 0; blue, 145 }  ,draw opacity=1 ][dash pattern={on 4.5pt off 4.5pt}] (290.91,82.87) .. controls (290.91,82.82) and (290.96,82.77) .. (291.01,82.77) .. controls (291.07,82.77) and (291.11,82.82) .. (291.11,82.87) .. controls (291.11,82.93) and (291.07,82.98) .. (291.01,82.98) .. controls (290.96,82.98) and (290.91,82.93) .. (290.91,82.87) -- cycle ;
	%Shape: Polygon Curved [id:ds49186281368908624] 
	\draw  [color={rgb, 255:red, 86; green, 0; blue, 145 }  ,draw opacity=1 ][fill={rgb, 255:red, 86; green, 0; blue, 145 }  ,fill opacity=0.1 ][dash pattern={on 4.5pt off 4.5pt}] (300.33,58.88) .. controls (327.33,50.88) and (324.67,96.13) .. (311,101) .. controls (297.33,105.88) and (268.33,100.88) .. (264.33,77.88) .. controls (260.33,54.88) and (273.33,66.88) .. (300.33,58.88) -- cycle ;

	% Text Node
	\draw (112.65,35.65) node [anchor=north west][inner sep=0.75pt]  [font=\footnotesize,color={rgb, 255:red, 0; green, 86; blue, 145 }  ,opacity=1 ]  {$V$};
	% Text Node
	\draw (76.25,63) node [anchor=north west][inner sep=0.75pt]  [font=\footnotesize,color={rgb, 255:red, 0; green, 145; blue, 105 }  ,opacity=1 ]  {$\textcolor[rgb]{0.34,0,0.57}{f( x)}$};
	% Text Node
	\draw (178,60) node [anchor=north west][inner sep=0.75pt]  [font=\footnotesize]  {$f^{-1}$};
	% Text Node
	\draw (290.65,40.65) node [anchor=north west][inner sep=0.75pt]  [font=\footnotesize,color={rgb, 255:red, 0; green, 86; blue, 145 }  ,opacity=1 ]  {$f^{-1}( V)$};
	% Text Node
	\draw (287.25,68) node [anchor=north west][inner sep=0.75pt]  [font=\footnotesize,color={rgb, 255:red, 86; green, 0; blue, 145 }  ,opacity=1 ]  {$x$};


\end{tikzpicture}

\end{center}

Veamos equivalencias de continuidad.

\begin{proposition}
	Sea $f : X \to Y$ y sea $x \in X$. Entonces, son equivalentes:
	\begin{enumerate}
		\item $f$ es continua en $x$.
		\item Si $x_n \longrightarrow x$, entonces $f(x_n) \longrightarrow f(x)$.
		\item Si $V \subseteq Y$ es un entorno de $f(x)$, entonces $f^{-1} (V)$ es un entorno de $x$.
	\end{enumerate}
\end{proposition}

Cuando decimos que una función $f$ es continua, nos referimos a que $f$ es continua en todo punto de su dominio.

\begin{proposition}
	Sea $f : X \to Y$. Entonces, son equivalentes:
	\begin{enumerate}
		\item $f$ es continua.
		\item Si $x_n \longrightarrow x$, entonces $f(x_n) \longrightarrow f(x)$ para todo $x \in X$ y sucesión $(x_n)_{n \in \mathbb{N}}$.
		\item $f^{-1} (A) \subseteq X$ es abierto para todo $A \subseteq Y$ abierto.
		\item $f^{-1} (B) \subseteq X$ es cerrado para todo $B \subseteq Y$ cerrado.
		\item $f( \overline{C} ) \subseteq \overline{f(C)}$ para todo $B \subseteq X$.
	\end{enumerate}
\end{proposition}

\begin{proof}
	\color{red} COMPLETAR
\end{proof}

\begin{proposition}
	La composición de funciones continuas es continua.
\end{proposition}

\section{Continuidad Uniforme}

\begin{definition}
	Decimos que $f : X \to Y$ es \emph{uniformemente continua} si
	\begin{equation*}
		\begin{split}
			\forall \varepsilon > 0, \exists \delta > 0 \text{ tal que } & d_X(x,x') < \delta \\ &\Rightarrow d_Y(f(x), f(x')) < \varepsilon, \forall x, x' \in X.
		\end{split}
	\end{equation*}
\end{definition}

\begin{remark}
	Las funciones uniformemente continuas mandan sucesiones de Cauchy a sucesiones de Cauchy.
\end{remark}

\begin{proposition}
	Sean $f, g : X \to Y$ continuas tales que $f |_A = g|_A$ con $A \subseteq X$ denso. Entonces, $g = g'$ en $X$.
\end{proposition}

\begin{proof}
	Sea $\Phi (x) = d(f(x), g(x))$. Por composición de funciones continuas, $\Phi$ es continua. Como $A \subseteq \Phi^{-1}(0)$ y $\left\{ 0 \right\}$ es cerrado, entonces $d(f(x), g(x)) = 0$ para todo $x \in \overline{D}$. Es decir, $f(x) = g(x)$ para todo $x \in X$.
\end{proof}


\section{Homeomorfismos}

Los homeomorfismo de cierta forma ``preservan'' la topología de un espacio.

\begin{definition}
	Decimos que $f : X \to Y$ es un \emph{homeomorfismo} si
	\begin{itemize}
		\item $f$ es continua.
		\item $f$ es biyectiva.
		\item $f^{-1}$ es continua.
	\end{itemize}
	Si existe un homeomorfismo de $X$ a $Y$, entonces decimos que son espacios \emph{homeomorfos}.
\end{definition}

También, definimos otros términos.

\begin{definition}
	Decimos que $f : X \to Y$ es
	\begin{itemize}
		\item \emph{abierta} si $f(A) \subseteq Y$ es abierto para todo $A \subseteq X$ abierto.
		\item \emph{cerrada} si $f(B) \subseteq Y$ es cerrado para todo $B \subseteq X$ cerrado.
	\end{itemize}
\end{definition}

\begin{remark}
	Son equivalentes
	\begin{itemize}
		\item homeomorfismo.
		\item continua y abierta.
		\item continua y cerrada.
	\end{itemize}
\end{remark}

\begin{example}
	Los espacios $\mathbb{R}$ y $(0, 1)$ son homeomorfos con la métrica usual.
\end{example}

En general, la idea de los homeomorfismos es que, de cierta forma, nos dicen que dos espacios métricos tienen las mismas propiedades. Por lo tanto, las propiedades \textit{topológicas} se mantienen con homeomorfismos. Es decir, los conceptos de abiertos, cerrados, puntos de acumulación, densos, continuidad sólo dependen de los abiertos y no la distancia del espacio métrico. (Esto lo habíamos visto antes y es la topología.)






\chapter{Completitud}

\section{Espacios métricos completos}

\begin{definition}
	Un espacio métrico $(X, d)$ se dice \emph{completo} si toda ucesión de Cauchy tiene límite en $X$.
\end{definition}

\begin{example}
	Sea $(X, \delta)$ un espacio métrico con la métrica discreta. Entonces, $(X, \delta)$ es completo.
\end{example}

\begin{proof}[Solución]
	Toda sucesión de Cauchy en $(X, \delta)$ es eventualmente constante. Por lo tanto, converge a un elemento de $X$.
\end{proof}

\begin{proposition}
	Sea $(X, d)$ un espacio métrico completo y sea $A \subseteq X$ cerrado. Entonces, $A$ es completo.
\end{proposition}

\begin{proof}
	Sea $(a_n)_{n \in \mathbb{N}}$ una sucesión de Cauchy. Entonces, existe un $x \in X$ tal que $a_n \longrightarrow x$. Como $A$ es cerrado, tenemos que $x \in A$. Por lo tanto, $A$ es completo.
\end{proof}

\begin{theorem}
	Sea $(X, d)$ un espacio métrico completo. Entonces, son equivalentes:
	\begin{enumerate}
		\item $X$ es completo.
		\item Toda familia $(C_n)_{n \in \mathbb{N}}$ de intervalos cerrados encajados tal que su diámetro tiende a $0$ tiene intersección no vacía. En particular, $\bigcap_{n \in \mathbb{N}} C_n = \left\{ x \right\}$, $x \in X$.
	\end{enumerate}
\end{theorem}

\begin{proof}
	\framebox{1. $\Rightarrow$ 2.} Sea $X$ completo y sea $(C_n)_{n \in \mathbb{N}}$ una familia de cerrados encajados con diámetro tendiendo a $0$. Tomamos una sucesión $(x_n)_{n \in \mathbb{N}}$ tal que $x_n \in C_n$. Veamos que es de Cauchy. Sean $n, m \in \mathbb{N}$. Entonces,
	\begin{equation*}
		d(x_n, x_m) \leq \diam C_n \longrightarrow 0.
	\end{equation*}
	Por lo tanto, como es de Cauchy en un completo, existe el límite de $x_n$, llamémoslo $x_0$.

	Como los intervalos son encajados, $(x_n)_{n \geq N}$ está en $C_N$, y como $C_N$ es cerrado, $x_0 \in C_N$, para todo $N \in \mathbb{N}$. O sea, $x \in \bigcap_{n \in \mathbb{N}} C_n$.

	\framebox{1. $\Rightarrow$ 2.} Supongamos 2. y sea $(x_n)_{n \in \mathbb{N}}$ una sucesión de Cauchy. Tomamos $C_m = \overline{\left\{ x_n \right\}}_{n \geq m}$. Todo $C_n$ es no vacío ya que $x_n \in C_n$ y claramente el diámetro tiende a $0$. Por lo tanto, existe $x \in \overline{\left\{ x_n \right\}}_{n \geq m}$. Como $d(x, x_n) \leq \diam C_n \longrightarrow 0$, $x_n \longrightarrow x$.
\end{proof}

\section{Completitud de espacios de funciones}

Algo que nos va a servir más adelante es probar la completitud del espacio de funciones acotadas.

\begin{theorem}
	Sea $X$ un conjunto no vacío e $(Y, d)$ un espacio métrico completo. Sea $B(X, Y) = \left\{ f : X \to Y \mid \diam \operatorname{Im} f < \infty \right\}$ junto con $d_{\infty} (f, g) = \sup_{x \in X} d(f(x), g(x))$. Entonces, $(B(X, Y), d_{\infty})$ es completo.
\end{theorem}

\begin{proof}
	Sea $(f_n)_{n \in \mathbb{N}}$. Sea $x \in X$ y consideramos la sucesión $(f_n (x))_{n \in \mathbb{N}}$ en $Y$.

	Veamos que $(f_n (x))_{n \in \mathbb{N}}$ es de Cauchy. Sean $m, n \in \mathbb{N}$. Como
	\begin{equation*}
		d(f_m (x), f_n (x)) \leq d_{\infty} (f_m, f_n) \longrightarrow 0,
	\end{equation*}
	ya que $d_{\infty} (f_m, f_n)$ es de Cauchy, $(f_n (x))_{n \in \mathbb{N}}$ es de Cauchy. Además, dado que $Y$ es completo, existe el límite de $(f_n(x))_{n \in \mathbb{N}}$. Por lo tanto, definimos $f(x) = \lim_{n \to \infty} f_n (x)$.

	Ya tenemos el candidato de límite, nos falta ver que $d_{\infty}(f, f_n) \longrightarrow 0$ y $f \in B(X, Y)$.

	Como $(f_n)_{n \in \mathbb{N}}$ es de Cauchy en $B(X, Y)$, entonces para todo $\varepsilon > 0$, existe $N \in \mathbb{N}$ tal que $d_{\infty} (f_m, f_n) \leq \varepsilon$ para todo $m, n \geq \mathbb{N}$. Tomamos límite en $m$ y obtenemos que
	\begin{equation*}
		d_{\infty} (f, f_n) \leq \varepsilon, \text{ para todo } n \geq N.
	\end{equation*}
	Por lo tanto, $f_n \longrightarrow f$.

	Veamos que $f$ es acotada. Sean $x, x' \in X$. Entonces,
	\begin{align*}
		d(f(x), f(x')) & \leq d(f(x), f_n (x)) + d(f_n (x), f_m (x')) + d(f_m(x'), f(x')) \\
		               & \leq \varepsilon + \varepsilon + \varepsilon.
	\end{align*}
	Por lo tanto, $f$ es acotada.
\end{proof}

\section{Teorema del punto fijo de Banach}

Primero una definición.

\begin{definition}
	Sea $X$ e $Y$ un espacios métricos. Sea $f : X \to Y$. Decimos que $f$ es una \emph{contracción} si existe $\lambda \in (0, 1)$ tal que
	\begin{equation*}
		d_Y(f(x), f(x')) \leq \lambda d_X(x, y), \text{ para todo }x, x' \in X.
	\end{equation*}
\end{definition}

\begin{remark}
	Esto es equivalente a pedir que $f$ sea Lipschitz con constante menor que $1$.
\end{remark}

\begin{theorem}[Teorema del Punto Fijo de Banach]
	Sea $(X,d)$ un espacio métrico completo y sea $f:X \to X$ una contracción. Entonces, existe un único punto $x_0 \in X$ tal que $f(x_0) = x_0$.
\end{theorem}

\begin{proof}
	Sea $x \in X$. Consideremos la sucesión $(f^n(x))_{n \in \mathbb{N}}$, definida como $f^1(x)=f(x)$, $f^2(x)=f(f(x))$, y así sucesivamente.
	Demostremos que $(f^n(x))_{n \in \mathbb{N}}$ es una sucesión de Cauchy.

	Sean $n, m \in \mathbb{N}$ con $n \ge m$. Por la desigualdad triangular, tenemos que
	\begin{align*}
		d(f^n(x), f^m(x)) & \le d(f^n(x), f^{n+1}(x)) + \dots + d(f^{m-1}(x), f^m(x)).
	\end{align*}
	Dado que $f$ es una contracción con constante $\lambda \in (0,1)$, se tiene
	$$
		d(f^k(x), f^{k+1}(x)) = d(f(f^{k-1}(x)), f(f^k(x))) \le \lambda d(f^{k-1}(x), f^k(x)).
	$$
	Aplicando esta propiedad repetidamente, obtenemos $d(f^k(x), f^{k+1}(x)) \le \lambda^k d(x, f(x))$.
	Sustituyendo esto en la desigualdad anterior:
	\begin{align*}
		d(f^n(x), f^m(x)) & \le \lambda^n d(x, f(x)) + \lambda^{n+1} d(x, f(x)) + \dots + \lambda^{m-1} d(x, f(x)) \\
		                  & = d(x, f(x)) (\lambda^n + \lambda^{n+1} + \dots + \lambda^{m-1})                       \\
		                  & = d(x, f(x)) \lambda^n (1 + \lambda + \lambda^2 + \dots + \lambda^{m-n-1})             \\
		                  & \le d(x, f(x)) \lambda^n \sum_{k=0}^{\infty} \lambda^k                                 \\
		                  & = \frac{\lambda^n d(x,f(x))}{1-\lambda}.
	\end{align*}
	Dado que $\lambda \in (0,1)$, cuando $n \to \infty$, $\lambda^n \to 0$. Por lo tanto, $\lim_{n \to \infty} d(f^n(x), f^m(x)) = 0$. Esto demuestra que $(f^n(x))_{n \in \mathbb{N}}$ es una sucesión de Cauchy.

	Como $(X,d)$ es un espacio métrico completo, toda sucesión de Cauchy converge en $X$. Así, existe un $x_0 \in X$ tal que $\lim_{n \to \infty} f^n(x) = x_0$.
	Dado que $f$ es una contracción, es continua. Por la continuidad de $f$, tenemos:
	$$f(x_0) = f\left(\lim_{n \to \infty} f^n(x)\right) = \lim_{n \to \infty} f(f^n(x)) = \lim_{n \to \infty} f^{n+1}(x) = x_0.$$
	Por lo tanto, $x_0$ es un punto fijo de $f$.

	Ahora, demostremos que este punto fijo es único.
	Supongamos que existen dos puntos fijos, $x_0$ y $x_0'$, tales que $f(x_0) = x_0$ y $f(x_0') = x_0'$.
	Como $f$ es una contracción, sabemos que:
	$$d(x_0, x_0') = d(f(x_0), f(x_0')) \le \lambda d(x_0, x_0').$$
	Reorganizando la desigualdad, obtenemos $d(x_0, x_0') (1 - \lambda) \le 0$.
	Dado que $\lambda \in (0,1)$, tenemos $1 - \lambda > 0$. Para que la desigualdad se cumpla, debe ser $d(x_0, x_0') = 0$.
	Por la definición de métrica, $d(x_0, x_0') = 0$ implica que $x_0 = x_0'$. Esto prueba la unicidad del punto fijo.
\end{proof}


\section{Espacios topológicamente completos}

Muchos de los resultados que vimos valen para no sólo los espacios métricos completos, sino también para los espacios métricos \textit{topológicamente completos}.

\begin{definition}
	Decimos que un espacio métrico es \emph{topológicamente completo} si tiene un homeomorfismo con un espacio completo.
\end{definition}

Las propiedades topológicas son invariantes por homeomorfismos. Por ejemplo:
\begin{itemize}
	\item Separabilidad.
	\item Cerrados y clausura.
	\item Abiertos e interior.
	\item Densidad.
	\item Convergencia.
\end{itemize}

\begin{example}
	El intervalo abierto $(0, 1)$ es topológicamente completo ya que la función $f : \mathbb{R} \to (0, 1)$ dada por $t \mapsto \frac{1}{1 + e^t}$ es un homeomorfismo.
\end{example}

Más en general, tenemos la siguiente proposición.

\begin{proposition}
	Sea $X$ un espacio métrico completo. Sea $U \subseteq X$ abierto. Entonces, $U$ es topológicamente completo.
\end{proposition}

\begin{proof}
	Si $U = X$ tomamos la identidad y ya estamos.

	Sea $U \subsetneq X$. Definimos $\varphi : U \to \mathbb{R}$ dada por $x \mapsto d(x, X \setminus U)$. Por ser distancia, es continua. Consideramos
	\begin{equation*}
		\operatorname{Graf}\left(\frac{1}{\varphi}\right) = \left\{ (x, \frac{1}{\varphi (x)}) \mid x \in U \right\} \subseteq X \times \mathbb{R}.
	\end{equation*}
	Como $\operatorname{Graf}(\frac{1}{\varphi})$ es un cerrado dentro de un completo, es completo.

	No es difícil probar que $\varphi$ es un homeomorfismo.\end{proof}


\section{Teorema de Baire}

Definimos previamente algunos conceptos.

\begin{definition}
	Sea $X$ un espacio métrico. Decimos que $A$ es \emph{abierto denso} si es abierto y denso (wow).
\end{definition}

\begin{lemma}
	Sean $U, V \subseteq X$ abiertos densos. Entonces, $U \cap V$ es abierto denso.
\end{lemma}

\begin{proof}
	Que $U \cap V$ es abierto es inmediato. Sea $W \subseteq X$ un abierto arbitrario. Entonces,
	\begin{equation*}
		W \cap (U \cap V) = (W \cap U) \cap V.
	\end{equation*}
	Como $W \cap U$ es un abierto no vacío y $V$ es denso, necesariamente $W \cap (U \cap V)$ es denso. Por lo tanto, es abierto denso.
\end{proof}

\begin{proposition}
	Sea $X$ un espacio métrico topológicamente completo no vacío. Entonces, la intersección numerable de abiertos densos es no vacía.
\end{proposition}

\begin{proof}
	Sea $\left\{ V_n \right\}_{n \in \mathbb{N}}$ una familia de abiertos densos de $X$. Existe una bola cerrada $\overline{B}_1 \subseteq V_1$ de diámetro menor que $1$. Entonces, $V_2 \cap \overline{B}_1$ es abierto denso en $\overline{B}_1$. Ahora, construimos una bola cerrada $\overline{B}_2 \subseteq V_2 \cap B_1$ de diámetro menor que $\frac{1}{n}$. Y así sucesivamente, nos queda una sucesión de bolas cerradas $\overline{B}_1, \overline{B}_2, \ldots$ tales que
	\begin{equation*}
		\overline{B}_1 \supseteq \overline{B}_2 \supseteq \dots \text{ y } \overline{B}_n \subseteq V_n \cap \overline{B}_{n-1}, \forall n \in \mathbb{N}.
	\end{equation*}
	Como el diámetro de las bolas cerradas tiende a cero, tiene un punto de intersección, ya que $X$ es topolígicamente completo. Por ende, $\bigcap_{n \in \mathbb{N}} V_n$ es no vacío.
\end{proof}

\begin{theorem}
	Sea $X$ un espacio métrico topológicamente completo no vacío. La intersección numerable de abiertos densos es denso.
\end{theorem}

\begin{proof}
	Sea $\left\{ V_n \right\}_{n \in \mathbb{N}}$ una familia de abiertos densos de $X$. Sea $U \subseteq X$ un abierto no vacío arbitrario. Como $U$ es topológicamente completo y $V_n \cap U \subseteq U$ es abierto denso de $U$ por la proposición anterior,
	\begin{equation*}
		\bigcap_{n \in \mathbb{N}} (V_n \cap U) \neq \varnothing \implies \left( \bigcap_{n \in \mathbb{N}} V_n  \right) \cap U \neq \varnothing.
	\end{equation*}
	Entonces, $\bigcap_{n \in \mathbb{N}} V_n$ es denso.
\end{proof}

Podemos ver otra formulación del toerema de Baire y definimos los conjuntos nunca densos.

\begin{definition}
	Sea $X$ un espacio métrico. Decimos que $A \subseteq X$ es \emph{nunca denso} si $(\overline{A})^{\circ} = \varnothing$.
\end{definition}

\begin{remark}
	El complemento de un nunca denso es abierto denso.
\end{remark}

\begin{theorem}
	Sea $X$ un espacio métrico topológicamente completo no vacío. La unión numerable de nunca densos tiene interior vacío.
\end{theorem}

\begin{proof}
	Esto es una reformulación de Baire.
\end{proof}

Otra formulación posible es.

\begin{theorem}
	Si la unión numerable de cerrados tiene interior no vacío, entonces por lo menos un cerrado tiene interior no vacío.
\end{theorem}

\begin{proof}
	Esto es el contrarrecíproco de la formulación anterior.
\end{proof}




\chapter{Compacidad}

\section{Definición y equivalencias}

\begin{definition}
	Sea $X$ un espacio métrico. Decimos que $X$ es \emph{compacto} si todo cubrimiento abierto admite un subcubrimiento finito.
\end{definition}

De cierta forma, los espacios métricos compactos son análogos a los conjuntos finitos. Es decir, muchas de las propiedades de los conjuntos finitos ---tomar mínimo, por ejemplo--- valen también para los compactos.

\begin{definition}
	Decimos que $X$ es \emph{totalmente acotado} si para todo $\varepsilon > 0$, existe un cubrimiento finito con bolas de radio menor o igual a $\varepsilon$.
\end{definition}

\begin{remark}
	Notar la sutileza de la definición. Si bien ser totalmente acotado implica ser acotado, la vuelta no es verdad.
\end{remark}

Damos equivalencias de compacidad.
\begin{theorem}
	Sea $X$ un espacio métrico. Son equivalentes:
	\begin{enumerate}
		\item $X$ es compacto.
		\item Toda sucesión de $X$ admite una subsucesión convergente.
		\item $X$ es totalmente acotado y completo.
	\end{enumerate}
\end{theorem}

\begin{proof}
	(1 $\Rightarrow$ 2) Sea $X$ compacto. Supongamos que existe una sucesión $(x_n)_{n \in \mathbb{N}}$ que no tiene ninguna subsucesión convergente. Por lo tanto, no tiene ningún punto de acumulación. Entonces, para cada $x \in X$, existe un $\varepsilon_x > 0$ tal que $B(x, \varepsilon_x)$ contiene únicamente finitos términos de $(x_n)_{n \in \mathbb{N}}$. Con esto, armamos el cubrimiento abierto $\mathcal{U} = \left\{ B(x, \varepsilon_x) \right\}_{x \in X}$. Por compacidad, existe un subcubrimiento finito. Esto es una contradicción, ya que el subcubrimiento finito cubriría a todos los términos de la sucesión, forzando a alguna de las bolas a contener infinitos términos, contradiciendo la elección de los $\varepsilon_x$.

	(2 $\Rightarrow$ 3) Sea $X$ tal que toda sucesión admite una subsucesión convergente. Para ver la completitud, basta con tomar una sucesión de Cauchy y tomar una subsucesión convergente.
	Veamos la total acotación. Supongamos que $X$ no es totalmente acotado. Entonces, existe $\varepsilon_0 > 0$ tal que no hay un cubrimiento por finitas bolas de radio $\varepsilon$. Construimos una sucesión que no admite una subsucesión convergente. Sea $x_1 \in X$. Sea $x_2 \in X \setminus B(x_1, \varepsilon_0)$. Seguimos inductivamente y obtenemos que $x_n \in X \setminus \left( \bigcup_{i < n} B(x_i, \varepsilon_0)\right)$. Nuestra sucesión cumple que para $i \neq j$, $d(x_i, x_j) \geq \varepsilon_0$. Por hipótesis, $(x_n)_{n \in \mathbb{N}}$ tiene una subsucesión convergente $(x_{n_k})_{k \in \mathbb{N}}$. Sin embargo, $(x_{n_k})$ no es de Cauchy ya que $d(x_{n_{i}}, x_{n_j}) \geq \varepsilon_0$, por ende no es convergente.

	(3 $\Rightarrow$ 1) Sea $X$ totalmente acotado y completo. Sea $\mathcal{U} = \left\{ U_i \right\}_{i \in I}$ un cubrimiento abierto de $X$ tal que no admite un subcubrimiento finito.

	Consideramos al cubrimiento finito por bolas abiertas $\left\{ B(x_n, 1) \right\}_{1 \leq n \leq N}$. Por lo menos una de estas bolas no puede ser cubierta por ningún subcubrimiento finito de $\mathcal{U}$. Llamemos a esta bola $B_1 = B(x_1, 1)$.

	Podemos cubrir a $B_1$ con finitas bolas de radio $\frac{1}{2}$ tales que sus centros distan como mucho a $1 + \frac{1}{2}$ de $x_1$. Por el mismo argumento de antes, una de estas bolas no puede ser cubierta por ningún subcubrimiento finito de $\mathcal{U}$. Llamemos a esta bola $B_2 = B(x_2, \frac{1}{2})$.

	Repetimos el mismo proceso de $B_1$ con $B_2$ y conseguimos la bola $B_3 = B(x_3, \frac{1}{2^2})$. Procediendo inductivamente, obtenemos la sucesión $(B_n)_{n \in \mathbb{N}}$ tal que $d(x_{n-1}, x_{n}) \leq \frac{1}{2^{n-1}} + \frac{1}{2^n}$ para todo $n \in \mathbb{N}$.

	\begin{center}
		
\begin{tikzpicture}[x=0.75pt,y=0.75pt,yscale=-1,xscale=1]
	%uncomment if require: \path (0,300); %set diagram left start at 0, and has height of 300

	%Shape: Ellipse [id:dp7658406221599783] 
	\draw  [color={rgb, 255:red, 86; green, 0; blue, 145 }  ,draw opacity=1 ][fill={rgb, 255:red, 86; green, 0; blue, 145 }  ,fill opacity=0.1 ][dash pattern={on 4.5pt off 4.5pt}] (0,75) .. controls (0,33.58) and (33.58,0) .. (75,0) .. controls (116.42,0) and (150,33.58) .. (150,75) .. controls (150,116.42) and (116.42,150) .. (75,150) .. controls (33.58,150) and (0,116.42) .. (0,75) -- cycle ;

	%Shape: Ellipse [id:dp6629907428161504] 
	\draw  [color={rgb, 255:red, 86; green, 0; blue, 145 }  ,draw opacity=1 ][fill={rgb, 255:red, 86; green, 0; blue, 145 }  ,fill opacity=0.1 ][dash pattern={on 4.5pt off 4.5pt}] (120,141.9) .. controls (120,120.85) and (136.86,103.79) .. (157.67,103.79) .. controls (178.47,103.79) and (195.33,120.85) .. (195.33,141.9) .. controls (195.33,162.94) and (178.47,180) .. (157.67,180) .. controls (136.86,180) and (120,162.94) .. (120,141.9) -- cycle ;
	%Shape: Ellipse [id:dp09734770391428005] 
	\draw  [color={rgb, 255:red, 86; green, 0; blue, 145 }  ,draw opacity=1 ][fill={rgb, 255:red, 86; green, 0; blue, 145 }  ,fill opacity=0.1 ][dash pattern={on 4.5pt off 4.5pt}] (191,131.95) .. controls (191,121.98) and (198.91,113.9) .. (208.67,113.9) .. controls (218.42,113.9) and (226.33,121.98) .. (226.33,131.95) .. controls (226.33,141.92) and (218.42,150) .. (208.67,150) .. controls (198.91,150) and (191,141.92) .. (191,131.95) -- cycle ;
	%Shape: Ellipse [id:dp1342431229797264] 
	\draw  [color={rgb, 255:red, 86; green, 0; blue, 145 }  ,draw opacity=1 ][fill={rgb, 255:red, 86; green, 0; blue, 145 }  ,fill opacity=0.1 ][dash pattern={on 4.5pt off 4.5pt}] (210,114) .. controls (210,108.97) and (214.03,104.9) .. (219.01,104.9) .. controls (223.99,104.9) and (228.02,108.97) .. (228.02,114) .. controls (228.02,119.03) and (223.99,123.1) .. (219.01,123.1) .. controls (214.03,123.1) and (210,119.03) .. (210,114) -- cycle ;

	% Text Node
	\draw (61,60) node [anchor=north west][inner sep=0.75pt]  [font=\footnotesize,color={rgb, 255:red, 86; green, 0; blue, 145 }  ,opacity=1 ]  {$x_{1}$};
	% Text Node
	\draw (75.5,74.5) node  [font=\footnotesize,color={rgb, 255:red, 86; green, 0; blue, 145 }  ,opacity=1 ]  {$\bullet $};
	% Text Node
	\draw (201.4,117) node [anchor=north west][inner sep=0.75pt]  [font=\footnotesize,color={rgb, 255:red, 86; green, 0; blue, 145 }  ,opacity=1 ]  {$x_{3}$};
	% Text Node
	\draw (208.78,131.83) node  [font=\footnotesize,color={rgb, 255:red, 86; green, 0; blue, 145 }  ,opacity=1 ]  {$\bullet $};
	% Text Node
	\draw (152.4,126.08) node [anchor=north west][inner sep=0.75pt]  [font=\footnotesize,color={rgb, 255:red, 86; green, 0; blue, 145 }  ,opacity=1 ]  {$x_{2}$};
	% Text Node
	\draw (157.92,141.64) node  [font=\footnotesize,color={rgb, 255:red, 86; green, 0; blue, 145 }  ,opacity=1 ]  {$\bullet $};
	% Text Node
	\draw (204.61,96.47) node [anchor=north west][inner sep=0.75pt]  [font=\footnotesize,color={rgb, 255:red, 86; green, 0; blue, 145 }  ,opacity=1 ]  {$x_{4}$};
	% Text Node
	\draw (219.07,111.94) node  [font=\footnotesize,color={rgb, 255:red, 86; green, 0; blue, 145 }  ,opacity=1 ]  {$\bullet $};
	% Text Node
	\draw (219,95.5) node  [font=\footnotesize,color={rgb, 255:red, 86; green, 0; blue, 145 }  ,opacity=1 ,rotate=-283.93]  {$\dotsc $};


\end{tikzpicture}

	\end{center}

	Consideramos la sucesión de centros $(x_n)_{n \in \mathbb{N}}$. Sean $m, n, N \in \mathbb{N}$ tales que $N \leq m \leq n$. Entonces,
	\begin{equation*}
		d(x_m, x_n) \leq d(x_m, x_{m+1}) + d(x_{m+1}, x_{m+2}) + \dots + d(x_{n-1}, x_n) \leq \frac{1}{2^{N-2}}.
	\end{equation*}
	Por lo tanto, $(x_n)$ es de Cauchy y como $X$ es completo, tiene límite $x$.

	Sea $U_i$ tal que $x \in U_i$. Dado que $U_i$ es abierto, existe un $r > 0$ tal que $B(x, r) \subseteq U_i$. Por lo tanto, existe un $N \in \mathbb{N}$ tal que $B_{N} \subseteq B(x, r) \subseteq U_i$, lo cual es absurdo por construcción de $(B_n)$.
\end{proof}

Si estamos tratando en $\mathbb{R}^n$, podemos dar una caracterización de compacidad.

\begin{theorem}[Heine–-Borel]
	Sea $A \subseteq \mathbb{R}^n$. Entonces, $A$ es compacto si y sólo si es cerrado y acotado.
\end{theorem}

\begin{proof}
	($\Rightarrow$) Supongamos que $A$ es compacto, es decir completo y totalmente acotado. Veamos que $A$ es cerrado y acotado. Como $A$ es completo, entonces es cerrado y como es totalmente acotado, en particular es acotado.

	($\Leftarrow$) Supongamos que $A$ es cerrado y acotado. Veamos que $A$ es compacto. Basta con ver que toda sucesión de puntos de $A$ admite una subsucesión convergente.

	Sea $(x_k)_{k in \mathbb{N}}$ una sucesión de puntos de $A$. Como $A$ es acotado, existe un $R > 0$ tal que $B(0, R)$ contiene a todos los puntos de la sucesión. Entonces, por el Teorema de Bolzano–-Weierstrass, existe una subsucesión convergente $(x_{n_k})_{k \in \mathbb{N}}$ que converge a un punto $x \in B(0, R)$. Como $A$ es cerrado, entonces $x \in A$. Por lo tanto, $A$ es compacto.
\end{proof}

\begin{remark}
	Este teorema no es válido en espacios métricos generales. Por ejemplo, el conjunto $\mathbb{Q} \cap [0,1]$ es cerrado y acotado en $\mathbb{Q}$, pero no es compacto, ya que $\mathbb{Q}$ no es completo.
\end{remark}

\section{Propiedades de compactos}

Veamos algunas propiedades de los compactos.

\begin{proposition}
	Sea $X$ compacto e $Y$ un espacio métrico y sea $f : X \to Y$ continua. Entonces, $f(X)$ es compacto.
\end{proposition}

\begin{proof}
	Sea $\left\{ V_i \right\}_{i \in I}$ un cubrimiento abierto de $f(X)$. Consideramos el cubrimiento $\left\{ f^{-1}(V_i) \right\}_{i \in I}$. Dado que $X$ es compacto, consideramos un subconjunto finito $J \subseteq I $ tal que forma un subcubrimiento finito. Ahora consideramos $\left\{ V_j \right\}_{j \in J}$ y vemos que es un cubrimiento de $f(X)$. Sea $y \in f(X)$, entonces existe un $x \in X$ tal que $f(x) = y$. Sabemos que $x \in f^{-1}(V_j)$ para algún $j \in J$. Por lo tanto, $y \in V_j$.
\end{proof}

\begin{proposition}
	Sea $X$ compacto e $Y$ un espacio métrico y sea $f : X \to Y$ continua. Entonces, $f$ es uniformemente continua.
\end{proposition}

\begin{proof}
	Supongamos que $f$ no es uniformemente continua. Es decir
	\begin{equation*}
		\exists \varepsilon > 0 \mid \forall \delta > 0, \exists x, y \mid d(x, y) < \delta \text{ pero } d(f(x), f(y)) \geq \varepsilon.
	\end{equation*}
	Tomamos sucesiones $(x_n)_{n \in \mathbb{N}}$ e $(y_n)_{n \in \mathbb{\mathbb{N}}}$ tales que existe $\varepsilon > 0$ tal que
	\begin{equation*}
		d(x_n, y_n) \to 0 \text{ pero } d(f(x_n), f(y_n)) \geq \varepsilon.
	\end{equation*}

	Por compacidad de $X$, puedo tomar dos subsucesiones convergentes de $(x_n)_{n \in \mathbb{N}}$ e $(y_n)\mathbb{n \in \mathbb{N}}$ que tienden a $x_0$ e $y_0$, respectivamente. Entonces, $d(x_0, y_0) = \lim_{k \to \infty} d(x_{n_k}, y_{n_k}) = 0$, entonces $x_0 = y_0$. Sin embargo, esto implica que $d(f(x_0), f(y_0)) = 0$, lo cual es absurdo ya que $d(f(x_{n_k}), f(y_{n_k})) \to d(f(x_0), f(y_0))$ pero $d(f(x_{n_k}), f(y_{n_k})) \geq \varepsilon$.
\end{proof}

Enunciamos el Teorema de Dini, que de cierta manera nos da una forma de probar convergencia de funciones continuas en espacios compactos dada convergencia puntual más algunas hipótesis adicionales.

\begin{theorem}[Dini]
	Sea $X$ compacto y sea $(f_n)_{n \in \mathbb{N}}$ una sucesión de funciones continuas $f_n : X \to \mathbb{R}$ tal que $f_n \to f$ puntualmente y $(f_n)_{n \in \mathbb{N}}$ es monótona. Entonces $(f_n)_{n \in \mathbb{N}}$ converge uniformemente a $f$.
\end{theorem}

\begin{proof}
	Sin pérdida de generalidad, supongamos que $(f_n)_{n \in \mathbb{N}}$ es monótona creciente. Sea $\varepsilon > 0$. Definimos $g_n = f - f_n$ y consideramos $A_n = g_n^{-1}((-\infty, \varepsilon))$. Como $f_n \to f$ puntualmente, entonces para todo $x \in X$, existe un $n \in \mathbb{N}$ tal que $g_n(x) < \varepsilon$. Por lo tanto, $X = \bigcup_{n \in \mathbb{N}} A_n$.
	
	Notemos también que $A_n$ es abierto, ya que $g_n$ es continua. Por lo tanto, $\left\{ A_n \right\}_{n \in \mathbb{N}}$ es un cubrimiento abierto de $X$. Como $X$ es compacto, existe un subcubrimiento finito $\left\{ A_{n_1}, A_{n_2}, \ldots, A_{n_k} \right\}$ tal que $X = A_{n_1} \cup A_{n_2} \cup \ldots \cup A_{n_k}$.

	Como $(f_n)_{n \in \mathbb{N}}$ es monótona creciente, entonces $(g_n)_{n \in \mathbb{N}}$ es monótona decreciente. Por lo tanto, $A_n \subseteq A_{n+1}$ para todo $n \in \mathbb{N}$. Entonces, $X = A_{n_k}$. Se sigue que, para todo $x \in X$, existe un $n \in \mathbb{N}$ tal que $g_n(x) < \varepsilon$. Es decir, $|f(x) - f_n(x)| < \varepsilon$ para todo $x \in X$ y para todo $n \geq n_k$. Por lo tanto, $(f_n)_{n \in \mathbb{N}}$ converge uniformemente a $f$.
\end{proof}


\chapter{Conexión}

\section{Conjuntos conexos}

Los espacios \textit{conexos} vendrían a ser espacios que están ``conectados'' ---valga la redundancia---. En particular, se nos va a hacer más fácil definir la no-conexión.

\begin{definition}
	Sea $(X, d)$ un espacio métrico. Decimos que $X$ \textbf{no} es \emph{conexo} si se cumple alguna de las siguientes:
	\begin{enumerate}
		\item Existe $B \subseteq X$ abierto y cerrado tal que $B \neq X$.
		\item Existen $U, V \subseteq X$ abiertos disjuntos no vacíos tales que $U \cup V = X$.
		\item Existen $A, B \subseteq X$ cerrados disjuntos no vacíos tales que $A \cup B = X$.
	\end{enumerate}
\end{definition}

\begin{remark}
	Las condiciones 1., 2. y 3. son equivalentes.
\end{remark}

\begin{proof}
	Sea $(X, d)$ un espacio métrico.

	(1 $\Rightarrow$ 2) Supongamos que existe $B \subseteq X$ abierto y cerrado no vacío tal que $B \neq X$. Consideramos $B \cup (X \setminus B) = X$ y basta con ver que ambos subconjuntos son abiertos, disjuntos y no vacíos. Por hipótesis, $B$ es abierto; como también es cerrado, entonces $X \setminus B$ es abierto. Como $B$ es un subconjunto propio de $X$, entonces su complemento es no vacío.

	(2 $\Rightarrow$ 3) Supongamos que existen $U, V \subseteq X$ abiertos disjuntos no vacíos tales que $U \cup V = X$. Consideramos $A = X \setminus U$ y $B = X \setminus V$. Ambos son cerrados al ser complementos de abiertos. Dado que $U \cup V = X$, obtenemos que $A \cap B = \varnothing$, por ende son disjuntos. Veamos que su unión es $X$. Sea $x \in X$. Entonces, o bien $x \in U$ o $x \in V$. Si $x \in U$, entonces $x \not \in V$, por lo tanto $x \in X \setminus V = B$. Análogamente, Si $x \in V$, entonces $x \not \in U$, por lo tanto $x \in X \setminus U = A$. Lo cual nos dice que $X = A \cup B$.

	(3 $\Rightarrow$ 1) Supongamos que existen $A, B \subseteq X$ cerrados disjuntos no vacíos tales que $A \cup B = X$. Como $A \cup B = X$, obtenemos que $B = X \setminus A$. Dado que $A$ y $B$ son disjuntos no vacíos, $B \neq X$. También, $B$ es cerrado y, como es el complemento de un cerrado, $B$ es abierto.
\end{proof}

Ahora sí definimos conexión.

\begin{definition}
	Sea $(X, d)$ un espacio métrico. Decimos que $X$ es \emph{conexo} si no se cumple ninguna de las propiedades anteriores.
\end{definition}

Algunos ejemplos.

\begin{example}
	\begin{enumerate}
		\item Los intervalos de $\mathbb{R}$ son conexos.
		\item Las bolas en $\mathbb{R}^n$ son conexas.
		\item El espacio con un elemento es conexo.
	\end{enumerate}
\end{example}

\section{Propiedades de conexos}

Vamos a caracterizar los conexos de $\mathbb{R}$, pero antes veamos un lema que nos va a ayudar.

\begin{lemma}
	Un subconjunto $X \subseteq \mathbb{R}$ es un intervalo si y sólo si, para todo $a, b \in X$ tales que $a \leq b$, se cumple que $[a, b] \subseteq X$.
\end{lemma}

\begin{proof}
	($\Rightarrow$) Sea $X \subseteq \mathbb{R}$ un intervalo con bordes en $x, y \in \mathbb{R}$ tal que $x \leq y$. Sean $a, b \in X$ tales que $a \leq b$. Sea $c \in [a, b]$. Como $x < a \leq c \leq b < y$, entonces $c \in X$. (La demostración es casi idéntica si el intervalo $X$ es abierto, semiabierto o cerrado.)

	($\Leftarrow$) Sea $X \subseteq \mathbb{R}$ tal que para todo $a, b \in X$ tales que $a \leq b$ se cumple que $[a, b] \subseteq X$. Consideramos $[\inf X, \sup X] \subseteq \mathbb{R}$. Todo $x \in X$ pertenece al intervalo. Sea $x \in [\inf X, \sup X]$. Entonces, si $x \not \in X$, como existen $a, b \in X$ tales que $a \leq x \leq b$, llegamos a un absurdo. (Lo mismo, la demostración es casi idéntica, pero técnicamente hay que separar por casos.)
\end{proof}

Veamos cómo son los conexos de $\mathbb{R}$.

\begin{theorem}
	Sea $X \subseteq \mathbb{R}$. Entonces, $X$ es conexo si y sólo si $X$ es un intervalo.
\end{theorem}

\begin{proof}
	($\Rightarrow$) Probamos por contrarrecíproco. Supongamos que $X$ no es un intervalo. Entonces, existen un $a, b \in X$ y $r \not \in X$ tales que $a < r < b$. Por lo tanto, $((\infty, r) \cap X) \cup ((r, \infty \cap X)) = X$, entonces $X$ no es conexo.
	($\Leftarrow$) Sea $X$ no conexo. Sean $A, B \subseteq X$ abiertos disjuntos no vacíos tales que $A \cup B = X$. Consideramos $a \in A$ y $b \in B$. Definimos $c = \sup \left\{ x \in [a, b] \cap A \right\}$. O bien $c \in A$ o $c \in B$. Si $c \in A$, entonces como $A$ es abierto, $c$ no es una cota superior. Si $c \in B$, entonces como $B$ es abierto, existe una cota superior menor que $c$.
\end{proof}

Las funciones continuas, al igual que con los compactos, manda conexos a conexos.

\begin{theorem}
	Sea $X$ un espacio métrico conexo e $Y$ un espacio métrico. Sea $f : X \to Y$ continua. Entonces, $f(X)$ es conexo.
\end{theorem}

\begin{proof}
	Supongamos que $f(X)$ no es conexo. Entonces, existen $A, B \subseteq Y$ abiertos disjuntos no vacíos tales que $f(X) = A \cup B$. Tomando preimagen, $X = f^{-1}(A) \cup f^{-1}(B)$, donde $f^{-1}(A)$ y $f^{-1}(B)$ son abiertos disjuntos no vacíos. Esto implica que $X$ no es conexo, lo cual es absurdo.
\end{proof}

Este teorema nos proporciona una nueva manera de demostrar que un espacio es conexo: probamos que es imagen de un conexo por una función continua.

La proposición de a continuación nos va a ser más útil adelante.

\begin{proposition}
	Sea $Y$ un espacio métrico. Sea $X \subseteq Y$ un subespacio métrico tal que existen $A, B \subseteq X$ abiertos disjuntos no vacíos tales que $A \cup B = X$. Entonces, existen $V_{A}, V_B \subseteq Y$ abiertos disjuntos no vacíos tales que $A = V_A \cap X$ y $B = V_B \cap X$.
\end{proposition}

\begin{proof}
	Como $A$ y $B$ son abiertos en la topología subespacio de $X$, existen abiertos $V_A, V_B \subseteq Y$ tales que $A = V_A \cap X$ y $B = V_B \cap X$.

	Ahora bien, puede pasar que $V_A \cap V_B \neq \varnothing$. Pero como $A \cap B = \varnothing$, tenemos que $(V_A \cap V_B) \cap X = \varnothing$.

	Queremos modificar $V_A$ y $V_B$ para que sean disjuntos en $Y$ sin cambiar su intersección con $X$. Definimos
	\begin{equation*}
		\widetilde{V}_A := V_A \setminus \overline{V_B}, \qquad \widetilde{V}_B := V_B \setminus \overline{V_A}.
	\end{equation*}
	Ambos conjuntos son abiertos como diferencia de abierto menos cerrado.

	Veamos que siguen cortando a $X$ donde queremos. Como $A = V_A \cap X$ y $A \cap \overline{V_B} = \varnothing$, se sigue que
	\begin{equation*}
		\widetilde{V}_A \cap X = (V_A \setminus \overline{V_B}) \cap X = V_A \cap X = A.
	\end{equation*}
	Análogamente, $\widetilde{V}_B \cap X = B$.

	Finalmente, veamos que $\widetilde{V}_A$ y $\widetilde{V}_B$ son disjuntos. Si $x \in \widetilde{V}_A \cap \widetilde{V}_B$, entonces $x \in V_A$ y $x \notin \overline{V_B}$, pero también $x \in V_B$ y $x \notin \overline{V_A}$, lo cual es absurdo ya que $x \in V_B$ implica $x \in \overline{V_B}$ y análogamente para $V_A$.

	Entonces, $\widetilde{V}_A$ y $\widetilde{V}_B$ son abiertos disjuntos tales que $A = \widetilde{V}_A \cap X$ y $B = \widetilde{V}_B \cap X$.
\end{proof}

Ahora vamos a ver otra forma de demostrar que un espacio métrico es conexo.

\begin{proposition}
	Sea $X$ un espacio métrico. Entonces, $X$ es conexo si y sólo si toda función $g : X \to \left\{ 0, 1 \right\}$ continua es constante.
\end{proposition}

\begin{proof}
	($\Rightarrow$) Sea $X$ conexo y sea $g : X \to \left\{ 0, 1 \right\}$ continua. Consideramos a las preimágenes $f^{-1}(0)$ y $f^{-1}(1)$. Sabemos que $f^{-1}(0)$ y $f^{-1}(1)$ son disjuntas. Por continuidad de $f$, también son cerradas. Por último, $f^{-1}(0) \cup f^{-1}(1) = X$. Entonces, para que $X$ sea conexo, necesariamente alguna de las preimágenes es vacío.

	($\Leftarrow$) Supongamos que toda función $g : X \to \left\{ 0, 1 \right\}$ continua es constante. Supongamos que $X$ no es conexo, entonces existe una separación con $U, V \subseteq X$ cerrados disjuntos no vacíos tales que $U \cup V = X$. Definimos $f : X \to \left\{ 0, 1 \right\}$ tal que $f(U) = \left\{ 0 \right\}$ y $f(V) = \left\{ 1 \right\}$. Esto es absurdo.
\end{proof}

El siguiente teorema nos da una condición para que la unión de conexos sea conexa. Pensemos por qué no vale en general que la unión de conexos sea conexa. La idea es que la unión de dos conexos que no se ``tocan'' no es conexa.

\begin{center}
	\begin{tikzpicture}[x=0.75pt,y=0.75pt,yscale=-1,xscale=1]
	%uncomment if require: \path (0,300); %set diagram left start at 0, and has height of 300

	%Shape: Polygon Curved [id:ds20022184980649105] 
	\draw  [color={rgb, 255:red, 0; green, 86; blue, 145 }  ,draw opacity=1 ][fill={rgb, 255:red, 0; green, 86; blue, 145 }  ,fill opacity=0.1 ][line width=0.75]  (47.85,39.3) .. controls (62.33,33.02) and (130.77,31.76) .. (116.3,44.33) .. controls (102.26,56.52) and (91.54,64.77) .. (101.55,75.96) .. controls (101.86,76.31) and (102.19,76.66) .. (102.54,77.01) .. controls (114.2,88.74) and (100.53,123.52) .. (79.7,121) .. controls (58.88,118.49) and (35.66,97.37) .. (30,90) .. controls (24.34,82.63) and (33.38,45.59) .. (47.85,39.3) -- cycle ;
	%Shape: Polygon Curved [id:ds5640401880667114] 
	\draw  [color={rgb, 255:red, 0; green, 86; blue, 145 }  ,draw opacity=1 ][fill={rgb, 255:red, 0; green, 86; blue, 145 }  ,fill opacity=0.1 ][line width=0.75]  (178.05,21.82) .. controls (192.52,15.53) and (240.75,45.29) .. (250.05,62.15) .. controls (259.34,79.02) and (233.05,102.82) .. (222.05,111.82) .. controls (211.05,120.82) and (187.05,124.48) .. (178.05,111.82) .. controls (169.05,99.15) and (163.57,28.1) .. (178.05,21.82) -- cycle ;
	%Curve Lines [id:da7601976291136298] 
	\draw  [dash pattern={on 4.5pt off 4.5pt}]  (138,10.33) .. controls (165,48.67) and (134,76.33) .. (132,102.33) .. controls (130,128.33) and (139,108.33) .. (150,132.33) ;

	% Text Node
	\draw (61,60) node [anchor=north west][inner sep=0.75pt]  [font=\footnotesize,color={rgb, 255:red, 0; green, 86; blue, 145 }  ,opacity=1 ]  {$A$};
	% Text Node
	\draw (196.05,62.02) node [anchor=north west][inner sep=0.75pt]  [font=\footnotesize,color={rgb, 255:red, 0; green, 86; blue, 145 }  ,opacity=1 ]  {$B$};
	% Text Node
	\draw (16,150) node [anchor=north west][inner sep=0.75pt]   [align=left] {};


\end{tikzpicture}

\end{center}

\begin{theorem}
	Sea $X$ un espacio métrico. Sea $\left\{ A_i \right\}_I$ una familia de subconjuntos conexos de $X$. Si existe $x \in X$ tal que, para todo $i \in I$, $x \in A_i$, entonces $\bigcup_{i \in I} A_i$ es conexo.
\end{theorem}

\begin{proof}
	Sea $g : \bigcup_{i \in I} A_j \to \left\{ 0, 1 \right\}$ una función continua cualquiera. Sea $x \in X$ tal que, para todo $i \in I$, $x \in A_i$. Entonces, consideramos $f(x)$. Si consideramos $f|_{A_i}$, entonces como es una función continua de un conexo a $\left\{ 0, 1 \right\}$, es constante. O sea, para todo $i \in I$, $f|_{A_i} = f(x)$. Entonces, $f$ es constante.
\end{proof}

\begin{remark}
	No hace falta que todos los conjuntos compartan un punto, con que para todo $\alpha, \beta \in I$, existan $i_1, i_2, \ldots, i_n \in I$ tales que
	\begin{equation*}
		A_{\alpha} \cap A_{i_1} \neq \varnothing, A_{i_1} \cap A_{i_2} \neq \varnothing, \ldots, A_{i_{n-1}} \cap A_{i_n} \neq \varnothing, A_{i_n} \cap A_{\beta} \neq \varnothing.
	\end{equation*}

	\begin{center}
		\begin{tikzpicture}[x=0.75pt,y=0.75pt,yscale=-1,xscale=1]
	%uncomment if require: \path (0,300); %set diagram left start at 0, and has height of 300

	%Shape: Polygon Curved [id:ds5691606692406826] 
	\draw  [color={rgb, 255:red, 0; green, 86; blue, 145 }  ,draw opacity=1 ][fill={rgb, 255:red, 0; green, 86; blue, 145 }  ,fill opacity=0.1 ][line width=0.75]  (47.85,39.3) .. controls (62.33,33.02) and (130.77,31.76) .. (116.3,44.33) .. controls (102.26,56.52) and (91.54,64.77) .. (101.55,75.96) .. controls (101.86,76.31) and (102.19,76.66) .. (102.54,77.01) .. controls (114.2,88.74) and (100.53,123.52) .. (79.7,121) .. controls (58.88,118.49) and (35.66,97.37) .. (30,90) .. controls (24.34,82.63) and (33.38,45.59) .. (47.85,39.3) -- cycle ;
	%Shape: Polygon Curved [id:ds015378215570174936] 
	\draw  [color={rgb, 255:red, 0; green, 86; blue, 145 }  ,draw opacity=1 ][fill={rgb, 255:red, 0; green, 86; blue, 145 }  ,fill opacity=0.1 ][line width=0.75]  (101,30) .. controls (115.48,23.72) and (187.75,43.14) .. (197.05,60) .. controls (206.34,76.86) and (180.05,100.67) .. (169.05,109.67) .. controls (158.05,118.67) and (134.05,122.33) .. (125.05,109.67) .. controls (116.05,97) and (86.52,36.28) .. (101,30) -- cycle ;
	%Shape: Polygon Curved [id:ds79267022071116] 
	\draw  [color={rgb, 255:red, 0; green, 86; blue, 145 }  ,draw opacity=1 ][fill={rgb, 255:red, 0; green, 86; blue, 145 }  ,fill opacity=0.1 ][line width=0.75]  (168,63) .. controls (182.48,56.72) and (248.7,32.14) .. (258,49) .. controls (267.3,65.86) and (241,89.67) .. (230,98.67) .. controls (219,107.67) and (200,90) .. (198,79) .. controls (196,68) and (153.52,69.28) .. (168,63) -- cycle ;
	%Curve Lines [id:da15620468455330305] 
	\draw [color={rgb, 255:red, 0; green, 86; blue, 145 }  ,draw opacity=1 ] [dash pattern={on 4.5pt off 4.5pt}]  (57,58) .. controls (71.07,47.45) and (91.29,28.38) .. (104.67,40.58) .. controls (118.04,52.78) and (160.24,84.53) .. (186.17,65.08) .. controls (212.1,45.64) and (215.17,96.58) .. (240.17,71.58) ;

	% Text Node
	\draw (61,60) node [anchor=north west][inner sep=0.75pt]  [font=\footnotesize,color={rgb, 255:red, 0; green, 86; blue, 145 }  ,opacity=1 ]  {$A_{1}$};
	% Text Node
	\draw (140.05,63.02) node [anchor=north west][inner sep=0.75pt]  [font=\footnotesize,color={rgb, 255:red, 0; green, 86; blue, 145 }  ,opacity=1 ]  {$A_{2}$};
	% Text Node
	\draw (16,150) node [anchor=north west][inner sep=0.75pt]   [align=left] {};
	% Text Node
	\draw (211,60) node [anchor=north west][inner sep=0.75pt]  [font=\footnotesize,color={rgb, 255:red, 0; green, 86; blue, 145 }  ,opacity=1 ]  {$A_{3}$};
\end{tikzpicture}

	\end{center}
\end{remark}

\begin{proposition}
	Sea $X$ un espacio métrico. Sea $A \subseteq X$ un subespacio métrico conexo denso. Entonces, $X$ es conexo.
\end{proposition}

\begin{proof}
	Sea $g : X \to \left\{ 0, 1 \right\}$ una función continua cualquiera. Entonces, $g|_A$ es constante. Por continuidad de $g$, la extensión de $g|_A$ a $X$ es igual a $g$. Por lo tanto, $g$ es constante y entonces $X$ es conexo.
\end{proof}

\begin{corollary}
	Sea $A$ conexo. Si $A \subseteq B \subseteq \overline{A}$, entonces $B$ es conexo.
\end{corollary}

\begin{proof}
	Como $B \subseteq \overline{A}$, entonces $A$ es denso en $B$. Por lo tanto, $B$ es conexo.
\end{proof}

Veamos un ejemplo poco intuitivo de aplicar el corolario.

\begin{example}
	En $\mathbb{R}^{2}$, consideramos al conjunto $X = (\bigcup_{n \in \mathbb{N}} \left\{ \frac{1}{n} \right\} \times [0, 1]) \cup [0, 1] \cup \{(0, 1)\}$.

	\begin{center}
		% GRÁFICO
\begin{tikzpicture}[xscale=5, yscale=3]
	% Configuración de los ejes
	\draw[->] (-0.1,0) -- (1.1,0) node[below] {$x$};
	\draw[->] (0,-0.1) -- (0,1.1) node[left] {$y$};

	% Marcas en los ejes
	% Se envuelve \xtext en $...$ para asegurar que el contenido matemático se interprete correctamente.
	% Se usa {1} en lugar de 1 para que el foreach lo trate como un argumento agrupado.
	\foreach \x/\xtext in {1/{1}, 0.5/{\frac{1}{2}}, 0.333/{\frac{1}{3}}, 0.25/{\frac{1}{4}}, 0.2/{\frac{1}{5}}}
	{
	\draw (\x,0.02) -- (\x,-0.02) node[below] {$\xtext$};
	}

	% Para indicar que se acumulan hacia 0
	\draw (0.1,0.02) -- (0.1,-0.02);
	\draw (0.05,0.02) -- (0.05,-0.02);

	\draw (0.02,1) -- (-0.02,1) node[left] {$1$};
	\draw (0.02,0.5) -- (-0.02,0.5) node[left] {$0.5$};

	% Parte 2: El segmento horizontal (0,1) en el eje x
	% Interpretamos (0,1) como el intervalo abierto en el eje x: {(x,0) | 0 < x < 1}
	\draw[line width=1.5pt, accentcolor] (0,0) -- (1,0);

	% Parte 1: Los segmentos verticales {1/n} x [0,1]
	% Dibujamos los primeros N dientes (por ejemplo, hasta n=5)
	\foreach \n in {1,2,3,4,5} {
			\pgfmathsetmacro{\xcoord}{1/\n}
			\draw[line width=1.5pt, accentcolor] (\xcoord,0) -- (\xcoord,1);
		}

	% Indicamos la acumulación de los dientes hacia el eje Y
	% con puntos suspensivos o una línea punteada que se difumina
	\node[blue, align=center, scale=0.8] at (0.11,0.5) {$\ldots \ldots$};
	\draw[line width=1.5pt, accentcolor, opacity=0.5] (0.02,0) -- (0.02,1); % Un "diente" simbólico muy cerca del eje Y

	% Etiqueta del conjunto
	\node[below right, accentcolor, scale=1.2] at (0.6,0.8) {$X$};

\end{tikzpicture}

	\end{center}

	Probar que $X$ es conexo.
\end{example}

\begin{proof}[Solución]
	Consideramos al conjunto $\bigcup_{n \in \mathbb{N}} \left\{ \frac{1}{n} \right\} \times [0, 1]$ y probamos que es conexo. Cada recta $\left\{ \frac{1}{n} \right\} \times [0, 1]$ es conexa, para todo $n \in \mathbb{N}$. Y la recta $[0, 1] \times \left\{ 0 \right\}$ es conexa e interseca a las rectas verticales en el punto $(\frac{1}{n}, 0)$. Por lo tanto, la unión es conexa.

	Como $(0, 1)$ es punto límite del conjunto, está en la clausura y entonces $X$ es conexo.
\end{proof}

\section{Componentes conexas}

Las componentes conexas de un espacio métrico las podemos pensar como los conjuntos conexos más grandes del espacio.

\begin{definition}
	Sea $X$ un espacio métrico y $x \in X$. Definimos a la \emph{componente conexa} de $x$ como
	\begin{equation*}
		C_x = \bigcup_{A \subseteq X \text{ conexo, } x \in A} A.
	\end{equation*}
\end{definition}

\begin{remark}
	Sean $x, y \in X$. Entonces, o bien $C_x \cap C_y = \varnothing$ o $C_x = C_y$. Y obviamente, $X = \cup_{x \in X} C_x$.
\end{remark}

A partir de esto, definimos los espacios totalmente disconexos.

\begin{definition}
	Sea $X$ un espacio métrico. Decimos que $X$ es \emph{totalmente disconexo} si toda componente conexa es un punto.
\end{definition}

\begin{example}
	El espacio métrico $(\mathbb{Q}, |\cdot|)$ es totalmente disconexo.
\end{example}

\begin{proof}[Solución]
	Sea $A \subseteq \mathbb{Q} \setminus \varnothing$. Entonces, existe un número irracional $x$ tal que $a < x < b$, para $a, b \in \mathbb{Q}$. Consideramos $(-\infty, x) \cap \mathbb{Q}$ y $(x, \infty) \cap \mathbb{Q})$. Ambos son abiertos disjuntos no vacíos. Entonces, ningún subconjunto con cardinal mayor a $1$ es conexo en $\mathbb{Q}$. Por lo tanto, como todo conjunto con cardinal $1$ es conexo, $\mathbb{Q}$ es totalmente disconexo.
\end{proof}












\part{Espacios normados}

\chapter{Espacios de funciones}

En este capítulo, cuando hablamos de funciones nos referimos únicamente a funciones acotadas. Recordemos cómo se definía este espacio métrico.

\begin{definition}
	Sea $Y$ un conjunto y $(Z, d^Z)$ un espacio métrico. Definimos el espacio de funciones de $Y$ a $Z$ acotadas como
	\begin{equation*}
		B(Y, Z) = \left\{ f : Y \to Z \mid f \text{ es acotada} \right\}.
	\end{equation*}
\end{definition}

De ahora en adelante, consideramos al espacio de funciones $(B(Y, Z), d_{\infty})$.

\section{Función evaluación y Ley Exponencial}

\begin{definition}
	Sea $X = B(Y, Z)$. Definimos las funciones
	\begin{enumerate}
		\item La evaluación: $\ev : X \times Y \to Z$ tal que $\ev(f, y) = f(y)$.
		\item La evaluación en un punto $y$: $\ev_y : X \to Z$ tal que $\ev_y(f) = f(y)$.
	\end{enumerate}
\end{definition}

\begin{remark}
	La función $\ev_y$ es continua.
\end{remark}

Ahora sea $Y$ también un espacio métrico. Entonces, podemos considerar alguna de las métricas
\begin{equation*}
	d_{X \times Y} ((f, y), (f', y')) =
	\begin{cases}
		d_X (f, f') + d_Y (y, y'), \text{ o,}
		\max \left\{d_X (f, f'), d_Y (y, y')  \right\}.
	\end{cases}
\end{equation*}
Cualquiera de estas dos verifica que una sucesión es convergente en $X \times Y$ si y sólo si es convergente coordenada a coordenada en $X$ e $Y$, respectivamente.

\begin{proposition}
	Sea $X$ el espacio de funciones de $Y$ a $Z$ continuas y acotadas. Entonces, $\ev : X \times Y \to Z$ es continua.
\end{proposition}

\begin{proof}
	Veamos la continuidad mediante sucesiones. Sea $(f_n, y_n)_{n \in \mathbb{N}}$ una sucesión convergente en $X \times Y$. Recordemos que $(f_n, y_n)_{n \in \mathbb{N}}$ converge si y sólo si $(f_n)_{n \in \mathbb{N}}$ e $(y_n)_{n \in \mathbb{N}}$ convergen. Sean respectivamente $f$ e $y$ los límites.

	Veamos que $\ev(f_n, y_n) \longrightarrow \ev(f, y)$. Calculamos
	\begin{align*}
		d_Z(f_n(y_n), f(y)) & \leq d_Z(f_n(y_n), f(y_n)) + d_Z(f(y_n), f(y))            \\
		                    & \leq \sup_{t \in Y} d_Z(f_n(t), f(t)) + d_Z(f(y_n), f(y)) \\
		                    & = d_X(f_n, f) + d_Z(f(y_n), f(y))
	\end{align*}

	Dado que $f_n \to f$ en $X$, el primer término $d_X(f_n, f) \longrightarrow 0$, y como $f$ es continua e $y_n \longrightarrow y$, $d_Z(f(y_n), f(y)) \longrightarrow 0$.
\end{proof}

Veamos la ley exponencial.

\begin{proposition}
	Sean $X$ e $Y$ espacios métricos y sea $K$ compacto. Entonces, existe una biyección natural entre $C(X \times K, Y)$ y $C(X, C(K, Y))$. En particular es
	\begin{align*}
		f : X \times K \to Y   & \mapsto \tilde{f} : X \to C(K, Y) \\
		\text{tal que }f(x, k) & = \tilde{f}(x)(k).
	\end{align*}
\end{proposition}

\begin{proof}
	\color{red} COMPLETAR
\end{proof}

\section{Equicontinuidad}

\begin{definition}
	Sean $X$ e $Y$ espacios métricos y sea $\mathcal{F} \subseteq C(X, Y)$ una familia de funciones continuas de $X$ a $Y$. Decimos que $\mathcal{F}$ es \emph{equicontinua} en $x_0 \in X$ si
	\begin{center}
		\begin{minipage}{0.9\linewidth}
			para todo $\varepsilon > 0$, existe $\delta > 0$ tal que $f(B_X(x_0, \varepsilon)) \subseteq B_Y(f(x_0, \varepsilon))$, para todo $f \in \mathcal{F}$.
		\end{minipage}
	\end{center}
	Si $\mathcal{F}$ es equicontinua en todo $X$, decimos que $\mathcal{F}$ es \emph{equicontinua}.
\end{definition}

\begin{remark}
	Como con la continuidad usual, podemos usar la distancia en vez de bolas.
\end{remark}

\begin{definition}
	Sean $X$ e $Y$ espacios métricos y sea $\mathcal{F} \subseteq C(X, Y)$. Decimos que $\mathcal{F}$ es \emph{uniformemente equicontinua} si
	\begin{center}
		\begin{minipage}{0.9\linewidth}
			para todo $\varepsilon > 0$, existe $\delta > 0$ tal que $d_X(x, x') < \delta$ implica $d_Y(f(x), f(x')) < \varepsilon$ para todo $x, x' \in X$ y $f \in \mathcal{F}$.
		\end{minipage}
	\end{center}
\end{definition}

\begin{remark}
	Si para todo $f \in \mathcal{F}$, $f$ es Lipschitz con la misma constante, entonces $\mathcal{F}$ es equicontinua.
\end{remark}

\begin{proposition}
	Sea $\left\{ f_n \right\}_{n \in \mathbb{N}}$ equicontinua que converge puntualmente a $f : X \to Y$. Entonces, $\overline{\left\{ f_n \right\}}_{n \in \mathbb{N}}$ es equicontinua.
\end{proposition}

\begin{proof}
	Sean $x_0 \in X$ y $\varepsilon > 0$. Por equicontinuidad de $\overline{\left\{ f_n \right\}}_{n \in \mathbb{N}}$, existe $\delta > 0$ tal que
	\begin{equation*}
		d(x_0, x) < \delta \implies d(f_n(x_0), f_n(x)) < \varepsilon, \quad \text{para todo }n \in \mathbb{N}.
	\end{equation*}
	Tomamos límite en $n \to \infty$,
	\begin{equation*}
		d(f(x_0), f(x)) \leq \varepsilon.
	\end{equation*}
	Por lo tanto, $\overline{\left\{ f_n \right\}}_{n \in \mathbb{N}}$ es equicontinua.
\end{proof}

\begin{remark}
	La familia $\mathcal{F}$ es equicontinua en $x_0$ si y sólo si
	\begin{equation*}
		\sup_{f \in \mathcal{F}} d(f(x), f(x_0)) \xrightarrow{x \to x_0} 0.
	\end{equation*}
\end{remark}

\begin{lemma}
	Sea $X$ un espacio métrico e $Y$ un espacio métrico completo y sea $A \subseteq X$ denso. Sea $\left\{ f_n \right\}_{n \in \mathbb{N}} \subseteq C(X, Y)$ equicontinua. Si $(f_n)_{n \in \mathbb{N}}$ converge puntualmente en $A$, entonces converge puntualmente en $X$.
\end{lemma}

\begin{proof}
	Sea $x \in X$. Veamos que $(f_n(x))_{n \in \mathbb{N}}$ es de Cauchy. Calculamos
	\begin{equation*}
		d(f_n(x), f_m(x)) \leq d(f_n(x), f_n(a)) + d(f_n(a), f_m(a)) + d(f_m(a), f_m(x)).
	\end{equation*}
	Por equicontinuidad de $(f_n(x))_{n \in \mathbb{N}}$, existe $\delta$ tal que
	\begin{equation*}
		d(x, a) < \delta \implies d(f_n(x), f_n(a)), d(f_m(a), f_m(x)) < \frac{\varepsilon}{3}.
	\end{equation*}
	También, como $(f_n(a))_{n \in \mathbb{N}}$ es de Cauchy, existe $N \in \mathbb{N}$ tal que
	\begin{equation*}
		d(f_n(a), f_m(a)) < \frac{\varepsilon}{3}, \quad \forall m, n \geq N.
	\end{equation*}
	Entonces,
	\begin{equation*}
		d(f_n(x), f_m(x)) \leq \frac{\varepsilon}{3} + \frac{\varepsilon}{3} + \frac{\varepsilon}{3},
	\end{equation*}
	probando que existe el límite puntual.
\end{proof}

\begin{lemma}
	Sea $K$ compacto e $Y$ un espacio métrico y sea $\left\{ f_n \right\}_{n \in \mathbb{N}}$ equicontinua. Si $\left( f_n \right)_{n \in \mathbb{N}}$ converge puntualmente, entonces converge uniformemente.
\end{lemma}

\begin{proof}
	Supongamos que $f$ no converge uniformemente. Entonces, existe $\varepsilon > 0$ y $\left\{ f_{n_k} \right\}_{k \in \mathbb{N}}$ y $(x_k)_{k \in \mathbb{N}}$ tal que
	\begin{equation*}
		d(f_{n_k}(x_k), f(x_k)) \geq \varepsilon.
	\end{equation*}
	Tomando una subsucesión convergente, podemos suponer que $(x_k)_{k \in \mathbb{N}}$ converge. Entonces,
	\begin{align*}
		\varepsilon \leq d(f_{n_k}(x_k), f_{n_k}(x_0)) + d(f_{n_k}(x_0), f(x_0)) + d(f(x_0), f(x_k)).
	\end{align*}
	Y siguiendo pasos similares a la demostración anterior, llegamos a un absurdo.
\end{proof}

Un resultado no tan sorprendente.

\begin{proposition}
	Sea $K$ compacto e $Y$ un espacio métrico y sea $\mathcal{F} \subseteq C(K, Y)$ equicontinua. Entonces, $\mathcal{F}$ es uniformemente equicontinua.
\end{proposition}

\begin{proof}
	Sea $\varepsilon > 0$. Como $\mathcal{F}$ es equicontinua, para cada $x \in K$, existe $\delta_x > 0$ tal que
	\begin{equation*}
		d_K(x, x') < \delta_x \implies d_Y(f(x), f(x')) < \varepsilon, \quad \forall x' \in K, f \in \mathcal{F}.
	\end{equation*}
	Dado que $K$ es compacto y $\{B(x, \delta_x)\}_{x \in K}$ forma un cubrimiento abierto, tomamos un subcubrimiento finito $\{B(x_n, \delta_{x_n})\}_{1 \leq n \leq N}$ y consideramos $\delta = \min_{1 \leq n \leq N} \delta_{x_n}$. Entonces, para todo $x \in K$,
	\begin{equation*}
		d_K(x, x') < \delta_x \implies d_Y(f(x), f(x')) < \varepsilon, \quad \forall x' \in K, f \in \mathcal{F}.
	\end{equation*}
	Por lo tanto, $\mathcal{F}$ es uniformemente continua.
\end{proof}

\begin{definition}
	Sea $\mathcal{F}$ una familia de funciones de $X$ a $Y$. Decimos que $F$ es \emph{equiacotada} si
	\begin{center}
		para todo $x \in X$, $\mathcal{F}(x) = \left\{ f(x) \mid f \in \mathcal{F} \right\}$ es totalmente acotado.
	\end{center}
	Además, decimos que $\mathcal{F}$ es \emph{totalmente equiacotada} si $\mathcal{F}(X) = \left\{ f(x) \mid x \in X, f \in \mathcal{F} \right\}$ es totalmente acotado.
\end{definition}

\begin{remark}
	Un espacio es totalmente acotado si y sólo si su clausura es compacta.
\end{remark}

Ahora vemos el teorema de Arzela--Ascoli.

\begin{theorem}
	Sea $K$ compacto e $Y$ completo. Sea $\mathcal{F} \subseteq C(K, Y)$. Son equivalentes:
	\begin{enumerate}
		\item $\mathcal{F}$ es equicontinua y equiacotada.
		\item $\mathcal{F}$ es totalmente acotado.
		\item $\mathcal{F}$ es uniformemente equicontinua y uniformemente equiacotado.
	\end{enumerate}
\end{theorem}

\begin{proof}
	(1 $\Rightarrow$ 2) Sea $\mathcal{F}$ equicontinua y equiacotada. Basta con ver que $\overline{\mathcal{F}}$ es compacto.

	Sea $(f_n)_{n \in \mathbb{K}}$ una sucesión en $\mathcal{F}$. Como $K$ es compacto, en particular, es separable. Por lo tanto, consideramos $A \subseteq K$ denso numerable. Queremos ver que existe una subsucesión $(f_{n_k})_{k \in \mathbb{N}}$ tal que converge puntualmente en $A$. Tomamos las sucesiones $(f_n(a_i))_{n \in \mathbb{N}}$ con $i \in \mathbb{N}$ fijo. Dado que $\mathcal{F}$ es equiacotada, $\left\{ f_n(a_i) \right\}_{n \in \mathbb{N}}$ es totalmente acotado y entonces $\overline{\left\{ f_n(a_i) \right\}}_{n \in \mathbb{N}}$ es compacto. Por lo tanto, podemos tomar un subconjunto $\mathcal{M} \subseteq \mathbb{N}$ tal que $(f_m(a_i))_{m \in \mathcal{M}}$ converge para todo $i \in \mathbb{N}$. Entonces, la subsucesión $(f_m)_{m \in \mathcal{M}}$ converge puntualmente en $A$.

	Dado que $(f_m)_{m \in \mathcal{M}}$ converge puntualmente en $A$ denso, entonces converge uniformemente.

	(2 $\Rightarrow$ 3) Sea $\mathcal{F}$ totalmente acotado. Entonces, $\overline{\mathcal{F}}$ es compacto. Por lo tanto, consideramos
	\begin{equation*}
		\mathcal{F}(K) = \ev (K \times \mathcal{F}) \subseteq \ev (K \times \overline{\mathcal{F}}).
	\end{equation*}
	Dado que $\ev$ es continua y $K \times \overline{\mathcal{F}}$ es compacto, $\mathcal{F}$ es totalmente acotado.

	Por la ley exponencial,
	\begin{equation*}
		C(K \times \overline{\mathcal{F}}, Y) \cong C(K, C(\overline{\mathcal{F}}, Y)).
	\end{equation*}
	Como $\ev : K \to C(\overline{F}, Y)$ es continua sobre un compacto, entonces es uniformemente continua. Por lo tanto, $\mathcal{F}$ es uniformemente equicontinua.

	(3 $\Rightarrow$ 1) Inmediato.
\end{proof}

\section{Sucesiones y series de funciones}

Recordemos las definiciones de convergencia puntual y uniforme.

\begin{definition}
	Sea $(f_n)_{n \in \mathbb{R}}$ una sucesión de funciones de $\mathbb{R}$ en $\mathbb{R}$ y $f : \mathbb{R} \to \mathbb{R}$. Decimos que converge \emph{puntualmente} a $f$ si para todo $x \in \mathbb{R}$ y $\varepsilon > 0$, existe $N \in \mathbb{N}$ tal que
	\begin{equation*}
		\left\lvert f(x) - f_n(x) \right\rvert < \varepsilon, \quad \text{para todo }n \geq N.
	\end{equation*}
\end{definition}

Y convergencia uniforme.

\begin{definition}
	Sea $(f_n)_{n \in \mathbb{R}}$ una sucesión de funciones de $\mathbb{R}$ en $\mathbb{R}$ y $f : \mathbb{R} \to \mathbb{R}$. Decimos que converge \emph{puntualmente} a $f$ si para todo $\varepsilon > 0$, existe $N \in \mathbb{N}$ tal que
	\begin{equation*}
		\left\lvert f(x) - f_n(x) \right\rvert < \varepsilon, \quad \text{para todo }n \geq N, x \in \mathbb{R}.
	\end{equation*}
\end{definition}

\begin{remark}
	La convergencia uniforme implica la convergencia puntual.
\end{remark}

Veamos algunas propiedades útiles.

\begin{proposition}
	Sea $(f_n)_{n \in \mathbb{N}}$ una sucesión de funciones reales.
	\begin{enumerate}
		\item \textbf{Continuidad del Límite Uniforme:} Si $(f_n)_{n \in \mathbb{N}}$ es una sucesión de funciones continuas definida en un conjunto $E$ y $(f_n)_{n \in \mathbb{N}}$ converge uniformemente a $f$ en $E$, entonces la función límite $f$ es continua en $E$.
		\item \textbf{Intercambio de Derivación y Límite:} Sea $(f_n)_{n \in \mathbb{N}}$ una sucesión de funciones diferenciables en un intervalo $[a,b]$. Si la sucesión de las derivadas $(f_n')_{n \in \mathbb{N}}$ converge uniformemente a una función $g$ en $[a,b]$ y existe un punto $x_0 \in [a,b]$ tal que la sucesión numérica $(f_n(x_0))_{n \in \mathbb{N}}$ converge (puntualmente), entonces la sucesión $(f_n)_{n \in \mathbb{N}}$ converge uniformemente a una función $f$ en $[a,b]$, $f$ es diferenciable en $[a,b]$, y su derivada es $f'(x) = g(x)$. Es decir, se puede intercambiar el límite con la derivación:
		      $$\left( \lim_{n \to \infty} f_n(x) \right)' = \lim_{n \to \infty} f_n'(x)$$
		\item \textbf{Intercambio de Integración y Límite:} Si $(f_n)_{n \in \mathbb{N}}$ es una sucesión de funciones integrables en un intervalo $[a,b]$ y $(f_n)_{n \in \mathbb{N}}$ converge uniformemente a $f$ en $[a,b]$, entonces la función límite $f$ es integrable en $[a,b]$ y se puede intercambiar el límite con la integración:
		      $$\lim_{n \to \infty} \int_a^b f_n(x) \, dx = \int_a^b \lim_{n \to \infty} f_n(x) \, dx = \int_a^b f(x) \, dx$$
	\end{enumerate}
\end{proposition}

\begin{proof}
	No lo demuestro.
\end{proof}

\section{Series de funciones}

\begin{definition}
	Sea $(f_n)_{n \in \mathbb{N}}$ una sucesión de funciones. Entonces, la serie $\sum_{n = 1}^{\infty} f_n(x)$ converge \emph{puntualmente} si la sucesión $(S_{N}(x))_{N \in \mathbb{N}}$ lo hace. Análogo para convergencia uniforme. Y converge absolutamente si la serie de $(\left\lvert f_n \right\rvert)_{n \in \mathbb{N}})$ lo hace.
\end{definition}

\begin{proposition}
	Sea $(f_n)_{n \in \mathbb{N}}$ una sucesión de funciones continuas tal que $\sum f_n(x)$ converge uniformemente. Entonces, $\sum f_n(x)$ es continua.
\end{proposition}

\begin{proof}
	Como $f_n$ es continua para todo $n \in \mathbb{N}$, entonces $S_n$ también lo es ya que es suma de continuas.
\end{proof}

Veamos criterios de convergencia puntual.

\begin{proposition}
	Sea $(f_n)_{n \in \mathbb{N}}$ una sucesión de funciones continuas. Consideramos $a_n = f_n(x)$.
	\begin{enumerate}
		\item \textbf{Criterios de D'Alembert y de la raíz:} \\Sea $L = \lim_{n \to \infty} \left\lvert \frac{a_{n+1}}{a_n} \right\rvert$ o $L = \limsup \sqrt[n]{\left\lvert a_n \right\rvert}$.
		      \begin{itemize}
			      \item Si $L < 1$, entonces la serie converge absolutamente.
			      \item Si $L > 1$, entonces la serie diverge.
			      \item Si $L = 1$, no sabemos nada.
		      \end{itemize}

		\item \textbf{Criterio de Liebniz:} Si $\lim_{n \to \infty} a_n = 0$ y es monotónica decreciente, entonces la serie alternada converge.
	\end{enumerate}
\end{proposition}


Y uno para la convergencia uniforme.

\begin{proposition}
	Sea $(f_n)_{n \in \mathbb{N}}$ una sucesión de funciones tal que existe $(M_n)_{n \in \mathbb{N}}$ una sucesión de reales tales que $\sum M_n$ converge y, para todo $x$, $\left\lvert f_n(x) \right\rvert \leq M_n$. Entonces, $\sum f_n$ converge uniformemente.
\end{proposition}

Por último, para series de potencias.

\begin{proposition}
	Sea $\sum a_n x^n$ una serie de potencias. Entonces, el radio de convergencia $R = \frac{1}{\limsup \sqrt[n]{\left\lvert a_n \right\rvert}}$. Es decir, en el intervalo $[-R, R]$, la serie converge uniformemente.
\end{proposition}





\chapter{Espacios normados}

\section{Definición y ejemplos}

Cuando hablamos de espacios normados, siempre vamos a pensar en espacios vectoriales sobre $\mathbb{R}$ o $\mathbb{C}$.

\begin{definition}
	Un \emph{espacio normado} es un par $(E, \left\lVert \cdot \right\rVert)$ donde $E$ es un espacio vectorial sobre $K$ ($\mathbb{R}$ o $\mathbb{C}$) y una norma $\left\lVert \cdot \right\rVert : E \to \mathbb{R}_{\ge 0}$ tal que
	\begin{enumerate}
		\item $\left\lVert x \right\rVert = 0$ si y sólo si $x = 0$.
		\item $\left\lVert \lambda x \right\rVert = \left\lvert \lambda \right\rvert \left\lVert x \right\rVert$, para todo $x \in E$, $\lambda \in K$.
		\item $\left\lVert x + y \right\rVert \leq \left\lVert x \right\rVert + \left\lVert y \right\rVert$, para todo $x, y \in E$.
	\end{enumerate}
\end{definition}

Junto con la norma del espacio, conseguimos una distancia dada por $d(x, y) = \left\lVert x - y \right\rVert$.

\begin{remark}
	La distancia inducida por la norma es invariante por traslaciones, es decir $d(x, y) = d(x + z, y + z)$ para todo $x, y, z \in E$.
\end{remark}

No tan complicado de demostrar:

\begin{remark}
	En espacios normados, $\overline{B}(x, r) = \overline{B(x, r)}$.
\end{remark}

\begin{proposition}
	La suma y el producto por escalar son continuos.
\end{proposition}

\begin{proof}
	Veamos la suma $+ : E \times E \to E$. Notamos que
	\begin{equation*}
		\left\lVert (x + x') - (y + y')\right\rVert \leq \left\lVert x - y \right\rVert + \left\lVert x' - y' \right\rVert.
	\end{equation*}

	Para el producto,
	\begin{equation*}
		\left\lVert kx - k' x' \right\rVert \leq \left\lvert k \right\rvert \left\lVert x - x' \right\rVert + \left\lvert k' - k \right\rvert \left\lVert x' \right\rVert.
	\end{equation*}
\end{proof}

\begin{remark}
	La suma y el producto por un escalar fijo son uniformemente continuas.
\end{remark}

Veamos algunos ejemplos de espacios normados:

\begin{example}
	\color{red} COMPLETAR.
\end{example}

\begin{definition}
	Un espacio normado es de \emph{Banach} si es completo.
\end{definition}

\begin{definition}
	Un espacio normado es de \emph{Hilbert} si tiene un producto interno que induce su norma.
\end{definition}


\section{Funciones lineales}

La definición es la misma que en Álgebra Lineal.

\begin{definition}
	Sean $E$ y $F$ espacios normados sobre $K$. Una función $f : E \to F$ es \emph{lineal} si
	\begin{enumerate}
		\item $f(x + y) = f(x) + f(y)$, para todo $x, y \in E$.
		\item $f(\lambda x) = \lambda f(x)$, para todo $x \in E$, $\lambda \in K$.
	\end{enumerate}
\end{definition}

Principalmente nos van a interesar las funciones continuas.

\begin{definition}
	Denotamos por $\mathcal{L}(E, F)$ al espacio (vectorial) de funciones \emph{lineales continuas} de $E$ a $F$.
\end{definition}

Probamos varias equivalencias útiles de la continuidad.

\begin{proposition}
	Sean $E$ y $F$ espacios normados, y sea $f : E \to F$ una función lineal. Son equivalentes:
	\begin{enumerate}
		\item $f$ es continuo en $0$;
		\item $\exists x_0 \in E$ tal que $f$ es continuo en $x_0$;
		\item $f$ es continuo;
		\item $f$ es uniformemente continuo;
		\item $\exists M > 0$ tal que $\|fx\| \le M\|x\|$ para todo $x \in E$ ($f$ es acotado);
		\item $\forall A \subset E$ acotado, $f(A)$ es acotado.
	\end{enumerate}
\end{proposition}

\begin{proof}
	No lo demuestro, pero tampoco es tan complicado.
\end{proof}

En particular, definimos la norma de una función lineal continua.

\begin{definition}
	Sean $E$ y $F$ espacios normados y $f : E \to F$ lineal continua. Entonces, definimos la \emph{norma} de $f$ como
	\begin{equation*}
		\left\lVert f \right\rVert = \sup_{\left\lVert x \right\rVert \leq 1} \left\lVert f(x) \right\rVert.
	\end{equation*}
\end{definition}

Y esto nos lleva a considerar el espacio $(\mathcal{L}, \left\lVert \cdot \right\rVert)$.

\begin{proposition}
	El par $(\mathcal{L}(E, F), \left\lVert \cdot \right\rVert)$ es un espacio normado.
\end{proposition}

\begin{proof}
	Realmente no es difícil.
\end{proof}

\begin{remark}
	Si $F$ es completo, entonces $\mathcal{L}(E, F)$ es de Banach.
\end{remark}

\begin{definition}
	Dos normas $\left\lVert \cdot \right\rVert_1$ y $\left\lVert \cdot \right\rVert_2$ de un espacio normado $E$ son \emph{equivalentes} si existe un homeomorfismo entre $(E, \left\lVert \cdot \right\rVert_1)$ y $(E, \left\lVert \cdot \right\rVert_2)$.
\end{definition}

\begin{remark}
	Esto es equivalente a decir que existen constantes $c, c' \in K$ tales que
	\begin{equation*}
		c \left\lVert x \right\rVert_2 \leq \left\lVert x \right\rVert_1 \leq c' \left\lVert x \right\rVert_2 \quad \text{para todo }x \in E.
	\end{equation*}
\end{remark}

La siguiente proposición relaciona el núcleo de un funcional lineal continuo con su continuidad.

\begin{proposition}
	Sean $E$ un espacio normado sobre $K$ y sea $f : E \to K$. Entonces, $f$ es continua si y sólo si $\ker f$ es cerrado.
\end{proposition}

\begin{proof}
	($\Rightarrow$) Por definición, $\ker f = f^{-1}(\left\{ 0 \right\})$ que claramente es cerrado.

	($\Leftarrow$) Si $\ker f = E$, entonces $f = 0$ y ya estamos. Supongamos que $\ker f \subsetneq E$. Por lo tanto, existe $x_0 \in E \setminus \ker f$ y entonces $\ker f \oplus \left\langle x_0 \right\rangle = E$.

	Sea $x \in E$. Podemos escribir a $x$ como
	\begin{equation*}
		x = \lambda x_0 + s \quad \text{donde } \lambda \in K, s \in \ker f.
	\end{equation*}
	Sin pérdida de generalidad, tomamos $\left\lVert x_0 \right\rVert = 1$.
	Buscamos $M \in \mathbb{R}$ tal que
	\begin{equation*}
		\left\lvert f(x) \right\rvert \leq M \left\lVert x \right\rVert \quad \text{para todo }x \in E.
	\end{equation*}
	Notamos que
	\begin{equation*}
		x = f(x) x_0 + (x - f(x) x_0).
	\end{equation*}
	Y $(x - f(x) x_0) \in \ker f$ ya que $f(x - f(x) x_0) = f(x) - f(x) = 0$. Ahora tomamos norma y obtenemos
	\begin{equation*}
		\left\lvert f(x) \right\rvert =
	\end{equation*}
	Para que la descomposición $x = f(x) x_0 + (x - f(x) x_0)$ sea general, primero debemos normalizar $x_0$ respecto a la función $f$. Como $\ker f \subsetneq E$, existe un $x_1 \in E$ tal que $f(x_1) \ne 0$. Tomamos $x_0 = \frac{x_1}{f(x_1)}$, de modo que $f(x_0) = 1$. Con esta elección, la descomposición es válida para cualquier $x \in E$.

	Llamemos $s_x = x - f(x) x_0$. Ahora, como por hipótesis $\ker f$ es un subespacio cerrado y $x_0 \notin \ker f$ (pues $f(x_0)=1$), la distancia de $x_0$ al núcleo es estrictamente positiva. Sea
	\begin{equation*}
		d = \inf_{s \in \ker f} \left\lVert x_0 - s \right\rVert > 0.
	\end{equation*}
	Para cualquier $x \in E$ con $f(x) \ne 0$, el vector $-s_x/f(x)$ también pertenece a $\ker f$. Por la definición de ínfimo, tenemos que:
	\begin{equation*}
		\left\lVert x_0 - \left( -\frac{s_x}{f(x)} \right) \right\rVert = \left\lVert x_0 + \frac{s_x}{f(x)} \right\rVert \ge d.
	\end{equation*}
	A partir de esta desigualdad, podemos acotar la norma de $x$:
	\begin{equation*}
		\left\lVert x \right\rVert = \left\lVert f(x) x_0 + s_x \right\rVert = \left\lvert f(x) \right\rvert \left\lVert x_0 + \frac{s_x}{f(x)} \right\rVert \ge \left\lvert f(x) \right\rvert d.
	\end{equation*}
	De esta relación podemos finalmente despejar $\left\lvert f(x) \right\rvert$:
	\begin{equation*}
		\left\lvert f(x) \right\rvert \le \frac{1}{d} \left\lVert x \right\rVert.
	\end{equation*}
	Esta desigualdad también es cierta si $f(x)=0$, pues daría $0 \le \frac{1}{d} \left\lVert x \right\rVert$. Al haber encontrado una constante $M = 1/d$ tal que $\left\lvert f(x) \right\rvert \le M \left\lVert x \right\rVert$ para todo $x \in E$, concluimos que $f$ es acotado y, por lo tanto, continuo.
\end{proof}

Una proposición útil.

\begin{proposition}
	Sea $E$ un espacio normado y $S, T \subseteq E$ subespacios vectoriales tales que $S$ es cerrado y $T$ tiene dimensión finita. Entonces, $S + T$ es cerrado.
\end{proposition}

\begin{proof}
	Probemos primero el caso de $\dim T = 1$ y luego procedemos por inducción. Dado que $\dim T = 1$, existe $x_0 \in E$ tal que $T = \left\langle x_0 \right\rangle$. Si $x_0 \in S$, entonces $S + T = S$ es cerrado.

	Supongamos que $x_0 \not \in S$. Sea $(y_n)_{n \in \mathbb{N}}$ una sucesión de $S \oplus \left\langle x_0 \right\rangle$ convergente. Entonces,
	\begin{equation*}
		y_n = \lambda_n x_0 + s_n.
	\end{equation*}
	Sea $\varphi : S \oplus \left\langle x_0 \right\rangle \to K$ tal que $\varphi(y) = \lambda$, donde $\lambda$ es el coeficiente de $x_0$. Es fácil ver que $\varphi$ es lineal y que $\ker \varphi = S$. Como el núcleo de $\varphi$ es cerrado, $\varphi$ es continua y además, como es lineal, entonces es uniformemente continua.

	Dado que $\varphi$ es uniformemente continua, $(\lambda_n)_{n \in \mathbb{N}}$ es de Cauchy en $K$, por ende converge a $\lambda \in K$. Por último, como $\lambda_n$ converge a $\lambda$ y $s_n$ converge a $s$, ya que $S$ es cerrado, entonces
	\begin{equation*}
		\lim_{n \to \infty} y_n = s + \lambda x_0 \in S \oplus \left\langle x_0 \right\rangle.
	\end{equation*}
	Por lo tanto, $S + T$ es cerrado.

	Si $\dim T > 1$, repetimos hasta completar.
\end{proof}


\section{Extensión de funcionales}

Nuestro objetivo es extender un funcional definido en un subespacio al espacio entero, pero preservando la norma.

\begin{theorem}
	Sea $E$ un espacio normado sobre $K$ y $S \subseteq E$ un subespacio vectorial. Sea $\varphi : S \to K$ lineal y continua. Entonces, existe $\tilde{\varphi} : E \to K$ lineal y continua tal que
	\begin{equation*}
		\tilde{\varphi}|_S = \varphi \quad \text{y} \quad \left\lVert \tilde{\varphi} \right\rVert = \left\lVert \varphi \right\rVert.
	\end{equation*}
\end{theorem}

\begin{proof}
	Usamos el lema de Zorn. Sea
	$$\wp = \left\{ (W, \varphi_W)  \mid S \subseteq W \subseteq E, \varphi_W|_S = \varphi, \left\lVert \varphi_W \right\rVert = \left\lVert \varphi \right\rVert \right\}$$
	y $(W, \varphi_W) \preceq (W', \varphi_{W'})$ si $W \subseteq W'$ y $\varphi_{W'}|_{W} = \varphi_{W}$.

	Sea $\mathcal{C} \subseteq \wp$ una cadena. Entonces, $W = \bigcup_{i \in I} W_i$ y $\varphi_W (x) = \varphi_{W_i}(v)$ es una cota superior de la cadena. Veamos que preserva la norma. Calculamos
	\begin{equation*}
		\left\lVert \varphi_{W}(x) \right\rVert = \left\lVert \varphi_{W_i}(x) \right\rVert \leq \left\lVert \varphi_{W_i} \right\rVert \left\lVert x \right\rVert = \left\lVert \varphi \right\rVert \left\lVert x \right\rVert.
	\end{equation*}
	Entonces, $\left\lVert \varphi_{W} \right\rVert = \left\lVert \varphi \right\rVert$.

	Por Zorn, existe un elemento maximal $(W, \varphi_{W})$. Necesariamente $W = E$, ya que sino extendemos con $W + \left\langle x \right\rangle$.
\end{proof}

Lo de extender con $W + \left\langle x \right\rangle$ lo vimos en clase y no tengo tiempo de escribirlo.

\begin{proposition}
	Sea $E$ un espacio normado sobre $\mathbb{R}$. Sean $v_1, v_2, \ldots $ numerables vectores con norma $1$. Si $E = \overline{\left\langle v_1, v_2, \ldots \right\rangle}$, entonces $E$ es separable.
\end{proposition}

\begin{proof}
	Consideramos a $E$ sobre $\mathbb{Q}$. Entonces, $E = \overline{\left\langle v_1, v_2, \ldots \right\rangle_{\mathbb{R}}} \subseteq \overline{\overline{\left\langle v_1, v_2, \ldots \right\rangle_{\mathbb{Q}}}}$, el cual es un conjunto numerable.
\end{proof}

\begin{remark}
	Tomando $e_n = (0, \ldots, 0, 1, 0 \ldots)$, entonces vemos que $\ell^p$ es separable y $\ell^{\infty}$ no.
\end{remark}

\begin{theorem}
	Si $E^*$ es separable, entonces $E$ es separable.
\end{theorem}

\begin{proof}
	\color{red} COMPLETAR
\end{proof}

\begin{lemma}
	Sean $E$ e $F$ espacios de Banach. Sea $T \in \mathcal{L}(E, F)$. Entonces, existe $M > 0$ y $r \in (0, 1)$ tales que
	\begin{center}
		\begin{minipage}{0.9\linewidth}
			si para todo $y \in F$, existe $x \in E$ tal que
			\begin{equation*}
				\left\lVert y - T(x) \right\rVert \leq r \left\lVert y \right\rVert \quad \text{ y } \quad \left\lVert x \right\rVert \leq M \left\lVert y \right\rVert,
			\end{equation*}
			entonces para todo $y \in F$, existe $x \in E$ tal que
			\begin{equation*}
				y = T(x) \quad \text{ y } \quad \left\lVert x \right\rVert \leq \frac{M}{1-r} \left\lVert y \right\rVert.
			\end{equation*}
		\end{minipage}
	\end{center}
\end{lemma}

\part{Ejercicios Surtidos}

\chapter{Ejercicios Surtidos}

Este capítulo es en preparación para el final. Contiene la resolución de distintos ejercicios que me fui encontrando en el camino y que me parecieron interesantes.

\begin{exercise}
    Todas las normas en $\mathbb{R}^n$ son equivalentes.
\end{exercise}

\begin{proof}
    Sean $\| \cdot \|$ y $\| \cdot \|'$ normas arbitrarias de $\mathbb{R}^n$. Recordemos que dos normas son equivalentes si existen constantes $C_1, C_2 > 0$ tales que para todo $x \in \mathbb{R}^n$ se cumple:
    \begin{equation*}
        C_1 \| x \| \leq \| x \|' \leq C_2 \| x \|.
    \end{equation*}
    Un camino más fácil es probar que toda norma es equivalente a $\| \cdot \|_{\infty}$, ya que la equivalencia de normas es transitiva.

    Veamos primero que existe una constante $C_1 > 0$ tal que para todo $x \in \mathbb{R}^n$ se cumple:
    \begin{equation*}
        \| x \| \leq C_1 \| x \|_{\infty}.
    \end{equation*}
    Consideramos
    \begin{align*}
        \| x \| = \| x_1 \cdot e_1 + \dots + x_n \cdot e_n \| &\leq |x_1| \cdot \| e_1 \| + \dots + |x_n| \cdot \| e_n \| \\
        &\leq \| x \|_{\infty} \cdot (\| e_1 \| + \dots + \| e_n \|),
    \end{align*}
    y como $\| e_1 \| + \dots + \| e_n \|$ es constante, entonces podemos tomar $C_1 = \| e_1 \| + \dots + \| e_n \|$, obteniendo así 
    \begin{equation*}
        \| x \| \leq C_1 \| x \|_{\infty}.
    \end{equation*}
    
    Ahora, veamos que existe una constante $C_2 > 0$ tal que para todo $x \in \mathbb{R}^n$ se cumple:
    \begin{equation*}
        \| x \|_{\infty} \leq C_2 \| x \|.
    \end{equation*}
    Definimos al conjunto $S = \{ x \in \mathbb{R}^n : \| x \| = 1 \}$, que es cerrado y acotado, por lo que es compacto. Definimos también la función $\varphi : S \to \mathbb{R}$ dada por $\varphi(x) = \| x \|_{\infty}$. Esta función es continua porque $\| x \|_{\infty}$ es una norma, y por lo tanto es continua en $\mathbb{R}^n$. Como $S$ es compacto, $\varphi$ alcanza su máximo en $S$ en un punto $x_0$.

    Todo $x \in \mathbb{R}^n$ puede escribirse como $x = \| x \| \cdot \frac{x}{\| x \|}$ si $\| x \| \neq 0$, y por lo tanto
    \begin{align*}
        \| x \|_{\infty} &= \left\| \frac{x}{\| x \|} \right\|_{\infty} \cdot \| x \| \\
        &\leq \varphi(x_0) \cdot \| x \|.
    \end{align*}
    Basta con tomar $C_2 = \varphi(x_0)$, y así obtenemos la desigualdad buscada.
\end{proof}

\begin{exercise}
    Hallar el cardinal de $C(\mathbb{R}, \mathbb{R})$, el espacio de funciones continuas de $\mathbb{R}$ a $\mathbb{R}$.
\end{exercise}

\begin{proof}
    Dado que $\mathbb{Q}$ es denso en $\mathbb{R}$, las funciones continuas en $\mathbb{R}$ están completamente determinadas por sus valores en $\mathbb{Q}$. Es decir, si $f, g \in C(\mathbb{R}, \mathbb{R})$ y $f(q) = g(q)$ para todo $q \in \mathbb{Q}$, entonces $f = g$. Por lo tanto, existe un biyección entre $C(\mathbb{R}, \mathbb{R})$ y $C(\mathbb{Q}, \mathbb{R})$. Entonces, basta con calcular el cardinal de $C(\mathbb{Q}, \mathbb{R})$.

    Enumeramos a $\mathbb{Q}$ como $\{ q_1, q_2, \ldots \}$. Para definir una función en $C(\mathbb{Q}, \mathbb{R})$, necesitamos asignar un valor real a cada $q_i$. Como $\mathbb{R}$ tiene cardinalidad $2^{\aleph_0}$, el número de funciones de $\mathbb{Q}$ a $\mathbb{R}$ es $(2^{\aleph_0})^{\aleph_0} = 2^{\aleph_0 \cdot \aleph_0} = 2^{\aleph_0}$. Por lo tanto, el cardinal de $|C(\mathbb{R}, \mathbb{R})| \leq 2^{\aleph_0}$. Y para ver que el cardinal es igual, el conjunto de funciones constantes es un subconjunto de $C(\mathbb{R}, \mathbb{R})$ y tiene cardinalidad $2^{\aleph_0}$, ya que cada función constante puede ser identificada con un número real. Por lo tanto, el cardinal de $C(\mathbb{R}, \mathbb{R})$ es exactamente $2^{\aleph_0}$.
\end{proof}

\begin{exercise}
    Sea $(X, d)$ un espacio métrico completo sin puntos aislados y sea $D \subseteq X$ un subconjunto denso numerable. Probar que $D$ no es un $G_\delta$.
\end{exercise}

\begin{proof}
    Lo probamos por absurdo. Supongamos que $D$ es un conjunto $G_\delta$, entonces
    \begin{equation*}
        D = \bigcap_{n=1}^{\infty} U_n,
    \end{equation*}
    donde cada $U_n$ es un conjunto abierto y como $D \subseteq U_n$, también es denso. Consideramos ahora el complemento de $D$,
    \begin{equation*}
        X \setminus D = \bigcup_{n=1}^{\infty} (X \setminus U_n).
    \end{equation*}
    Cada $X \setminus U_n$ es el compleme de un conjunto abierto denso, por lo tanto es nunca denso, entonces tiene interior vacío. Por el Teorema de Baire, la unión numerable de nunca densos tiene interior vacío.           
\end{proof}

\begin{exercise}
    Probar que:
    \begin{enumerate}
        \item[(a)] Las componentes conexas de un espacio métrico son cerradas.
        \item[(b)] Si el espacio tiene finitas componentes conexas, las componentes son abiertas.
        \item[(c)] Dar un ejemplo de un espacio métrico con una componente conexa no abierta.
        \item[(d)] Si $f : E \to \mathbb{Z}$ es continua, entonces $f$ es constante en cada componente conexa de $E$.
    \end{enumerate}
\end{exercise}

\begin{proof}[Solución]
    (a) Sea $C$ una componente cerrada de un espacio métrico $X$. Probamos que la clausura de un conjunto conexo es conexo. 

    Sea $A \subseteq X$ un conjunto conexo. Para ver que $\overline{A}$ es conexo, basta con probar que $\overline{A}$ no puede ser escrito como la unión disjunta de dos conjuntos abiertos no vacíos. Supongamos que $\overline{A} = U \cup V$ con $U$ y $V$ abiertos y no vacíos, y $U \cap V = \emptyset$. Entonces, $U \cap A$ y $V \cap A$ son conjuntos abiertos no vacíos en $A$, y por lo tanto, $A$ se puede escribir como la unión disjunta de estos dos conjuntos, lo cual contradice la conexidad de $A$. Por lo tanto, $\overline{A}$ es conexo.

    Usando esto, dado que $C$ es una componente conexa, como $C \cap \overline{C} \neq \emptyset$ y $\overline{C}$ es conexo, entonces $\overline{C} \subseteq C$. Por lo tanto, $C$ es cerrado.

    (b) Sea $X$ un espacio métrico con finitas componentes conexas. Podemos escribir a $X$ como la unión disjunta de sus componentes conexas,
    \begin{equation*}
        X = C_1 \cup C_2 \cup \dots \cup C_n,
    \end{equation*}
    donde cada $C_i$ es una componente conexa. En particular, como cada $C_i$ es cerrado y hay finitas componentes conexas, $\bigcup_{j \neq i} C_j$ es cerrado. Por lo tanto, su complemento 
    \begin{equation*}
        X \setminus \bigcup_{j \neq i} C_j = C_i
    \end{equation*}
    es abierto. Entonces, cada componente conexa es abierta.

    (c) Un ejemplo de un espacio métrico con una componente conexa no abierta es el conjunto $\{ \frac{1}{n} \mid n \in \mathbb{N} \} \cup \{ 0 \}$. Sea $r > 0$. La bola abierta $B(0, r)$ contiene a algún $\frac{1}{N}$ para $N$ suficientemente grande, por lo tanto $\{ \frac{1}{n} \mid n \in \mathbb{N} \} \cup \{ 0 \}$ no es abierto.
    
    (d) Sea $f : E \to \mathbb{Z}$ continua. Sea $C$ una componente conexa de $E$. Como $C$ es conexa, no se puede expresar como unión de abiertos disjuntos no vacíos. Recordemos que todo subconjunto de $\mathbb{Z}$ es abierto, por lo tanto su preimagen es abierta. Supongamos que $f|_C$ no es constante, entonces podemos expresar a $C$ como la unión disjunta de dos conjuntos abiertos no vacíos, $f|_C^{-1}(a)$ y $f|_C^{-1}(b)$, donde $a, b \in \mathbb{Z}$ y $a \neq b$. Esto contradice que $C$ sea conexo. Por lo tanto, $f|_C$ es constante.
\end{proof}

\end{document}
